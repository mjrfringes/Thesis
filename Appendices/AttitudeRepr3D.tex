\chapter{Attitude representation in three dimensions}
\label{sec:attituderepresentation}
%\subsection{Definitions and conventions}

There are three common representations of the orientation, or \textit{attitude}, of an object in a 3-dimensional Euclidian reference frame: in the following we will discuss the Tait-Bryan angles (which are very similar to, and sometimes confused with proper Euler angles), rotation matrices, and quaternions. All of them can be understood as a rotation of the initial reference frame $I = \{\I,\J,\K\}$ into the object's local reference frame $L = \{\i,\j,\k\}$. The reference frame $I$ is assumed to be fixed while $L$ is allowed to move. We can write each unit vector as follows: $\I = \fromto{}{I}[1,0,0]^T$, $\J = \fromto{}{I}[0,1,0]^T$, $\K = \fromto{}{I}[0,0,1]^T$, and $\i = \fromto{}{L}[1,0,0]^T$, $\j = \fromto{}{L}[0,1,0]^T$, $\k = \fromto{}{L}[0,0,1]^T$. $\{\I,\J,\K\}$ and $\{\i,\j,\k\}$ are  orthonormal bases to $I$ and $L$, respectively. The subscript before the vector indicates in which reference frame the vector is expressed, and the $T$ after the vector indicates the transpose operation. We will keep this formalism for all vectors and matrices in this work.

\section{Tait-Bryan/Euler angles}
\label{sec:Tait-Bryan}
The Tait-Bryan formalism [CITE] corresponds to a sequence of three angles, each corresponding to a rotation about one of the object's main axes: these are also called "intrinsic" rotations. They differ from "extrinsic" rotation, sometimes called "Euler angles", which correspond to a rotation about one of the axes of the global (fixed) reference frame. In the following, we will focus on using exclusively intrinsic rotations, as they are more intuitive. Note that sometimes people call this formalism "Euler angles" as well, so it is important to understand how this works.
With this formalism, we start in the global reference frame and rotate the reference frame three times to end up in the 
\textit{body} reference frame, which describes the final orientation of an object. We will most often choose a well-known sequence of rotation such as the $z-y'-x''$ order, which corresponds to the angles used to describe the heading, elevation and bank of an aircraft with respect to a reference frame attached to the Earth, for example the North-East-Down reference frame. The first rotation about $\k$ will transform $I$ into $L'$. The second rotation, about the $\j$ axis of the rotated frame $L'$, transforms $L'$ into $L''$. The third and last rotation, about the $\i$ axis of $L''$, will transform $L''$ into the final orientation, $L$, of the object (see Fig.~\ref{fig:3simpleRotate}).

This sequence of rotation can be used to represent the rotation matrix that describes the attitude of an image of the sky. Celestial coordinates are usually given in terms of right ascension, declination. To fully describe the image of a patch of sky, we need another degree of freedom, which is the roll of the image about the boresight. When given these three angles: RA, DEC, and ROLL, one can reconstruct the attitude using the Tait-Bryan angles in the $z-y'-x''$ order, where the first, second and third elementary rotations correspond to the rotations in right ascension, declination and roll, respectively.



\section{Rotation matrices}
\label{sec:rotationMatrices}
Perhaps the most common way to express the orientation of an object within a given reference frame is to use the matrix that describes the rotation from one reference frame to the other. Since rotations are linear transformations of $\RealNumbers^3$, there always exists a matrix to represent it. If we choose an orthonormal basis to $\RealNumbers^3$, matrices representing rotations are $3\times 3$ orthogonal matrices. When given the traditional matrix multiplication operation, $3\times 3$ orthogonal matrices with determinant of $+1$ form a group which is an isomorphism of the group of all 3-D rotations of Euclidian space (subsequently called SO(3) for "special orthogonal group") [REFERENCE?]: it means that each rotation can always be represented by exactly one $3\times 3$ orthogonal matrix. This theorem is the mathematical translation of the sometimes obvious intuition that rotation matrices always exist, are unique for a given rotation, and that the composition of two rotations is still a rotation. It also expresses the requirement that the corresponding rotation matrices have a determinant of $+1$, which can be useful when we consider numerical implementations of these matrices, as rounding errors might require a periodic normalization of the matrices to ensure they stay in this group. Note that the group of rotation is a cyclic group, since a rotation of an angle $\theta$ is the same as a rotation of $\theta+2\pi$.

We are interested in matrices describing rotations of entire coordinate systems, which are also called \textit{passive} rotations. This is different from matrices describing rotations of vectors within a given coordinate system (called \textit{active} rotations), and an important distinction that can often lead to confusion. Let's suppose that we have an initial coordinate system $I$ of basis $\{\I,\J,\K\}$, and a second coordinate system $L$ of basis $\{\i,\j,\k\}$. For example, this applies when $L$ is the body reference frame, and we want to understand its orientation with respect to an initial reference frame, such as the inertial reference frame. The basis vectors of $L$ can all be expressed by a linear combination of the basis vector of $I$. This transformation can be described using the \textit{direction cosine matrix}, which has the following expression:

\begin{equations}
\fromto{I}{L}\R = 
\begin{bmatrix}  \I\cdot\i & \J\cdot\i  & \K\cdot\i\\
                \I\cdot\j &\J\cdot\j  & \K\cdot\j \\
				\I\cdot\k & \J\cdot\k  & \K\cdot\k
\end{bmatrix}.
\end{equations}

The columns of this matrix correspond to the expression of the basis vectors of $I$ expressed in the basis of $L$. This is what we call the \textit{rotation matrix} between $I$ and $L$, and transforms vectors expressed in $I$ into their representation in $L$. With this convention, the matrix pre-multiplies the vector. For example, if we have some vector $\fromto{}{I}\vectors{u}$ expressed in the initial reference frame $I$, its expression in the reference frame $L$ will be $\fromto{}{L}\vectors{u} = \fromto{I}{L}\R \fromto{}{I}\vectors{u}$.

%Most often, a rotation matrix is defined as the matrix $\R$ that \textit{rotates} a vector from $L$ to $I$, and can be constructed by expressing the unit vectors $\i$, $\j$ and $\k$ of $L$ in the reference frame of $I$. These three vectors form the columns of the rotation matrix $\R$. So, if we construct the matrix $\R = [\fromto{}{I}\i,\fromto{}{I}\j,\fromto{}{I}\k]$, we notice that this matrix would multiply vectors expressed in the local reference frame $L$ and express them in the reference frame $I$. In the rest of this work we will always refer to $\fromto{L}{I}\R$ as this rotation matrix, where the subscript designates the starting reference frame (the reference frame in which the vectors that post-multiply the matrix will be expressed), and the superscript designates the reference frame in which the resulting vector will be expressed. 

Note that the rotation matrix $\fromto{I}{L}\R$ is an orthogonal matrix of determinant $+1$: each columns are orthogonal with each other and of unit norm. Hence, the inverse of this matrix is its transpose, which also corresponds to the rotation of a vector from frame $I$ to frame $L$: $(\fromto{I}{L}\R)^{-1} = (\fromto{I}{L}\R)^T = \fromto{L}{I}\R$.

%This convention is widespread in the literature, but can be source of confusion. A given vector can be thought of as being \textit{rotated} from one reference frame to another, as described above, which is also called an \textit{active transformation}; alternatively, one can think of the vector staying still, and the coordinate systems rotating, which is also called a \textit{passive transformation}. This is the representation that is the most useful for our purpose, and will lead to simplifications down the road, especially when this formalism is related to quaternions. 

%What is the matrix that corresponds to the \textit{passive} transformation from $I$ to $L$? In other words, what is the matrix that will pre-multiply vectors originally expressed in $I$ and express them in $L$? It is precisely the inverse (or transpose) of the matrix $\R = [\fromto{}{I}\i,\fromto{}{I}\j,\fromto{}{I}\k]$ described earlier, so that $\fromto{}{L}\vectors{u} = \fromto{I}{L}\R \fromto{}{I}\vectors{u}$. 

Let's take an example and consider the unit vector $\fromto{}{I}\vectors{u} = \fromto{}{I}(1, 0, 0)$, expressed in $I$ originally. Now, let's rotate the coordinate frame $I$ by an angle $\theta$ with respect to the axis $\k$. The new reference frame is $L' = \{\i',\j',\k'\}$. For simplification, let's consider that $\theta = +90$~degrees. It is clear that the vector $\i$ is now equal to $-\j'$, and $\fromto{}{L'}\i = \fromto{}{L'}(0,-1,0)$. 

\begin{figure}
\begin{center}
\tdplotsetmaincoords{0}{0}

%start tikz picture, and use the tdplot_main_coords style to implement the display 
%coordinate transformation provided by 3dplot
\begin{tikzpicture}[scale=4.5,tdplot_main_coords,label/.style={%
   postaction={ decorate,%transform shape,
   decoration={ markings, mark=at position .8 with \node #1;}}}]

\def\rvec{0.8}
\coordinate (O) at (0,0,0);
%draw the main coordinate system axes
\draw[line width=0.3mm,->] (0,0,0) -- (\rvec,0,0) node[below,align=left]{$\X$};
\draw[line width=0.3mm,->] (0,0,0) -- (0,\rvec,0) node[above left]{$\Y$};
\draw[line width=0.3mm] (0,0,0) -- (0,0,\rvec) node[anchor=north]{$\Z=\z'$};

\def\rotangle{30}
\tdplotsetrotatedcoords{0}{0}{\rotangle}
\draw[line width=0.3mm,->,tdplot_rotated_coords,color=blue] (0,0,0) -- (\rvec,0,0) node[below,align=left]{$\x'$};
\draw[line width=0.3mm,->,tdplot_rotated_coords,color=blue] (0,0,0) -- (0,\rvec,0) node[above left]{$\y'$};

\tdplotdrawarc[tdplot_main_coords,->]{(0,0,0)}{\rvec/3}{0}{\rotangle}{align=center,anchor=west}{$\theta$}


\end{tikzpicture}
\caption[Rotation of 30~degrees about $\Z$]{The $\{\x',\y',\z'\}$ reference frame (in blue) is rotated with respect to $\{\X,\Y,\Z\}$ (in black). The rotation is about the axis $\Z$ by an angle $\theta=30$~degrees.}
\label{fig:TopViewSimpleRotate}
\end{center}
\end{figure}

In the more general case, let's suppose that the local reference frame $L'$ is rotated by an angle $\theta$ about the $\k$ axis (Fig.~ \ref{fig:simpleRotate}) with respect to the reference frame $I$. The convention we adopt sets the rotation matrix for this transformation as being:
\begin{equations}
\fromto{I}{L'}\R = \R_\k(\theta) = 
\begin{bmatrix} \cos\theta & \sin\theta & 0 \\
				-\sin\theta & \cos\theta & 0 \\
                0 & 0 & 1
\end{bmatrix},
\end{equations}
where $\k$ indicates the third axis of the current basis ($\i$ and $\j$ represent the first and second axes, respectively). This will transform vectors from $I$ to $L'$. Suppose now that we further rotate our reference frame by an angle $\phi$ about the newly-rotated $\j'$ axis. The rotation for this elementary transformation is:
\begin{equations}
\fromto{L'}{L''}\R = \R_\j(\phi) = 
\begin{bmatrix} \cos\phi  & 0 & -\sin\phi\\
                0  & 1 & 0 \\
				\sin\phi  & 0 & \cos\phi
\end{bmatrix}.
\end{equations}
And let's do one last rotation about $\i''$, of an angle $\psi$, for which the transformation matrix is:

\begin{equations}
\fromto{L''}{L}\R = \R_\i(\psi) = 
\begin{bmatrix}  1& 0  & 0\\
                0 &\cos\psi  &  \sin\psi \\
				0 & -\sin\psi  &  \cos\psi
\end{bmatrix}.
\end{equations}

The matrix that corresponds to the active transformation of $I$ to $L$ will multiply vectors expressed in $I$ and express them in $L$. Hence, this matrix can be written:
\begin{equations}
\fromto{I}{L}\R = 
\fromto{L''}{L}\R\fromto{L'}{L''}\R\fromto{I}{L'}\R = \R_\i(\psi)\R_\j(\phi)\R_\k(\theta),
\end{equations}
where we pre-multiply the matrix for each consecutive rotation of reference frames. This corresponds to the "natural order" of rotations [CITE], and is especially relevant when related to quaternions. While the first axis of rotation, $\k$, is defined in the initial reference frame, it is important to realize that the axes corresponding to the second and third rotations are defined in the intermediate frames $L'$ and $L''$, respectively. We can understand this by thinking that the transformations follow the \textit{body}, as each rotation is done in the body reference frame, and is a particularly useful approach to our problem.


\def\rvec{0.8}
\begin{figure}
\begin{center}
\tdplotsetmaincoords{70}{110}

%start tikz picture, and use the tdplot_main_coords style to implement the display 
%coordinate transformation provided by 3dplot
\begin{tikzpicture}[scale=4.5,tdplot_main_coords,label/.style={%
   postaction={ decorate,%transform shape,
   decoration={ markings, mark=at position .8 with \node #1;}}}]


\coordinate (O) at (0,0,0);
%draw the main coordinate system axes
\draw[line width=0.3mm,->] (0,0,0) -- (\rvec,0,0) node[below,align=left]{$\X$};
\draw[line width=0.3mm,->] (0,0,0) -- (0,\rvec,0) node[above left]{$\Y$};
\draw[line width=0.3mm,->] (0,0,0) -- (0,0,\rvec) node[above] {$\K = \k'$} node[midway]{\AxisRotator[rotate=-90,->]};


\def\rotangle{-30}
\tdplotsetrotatedcoords{\rotangle}{0}{0}
\draw[line width=0.3mm,->,tdplot_rotated_coords,color=blue] (0,0,0) -- (\rvec,0,0) node[below,align=left]{$\x'$};
\draw[line width=0.3mm,->,tdplot_rotated_coords,color=blue] (0,0,0) -- (0,\rvec,0) node[above left]{$\y'$};

\tdplotdrawarc[tdplot_main_coords,->]{(0,0,0)}{\rvec/3}{0}{\rotangle}{align=center,anchor=north east}{$\theta$}


\end{tikzpicture}
\caption[Rotation of -30~degrees about $\Z$]{The $\{\x',\y',\z'\}$ reference frame (in blue) is rotated with respect to $\{\X,\Y,\Z\}$ (in black). The rotation is about the axis $\Z$ by an angle $\theta=-30$~degrees.}
\label{fig:simpleRotate}
\end{center}
\end{figure}
\begin{figure}
\begin{center}
\tdplotsetmaincoords{70}{110}

%start tikz picture, and use the tdplot_main_coords style to implement the display 
%coordinate transformation provided by 3dplot
\begin{tikzpicture}[scale=4.5,tdplot_main_coords,label/.style={%
   postaction={ decorate,%transform shape,
   decoration={ markings, mark=at position .8 with \node #1;}}}]


\coordinate (O) at (0,0,0);
%draw the main coordinate system axes
\draw[line width=0.3mm,->] (0,0,0) -- (\rvec,0,0) node[below,align=left]{$\X$};
\draw[line width=0.3mm,->] (0,0,0) -- (0,\rvec,0) node[above left]{$\Y$};
\draw[line width=0.3mm,->] (0,0,0) -- (0,0,\rvec) node[anchor=east]{$\Z=\z'$};

% first rotation about Z
\def\rotangleZ{-30}
\def\rotangleY{45}
\def\rotangleX{15}
\tdplotsetrotatedcoords{\rotangleZ}{0}{0}
\draw[line width=0.3mm,->,tdplot_rotated_coords,color=blue] (0,0,0) -- (\rvec,0,0) node[below,align=left]{$\x'$};
\draw[line width=0.3mm,->,tdplot_rotated_coords,color=blue] (0,0,0) -- (0,\rvec,0) node[below left]{$\y'=\y''$};

\tdplotdrawarc[tdplot_main_coords,->]{(0,0,0)}{\rvec/3}{0}{\rotangleZ}{align=center,anchor=north east}{$\theta$}
\tdplotsetrotatedthetaplanecoords{0}
\tdplotdrawarc[tdplot_rotated_coords,->]{(0,0,0)}{\rvec/2}{90}{90-\rotangleY}{align=center,anchor=east}{$\phi$}
%\draw[tdplot_rotated_coords,->] (\rvec/3,0,0) arc (0:90:\rvec/3);


% second rotation about y'
\tdplotsetrotatedcoords{\rotangleZ}{-\rotangleY}{0}
\draw[line width=0.3mm,->,tdplot_rotated_coords,color=red] (0,0,0) -- (\rvec,0,0) node[above,align=left]{$\x'' = \x$};
%\draw[line width=0.3mm,->,tdplot_rotated_coords,color=red] (0,0,0) -- (0,\rvec,0);
\draw[line width=0.3mm,->,tdplot_rotated_coords,color=red] (0,0,0) -- (0,0,\rvec) node[above right]{$\z''$};

\end{tikzpicture}
\caption[Two consecutive elementary rotations]{The $\{\x'',\y'',\z''\}$ reference frame (in red) is rotated with respect to $\{\x',\y',\z'\}$ (in blue). The rotation is about the axis $\y'$ by an angle $\phi=-45$~degrees.}
\label{fig:3simpleRotate}
\end{center}
\end{figure}
\begin{figure}
\begin{center}
\tdplotsetmaincoords{70}{110}

%start tikz picture, and use the tdplot_main_coords style to implement the display 
%coordinate transformation provided by 3dplot
\begin{tikzpicture}[scale=4.5,tdplot_main_coords,label/.style={%
   postaction={ decorate,%transform shape,
   decoration={ markings, mark=at position .8 with \node #1;}}}]


\coordinate (O) at (0,0,0);
%draw the main coordinate system axes
\draw[line width=0.3mm,->] (0,0,0) -- (\rvec,0,0) node[below,align=left]{$\X$};
\draw[line width=0.3mm,->] (0,0,0) -- (0,\rvec,0) node[above left]{$\Y$};
\draw[line width=0.3mm,->] (0,0,0) -- (0,0,\rvec) node[anchor=east]{$\Z=\z'$};

% first rotation about Z
\def\rotangleZ{-30}
\def\rotangleY{45}
\def\rotangleX{15}
\tdplotsetrotatedcoords{\rotangleZ}{0}{0}
\draw[line width=0.3mm,->,tdplot_rotated_coords,color=blue] (0,0,0) -- (\rvec,0,0) node[below,align=left]{$\x'$};
\draw[line width=0.3mm,->,tdplot_rotated_coords,color=blue] (0,0,0) -- (0,\rvec,0) node[below left]{$\y'=\y''$};

\tdplotdrawarc[tdplot_main_coords,->]{(0,0,0)}{\rvec/3}{0}{\rotangleZ}{align=center,anchor=north east}{$\theta$}
\tdplotsetrotatedthetaplanecoords{0}
\tdplotdrawarc[tdplot_rotated_coords,->]{(0,0,0)}{\rvec/2}{90}{90-\rotangleY}{align=center,anchor=east}{$\phi$}
%\draw[tdplot_rotated_coords,->] (\rvec/3,0,0) arc (0:90:\rvec/3);


% second rotation about y'
\tdplotsetrotatedcoords{\rotangleZ}{-\rotangleY}{0}
\draw[line width=0.3mm,->,tdplot_rotated_coords,color=red] (0,0,0) -- (\rvec,0,0) node[above,align=left]{$\x'' = \x$};
%\draw[line width=0.3mm,->,tdplot_rotated_coords,color=red] (0,0,0) -- (0,\rvec,0);
\draw[line width=0.3mm,->,tdplot_rotated_coords,color=red] (0,0,0) -- (0,0,\rvec) node[above right]{$\z''$};

% third rotation about x''
\tdplotsetrotatedcoords{\rotangleZ}{-\rotangleY+90}{\rotangleX+90}
\draw[line width=0.3mm,->,tdplot_rotated_coords,color=green] (0,0,0) -- (\rvec,0,0) node[above,align=left]{$\y$};
\draw[line width=0.3mm,->,tdplot_rotated_coords,color=green] (0,0,0) -- (0,\rvec,0) node[above left]{$\z$};
%\draw[line width=0.3mm,->,tdplot_rotated_coords,color=green] (0,0,0) -- (0,0,\rvec) ;

%\tdplotsetrotatedcoords{\rotangleZ}{-\rotangleY-90}{90+\rotangleX}
%\tdplotsetrotatedthetaplanecoords{180}
\tdplotdrawarc[tdplot_rotated_coords,->]{(0,0,0)}{2*\rvec/3}{90-\rotangleX}{90}{align=center,anchor=south}{$\psi$}


\end{tikzpicture}
\caption[Three consecutive elementary rotations]{The $\{\x,\y,\z\}$ reference frame (in green) is rotated with respect to $\{\x'',\y'',\z''\}$ (in red). The rotation is about the axis $\x''$ by an angle $\psi=15$~degrees.}
\label{fig:3simpleRotate}
\end{center}
\end{figure}


\section{Quaternions}
\label{sec:Quaternions}
Quaternions are a more modern way to describe the orientation of a reference frame with respect to another, and are today widely used to describe spacecraft orientation [QUOTE]. From a strictly mathematical point of view, quaternions form a normed algebra over the real numbers that is an extension of traditional complex numbers. The quaternion normed algebra has four dimensions, instead of just two for the complex numbers. At its fundamental level, the basis for the quaternion algebra consists of one real axis and three imaginary axes \{1, $\i$, $\j$, $\k$\}. Like complex numbers (which have a basis \{1, $\i$\}), there are fundamental relations between the basis elements that govern the multiplication operation, such as the well known identity $\i^2=-1$, that we will discuss at length later in this section. In this document, we will write a quaternion using one of the following equivalent notations (Coutsias 1999) and (Schmidt 2001) :
\begin{equations}
\quat{q} = q_r\times 1 + q_i\i + q_j\j + q_r\k = q_r + \vectors{q} =
\begin{bmatrix}
q_i\\
q_j\\
q_k\\
q_r
\end{bmatrix} = \begin{bmatrix}
\vectors{q}\\
q_r
\end{bmatrix} = \begin{bmatrix}  \vectors{q}^T & q_r\end{bmatrix}^T,
\end{equations}
where we make a clear distinction between the quaternion's real part $q_r$, and its 3-dimensional imaginary part that we choose to represent as a vector $\vectors{q} =q_i\i + q_j\j + q_r\k$. Like complex numbers, quaternion have a conjugate operation, which negates the imaginary part: $\quat{q}^* = \begin{bmatrix}  -\vectors{q}^T & q_r\end{bmatrix}^T$.

Quaternions are interesting beyond their pure mathematical definition because the subset of quaternions of unit norm can be used to represent a coordinate frame rotation in three dimensions. The Euler rotation theorem states that any coordinate frame rotation can be described by a rotation of an angle $\theta$ about an appropriately-chosen unit vector $\vectors{u} = x\i + y\j + z\k$ (also called the "Euler axis" or "Euler vector"). This formalism has 3 degrees of freedom, the minimum needed to describe a rotation between two reference frames: two degrees of freedom in the vector (which is constrained to be of unit norm), and one in the rotation angle. If we encode this information in a quaternion using Euler's exponential notation for vectors [need references here], this precisely defines the quaternion:
\begin{equations}
\quat{q} = \exp\left[\frac{\theta}{2}(x\i + y\j + z\k)\right] = \cos\frac{\theta}{2} + (x\i + y\j + z\k)\sin\frac{\theta}{2}.
\end{equations}
This quaternion completely describes the rotation between the two reference frames and has unit norm. Conversely, every quaternion of unit norm can be decomposed like this and represent a rotation in three-dimensional Euclidian space. Like rotation matrices, the unit quaternions form a group under the quaternion multiplication operation, which is isomorphic to the special unitary group SU(2) [reference]. It is known that SU(2) is a surjective 2:1 homomorphism of SO(3). This means that each element in SO(3) can be described by exactly two elements in SU(2), or equivalently, two distinct unit quaternions: the quaternion $\quat{q}$, and its opposite $-\quat{q}$.

Quaternions use 4 numbers to describe 3 degrees of freedom: an advantage over matrices (9 elements), but an apparent disadvantage over Tait-Bryan angles, which consist of an optimal number of 3 elements. However, Tait-Bryan angles can be shown to exhibit a phenomenon called \textit{gimbal lock} [references], which leads to a degeneracy when describing the set of angles corresponding to rotations when the pitch angle (second rotation angle, about $\j$) is $\pm\pi/2$. This creates situations where some rotations and sequences of rotation would have to be avoided by fear of creating numerical issues caused by gimbal lock [footnote: for an example of gimbal lock, refer to the Apollo program!]. Quaternions, while needing an extra number to represent the rotation, are free of this concern. This is one of the main reasons that they were originally preferred to Tait-Bryan angles early in the spaceflight era [WERTZ 1985?]. [talk more about the advantages of quaternions: number of multiplications, etc - see Shuster et al 1993]

\subsection{Quaternion multiplication}

In order to form the unit quaternion group, one has to define an appropriate multiplication operation. We warn that the formulation that we use and present in the next few paragraphs does not correspond to the commonly accepted rules for quaternion operations (also called "Hamilton notation", from W. R. Hamilton who is attributed the discovery of quaternions). We use a formalism that was popularized by Caley [CITE] and adopted in most of the aerospace community [cite JPL quaternion standard], mostly to describe the orientation of satellites in inertial space. Its main advantage is that consecutive transformations using quaternions consist of multiplying elementary quaternions in a "natural order", exactly in the same order as the rotation matrices mentioned at the end of the previous section. 

[Add more philosophical dwelling on the hows and whys of this notation; including references still on page]

To avoid confusion, we will not mention the original Hamilton rules in this work. Instead, we define the quaternion elementary multiplication rules as follows[CITE]:
\begin{equations}
\label{eq:QuaternionRules}
\begin{split}
 \i^2 = \j^2 = \k^2 &= -1;\\
 \j\i = -\i\j &= \k;\\
 \k\j = -\j\k &= \i;\\
 \i\k = -\k\i &= \j.
\end{split}
\end{equations}

Using the relations in Eq.~\ref{eq:QuaternionRules}, we define the general quaternion multiplication operator $\otimes$:
\begin{equations}
\begin{split}
\quat{p}\otimes\quat{q} & =  (p_r + p_i\i + p_j\j + p_r\k)\times(q_r + q_i\i + q_j\j + q_r\k) \\
& =  (p_rq_r - p_iq_i - p_jq_j- p_kq_k)\\
& \;\;\;\;\;\; +  (p_rq_i + p_iq_r - p_jq_k + p_kq_j)\i\\
& \;\;\;\;\;\; +  (p_rq_j + p_jq_r - p_kq_i + p_iq_k)\j\\
& \;\;\;\;\;\; +  (p_rq_k + p_kq_r - p_iq_j + p_jq_i)\k\\
\end{split}.
\end{equations}

To express a vector $\fromto{}{I}\vectors{v} = \fromto{}{I}(x,y,z)$ in the new frame $L$, we construct a purely imaginary quaternion from this vector: $\quat{q}_\vectors{v} = 0 + x\i + y\j + z\k$, and we use the quaternion multiplication to obtain:
\begin{equations}
\begin{bmatrix} \fromto{}{L}\vectors{v}\\ 0 \end{bmatrix}= \fromto{I}{L}\quat{q}\otimes\quat{q}_\vectors{v}\otimes\fromto{I}{L}\quat{q}^{-1},
\end{equations} 
and extract the vector $\fromto{}{L}\vectors{v}$ from the resulting quaternion.

Note that the quaternion inverse operation for quaternions of unit norm is the same as the conjugate operation. 

\subsection{Relationship with matrices and elementary quaternions}
Using this formalism, a quaternion is behaving in the same way as the corresponding \textit{passive} transformation matrix to describe a reference frame rotation. This means that consecutive rotations are multiplying in the "natural order", which makes it more intuitive.

For example, let's consider the elementary rotation described in Fig~\ref{fig:simpleRotate} that represents a rotation of the initial reference frame $I$ into a reference frame $L$ about $\k$. Using the "left-hand" rule, the angle $\theta$ of rotation about $\k$ is now $\theta = +30$~degrees. This quaternion is $\fromto{I}{L}\quat{q} = \quat{q}_\k(\theta) = \cos\frac{\theta}{2} + \sin\frac{\theta}{2}\k$, and represents the same rotation as the passive rotation matrix $\R_\k(\theta)$ discussed in Section \ref{subsec:rotationMatrices}. [WORK OUT THE MATHS FOR THE VECTOR TRANSFORMATION] If the rotation of the reference frame is described by three consecutive rotations of angles $\theta$, $\phi$ and $\psi$ about $\k$, $\j'$, and $\i''$, respectively [see some figure], we can write:
\begin{equations}
\begin{split}
\fromto{I}{L}\quat{q} & =  \quat{q}_{\i}(\psi)\quat{q}_{\j}(\phi)\quat{q}_\k(\theta) \\
& = \begin{bmatrix}
0\\
\sin\frac{\psi}{2}\\
0\\
\cos\frac{\psi}{2}
\end{bmatrix}\begin{bmatrix}
0\\
0\\
\sin\frac{\phi}{2}\\
\cos\frac{\phi}{2}
\end{bmatrix}\begin{bmatrix}
0\\
0\\
\sin\frac{\theta}{2}\\
\cos\frac{\theta}{2}
\end{bmatrix}\\
\end{split},
\end{equations}

which forms a quaternion that is equivalent to the rotation matrix multiplication $ \R_\i(\psi)\R_\j(\phi)\R_\k(\theta).$
Note that the order of the quaternions is the same as the order of the matrices. This is one of the advantages of choosing this "natural order" convention [cite Shuster].


\section{Quaternion derivative and integration}

Properly defining the derivative and integral of quaternions is necessary for our purpose. We will need a derivative to describe our dynamic system as its orientation changes over time; and we will need to integrate (or \textit{propagate}) those equations to find a numerical solution to the attitude estimation problem.

In the following, we consider the body reference frame $L(t)$ which evolves as a function of time with respect to a fixed, inertial reference frame $I$. 

The mathematical derivations leading to those results can be found elsewhere [CITE]. Over an infinitesimal time step $dt$, the local frame is rotating by an angular vector $\delta\boldsymbol{\theta}$. The instantaneous angular velocity, expressed in the body reference frame $\L(t)$, is $\fromto{}{L(t)}\gyroVec(t) = \lim_{dt\to 0}\frac{\delta\boldsymbol{\theta}}{\delta t}$. It can be shown [CITE TRAWNY] that with this formalism, the quaternion derivative is defined using either a quaternion multiplication, or an equivalent matrix multiplication:
\begin{equations}
\fromto{I}{L(t)}\dotAttitude(t) = \frac{1}{2}\begin{bmatrix} \gyroVec \\ 0 \end{bmatrix}\otimes \fromto{I}{L(t)}\Attitude =  \frac{1}{2}\matOmega(\gyroVec)\fromto{I}{L(t)}\Attitude,
\label{eq:quaternionDerivation}
\end{equations}

where the matrix
\begin{equations}
\matOmega(\gyroVec) = 
\begin{bmatrix}
0 & \omega_z & -\omega_y & \omega_x \\
-\omega_z & 0 & \omega_x & \omega_y \\
\omega_y & -\omega_x & 0 & \omega_z \\
-\omega_x & -\omega_y & -\omega_z & 0
\end{bmatrix}
\label{eq:Omega}
\end{equations}

is going to play an important in the later sections. 

The integrator formulas are derived in []. The problem is to find a matrix $\thetaMat$ to integrate a quaternion $\fromto{I}{L(t)}\Attitude(t)$, and estimate attitude at time $t+\Deltat$, knowing the instantaneous angular velocity $\gyroVec(t)$: 
\begin{equations}
\Attitude(t+\Deltat) = \thetaMat(t,t+\Deltat)\Attitude(t)
\end{equations}
A zeroth-order solution assumes that the angular velocity $\gyroVec$ is a constant over the timestep $\Deltat$, an important special case since it describes the typical discrete representation that we will use in our software. The solution can be expressed as:

\begin{equations}
\thetaMat(t,t+\Deltat) \equiv \thetaMat(\Delta t) = \exp\left(\frac{1}{2}\matOmega(\gyroVec)\Deltat\right),
\label{eq:quaternionIntegration}
\end{equations}
where the matrix exponential is defined using a Taylor expansion [give reference for that]. 


A first-order solution is given in [TRAWNY] and uses knowledge of two previous $\gyroVec$ values to estimate the integral.

\section{Covariance matrices in different reference frames}
In the following, we will be describing our attitude using quaternions or rotation matrices in a Kalman filter with a state-space representation. This means that we will be make estimates of physical quantities, as well as estimates of our estimation error. These errors are represented using covariance matrices. 

Covariance matrices contain information about the cross-correlation of the variables in the state vector. The diagonal elements represent the auto-covariance of a given variable, while the terms off the diagonal indicate the degree of covariance (or correlation) between the different variables. For example, we will have three gyroscopes which will be mounted orthogonally from each other, each measuring the angular velocity about three different axes. In the ideal case, all gyroscopes are independent and the covariance matrix associated to the set of three angular velocities is diagonal, with the variances of each gyroscope on the diagonal.

If we rotate the gyroscopes' frame with a rotation matrix $\R$, the new covariance matrix $\noiseCovMat$ needs to be rotated as well: $\noiseCovMat' = \R\noiseCovMat\R^T$. 

\subsection{Small angle approximation}

Quaternions become more intuitive in the small angle approximation. Indeed, when all angles are small with respect to $\pi$, we can write:
\begin{equations}
\fromto{I}{L}\Attitude \approx \begin{bmatrix}  \frac{1}{2}\delta\boldsymbol{\theta} \\ 1 \end{bmatrix},
\end{equations}
where $\frac{1}{2}\delta\boldsymbol{\theta} = \frac{1}{2}[\delta\theta_i,  \delta\theta_j, \delta\theta_k]^T$ corresponds to three small rotations about all the three axes of the initial reference frame. Because we are in the small angle approximation, the order of the rotations does not matter. Hence, if the imaginary part of a quaternion has small values $q_i, q_j, q_k$, and if $q_r\approx 1$, this quaternion represents a rotation of the reference frame by an angle $\delta\theta_i = 2q_i$, then by an angle $\delta\theta_j = 2q_j$, and finally by an angle $\delta\theta_k = 2q_k$, where all the angles are expressed in radians.

More simplifications can also be found. For example, in the limit where $\gyroVec \to \boldsymbol{0}$, the matrix exponential defined in Eq.~\ref{eq:quaternionIntegration} simplifies into:
\begin{equations}
\thetaMat(\Deltat) \stackrel{\gyroVec \to \boldsymbol{0}}{=} \I_{4\times 4} + \frac{\Deltat}{2}\matOmega(\gyroVec).
\label{eq:quaternionIntegrationSimple}
\end{equations}



%\subsection{Examples} (believe me, examples are very very important when people read your stuff – if only other people had put examples in their write-ups!)
%\subsection{Implementation in software}
%As part of this effort, we wrote quaternion libraries in C++, Python, and Labview. In all cases, we use double floating point representation for the quaternion and its associated operations, to ensure that we do not lose information due to numerical round-off errors. Indeed
