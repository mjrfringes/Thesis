\documentclass{standalone}

\begin{document}

\chapter{The PID control loop}
\label{ap:PID}

Before we elaborate on the control architecture of the entire system, let's first discuss the elementary controls block: the PID.

A Proportional-Integral-Derivative (PID) control loop is one of the most basic, yet most used method to build systems with active control. The problem that these systems try to solve is simply to make an object reach a desired state: a sensor is used to measure the current state, and the difference between the desired state and the current state is fed to an apparatus capable of changing the state. Most commonly, this uses motors and either position or velocity sensors, but it can also be used for example for temperature control in a cryogenic environment, where heaters are used to change the temperature. For simplicity, in the rest of this work, we will always consider a loop with sensors and actuators. 

In its most simple expression, the PID can be reduced to a simple proportional loop. That is, the command is proportional to the error between the desired and measured state. The value of this proportional coefficient usually sets the dynamics of the response, as a large proportional gain $\Kp$ will mean that even a small deviation from our desired state will trigger a large response. Sometimes, a purely proportional system can lack stability.

A proportional-derivative loop adds the information of the speed at which the error varies. If the error is growing quickly, we can increase our command. If the error is being reduced quickly, it is time to slow down the command to avoid overshooting our target. This uses the time derivative of the error that multiplies a gain, $\Kd$, and has the effect to damp the motion. A PD loop usually will help with the system's stability.

But even then, a proportional-derivative does not guarantee that you will reach your desired state. We then complete the PID loop with an integral gain $\Ki$, which multiplies the integral of the error over some length of time. While the $\Kp$ and $\Kd$ gains mostly control the dynamics of the response, the integral term will control the steady-state error and ensure it converges to zero. While useful, this term needs to be considered with precaution, as some situations can lead to a diverging response.

\begin{figure}[!ht]
	\centering
	\includestandalone{Figures/SimplePID}
	\caption{}
	\label{fig:SimplePID}
    \end{figure}



A simple PID loop diagram is shown in Fig~\ref{fig:SimplePID}, with the desired input state at the entrance of the loop and the real state at the output of the loop. It is often the case that the state cannot be directly measured: this require the use of an \textit{estimator} or \textit{observer}, in which various indirect measurements will feed a mathematical model of the system to estimate its parameters. The relevant example for us is a scenario where we only measure a velocity measurement, while we want to close the loop on the position. Simply put, we know that the position has an integral relationship with the velocity, and the observer's role is to estimate the integration constants.

The estimator is also used to realize \textit{sensor fusion}. This consists of combining various types of measurements to provide the best estimate of the state to feed back to the control loop. The various measurements often happen at different discrete rates, with different lag times, which can lead to rather complex implementations. One of the most well-known estimation algorithms is the Kalman filter, which we will discuss at length in Section []. 

For BETTII, each subsystem has its own PID control loop. Each PID loop structure consists of 7 variables: the $\Kp$, $\Kd$, $\Ki$ gains, and overall scaling factor, an upper and a lower limit on the command, and a boolean value that is used to reset the content of the integral term used to multiply $\Ki$. 

\end{document}