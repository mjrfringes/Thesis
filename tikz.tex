\usepackage{tikz,tikz-3dplot}
\usepackage{xifthen} % if then statements
%\usepackage{siunitx} % getting rid of decimal parts
\usepackage{adjustbox} % this is to allow the tikz image to span over two columns
\usepackage[nointegrals]{wasysym}	% For Aries symbol
\usepackage{pgfplots}
\usetikzlibrary{positioning,arrows,calc,intersections,fadings,decorations.pathreplacing,shapes.geometric,decorations.markings}
\usetikzlibrary{shapes,arrows}

\newcommand{\AxisRotator}[1][rotate=0]{%
    \tikz [x=0.25cm,y=0.60cm,line width=.2ex,-stealth,#1] \draw (0,0) arc (-150:150:1 and 1);%
}

%% helper macros
\newcommand\pgfmathsinandcos[3]{%
  \pgfmathsetmacro#1{sin(#3)}%
  \pgfmathsetmacro#2{cos(#3)}%
}
\newcommand\LongitudePlane[3][current plane]{%
  \pgfmathsinandcos\sinEl\cosEl{#2} % elevation
  \pgfmathsinandcos\sint\cost{#3} % azimuth
  \tikzset{#1/.style={cm={\cost,\sint*\sinEl,0,\cosEl,(0,0)}}}
}
\newcommand\LatitudePlane[3][current plane]{%
  \pgfmathsinandcos\sinEl\cosEl{#2} % elevation
  \pgfmathsinandcos\sint\cost{#3} % latitude
  \pgfmathsetmacro\yshift{\cosEl*\sint}
  \tikzset{#1/.style={cm={\cost,0,0,\cost*\sinEl,(0,\yshift)}}} %
}
\newcommand\DrawLongitudeCircleName[3][1]{
  \LongitudePlane{\angEl}{#2}
  \tikzset{current plane/.prefix style={scale=#1}}
   % angle of "visibility"
  \pgfmathsetmacro\angVis{atan(sin(#2)*cos(\angEl)/sin(\angEl))} %
  \draw[name path=#3,current plane] (\angVis:1) arc (\angVis:\angVis+180:1);
  \draw[name path=dashed#3,current plane,dashed] (\angVis-180:1) arc (\angVis-180:\angVis:1);
}
\newcommand\DrawLatitudeCircleName[3][2]{
  \LatitudePlane{\angEl}{#2}
  \tikzset{current plane/.prefix style={scale=#1}}
  \pgfmathsetmacro\sinVis{sin(#2)/cos(#2)*sin(\angEl)/cos(\angEl)}
  % angle of "visibility"
  \pgfmathsetmacro\angVis{asin(min(1,max(\sinVis,-1)))}
  \draw[name path=#3,current plane] (\angVis:1) arc (\angVis:-\angVis-180:1);
  \draw[name path=dashed#3,current plane,dashed] (180-\angVis:1) arc (180-\angVis:\angVis:1);
}

\newcommand\DrawLatitudeCircle[2][1]{
  \LatitudePlane{\angEl}{#2}
  \tikzset{current plane/.prefix style={scale=#1}}
  \pgfmathsetmacro\sinVis{sin(#2)/cos(#2)*sin(\angEl)/cos(\angEl)}
  % angle of "visibility"
  \pgfmathsetmacro\angVis{asin(min(1,max(\sinVis,-1)))}
  \draw[current plane,thin,black] (\angVis:1) arc (\angVis:-\angVis-180:1);
  \draw[current plane,thin,dashed] (180-\angVis:1) arc (180-\angVis:\angVis:1);
}%Defining functions to draw limited latitude circles (for the red mesh)

% Redefine rotation sequence for tikz3d-plot to z-y-x
\newcommand{\tdseteulerxyz}{
  \renewcommand{\tdplotcalctransformrotmain}{%
    %perform some trig for the Euler transformation
      \tdplotsinandcos{\sinalpha}{\cosalpha}{\tdplotalpha}
    \tdplotsinandcos{\sinbeta}{\cosbeta}{\tdplotbeta}
    \tdplotsinandcos{\singamma}{\cosgamma}{\tdplotgamma}
    %
      \tdplotmult{\sasb}{\sinalpha}{\sinbeta}
    \tdplotmult{\sasg}{\sinalpha}{\singamma}
    \tdplotmult{\sasbsg}{\sasb}{\singamma}
    %
      \tdplotmult{\sacb}{\sinalpha}{\cosbeta}
    \tdplotmult{\sacg}{\sinalpha}{\cosgamma}
    \tdplotmult{\sasbcg}{\sasb}{\cosgamma}
    %
      \tdplotmult{\casb}{\cosalpha}{\sinbeta}
    \tdplotmult{\cacb}{\cosalpha}{\cosbeta}
    \tdplotmult{\cacg}{\cosalpha}{\cosgamma}
    \tdplotmult{\casg}{\cosalpha}{\singamma}
    %
      \tdplotmult{\cbsg}{\cosbeta}{\singamma}
    \tdplotmult{\cbcg}{\cosbeta}{\cosgamma}
    %
      \tdplotmult{\casbsg}{\casb}{\singamma}
    \tdplotmult{\casbcg}{\casb}{\cosgamma}
    %
      %determine rotation matrix elements for Euler transformation
      \pgfmathsetmacro{\raaeul}{\cacb}
    \pgfmathsetmacro{\rabeul}{\casbsg - \sacg}
    \pgfmathsetmacro{\raceul}{\sasg + \casbcg}
    \pgfmathsetmacro{\rbaeul}{\sacb}
    \pgfmathsetmacro{\rbbeul}{\sasbsg + \cacg}
    \pgfmathsetmacro{\rbceul}{\sasbcg - \casg}
    \pgfmathsetmacro{\rcaeul}{-\sinbeta}
    \pgfmathsetmacro{\rcbeul}{\cbsg}
    \pgfmathsetmacro{\rcceul}{\cbcg}
  }
}

%% document-wide tikz options and styles

\tikzset{%
  >=latex, % option for nice arrows
  inner sep=0pt,%
  outer sep=1pt,%
  mark coordinate/.style={inner sep=0pt,outer sep=0pt,minimum size=4pt,
  fill=black,circle},%
	sundot/.style={
	fill, color=yellow, circle, inner sep=3.5pt}
}
\def\starcamcolor{purple}
\def\celestialcolor{blue}
\def\gondolacolor{orange}
\def\telcolor{black}
