\documentclass{standalone}
\usepackage{tikz}
\usetikzlibrary{positioning,intersections,calc,fadings,decorations.pathreplacing}


\begin{document}
\linespread{1}

\begin{tikzpicture}
\pgfmathsetmacro{\thetaX}{80}
\pgfmathsetmacro{\phiZ}{-125}
\tdseteulerxyz

% general setup
\tdplotsetmaincoords{\thetaX}{\phiZ}
\pgfmathsetmacro{\REarth}{4}
\pgfmathsetmacro{\RHorizon}{6}
\pgfmathsetmacro{\starAz}{30-125}
\pgfmathsetmacro{\starEl}{45}
\pgfmathsetmacro{\starRoll}{-15}
\pgfmathsetmacro{\starcamElevation}{-45}
\pgfmathsetmacro{\telElevation}{-25}

\coordinate (O) at (0,0,0);

% draw main axes
\draw[tdplot_main_coords,->,thick,\gondolacolor] (O) -- (0,0,\REarth) node[above]{$\vectors{z}_\gyro$};
\draw[tdplot_main_coords,->,thick,\gondolacolor] (O) -- (0,\REarth,0) node[above]{$\vectors{y}_\gyro$};
\draw[tdplot_main_coords,->,thick,\gondolacolor] (O) -- (\REarth,0,0) node[above]{$\vectors{x}_\gyro$};
\node[tdplot_main_coords,draw,fill,minimum size=3pt,circle] at (O){};



% draw horizon
\tdplotdrawarc[tdplot_main_coords,thin]{(O)}{\RHorizon}{0}{360}{below=1em, right,align=center}{Local \\ horizon};
\filldraw[tdplot_main_coords,\celestialcolor, fill opacity=0.2] (O) circle (\RHorizon);


% draw azimuth arc
\tdplotsetrotatedcoords{90}{0}{0}
\tdplotsetrotatedthetaplanecoords{90}
\tdplotdrawarc[tdplot_rotated_coords,thick,\telcolor,->]{(O)}{0.6*\REarth}{0}{\starAz}{above}{$\Az$};

\tdplotsetrotatedcoords{0}{\telElevation}{0}
\draw[tdplot_rotated_coords,->,thick,\telcolor] (O) -- (\REarth,0,0) node[above]{$\vectors{x}_\tel$};
\tdplotsetrotatedthetaplanecoords{0}
\tdplotdrawarc[tdplot_rotated_coords,thick,\telcolor,<-]{(O)}{0.6*\REarth}{90}{90-\telElevation}{left,align=center}{$\El$};

% draw elevation arc
\tdplotsetrotatedcoords{\starAz}{-\starEl}{0}
\draw[tdplot_rotated_coords,dashed,\celestialcolor] (O) -- (\RHorizon,0,0);
\tdplotsetrotatedcoords{\starAz}{0}{0}
\draw[tdplot_rotated_coords,dashed] (O) -- (\RHorizon,0,0);
\tdplotsetrotatedthetaplanecoords{0}
\tdplotdrawarc[tdplot_rotated_coords,thick,\telcolor,->]{(O)}{0.6*\REarth}{90}{90-\starEl}{right=0.3em,align=center}{desired \\ elevation};
\tdplotdrawarc[tdplot_rotated_coords,thin,gray!20]{(O)}{\RHorizon}{90}{-90}{}{};


%draw star and reference frame
\tdplotsetrotatedcoords{\starAz}{-\starEl}{\starRoll}
\node[tdplot_rotated_coords,draw=black,fill=yellow,star,star points=10,minimum size=10pt] at (\RHorizon,0,0){}; %star
\begin{scope}[tdplot_rotated_coords,shift={(\RHorizon,0,0)}]
\draw[tdplot_rotated_coords,->,\celestialcolor] (0,0,0) -- (2,0,0)node[above]{$\vectors{x}$};
\draw[tdplot_rotated_coords,->,\celestialcolor] (0,0,0) -- (0,2,0)node[above]{$\vectors{y}$};
\draw[tdplot_rotated_coords,dashed,\celestialcolor] (0,0,0) -- (0,-2,0) node[above right,rotate=2,midway]{constant DEC};
\draw[tdplot_rotated_coords,->,\celestialcolor] (0,0,0) -- (0,0,2)node[above]{$\vectors{z}$};
\draw[tdplot_rotated_coords,dashed,\celestialcolor] (0,0,0) -- (0,0,-2)
node[above ,rotate=-43,align=center,\celestialcolor]{constant RA};
\end{scope}

% draw star camera and coordinates
\pgfmathsetmacro{\sizeStarcamFrame}{1.5}
\tdplotsetrotatedcoords{0}{\starcamElevation}{0}
\draw[tdplot_rotated_coords,\starcamcolor,thick] (0,0,0) -- (\RHorizon,0,0);
\begin{scope}[tdplot_rotated_coords,shift={(\RHorizon,0,0)}]
\draw[tdplot_rotated_coords,\starcamcolor] (0,-\sizeStarcamFrame,-\sizeStarcamFrame)  -- (0,-\sizeStarcamFrame,\sizeStarcamFrame);
\draw[tdplot_rotated_coords,\starcamcolor] (0,-\sizeStarcamFrame,-\sizeStarcamFrame) -- (0,\sizeStarcamFrame,-\sizeStarcamFrame);
\draw[tdplot_rotated_coords,\starcamcolor] (0,\sizeStarcamFrame,\sizeStarcamFrame) -- (0,-\sizeStarcamFrame,\sizeStarcamFrame);
\draw[tdplot_rotated_coords,\starcamcolor] (0,\sizeStarcamFrame,\sizeStarcamFrame)node[above left,align=center]{Star camera \\ field of view} -- (0,\sizeStarcamFrame,-\sizeStarcamFrame);
\draw[tdplot_rotated_coords,->,\starcamcolor,thick] (0,0,0) -- (2,0,0)node[above]{$\vectors{x}_\starcam$};
\draw[tdplot_rotated_coords,->,\starcamcolor,thick] (0,0,0) -- (0,2,0)node[above=0.5em]{$\vectors{y}_\starcam$};
\draw[tdplot_rotated_coords,->,\starcamcolor,thick] (0,0,0) -- (0,0,4) node[above]{$\vectors{z}_\starcam$};
\end{scope}

% plot celestial coordinates
\tdplotsetrotatedcoords{0}{\starcamElevation}{\starRoll}
\begin{scope}[tdplot_rotated_coords,shift={(\RHorizon,0,0)}]

%\draw[tdplot_rotated_coords,->,\celestialcolor] (0,0,0) -- (2,0,0);
\node[tdplot_rotated_coords,right=1em,\starcamcolor] at (0,0,0){(RA,DEC)$_\starcam$};
\draw[tdplot_rotated_coords,->,\celestialcolor] (0,0,0) -- (0,2,0)node[left]{$\vectors{y}$};
\draw[tdplot_rotated_coords,\celestialcolor,dashed] (0,0,0) -- (0,-2,0) node[below,rotate=22,right=0.1em,midway]{};
\draw[tdplot_rotated_coords,->,\celestialcolor] (0,0,0) -- (0,0,4)node[above]{$\vectors{z}$};
\draw[tdplot_rotated_coords,\celestialcolor,dashed] (0,0,0) -- (0,0,-2)
node[above left,rotate=90,align=center,midway]{};
\node[draw,fill,circle,minimum size=3pt,tdplot_rotated_coords,\starcamcolor] at (0,0,0){};
\end{scope}
\tdplotsetrotatedthetaplanecoords{0}
\tdplotdrawarc[tdplot_rotated_coords,thin,gray!20]{(O)}{\RHorizon}{90-\starcamElevation+1}{-90-\starcamElevation+1}{}{};

\tdplotsetrotatedcoords{0}{90+\starcamElevation}{0}
\tdplotsetrotatedthetaplanecoords{90}
\tdplotdrawarc[tdplot_rotated_coords,\starcamcolor,<-,thick]{(0,0,\RHorizon)}{3}{270}{270+\starRoll}{right=1.0em}{ROLL$_\starcam$};


%\tdplotsetrotatedthetaplanecoords{90}
%\tdplotdrawarc[tdplot_rotated_coords,thin,->]{(O)}{\RHorizon}{0}{180}{above right}{RA};



\end{tikzpicture}

\end{document}