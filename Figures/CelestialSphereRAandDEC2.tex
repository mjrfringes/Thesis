\documentclass{standalone}
\usepackage{tikz}
\usetikzlibrary{positioning,intersections,calc,fadings,decorations.pathreplacing}
\def\starcamcolor{purple}
\def\celestialcolor{blue}


\begin{document}
% set up the main coordinate frame
\pgfmathsetmacro{\thetaXt}{80}
\pgfmathsetmacro{\phiZ}{140}
\tdplotsetmaincoords{\thetaXt}{\phiZ}

% rotate reference frames according to the Z, Y, X order instead of the default Z, Y, Z
\tdseteulerxyz

\begin{tikzpicture}[tdplot_main_coords] 

% Some definitions
\pgfmathsetmacro{\REarth}{4} % radius of Earth
\pgfmathsetmacro{\RSphere}{6} % radius of celestial sphere
\pgfmathsetmacro{\starRA}{30} % RA of star, in deg
\pgfmathsetmacro{\starDEC}{65} % DEC of star, in dec
\pgfmathsetmacro{\ecliptic}{23.5} % angle between ecliptic and celestial planes
\pgfmathsetmacro{\thetaZ}{115} % Z rotation of payload
\pgfmathsetmacro{\thetaY}{45} % Y rotation of payload
\pgfmathsetmacro{\thetaX}{15} % X rotation of payload
\pgfmathsetmacro{\elevationangle}{55} % elevation angle of star camera
\pgfmathsetmacro{\gondolaCSsize}{1} % size of gondola reference frame axes

\coordinate[mark coordinate] (O) at (0,0,0);

% star coordinate
\tdplotsetcoord{S}{\RSphere}{90-\starDEC}{\starRA}
\tdplotsetcoord{S2}{\REarth}{90-\starDEC}{\starRA}
\tdplotsetcoord{SRA}{\REarth}{90}{\starRA}
\coordinate[mark coordinate] (a) at (S2);
\draw[\celestialcolor] (O) -- (SRA); % projection of star
\draw[\celestialcolor] (O) -- (S); % star

% main axes
\draw[name path=Zrotaxis,->,thick] (O) -- (0,0,1.2*\RSphere); % Axis of rotation of the Earth
\draw[->,thick] (O) -- (1.5*\RSphere,0,0); % axis to vernal equinox

% Earth
\filldraw[name path=earth,tdplot_screen_coords,ball color=blue, fill opacity=0.1] (O) circle (\REarth);
\node[tdplot_screen_coords,above] at (4,-2.5) {Earth}; % Earth label

% draw equator
%\tdplotdrawarc[tdplot_main_coords,blue]{(O)}{\REarth}{90-\phiZ}{270-\phiZ}{}{}; 
%\tdplotdrawarc[tdplot_main_coords,dashed,blue]{(O)}{\REarth}{90-\phiZ}{-\phiZ-90}{}{};

% celestial sphere
\filldraw[tdplot_screen_coords,ball color=white, fill opacity=0.05,name path=celsphere] (O) circle (\RSphere);
\node[tdplot_screen_coords,above] at (4,\RSphere-0.5) {Celestial sphere}; % celestial sphere label
\tdplotsetthetaplanecoords{90} % rotate the theta plane
\tdplotdrawarc[name path=celeq,tdplot_main_coords,thick,blue]{(O)}{\RSphere}{90-\phiZ}{275-\phiZ}{anchor=north}{Celestial equator}; % draw equator front
\tdplotdrawarc[tdplot_main_coords,dashed,thick,blue]{(O)}{\RSphere}{90-\phiZ}{-\phiZ-90}{}{};% draw equator back
\tdplotdrawarc[draw=none,name path=celeq2,tdplot_main_coords,thick,blue]{(O)}{\RSphere}{0}{360}{}{}; % create full 360 degree equator line, but don't draw it

% draw star
\node[draw=black,fill=yellow,star,star points=10,minimum size=10pt] at (S){};

% RA star arc
\tdplotsetrotatedcoords{90}{0}{0}
\tdplotsetrotatedthetaplanecoords{90}
\tdplotdrawarc[tdplot_rotated_coords,red,->]{(O)}{0.8*\REarth}{0}{\starRA}{above=0.5em}{RA};

% DEC star arc
\tdplotsetrotatedcoords{\starRA}{0}{0}
\tdplotsetrotatedthetaplanecoords{0}
\tdplotdrawarc[tdplot_rotated_coords,red,->]{(O)}{0.8*\REarth}{90}{90-\starDEC}{below,rotate=80}{DEC};
% meridian that goes through the star's projection on the Earth
%\tdplotdrawarc[tdplot_rotated_coords]{(O)}{\REarth}{180}{0}{}{}; % front
%\tdplotdrawarc[tdplot_rotated_coords,dashed]{(O)}{\REarth}{180}{360}{}{}; % back
\tdplotdrawarc[draw=none,name path=meridian,tdplot_rotated_coords]{(O)}{\REarth}{0}{360}{}{}; % for labeling only
\path[name intersections={of= meridian and Zrotaxis,by={northpole}}]; % find north pole
\node[tdplot_rotated_coords,draw,fill,minimum size=3pt,circle] at (northpole){}; % mark it
\node[tdplot_rotated_coords,above right] at (northpole){North pole}; % label it

% % ROLL arc
% \tdplotsetrotatedcoords{\starRA}{90-\starDEC}{0}
% \tdplotsetrotatedthetaplanecoords{90}
% \tdplotdrawarc[tdplot_rotated_coords,red,->]{(0,0,5.5)}{0.1*\REarth}{-40}{260}{left=1.0em}{ROLL};

% ecliptic plane
\tdplotsetrotatedcoords{0}{0}{90-\ecliptic}
\tdplotsetrotatedthetaplanecoords{0}
\tdplotdrawarc[name path=ecl,tdplot_rotated_coords,thick,blue]{(O)}{\RSphere}{180-\phiZ}{360-\phiZ}{anchor=north,rotate=-\ecliptic}{Ecliptic plane}; % front
\tdplotdrawarc[tdplot_rotated_coords,dashed,thick,blue]{(O)}{\RSphere}{180-\phiZ}{-\phiZ}{}{}; % back

% intersections
\path[name intersections={of= celeq and ecl,by={vereq}}]; % find vernal equinox point
\coordinate[mark coordinate] (a) at (vereq); % mark it
\node[align=center,below=0.6em] at (vereq) {$\boldsymbol{\aries}$ \\ Vernal \\ equinox}; % label it

% draw payload reference frame at center of sphere
\tdplotsetrotatedcoords{\thetaZ}{\thetaY}{\thetaX}
\draw[tdplot_rotated_coords,orange,->,thick] (O) -- (0,0,\gondolaCSsize) node[above left]{$\vectors{z}_\gyro$};
\draw[tdplot_rotated_coords,orange,->,thick] (O) -- (0,\gondolaCSsize,0) node[right]{$\vectors{y}_\gyro$};
\draw[tdplot_rotated_coords,orange,->,thick] (O) -- (\gondolaCSsize,0,0) node[above right]{$\vectors{x}_\gyro$};

% % draw star camera elevation arc
%\tdplotsetrotatedthetaplanecoords{0}
%\tdplotdrawarc[tdplot_rotated_coords,thin,purple,->]{(O)}{0.1*\RSphere}{90}{90-\elevationangle}{right}{elevation}; 

\tdplotsetrotatedcoords{\thetaZ}{\thetaY}{0}
\tdplotsetrotatedthetaplanecoords{0}
% meridian going through payload location
\tdplotdrawarc[tdplot_rotated_coords,thin]{(O)}{\REarth}{-10-\thetaY}{170-\thetaY}{}{}; 
\tdplotdrawarc[tdplot_rotated_coords,thin,dashed]{(O)}{\REarth}{170-\thetaY}{360-\thetaY}{}{};

% plot the inertial reference frame at the star camera field
\tdplotsetrotatedcoords{\thetaZ}{\thetaY}{0}




% draw star camera and coordinates
\pgfmathsetmacro{\sizeStarcamFrame}{1}
\tdplotsetrotatedcoords{\thetaZ}{\thetaY-\elevationangle}{\thetaX}
\draw[tdplot_rotated_coords,\starcamcolor,thick] (0,0,0) -- (\RSphere,0,0);
\begin{scope}[tdplot_rotated_coords,shift={(\RSphere,0,0)}]
\draw[tdplot_rotated_coords,\starcamcolor] (0,-\sizeStarcamFrame,-\sizeStarcamFrame)  -- (0,-\sizeStarcamFrame,\sizeStarcamFrame);
\draw[tdplot_rotated_coords,\starcamcolor] (0,-\sizeStarcamFrame,-\sizeStarcamFrame)  -- (0,\sizeStarcamFrame,-\sizeStarcamFrame) node[below right,align=center]{Star camera \\ field of view};
\draw[tdplot_rotated_coords,\starcamcolor] (0,\sizeStarcamFrame,\sizeStarcamFrame) -- (0,-\sizeStarcamFrame,\sizeStarcamFrame);
\draw[tdplot_rotated_coords,\starcamcolor] (0,\sizeStarcamFrame,\sizeStarcamFrame) -- (0,\sizeStarcamFrame,-\sizeStarcamFrame);
\draw[tdplot_rotated_coords,->,\starcamcolor,thick] (0,0,0) -- (3,0,0)node[above]{$\vectors{x}_\starcam$};
\draw[tdplot_rotated_coords,->,\starcamcolor,thick] (0,0,0) -- (0,3,0)node[above]{$\vectors{y}_\starcam$};
\draw[tdplot_rotated_coords,->,\starcamcolor,thick] (0,0,0) -- (0,0,3) node[above]{$\vectors{z}_\starcam$};
\end{scope}

% plot celestial coordinates of star camera field
\tdplotsetrotatedcoords{\thetaZ}{\thetaY-\elevationangle}{0}
\begin{scope}[tdplot_rotated_coords,shift={(\RSphere,0,0)}]
\node[tdplot_rotated_coords,above right,red] at (0,0,0){}; 
\draw[tdplot_rotated_coords,->,\celestialcolor] (0,0,0) -- (0,3,0);
\draw[tdplot_rotated_coords,->,\celestialcolor] (0,0,0) -- (0,0,3);
\node[draw,fill,circle,minimum size=3pt,tdplot_rotated_coords,red] at (0,0,0){};
\end{scope}
\tdplotsetrotatedthetaplanecoords{0}
\tdplotdrawarc[name path=meridianstarcam,tdplot_rotated_coords,thin,blue,draw=none]{(O)}{\RSphere}{0}{180}{}{};
\path[name intersections={of= celeq and meridianstarcam,by={celstarcam}}];%, 
%\node[draw,fill,circle,minimum size=3pt,tdplot_rotated_coords,red] at (celstarcam){};
\draw[tdplot_rotated_coords,thin] (O) -- (celstarcam);

% RA starcam arc
\tdplotsetrotatedcoords{90}{0}{0}
\tdplotsetrotatedthetaplanecoords{90}
\tdplotdrawarc[tdplot_rotated_coords,red,->,thick]{(O)}{1*\REarth}{0}{\thetaZ}{below=2em, right=5em}{\small RA$_\starcam$};

% DEC starcam arc
\tdplotsetrotatedcoords{\thetaZ}{0}{0}
\tdplotsetrotatedthetaplanecoords{0}
\tdplotdrawarc[tdplot_rotated_coords,red,->,thick]{(O)}{1*\REarth}{90}{90-\elevationangle+\thetaY}{left=0.2em}{\small DEC$_\starcam$};
\tdplotdrawarc[tdplot_rotated_coords,thin]{(O)}{1*\RSphere}{90}{90-\elevationangle+\thetaY}{}{};

% roll angle
\tdplotsetrotatedcoords{\thetaZ}{90-\elevationangle+\thetaY}{0}
\tdplotsetrotatedthetaplanecoords{90}
\tdplotdrawarc[tdplot_rotated_coords,red,->,thick]{(0,0,\RSphere)}{2.5}{270}{270+\thetaX}{right=0.4em}{ROLL$_\starcam$};

% translated gondola reference frame
\tdplotsetrotatedcoords{\thetaZ}{\thetaY}{\thetaX}
\tdplotsetcoord{G}{\REarth}{\thetaY}{\thetaZ}
\tdplotsetrotatedcoordsorigin{(G)};

% plot local horizon
\filldraw[tdplot_rotated_coords,blue, fill opacity=0.2] (G) circle (0.2*\REarth);
\node[left=1.5em,align=center,blue] at (G){Local \\ horizon};

%\coordinate[mark coordinate] (T) at (G);
\draw[tdplot_rotated_coords,orange,->,thick] (G) --  (0,0,\gondolaCSsize) node[above,align=center]{Gondola \\ frame \\ $\vectors{z}_\gyro$};
\draw[tdplot_rotated_coords,orange,->,thick] (G) -- (0,\gondolaCSsize,0) node[right] {$\vectors{y}_\gyro$};
\draw[tdplot_rotated_coords,orange,->,thick] (G) -- (\gondolaCSsize,0,0) node[right] {$\vectors{x}_\gyro$};
\node[draw,fill,minimum size=3pt,circle,tdplot_rotated_coords] at (G){};
\tdplotsetrotatedcoords{\thetaZ}{\thetaY-\elevationangle}{\thetaX}
\draw[tdplot_rotated_coords,->,purple,thick] (G) -- (\gondolaCSsize,0,0) node[below=0.2em,align=center]{$\vectors{x}_\starcam$};


\end{tikzpicture}
\end{document} 