\begin{figure}
\begin{center}
\tdplotsetmaincoords{70}{110}

%start tikz picture, and use the tdplot_main_coords style to implement the display 
%coordinate transformation provided by 3dplot
\begin{tikzpicture}[scale=4.5,tdplot_main_coords,label/.style={%
   postaction={ decorate,%transform shape,
   decoration={ markings, mark=at position .8 with \node #1;}}}]


\coordinate (O) at (0,0,0);
%draw the main coordinate system axes
\draw[line width=0.3mm,->] (0,0,0) -- (\rvec,0,0) node[below,align=left]{$\X$};
\draw[line width=0.3mm,->] (0,0,0) -- (0,\rvec,0) node[above left]{$\Y$};
\draw[line width=0.3mm,->] (0,0,0) -- (0,0,\rvec) node[anchor=east]{$\Z=\z'$};

% first rotation about Z
\def\rotangleZ{-30}
\def\rotangleY{45}
\def\rotangleX{15}
\tdplotsetrotatedcoords{\rotangleZ}{0}{0}
\draw[line width=0.3mm,->,tdplot_rotated_coords,color=blue] (0,0,0) -- (\rvec,0,0) node[below,align=left]{$\x'$};
\draw[line width=0.3mm,->,tdplot_rotated_coords,color=blue] (0,0,0) -- (0,\rvec,0) node[below left]{$\y'=\y''$};

\tdplotdrawarc[tdplot_main_coords,->]{(0,0,0)}{\rvec/3}{0}{\rotangleZ}{align=center,anchor=north east}{$\theta$}
\tdplotsetrotatedthetaplanecoords{0}
\tdplotdrawarc[tdplot_rotated_coords,->]{(0,0,0)}{\rvec/2}{90}{90-\rotangleY}{align=center,anchor=east}{$\phi$}
%\draw[tdplot_rotated_coords,->] (\rvec/3,0,0) arc (0:90:\rvec/3);


% second rotation about y'
\tdplotsetrotatedcoords{\rotangleZ}{-\rotangleY}{0}
\draw[line width=0.3mm,->,tdplot_rotated_coords,color=red] (0,0,0) -- (\rvec,0,0) node[above,align=left]{$\x'' = \x$};
%\draw[line width=0.3mm,->,tdplot_rotated_coords,color=red] (0,0,0) -- (0,\rvec,0);
\draw[line width=0.3mm,->,tdplot_rotated_coords,color=red] (0,0,0) -- (0,0,\rvec) node[above right]{$\z''$};

\end{tikzpicture}
\caption[Two consecutive elementary rotations]{The $\{\x'',\y'',\z''\}$ reference frame (in red) is rotated with respect to $\{\x',\y',\z'\}$ (in blue). The rotation is about the axis $\y'$ by an angle $\phi=-45$~degrees.}
\label{fig:3simpleRotate}
\end{center}
\end{figure}