\documentclass{standalone}
\usepackage{tikz}
\usetikzlibrary{positioning,intersections,calc,fadings,decorations.pathreplacing}


\begin{document}

\begin{tikzpicture}[cross/.style={path picture={ 
  \draw[black,thick]
(path picture bounding box.south east) -- (path picture bounding box.north west) (path picture bounding box.south west) -- (path picture bounding box.north east);
}},label/.style={%
   postaction={ decorate,transform shape,
   decoration={ markings, mark=at position .5 with \node #1;}}}]
\tikzstyle{every node}=[font=\footnotesize]
%%% draw the star camera frame box
\def\starcamerotateangle{25}
\def\inertialrotateangle{30}
\def\telescoperotateangle{0}
\def\sizeStarcamFrame{2}
\def\colortelaxes{\celestialcolor!20}

\def\angledist{1}
%\draw (0,0) rectangle (13,10);
\tikzset{>=latex}

%%% drawing the inertial reference frame
\begin{scope}[shift={(8,7.5)},rotate=\inertialrotateangle]
\draw[->,\celestialcolor] (7,0) -- (-7,0) node[above,pos=0.9,rotate=\inertialrotateangle] {increasing RA} node[below,pos=0.9,rotate=\inertialrotateangle] {line of constant DEC};
\coordinate (RAnode) at (-\angledist,0);
\draw[->,\celestialcolor] (0,-1.5) -- (0,4.5) node[above,pos=0.8,rotate=\inertialrotateangle-90] {increasing DEC} node[below,pos=0.8,rotate=\inertialrotateangle-90] {line of constant RA} ;
\coordinate (DECnode) at (0,\angledist);
%%% drawing the star camera reference frame
\begin{scope}[rotate=-\starcamerotateangle]
\draw[\starcamcolor] (-\sizeStarcamFrame,-\sizeStarcamFrame)  -- (-\sizeStarcamFrame,\sizeStarcamFrame);
\draw[\starcamcolor] (-\sizeStarcamFrame,-\sizeStarcamFrame) -- (\sizeStarcamFrame,-\sizeStarcamFrame);
\draw[\starcamcolor] (\sizeStarcamFrame,\sizeStarcamFrame) -- (-\sizeStarcamFrame,\sizeStarcamFrame);
\draw[\starcamcolor] (\sizeStarcamFrame,\sizeStarcamFrame)node[above left,align=center]{Star camera \\ field of view} -- (\sizeStarcamFrame,-\sizeStarcamFrame);

\node[circle,draw,cross,minimum size=10pt,thick,\starcamcolor] (center) at (0,0) {};
\node[right,align=left,\starcamcolor] at (0.2,-0.1) {$\vectors{x}_\starcam$= star camera \\ boresight vector};
\draw[->,line width=0.2mm,\starcamcolor] (0,0) -- (0,3);
\node[right,\starcamcolor] at (0,3) {$\vectors{z}_\starcam$};
\draw[->,line width=0.2mm,\starcamcolor] (0,0) -- (-3,0);
\node[above,\starcamcolor] at (-3,0) {$\vectors{y}_\starcam$};
\coordinate (znode) at (0,\angledist);
\coordinate (ynode) at (-\angledist,0);
%%% need to rotate angles once more
%\begin{scope}[rotate=\telescoperotateangle]
%\node[circle,fill,draw] (sctarget) at (3.5,2.5) {};
%\node[above,align=center] at (3.5,2.6) {Star camera \\ target};
%\end{scope}

\end{scope}
%%% drawing roll angle
\draw[->,label={[above]{Roll}},\starcamcolor] (DECnode) to [bend left] (znode);
\end{scope}

%%% drawing the star camera reference frame




%%% drawing the telescope's reference frame
\begin{scope}[shift={(7,0)},rotate=\inertialrotateangle]
\begin{scope}[rotate=\telescoperotateangle-\inertialrotateangle]
%\node[circle,draw, fill, minimum size=3pt] (telescope center) at (0,0) {};
\node[circle,cross,draw,minimum size=10pt,thick] (telescope center) at (0,0) {};
\node[right,align=left,rotate=\telescoperotateangle] at (0.2,-0.5) {$\vectors{x}_\tel=$ telescope \\  line of sight};
\draw[->,line width=0.2mm] (0,0) -- (-3,0);
\draw[line width=0.2mm] (0,0) -- (6,0);
%\node[right,rotate=\telescoperotateangle] at (0,4) {$\vectors{z}_\tel$};
\draw[->,line width=0.2mm] (0,0) -- (0,4)node[above right]{$\vectors{z}_\tel$};
\node[below right,rotate=\telescoperotateangle,align=center] at (-3,0) {$\vectors{y}_\tel$ = baseline \\ vector};
%\node[below] at (-3,0) {$\uVec{y, \tel}$};
%%% defining desired target and associated quantities
\node[draw=black,fill=yellow,star,star points=10,minimum size=10pt] (target) at (2.5,3.5) {};
\node[above right,align=center] at (target) {Telescope \\ target};
\coordinate (ztarget) at (0,3.5);
\coordinate (ytarget) at (2.5,0);
\draw[dashed] (target) --  (ztarget) node[below right,near end,rotate=\telescoperotateangle] {$\Delta\crossEl$};
\draw[dashed] (target) -- node[right,near end,rotate=\telescoperotateangle] {$\approx\Delta\El$}  (ytarget);
\draw[fill] (ztarget) circle [radius=0.5mm];
\draw[fill] (ytarget) circle [radius=0.5mm];
%\node[below right] at (ytarget) {$\Delta\xEl=\Delta$xEl};
%\node[below right] at (ztarget) {$\Delta\El=\Delta$El};
\draw[->,line width=0.3mm] (telescope center) -- (target) node [below,midway,sloped,rotate=\telescoperotateangle,align=center] {error vector\\};
\end{scope}
\draw[color=\colortelaxes] (9,0) -- (-2,0);
\draw[color=\colortelaxes] (0,-1.2) -- (0,5);
%\node[circle,draw, fill, minimum size=3pt] (telescope center) at (0,0) {};

%%% find out the coordinates of the target in the rotated reference frame
\path let \p{test} = (target) in coordinate (target_y) at (\x{test},0);
\path let \p{test} = (target) in coordinate (target_z) at (0,\y{test});
\draw[dashed,color=\colortelaxes] (target) -- node[right,rotate=\inertialrotateangle] {$\Delta$DEC} (target_y);
\draw[dashed,color=\colortelaxes] (target) -- node[below,rotate=\inertialrotateangle] {$\Delta$RA} (target_z);
\draw[fill,color=\colortelaxes] (target_y) circle [radius=0.5mm];
\draw[fill,color=\colortelaxes] (target_z) circle [radius=0.5mm];
%\node[right] at (target_y) {$\Delta$RA};
%\node[left] at (target_z) {$\Delta$DEC};
\end{scope}

\end{tikzpicture}

\end{document}