%%%%%%%%%%%%%%%%%%%%%%%%%%%%%%%%%%%%%%%%%
% Masters/Doctoral Thesis 
% LaTeX Template
% Version 2.2 (21/11/15)
%
% This template has been downloaded from:
% http://www.LaTeXTemplates.com
%
% Version 2.x major modifications by:
% Vel (vel@latextemplates.com)
%
% This template is based on a template by:
% Steve Gunn (http://users.ecs.soton.ac.uk/srg/softwaretools/document/templates/)
% Sunil Patel (http://www.sunilpatel.co.uk/thesis-template/)
%
% Template license:
% CC BY-NC-SA 3.0 (http://creativecommons.org/licenses/by-nc-sa/3.0/)
%
%%%%%%%%%%%%%%%%%%%%%%%%%%%%%%%%%%%%%%%%%

%----------------------------------------------------------------------------------------
%	PACKAGES AND OTHER DOCUMENT CONFIGURATIONS
%----------------------------------------------------------------------------------------

\documentclass[
11pt, % The default document font size, options: 10pt, 11pt, 12pt
%oneside, % Two side (alternating margins) for binding by default, uncomment to switch to one side
english, % ngerman for German
doublespacing, % Single line spacing, alternatives: onehalfspacing or doublespacing
%draft, % Uncomment to enable draft mode (no pictures, no links, overfull hboxes indicated)
nolistspacing, % If the document is onehalfspacing or doublespacing, uncomment this to set spacing in lists to single
liststotoc, % Uncomment to add the list of figures/tables/etc to the table of contents
toctotoc, % Uncomment to add the main table of contents to the table of contents
%parskip, % Uncomment to add space between paragraphs
%nohyperref, % Uncomment to not load the hyperref package
headsepline, % Uncomment to get a line under the header
]{MastersDoctoralThesis} % The class file specifying the document structure

\usepackage{aas_macros} % required for bibliography
\usepackage[utf8]{inputenc} % Required for inputting international characters
\usepackage[T1]{fontenc} % Output font encoding for international characters
\usepackage{standalone}
%\usepackage{palatino} % Use the Palatino font by default
%\usepackage{avant}
%\usepackage{newcent}
\usepackage[final]{pdfpages} % for PDF support
\usepackage{hhline}  % fancy lines in tables
\usepackage{tablefootnote}  % footnotes in tables
\usepackage{verbatim} % comments
\usepackage{epigraph} % quotes

\usepackage[titles]{tocloft} % table of contents indents
\renewcommand*\cftchapnumwidth{2em}

\setlength\epigraphwidth{8cm}
\setlength\epigraphrule{0pt}

% to see what textwidth is
\usepackage{printlen}
\usepackage{layouts}

% some figure packages
\usepackage{subcaption}
\captionsetup[subfigure]{labelfont=rm}

% deals with spaces after newcommand
\usepackage{xspace}


\renewcommand\thechapter{\Roman{chapter}}
\renewcommand\thesection{\thechapter.\arabic{section}}
\renewcommand\thesubsection{\thesection.\arabic{subsection}}
\renewcommand\thesubsubsection{\thesubsection.\arabic{subsubsection}}
\renewcommand\theparagraph{\thesubsubsection.\alph{paragraph}}

\makeatletter
     \renewcommand*\l@figure{\@dottedtocline{1}{1em}{3.2em}}
     \renewcommand*\l@table{\@dottedtocline{1}{1em}{3.2em}}
\makeatother

\usepackage[backend=bibtex,style=authoryear,natbib=true,maxcitenames=1]{biblatex} % User the bibtex backend with the authoryear citation style (which resembles APA)

% tikz
\usepackage{tikz,tikz-3dplot}
\usepackage{xifthen} % if then statements
%\usepackage{siunitx} % getting rid of decimal parts
\usepackage{adjustbox} % this is to allow the tikz image to span over two columns
\usepackage[nointegrals]{wasysym}	% For Aries symbol
\usepackage{pgfplots}
\usetikzlibrary{positioning,arrows,calc,intersections,fadings,decorations.pathreplacing,shapes.geometric,decorations.markings}
\usetikzlibrary{shapes,arrows}

\newcommand{\AxisRotator}[1][rotate=0]{%
    \tikz [x=0.25cm,y=0.60cm,line width=.2ex,-stealth,#1] \draw (0,0) arc (-150:150:1 and 1);%
}

%% helper macros
\newcommand\pgfmathsinandcos[3]{%
  \pgfmathsetmacro#1{sin(#3)}%
  \pgfmathsetmacro#2{cos(#3)}%
}
\newcommand\LongitudePlane[3][current plane]{%
  \pgfmathsinandcos\sinEl\cosEl{#2} % elevation
  \pgfmathsinandcos\sint\cost{#3} % azimuth
  \tikzset{#1/.style={cm={\cost,\sint*\sinEl,0,\cosEl,(0,0)}}}
}
\newcommand\LatitudePlane[3][current plane]{%
  \pgfmathsinandcos\sinEl\cosEl{#2} % elevation
  \pgfmathsinandcos\sint\cost{#3} % latitude
  \pgfmathsetmacro\yshift{\cosEl*\sint}
  \tikzset{#1/.style={cm={\cost,0,0,\cost*\sinEl,(0,\yshift)}}} %
}
\newcommand\DrawLongitudeCircleName[3][1]{
  \LongitudePlane{\angEl}{#2}
  \tikzset{current plane/.prefix style={scale=#1}}
   % angle of "visibility"
  \pgfmathsetmacro\angVis{atan(sin(#2)*cos(\angEl)/sin(\angEl))} %
  \draw[name path=#3,current plane] (\angVis:1) arc (\angVis:\angVis+180:1);
  \draw[name path=dashed#3,current plane,dashed] (\angVis-180:1) arc (\angVis-180:\angVis:1);
}
\newcommand\DrawLatitudeCircleName[3][2]{
  \LatitudePlane{\angEl}{#2}
  \tikzset{current plane/.prefix style={scale=#1}}
  \pgfmathsetmacro\sinVis{sin(#2)/cos(#2)*sin(\angEl)/cos(\angEl)}
  % angle of "visibility"
  \pgfmathsetmacro\angVis{asin(min(1,max(\sinVis,-1)))}
  \draw[name path=#3,current plane] (\angVis:1) arc (\angVis:-\angVis-180:1);
  \draw[name path=dashed#3,current plane,dashed] (180-\angVis:1) arc (180-\angVis:\angVis:1);
}

\newcommand\DrawLatitudeCircle[2][1]{
  \LatitudePlane{\angEl}{#2}
  \tikzset{current plane/.prefix style={scale=#1}}
  \pgfmathsetmacro\sinVis{sin(#2)/cos(#2)*sin(\angEl)/cos(\angEl)}
  % angle of "visibility"
  \pgfmathsetmacro\angVis{asin(min(1,max(\sinVis,-1)))}
  \draw[current plane,thin,black] (\angVis:1) arc (\angVis:-\angVis-180:1);
  \draw[current plane,thin,dashed] (180-\angVis:1) arc (180-\angVis:\angVis:1);
}%Defining functions to draw limited latitude circles (for the red mesh)

% Redefine rotation sequence for tikz3d-plot to z-y-x
\newcommand{\tdseteulerxyz}{
  \renewcommand{\tdplotcalctransformrotmain}{%
    %perform some trig for the Euler transformation
      \tdplotsinandcos{\sinalpha}{\cosalpha}{\tdplotalpha}
    \tdplotsinandcos{\sinbeta}{\cosbeta}{\tdplotbeta}
    \tdplotsinandcos{\singamma}{\cosgamma}{\tdplotgamma}
    %
      \tdplotmult{\sasb}{\sinalpha}{\sinbeta}
    \tdplotmult{\sasg}{\sinalpha}{\singamma}
    \tdplotmult{\sasbsg}{\sasb}{\singamma}
    %
      \tdplotmult{\sacb}{\sinalpha}{\cosbeta}
    \tdplotmult{\sacg}{\sinalpha}{\cosgamma}
    \tdplotmult{\sasbcg}{\sasb}{\cosgamma}
    %
      \tdplotmult{\casb}{\cosalpha}{\sinbeta}
    \tdplotmult{\cacb}{\cosalpha}{\cosbeta}
    \tdplotmult{\cacg}{\cosalpha}{\cosgamma}
    \tdplotmult{\casg}{\cosalpha}{\singamma}
    %
      \tdplotmult{\cbsg}{\cosbeta}{\singamma}
    \tdplotmult{\cbcg}{\cosbeta}{\cosgamma}
    %
      \tdplotmult{\casbsg}{\casb}{\singamma}
    \tdplotmult{\casbcg}{\casb}{\cosgamma}
    %
      %determine rotation matrix elements for Euler transformation
      \pgfmathsetmacro{\raaeul}{\cacb}
    \pgfmathsetmacro{\rabeul}{\casbsg - \sacg}
    \pgfmathsetmacro{\raceul}{\sasg + \casbcg}
    \pgfmathsetmacro{\rbaeul}{\sacb}
    \pgfmathsetmacro{\rbbeul}{\sasbsg + \cacg}
    \pgfmathsetmacro{\rbceul}{\sasbcg - \casg}
    \pgfmathsetmacro{\rcaeul}{-\sinbeta}
    \pgfmathsetmacro{\rcbeul}{\cbsg}
    \pgfmathsetmacro{\rcceul}{\cbcg}
  }
}

%% document-wide tikz options and styles

\tikzset{%
  >=latex, % option for nice arrows
  inner sep=0pt,%
  outer sep=1pt,%
  mark coordinate/.style={inner sep=0pt,outer sep=0pt,minimum size=4pt,
  fill=black,circle},%
	sundot/.style={
	fill, color=yellow, circle, inner sep=3.5pt}
}
\def\starcamcolor{purple}
\def\celestialcolor{blue}
\def\gondolacolor{orange}
\def\telcolor{black}






\newcommand{\savedx}{0}
\newcommand{\savedy}{0}
\newcommand{\savedz}{0}
\newcommand{\bettii}[2]%
{   
\pgfmathsetmacro{\boxsize}{#1}
\coordinate (O) at (0,0,0);
\coordinate (Ox) at (#1,0,0);
\coordinate (Oy) at (0,#1,0);
\coordinate (Oz) at (0,0,#1);

\coordinate (a) at (0.5*\boxsize,-4.5*\boxsize,0);
\coordinate (b) at (0.5*\boxsize,4.5*\boxsize,0);
\coordinate (c) at (-0.5*\boxsize,-4.5*\boxsize,0);
\coordinate (d) at (-0.5*\boxsize,4.5*\boxsize,0);
\coordinate (e) at (0.5*\boxsize,-3.5*\boxsize,-\boxsize);
\coordinate (f) at (0.5*\boxsize,3.5*\boxsize,-\boxsize);
\coordinate (g) at (-0.5*\boxsize,-3.5*\boxsize,-\boxsize);
\coordinate (h) at (-0.5*\boxsize,3.5*\boxsize,-\boxsize);
\coordinate (zc1) at (0.5*\boxsize,0.5*\boxsize,\boxsize);
\coordinate (zc2) at (0.5*\boxsize,-0.5*\boxsize,\boxsize);
\coordinate (zc3) at (-0.5*\boxsize,-.5*\boxsize,\boxsize);
\coordinate (zc4) at (-0.5*\boxsize,.5*\boxsize,\boxsize);
\coordinate (Oc1) at (0.5*\boxsize,0.5*\boxsize,0);
\coordinate (Oc2) at (0.5*\boxsize,-0.5*\boxsize,0);
\coordinate (Oc3) at (-0.5*\boxsize,-.5*\boxsize,0);
\coordinate (Oc4) at (-0.5*\boxsize,.5*\boxsize,0);
\coordinate (Bc1) at (0.5*\boxsize,0.5*\boxsize,-\boxsize);
\coordinate (Bc2) at (0.5*\boxsize,-0.5*\boxsize,-\boxsize);
\coordinate (Bc3) at (-0.5*\boxsize,-.5*\boxsize,-\boxsize);
\coordinate (Bc4) at (-0.5*\boxsize,.5*\boxsize,-\boxsize);
\coordinate (Dc1) at (0.5*\boxsize,1.5*\boxsize,0);
\coordinate (Dc2) at (0.5*\boxsize,-1.5*\boxsize,0);
\coordinate (Dc3) at (-0.5*\boxsize,-1.5*\boxsize,0);
\coordinate (Dc4) at (-0.5*\boxsize,1.5*\boxsize,0);




   \draw[thin,#2] (a)--(b) (c)--(d) (a)--(c) (b)--(d) (e)--(f) (g)--(h) (e)--(g) (f)--(h) (a)--(e) (c)--(g) (b)--(f) (d)--(h) (zc1) -- (zc2) (zc2) -- (zc3) (zc3) -- (zc4) (zc4) -- (zc1) (zc1) -- (Dc1) (zc2) -- (Dc2) (zc3) -- (Dc3) (zc4) -- (Dc4) (Dc1) -- (Dc4) (Dc2) -- (Dc3) (zc1) -- (Bc1) (zc2) -- (Bc2) (zc3) -- (Bc3) (zc4) -- (Bc4) (Bc1) -- (Bc4) (Bc2) -- (Bc3) ;
   \draw[thick](zc1) -- (Oc2) (zc4) -- (Oc3) (Oc3) -- (Bc4) (Oc2) -- (Bc1);
%   \draw[->,thick] (O) -- (Ox) node[right] {x};
%   \draw[->,thick] (O) -- (Oy) node[right] {y};
%   \draw[->,thick] (O) -- (Oz) node[right] {z};
%    \fill (a) circle (0.1cm);
%    \fill (d) ++(0.1cm,0.1cm) rectangle ++(-0.2cm,-0.2cm);



}


\tikzset{
%Define standard arrow tip
>=stealth',
%Define style for different line styles
help lines/.style={dashed, thick},
axis/.style={<->},
important line/.style={thick},
connection/.style={thick, dotted},
}



\usepackage{amsmath,amssymb,amstext}
\usepackage{mathtools}
\usepackage{xcolor}
\usepackage{graphicx}
\usepackage{multirow}
\usepackage{pdflscape} % for landscape pages!
% number paragraphs
\setcounter{secnumdepth}{5}
\usepackage{titlesec}
\titleformat{\paragraph}
{\normalfont\normalsize\bfseries}{\theparagraph}{1em}{}
\titlespacing*{\paragraph}
{0pt}{3.25ex plus 1ex minus .2ex}{1.5ex plus .2ex}


%\usepackage{aas_macros}
%\usepackage{natbib}
%\bibliographystyle{apj}
% units
\DeclareSIUnit\deg{deg}
\DeclareSIUnit\arcsec{arcsec}
\DeclareSIUnit\pixel{pixel}
\DeclareSIUnit\jansky{Jy}
\DeclareSIUnit\Jy{Jy}
\DeclareSIUnit\beam{beam}
\DeclareSIUnit\arcmin{arcmin}
\DeclareSIUnit\parsec{pc}
\DeclareSIUnit\pc{pc}
\DeclareSIUnit\au{au}
\DeclareSIUnit\lightyear{ly}
\DeclareSIUnit\year{yr}
\DeclareSIUnit\micron{\micro\meter}
\DeclareSIUnit\um{\micro\meter}
\DeclareSIUnit\Msun{\ensuremath{\textrm{M}_\odot}}
\DeclareSIUnit\Lsun{\ensuremath{\textrm{L}_\odot}}

% nice tables
\usepackage{booktabs}
\usepackage{enumitem}
\usepackage{array}
\newcolumntype{P}[1]{>{\centering\arraybackslash}p{#1}}

%\addbibresource{references.bib} % The filename of the bibliography
\bibliography{references.bib,n2071refs.bib,extrabib.bib}

\usepackage[autostyle=true]{csquotes} % Required to generate language-dependent quotes in the bibliography

%----------------------------------------------------------------------------------------
%	MARGIN SETTINGS
%----------------------------------------------------------------------------------------
\renewcommand{\baselinestretch}{2}
\setlength{\textwidth}{5.9in}
\setlength{\textheight}{9in}
\setlength{\topmargin}{-.50in}
%\setlength{\topmargin}{0in}    %use this setting if the printer makes the the top margin 1/2 inch instead of 1 inch
\setlength{\oddsidemargin}{.55in}
\setlength{\parindent}{.4in}

% \geometry{
% 	paper=letterpaper, % Change to letterpaper for US letter
% 	inner=2.5cm, % Inner margin
% 	outer=3.8cm, % Outer margin
% 	bindingoffset=2cm, % Binding offset
% 	top=1.5cm, % Top margin
% 	bottom=1.5cm, % Bottom margin
% 	%showframe,% show how the type block is set on the page
% }

%----------------------------------------------------------------------------------------
%	THESIS INFORMATION
%----------------------------------------------------------------------------------------

\thesistitle{BETTII: A pathfinder for high angular resolution observations of star-forming regions in the far-infrared} % Your thesis title, this is used in the title and abstract, print it elsewhere with \ttitle
\supervisor{Dr. Lee G.\textsc{Mundy}} % Your supervisor's name, this is used in the title page, print it elsewhere with \supname
\examiner{} % Your examiner's name, this is not currently used anywhere in the template, print it elsewhere with \examname
\degree{Doctor of Philosophy} % Your degree name, this is used in the title page and abstract, print it elsewhere with \degreename
\author{Maxime J. \textsc{Rizzo}} % Your name, this is used in the title page and abstract, print it elsewhere with \authorname
\addresses{} % Your address, this is not currently used anywhere in the template, print it elsewhere with \addressname

\subject{Astronomy} % Your subject area, this is not currently used anywhere in the template, print it elsewhere with \subjectname
\keywords{} % Keywords for your thesis, this is not currently used anywhere in the template, print it elsewhere with \keywordnames
\university{\href{http://www.umd.edu}{University of Maryland, College Park}} % Your university's name and URL, this is used in the title page and abstract, print it elsewhere with \univname
\department{\href{http://department.university.com}{Department of Astronomy}} % Your department's name and URL, this is used in the title page and abstract, print it elsewhere with \deptname
%\group{\href{http://researchgroup.university.com}{Research Group Name}} % Your research group's name and URL, this is used in the title page, print it elsewhere with \groupname
\faculty{\href{http://faculty.university.com}{Faculty Name}} % Your faculty's name and URL, this is used in the title page and abstract, print it elsewhere with \facname

\hypersetup{pdftitle=\ttitle} % Set the PDF's title to your title
\hypersetup{pdfauthor=\authorname} % Set the PDF's author to your name
\hypersetup{pdfkeywords=\keywordnames} % Set the PDF's keywords to your keywords
%\pagestyle{plain} 

\begin{document}

% ------------------------------------------------------------------------------------
% Chapter 1
% ------------------------------------------------------------------------------------

\newcommand{\Spitzer}{\textit{Spitzer}\xspace}
\newcommand{\WISE}{WISE\xspace}
\newcommand{\Herschel}{\textit{Herschel}\xspace}
\newcommand{\JWST}{JWST\xspace}
\newcommand{\OPD}{\ensuremath{\textrm{OPD}}\xspace}
\newcommand{\Lc}{\ensuremath{L_c}\xspace}


\newlength\mylen
\settowidth\mylen{\textbullet}
\addtolength\mylen{-3mm}

\newcommand{\Pext}{\ensuremath{P_\textrm{ext}}\xspace}
\newcommand{\Mcrit}{\ensuremath{M_\textrm{crit}}\xspace}
\newcommand{\tff}{\ensuremath{\textrm{t}_\textrm{ff}}\xspace}
\newcommand{\ts}{\ensuremath{\textrm{t}_\textrm{s}}\xspace}
\newcommand{\cs}{\ensuremath{c_\textrm{s}}\xspace}
\newcommand{\lambdaJ}{\ensuremath{\lambda_\textrm{J}}\xspace}
\newcommand{\MJ}{\ensuremath{M_\textrm{J}}\xspace}

\newcommand{\foverlap}{\ensuremath{f_\textrm{overlap}}\xspace}
\newcommand{\Vloss}{\ensuremath{V_\textrm{loss}}\xspace}
\newcommand{\sigWFE}{\ensuremath{\sigma_\textrm{WFE}}\xspace}
\newcommand{\sigOPD}{\ensuremath{\sigma_\textrm{OPD}}\xspace}
\newcommand{\sigtt}{\ensuremath{\sigma_\textrm{tt}}\xspace}
\newcommand{\renv}{\ensuremath{r_\textrm{env}}\xspace}
\newcommand{\Menv}{\ensuremath{M_\textrm{env}}\xspace}
\newcommand{\Mdotenv}{\ensuremath{\dot{M}_\textrm{env}}\xspace}
\newcommand{\Lbol}{\ensuremath{L_\textrm{bol}}\xspace}
\newcommand{\Tbol}{\ensuremath{T_\textrm{bol}}\xspace}
\newcommand{\Av}{\ensuremath{A_V}\xspace}

\newcommand{\Knu}{\ensuremath{\kappa_\nu}\xspace}
\newcommand{\Bnu}{\ensuremath{B_\nu}\xspace}
\newcommand{\Fnu}{\ensuremath{F_\nu}\xspace}
\newcommand{\Snu}{\ensuremath{S_\nu}\xspace}
\newcommand{\ext}{\ensuremath{\textrm{ext}}\xspace}
\newcommand{\Cext}{\ensuremath{C_\ext}\xspace}
\newcommand{\dust}{\ensuremath{\textrm{dust}}\xspace}

\newcommand{\QE}{\ensuremath{\textrm{QE}}\xspace}
\newcommand{\MDLF}{\ensuremath{\textrm{MDLF}}\xspace}
\newcommand{\MDFD}{\ensuremath{\textrm{MDFD}}\xspace}
\newcommand{\Tint}{\ensuremath{T_\textrm{int}}\xspace}
\newcommand{\Nscan}{\ensuremath{N_\textrm{scan}}\xspace}
\newcommand{\Smin}{\ensuremath{S_\textrm{min}}\xspace}
\newcommand{\Rnir}{\ensuremath{R_\textrm{NIR}}\xspace}
\newcommand{\sigRON}{\ensuremath{\sigma_\textrm{RON}}\xspace}
\newcommand{\Nph}{\ensuremath{N_\textrm{ph}}\xspace}
\newcommand{\Nel}{\ensuremath{N_{\textrm{e}^-}}\xspace}
\newcommand{\FBW}{\ensuremath{\textrm{FBW}}\xspace}


\newcommand{\sigman}{\ensuremath{\sigma^\textrm{man}}\xspace}
\newcommand{\sigstd}{\ensuremath{\sigma^\textrm{std}}\xspace}
\newcommand{\sigth}{\ensuremath{\sigma^\textrm{th}}\xspace}

\newcommand{\Rpercent}{\ensuremath{R_{\%}}\xspace}
\newcommand{\Rfifty}{\ensuremath{R_{50}}\xspace}

\newcommand{\Rstar}{\ensuremath{R_{\star}}\xspace}
\newcommand{\Fmod}{\ensuremath{F_\textrm{mod}}\xspace}
\newcommand{\Fobs}{\ensuremath{F_\textrm{obs}}\xspace}
\newcommand{\Mstar}{\ensuremath{M_{\star}}\xspace}
\newcommand{\Tstar}{\ensuremath{T_{\star}}\xspace}
\newcommand{\Lstar}{\ensuremath{L_{\star}}\xspace}
\newcommand{\Rdiskmax}{\ensuremath{R_\textrm{disk}^\textrm{max}}\xspace}
\newcommand{\Rdiskmin}{\ensuremath{R_\textrm{disk}^\textrm{min}}\xspace}
\newcommand{\Mdisk}{\ensuremath{M_\textrm{disk}}\xspace}
\newcommand{\Renvmax}{\ensuremath{R_\textrm{env}^\textrm{max}}\xspace}
\newcommand{\Renvmin}{\ensuremath{R_\textrm{env}^\textrm{min}}\xspace}


% ------------------------------------------------------------------------------------
% Chapter 2
% ------------------------------------------------------------------------------------

% Units
\def\um{\si{\micro\meter}}
%\newcommand{\um}{\si{\micro\meter}}

\def\asec{\textrm{arcsec}}
\def\cm1{\ensuremath{\textrm{cm}^{-1}}}

%angle definitonis
\def\shat0{\hat{\textbf{s}}_0}
%\def\Ds{\Delta\textbf{s}}
\def\Ds{\xi}
\def\urad{\ensuremath{\mu\textrm{rad}}}

% Interferogram notations
\def\samp{s}
\def\baseline{\mathbf{B}}
\def\Bspec{B}
\def\F{\mathcal{F}}
\def\I{\textit{I}}
\def\Im{\I_\textrm{measured}}
\def\mI{\mathcal{I}}
\def\xn{x_n}
\def\xnp{x_{n'}}
\def\pxn{(\xn)}
\def\pxnp{(\xnp)}
\def\Ixn{\mI\pxn}
\def\Ikxn{\mI_k\pxn}
\def\Iksxn{\mI_k^{*}\pxnp}
\def\Vb{\mathcal{V}_\baseline}
\def\Phib{\Phi_\mathbf{B}}
\def\Phibp{\Phi_\textrm{bp}}
\def\Vi{\mathcal{V}_\textrm{i}}
\def\Phii{\Phi_\mathbf{i}}
\def\V{\mathcal{V}}
\def\S{\mathcal{S}}
\def\R{\mathcal{R}}
\def\oS{\overline{\S}}
\def\etaBS{\eta_{\textrm{b}}}
\def\etamf{\eta_\textrm{mf}}
\def\etao{\eta_o}
\def\s{\sigma}
\def\sig{\Delta}
\def\ps{(\s)}
\def\pms{(-\s)}
\def\Be{\B_e}
\def\Phir{\Phi_r}
\def\etaD{\eta_D}
\def\Tbp{\mathcal{T}_\textrm{bp}}
\def\Area{\mathcal{A}}
\def\F{\mathcal{F}}
\def\Ahat{\hat{\mathcal{A}}}
\def\Dx{dx}
\def\Dt{dt}
\def\D{d}
\def\intinf{\int_{-\infty}^{+\infty}}
\def\sinc{\textrm{sinc}}
\def\real{\textrm{Re}}
\def\imag{\textrm{Im}}
\def\FT{\mathcal{F}}
\def\DFT{\mathbf{DFT}}
\def\dsig{\delta\s}
\def\varPhir{\sig^2_\Phi}
\def\sigPhir{\sig_\Phi}
\def\Dopd{\sig_{\textrm{OPD}}}
\def\varopd{\sig^2_{\textrm{OPD}}}
\def\wk{w_k}
\def\O{\mathcal{O}}

% Noise definitons
\def\SNR{\ensuremath{\textrm{SNR}}\xspace}
\def\SNRnp{\SNR_\textrm{np}}
\def\ni{n_\mI}
\def\nA{n_+}
\def\nB{n_-}
\def\NEP{\textrm{NEP}}
\def\NEPtot{\NEP_\textrm{tot}}
\def\NEPph{\NEP_\textrm{ph}}
\def\NEPdet{\NEP_\textrm{det}}
\def\NEPsou{\NEP_\textrm{sou}}
\def\VAR{\mathbf{VAR}}
\def\varI{\sig_\mI^2}
\def\sigI{\sig_\mI}
\def\varspec{\sig_\mathcal{S}^2}
\def\varspectot{\sig^2_{\mathcal{S}\textrm{,tot}}}
\def\sigspec{\sig_\mathcal{S}}
\def\sigk{\s_k}
\def\mK{\mathcal{K}}
\def\sigalpha{\sig_\alpha}
\def\sigL{\sig_L}



\def\delay{\delta}
\def\delaytot{\delay_\textrm{tot}}
\def\inst{\textrm{inst}}
\def\col{\textrm{col}}

\def\delayint{\delay_\textrm{int}}
\def\delayext{\delay_\textrm{ext}}
\def\thetacrossel{\theta_\textrm{cross-El}}






% ------------------------------------------------------------------------------------
% Chapter 3
% ------------------------------------------------------------------------------------

% change spacing in equations
%\newenvironment{equations}{\begin{doublespace}\begin{equation}}{\end{equation}\end{doublespace}}
%\newenvironment{eqnarrays}{\begin{doublespace}\begin{eqnarray}}{\end{eqnarray}\end{doublespace}}
\newenvironment{equations}{\renewcommand*{\arraystretch}{0.75}\begin{equation}}{\end{equation}}
\newenvironment{eqnarrays}{\renewcommand*{\arraystretch}{0.75}\begin{eqnarray}}{\end{eqnarray}}

\def\Az{\ensuremath{\phi_\textrm{Az}}}
\def\dAz{\ensuremath{\dot{\phi}_\textrm{Az}}}
\def\El{\ensuremath{\phi_\textrm{El}}}
\def\crossEl{\ensuremath{\phi_{\times\textrm{El}}}}
\def\dOPD{\ensuremath{\dot{\textrm{OPD}}}}

% Reference frames
\def\G{\textrm{\textbf{G}}}
\def\L{\textrm{\textbf{L}}}
\def\X{\textrm{\textbf{I}}}
\def\Y{\textrm{\textbf{J}}}
\def\Z{\textrm{\textbf{K}}}
\def\x{\textrm{\textbf{i}}}
\def\y{\textrm{\textbf{j}}}
\def\z{\textrm{\textbf{k}}}
\def\I{\textrm{\textbf{I}}}
\def\J{\textrm{\textbf{J}}}
\def\K{\textrm{\textbf{K}}}
\def\i{\textrm{\textbf{i}}}
\def\j{\textrm{\textbf{j}}}
\def\k{\textrm{\textbf{k}}}
\def\M{\textrm{\textbf{M}}}
%\def\R{\textrm{\textbf{R}}}
\def\RealNumbers{\boldsymbol{\mathcal{R}}}

% General definitions
\def\target{\textrm{target}}
\def\dt{\ensuremath{\textrm{dt}}}
\def\Deltat{\ensuremath{\Delta t}}
\def\sig{\ensuremath{\sigma}}
\newcommand{\Ang}[3]{{#1}^{#2}_{#3}} % angle symbol, raw/true/new (superscript), sensor symbol (subscript)
\newcommand{\hatAng}[3]{\hat{{#1}}^{#2}_{#3}} % 
\newcommand{\dotAng}[3]{\dot{{#1}}^{#2}_{#3}} % 
\newcommand{\hatdotAng}[3]{\hat{\dot{{#1}}}^{#2}_{#3}} % 
\newcommand{\angvar}[1]{\sig^2_{#1}} % angle symbol
\newcommand{\invvar}[1]{\sig^{-2}_{#1}} % angle symbol
\newcommand{\at}[1]{|_{#1}} % this adds the loop iteration number
\newcommand{\on}[1]{^{#1}} % to indicate the loop at which the data is received

\newcommand{\rotaxis}[2] % the angle, the axis of rotation
{\mathcal{R}(#1,#2)}
\newcommand{\rotmat}[2] % First argument is where we start (the vector that this matrix  multiplies is in that coordinate system), second argument is where we end up (resulting vector is in that coordinate system)
{\mathcal{R}|_{#1}^{#2}}

% rotations
%\newcommand{\fromto}[2]{\prescript{#1}{#2}}
\newcommand{\fromto}[2]{{}_{#1}^{#2}}



\def\urad{\mu\textrm{rad}}
\def\xEl{\Phi}
%\def\El{\Psi}
\def\xElHat{\hatAng{\xEl}{}{}}
\def\varxElHat{\angvar{\hatAng{\xEl}{}{}}}
\def\invVarxElHat{\invvar{\hatAng{\xEl}{}{}}}

% Star camera definitions
\def\new{\textrm{new}}
\def\raw{\textrm{raw}}
\def\true{\textrm{true}}
\def\starcam{\textrm{sc}}
\def\xElsc{\ang{\xEl}{}{\starcam}}
\def\xElnew{\ang{\xEl}{}{\starcam}}
\def\varxElnew{\var{\ang{\xEl}{}{\starcam}}}
\def\invVarxElnew{\invvar{\ang{\xEl}{}{\starcam}}}

% telescope definitions
\def\tel{\textrm{tel}}

% Gyro definitions
\def\gyro{\textrm{g}}
\def\xElGyro{\ang{\xEl}{}{\gyro}}
\def\vxEl{\dotAng{\xEl}{}{}}
\def\vxElGyro{\hatdotAng{\xEl}{}{\gyro}}
\def\vxElGyroRaw{\dotAng{\xEl}{}{\gyro}}
\def\varVxElGyro{\angvar{\vxElGyro}}
\def\gyroBias{\hat{b}}
\def\newGyroBias{b}
\def\gyroVec{\boldsymbol\omega}
\def\EstGyroVec{\hat{\gyroVec}}
\def\gyroVecMeas{\gyroVec^\textrm{meas}}
\def\matOmega{\boldsymbol\Omega}

% fine guidance sensor defitions
\def\fine{\textrm{fgs}}

% Vectors
\newcommand{\units}[1]{{}_{[\textrm{#1}]}}
\newcommand{\vectors}[1]{\textrm{\textbf{#1}}}
\newcommand{\vect}[3]{ \begin{bmatrix} #1 \\ #2 \\ #3 \end{bmatrix}}
\newcommand{\vecttwod}[2]{ \begin{bmatrix} #1 \\ #2 \end{bmatrix}}
\newcommand{\mattwod}[4]{ \begin{bmatrix} #1 & #2 \\ #3 & #4 \end{bmatrix}}
\newcommand{\cov}[1]{\textrm{cov}\left[ #1 \right]}
\def\idmat{\boldsymbol{\mathcal{I}}}

\def\HRG{H1RG}

% Mechanical
\def\inertia{\textrm{\textbf{J}}}
\def\inertial{I}
\def\inertiaVec{\textbf{\inertia}}
\def\angpos{\theta}
\def\angvel{\omega}
\def\angvelvec{\vectors{\angvel}}
\def\angvelvecdot{\dot{\vectors{\angvel}}}
\def\angposvec{\vectors{\angpos}}
\newcommand{\uVec}[1]{\vectors{u}_{#1}}
\def\Yaw{\psi}
\def\Pitch{\theta}
\def\Roll{\phi}

% azimuth & CCMG
\def\CCMG{\textrm{CCMG}}
\def\torque{\mathcal{T}}
\def\MCCMGz{M_{\CCMG,z}}
\def\MCCMG{M_{\CCMG}}
\def\velstepper{\textrm{n}}
\def\thetaz{\theta_z}
\def\exttorques{\torque_{\textrm{ext}}}
\def\ccmgtorque{\torque_{\CCMG}}
\def\momdumptorque{\torque_{\textrm{M. Dump}}}

\def\Kp{\textrm{K}_\textrm{p}}
\def\Kd{\textrm{K}_\textrm{d}}
\def\Ki{\textrm{K}_\textrm{i}}

% Kalman filter
\newcommand{\quat}[1]{\bar{#1}}
\newcommand{\quatVec}[1]{\boldsymbol{#1}}
\def\state{x}
\def\stateVec{\vectors{\state}}
\def\EstStateVec{\hat{\stateVec}}
\def\attitudeLetter{q}
\def\Attitude{\quat{\attitudeLetter}}
\def\EstQuatVec{\hat{\quatVec{\attitudeLetter}}}
\def\EstDeltaQuatVec{\hat{\delta\quatVec{\attitudeLetter}}}
\def\DeltaQuatVec{\delta\quatVec{\attitudeLetter}}
\def\EstAttitude{\hat{\Attitude}}
\def\dotAttitude{\dot{\Attitude}}
\def\dotEstAttitude{\dot{\EstAttitude}}
\def\bias{\vectors{b}}
\def\EstBias{\hat{\bias}}
\def\nBias{\textbf{n}_b}
\def\nGyros{\textbf{n}_g}
\def\nStarcam{\textbf{n}^{SC}}
\def\nProcess{\textbf{n}^\textrm{process}}
\def\nMeas{\textbf{n}^\textrm{meas}}
\def\EstErrorState{\hat{\tilde{\stateVec}}}
\def\ErrorState{\tilde{\stateVec}}
\def\deltaTheta{\delta\boldsymbol\theta}
\def\deltaBias{\Delta\bias}
\def\EstDeltaTheta{\hat{\deltaTheta}}
\def\EstDeltaBias{\hat{\deltaBias}}
\def\uCommand{\vectors{u}}
\def\wNoise{\vectors{w}}
\def\vNoise{\vectors{v}}
\def\zMeasurement{\tilde{\vectors{z}}}
\def\StateTransitionMat{\boldsymbol{\Phi}}
\def\Fc{\textrm{\textbf{F}}}
\def\Gc{\textrm{\textbf{G}}}
\def\bzero{\boldsymbol{0}}
\def\bI{\boldsymbol{I}}
\def\omegaCross{\lfloor \EstGyroVec_\times \rfloor}
\def\measErrMat{\textrm{\textbf{H}}}
\def\noiseCovMat{\textrm{\textbf{Q}}}
\def\N{\textrm{\textbf{N}}}
\def\stateCovMat{\textrm{\textbf{P}}}
\def\measCovMat{\textrm{\textbf{R}}}
\def\measErrCovMat{\textrm{\textbf{S}}}
\def\KalmanGain{\textrm{\textbf{K}}}
\def\A{\textrm{\textbf{A}}}
\def\B{\textrm{\textbf{B}}}
\def\C{\textrm{\textbf{C}}}
\def\thetaMat{\boldsymbol\Theta}

% modes
\def\SAFE{\textrm{SAFE}}
\def\BRAKE{\textrm{BRAKE}}
\def\SLEW{\textrm{SLEW}}
\def\TRACK{\textrm{TRACK}}
\def\ACQUIRE{\textrm{ACQUIRE}}
\def\LOCKED{\textrm{LOCKED}}

\newcommand{\boop}{\ensuremath{boop} }
\newcommand{\boopRT}{\ensuremath{boop_\textrm{RT}} }
\newcommand{\boopFPGA}{\ensuremath{boop_\textrm{FPGA}} }
\newcommand{\ford}{\ensuremath{ford} }
\newcommand{\heartbeat}{\SI{100}{\hertz}}


\def\gyroVecSC{\gyroVec^\starcam}
\def\gyroVecMeas{\gyroVec^\textrm{meas}}

%%%%% Chapter 4

%\def\Lsun{L_\odot}
%\def\Msun{M_\odot}
%\def\Rsun{R_\odot}
%\def\Rstar{R_\star}





%----------------------------------------------------------------------------------------
%	ABSTRACT PAGE
%----------------------------------------------------------------------------------------
\thispagestyle{empty}
\hbox{\ }

\renewcommand{\baselinestretch}{1}
\small \normalsize

\begin{center}
\large{{ABSTRACT}} 

\vspace{3em} 

\end{center}
\hspace{-.15in}
\begin{tabular}{ll}
Title of dissertation:    & {\large  BETTII: A PATHFINDER FOR HIGH ANGULAR }\\
&				      {\large  RESOLUTION OBSERVATIONS OF STAR-FORMING } \\
&				      {\large REGIONS IN THE FAR-INFRARED} \\
\ \\
&                          {\large  Maxime J. Rizzo, Doctor of Philosophy, 2016} \\
\ \\
Dissertation directed by: & {\large  Professor Lee Mundy, Department of Astronomy} \\
&  				{\large	 and Dr. Stephen Rinehart, NASA GSFC } \\
\end{tabular}

\vspace{3em}

\renewcommand{\baselinestretch}{2}
\large \normalsize

In this thesis, we study clustered star formation in nearby star clusters and discuss how high angular resolution observations in the far-infrared regime could help us understand these important regions of stellar birth. We use the increased angular resolution from the FORCAST instrument on the SOFIA airborne observatory to study 10 nearby star-forming regions, and discuss the physical properties of sources in these regions that we can infer from radiative transfer modeling using these new observations. We discuss the design of BETTII, a pathfinder balloon-borne interferometer which will provide significantly better angular resolution in the far-infrared regime, and pave the way for future space-borne observatories. We elaborate on the details of BETTII's core technique, called Double-Fourier interferometry, and how to accurately compute the sensitivity of instruments which use this technique. Finally, we show our design and implementation results of the control system of the BETTII payload, which, as the first flying interferometer, proposes unique challenges.

%----------------------------------------------------------------------------------------
%	TITLE PAGE
%----------------------------------------------------------------------------------------

% \begin{titlepage}
% \begin{center}

% \textsc{\LARGE \univname}\\[1.5cm] % University name
% \textsc{\Large Doctoral Thesis}\\[0.5cm] % Thesis type

% \HRule \\[0.4cm] % Horizontal line
% {\huge \bfseries \ttitle}\\[0.4cm] % Thesis title
% \HRule \\[1.5cm] % Horizontal line
 
% \begin{minipage}{0.4\textwidth}
% \begin{flushleft} \large
% \emph{Author:}\\
% \href{http://www.johnsmith.com}{\authorname} % Author name - remove the \href bracket to remove the link
% \end{flushleft}
% \end{minipage}
% \begin{minipage}{0.4\textwidth}
% \begin{flushright} \large
% \emph{Supervisors:} \\
% {\supname \\
% Dr. Stephen A. Rinehart} % Supervisor name - remove the \href bracket to remove the link  

% \end{flushright}
% \end{minipage}\\[3cm]
 
% \large \textit{A thesis submitted in fulfillment of the requirements\\ for the degree of \degreename}\\[0.3cm] % University requirement text
% \textit{in the}\\[0.4cm]
% \deptname\\[2cm] % Research group name and department name
 
% {\large \today}\\[4cm] % Date
% %\includegraphics{Logo} % University/department logo - uncomment to place it
 
% \vfill
%\end{center}
% \end{titlepage}

 \begin{titlepage}

%\thispagestyle{empty}
\hbox{\ }
\vspace{1in}
\renewcommand{\baselinestretch}{1}
\small\normalsize
\begin{center}

\large{\ttitle }\\

% Strastospheric instrumentation towards high angular resolution in the far-infrared
% Balloon-borne instrumentation towards far-infrared, free-flying interferometry
\ \\
\ \\
\large{by} \\
\ \\
\large{Maxime J. Rizzo}%Your full name as it appears in University records.
\ \\
\ \\
\ \\
\ \\
\normalsize
Dissertation submitted to the Faculty of the Graduate School of the \\
University of Maryland, College Park in partial fulfillment \\
of the requirements for the degree of \\
Doctor of Philosophy \\
2016

\vspace{7.5em}
\end{center}

\noindent Advisory Committee: \\
Professor Lee G. Mundy, Chair/Advisor (UMD)\\
Dr. Stephen A. Rinehart, Co-Advisor (NASA GSFC)\\
Professor Andrew Harris (UMD)\\
Dr. Alison Nordt (Lockheed Martin)\\
Dr. Mark Wolfire (UMD)\\
Professor Eun-Suk Seo, Graduate Dean's representative (UMD)

\end{titlepage}

\frontmatter % Use roman page numbering style (i, ii, iii, iv...) for the pre-content pages

%----------------------------------------------------------------------------------------
%	DEDICATION
%----------------------------------------------------------------------------------------

\dedicatory{To Michelle, my parents, and my brother.} 


%----------------------------------------------------------------------------------------
%	ACKNOWLEDGEMENTS
%----------------------------------------------------------------------------------------

\begin{acknowledgements}
\addchaptertocentry{\acknowledgementname} % Add the acknowledgements to the table of contents

First and foremost, I would like to thank my advisors Lee~Mundy and Stephen~Rinehart for continuously supporting me during this thesis. Their guidance, knowledge and the trust they have put in me have been essential to the success of this work. Thanks for allowing me to do my first steps as a young instrument scientist, and giving me the opportunity to work on the BETTII project since day one. Special thanks to Aki Roberge for her continued mentorship, and to Richard Lyon, who originally inspired into getting my Ph.D. during my first internship at Goddard in 2008. I would also like to thank Andrew Harris, Alison Nordt, Eun-Suk Seo, and Mark Wolfire for accepting to serve on my thesis committee.

Much of what is written in this thesis is a result of discussions and reflections with members of the BETTII team, who have been supporting me during my entire graduate work. On the technical side, Dale Fixsen and Dominic Benford have inspired me tremendously with their all-around knowledge of physics and engineering and how they apply to astronomical instrumentation. I thank the rest of the team for the respect and trust they have given me: David Leisawitz, Stephen Maher, Eric Mentzell, Elmer Sharp, Robert Silverberg, Johannes Staguhn; but also Arnab Dhabal, Jordi Vila, Todd Veach and Roser Juanola-Parramon who, as graduate students and postdocs, have shared this wonderful experience with me. This extends to the entire Goddard community which has always been a friendly and welcoming group to work with.

Working on BETTII would not have been the same without the community of the UMD Department of Astronomy. I shared my experience with wonderful classmates, who always provided friendship, technical assistance, and support when needed. I especially want to thank Vicki Toy and John Capone for being my Goddard crew, and of course my office mate Taro Shimizu for his patience and assistance with some aspects of my work. In addition, I wanted to thank my professors for always having their door open for my questions, in particular Richard Mushotzky, and Alberto Bolatto. Finally, I have had a fantastic experience with the research scientists and staff of the department. In particular I wanted to thank Peter Teuben and Adrienne Newman for their patience in assisting me all the time.

I am grateful for my friends and family in France, in the U.S., and elsewhere who have continually provided encouragements and support, despite the long distances that often separate us in this world that still doesn't seem connected enough. I want to thank especially my parents and my brother, who have been strong supporters of this endeavor. Thanks to Pauline Simonet, Jean-Baptiste Vallart, and the rest of my SUPAERO crew with who I learned so much. Thanks to Simon Oudin and Adrien Pavageau for those special moments together. Thanks to my old friends in Alsace, C\'{e}dric \& No\'{e}mie, Thibaut \& Jen, Adrien \& Aur\'{e}, and Ad\'{e}, for holding the fort and never missing an occasion to get together when I come home. 

Last, but not least, no one deserves more gratitude than my wife Michelle, who has given me her unconditional support, relentless encouragements, and has always been here for me since the day I met her. Michelle, this work would have been impossible without you.


\end{acknowledgements}

%----------------------------------------------------------------------------------------
%	QUOTATION PAGE
%----------------------------------------------------------------------------------------

%\vspace*{0.2\textheight}
%
%\noindent\enquote{\itshape Thanks to my solid academic training, today I can write hundreds of words on virtually any topic without possessing a shred of information, which is how I got a good job in journalism.}\bigbreak
%
%\hfill Dave Barry


%----------------------------------------------------------------------------------------
%	LIST OF CONTENTS/FIGURES/TABLES PAGES
%----------------------------------------------------------------------------------------

\tableofcontents % Prints the main table of contents

\listoftables % Prints the list of tables

\listoffigures % Prints the list of figures


%----------------------------------------------------------------------------------------
%	ABBREVIATIONS
%----------------------------------------------------------------------------------------

%\begin{abbreviations}{ll} % Include a list of abbreviations (a table of two columns)
%
%\textbf{LAH} & \textbf{L}ist \textbf{A}bbreviations \textbf{H}ere\\
%\textbf{WSF} & \textbf{W}hat (it) \textbf{S}tands \textbf{F}or\\
%
%\end{abbreviations}

%----------------------------------------------------------------------------------------
%	PHYSICAL CONSTANTS/OTHER DEFINITIONS
%----------------------------------------------------------------------------------------

%\begin{constants}{lr@{${}={}$}l} % The list of physical constants is a three column table

% The \SI{}{} command is provided by the siunitx package, see its documentation for instructions on how to use it

%	Speed of Light & $c_{0}$ & \SI{2.99792458e8}{\meter\per\second} (exact)\\
%Constant Name & $Symbol$ & $Constant Value$ with units\\

%\end{constants}

%----------------------------------------------------------------------------------------
%	SYMBOLS
%----------------------------------------------------------------------------------------

%\begin{symbols}{lll} % Include a list of Symbols (a three column table)

%$a$ & distance & \si{\meter} \\
%$P$ & power & \si{\watt} (\si{\joule\per\second}) \\
%Symbol & Name & Unit \\

%\addlinespace % Gap to separate the Roman symbols from the Greek

%$\omega$ & angular frequency & \si{\radian} \\

%\end{symbols}


%----------------------------------------------------------------------------------------
%	THESIS CONTENT - CHAPTERS
%----------------------------------------------------------------------------------------

\mainmatter % Begin numeric (1,2,3...) page numbering

\pagestyle{thesis} % Return the page headers back to the "thesis" style

% Include the chapters of the thesis as separate files from the Chapters folder
% Uncomment the lines as you write the chapters
% loads up the file with all the definitions
%% Chapter 1
\chapter*{Introduction}
\label{chap:introduction}

The work presented in this thesis is centered around the design, development, and testing of an astronomical balloon-borne telescope called BETTII: the Balloon Experimental Twin Telescope for Infrared Interferometry. Developed at NASA Goddard Space Flight Center, this instrument is exploring a new observation technique called "Double-Fourier" interferometry, which could lead to future space-borne telescopes with very high angular resolution in the far-infrared regime. Various fields in astronomy would benefit from such enhanced capability, as demonstrated by the success of far-infrared single-aperture telescopes such as \WISE, \Spitzer and \Herschel. 

More than just a pathfinder, BETTII is a scientific instrument in its own right. For its first flights, it will study regions of clustered star formation with unprecedented details, providing almost an order of magnitude better spatial resolution than any existing or past far-IR facility.

This thesis describes some aspects of my involvement with BETTII as well as my contributions to the scientific field of clustered star formation using the only existing far-IR facility, SOFIA. The document is organized as follows:
\begin{itemize}
\item Chapter~\ref{chap:StarFormation} describes the framework and current understanding of how stars are forming in clusters, and lays out the key tools that we use to study these regions.
\item Chapter~\ref{chap:SOFIA} is a study of nearby star-forming clusters using new data that we obtained with the SOFIA observatory. SOFIA offers moderately high angular resolution, which we attempt to use to improve the study of the brightest, densest regions of star formation. This work is to be submitted for publication shortly after the conclusion of this dissertation. 
\item Chapter~\ref{chap:BETTII} describes the physical principles of interferometry which drive the design of the balloon instrument. We discuss how sensitive the instrument will be and identify scientific targets and calibrators that are suitable for our first flights.
\item Chapter~\ref{chap:phasenoisepaper} is a standalone, refereed publication that was published in 2015 on the spectral sensitivity of double-Fourier interferometers in general. It proposes a mathematical framework to analyze the sensitivity of such instruments to various types of noise sources. We apply those findings to the case of BETTII.
\item Chapter~\ref{chap:controls} discusses the design of the control system for BETTII, which presents unique challenges compared to any other balloon-borne instrument. We also discuss the controls algorithm that is used in flight to properly estimate the orientation of the payload, a key requirement to achieve successful interferometry.
\item Chapter~\ref{chap:implementation} shows results of the implementation that we discusses in the previous chapter. This consists of laboratory and on-sky testing of the observatory. We discuss the expected performance at float.
\item Chapter~\ref{chap:conclusion} summarizes our findings and discusses the path forward for the BETTII project.
\end{itemize}


\chapter{Star formation in clustered environments} % Main chapter title

\label{chap:StarFormation}

%----------------------------------------------------------------------------------------


\section{Molecular Clouds}

Molecular clouds are the dense regions of the interstellar medium (ISM) where stars are forming. They contain about half the mass of the ISM in $<2$\% of its volume \citep{Ferriere:2001gv}. High densities ($n > \SI{100}{\centi\meter\raiseto{-3}}$) of mostly molecular hydrogen and low temperatures ($< \SI{20}{\kelvin}$) distinguish molecular clouds from the various other components of the ISM in galaxies: the Warm Neutral Medium (WNM), the Warm Ionized Medium (WIM), and the other cold phase of the ISM, the Cold Neutral Medium (CNM), which is thought to be the parent region in which molecular clouds are formed \citep{Kennicutt:2012ey}. In addition to molecular hydrogen, molecular clouds also contain Helium (10\% by number), dust ($\sim 1\%$ by mass), CO ($\sim \num{1e-4}$ by number), and traces of many other molecules.

Observations of molecular clouds reveal that they are highly structured with often a filamentary pattern [cite shaye?]. While the literature proposes multiple classifications for the various structures found in molecular clouds, we choose to focus only on two structures which are key to this work: clusters, which are more local associations of stars in virial equilibrium \citep{Lada:2003il}; and dense cores, which are sites where stars form individually or in systems of small multiples \citep{Williams:2000wl}. Clusters are formed of multiple cores, but cores can also be found outside of clusters, in the field. In the classical picture, clouds are thought to fragment into clusters, which still contain many times the Jeans' mass - the minimum mass for gravitationally-bound cores \citep{Larson:1994cj} which are also called prestellar cores \citep{DiFrancesco:2007vg}.

%[The structure of clusters itself is of great interest to study the mechanisms at stake in star formation, because it gives information about the cluster's origins \citep{Lada:2003il}. Clusters are believed to be relatively short-lived, because the stars it forms can disrupt its gravitational equilibrium. Typical timescales for clusters are on the order of \SI{100}{\mega\year} \citep{Lada:2003il}. ]

Approximately 60\% of all stars are thought to form in embedded, young stellar clusters of 1-\SI{3}{\mega\year} with 100 or more stars \citep{Porras:2003kxa,Allen:2007wqa}. These $>$100 star clusters have characteristic sizes of 2-4 parsecs (pc) with peak surface densities of 100-\si{1000} stars per square parsec and a typical median distance between nearest neighbor young stellar objects (YSOs) of 0.03 to 0.06 pc \citep{Gutermuth:2009gca}.

Because star-forming clusters are surrounded by interstellar matter from the parent molecular cloud, they usually cannot be studied at optical wavelengths, due to the large obscuration from dust grains along the line of sight. Infrared observations can be used to probe these structures since the dust can acquire sufficient temperature to emit thermally from the mid-infrared to millimeter and radio wavelengths. 

The high density of YSOs within clusters, combined with their typical separations of few hundredths of parsecs requires a high angular resolution in order to capture the relevant spatial scales at which accretion mechanisms are occurring to give the star its final mass.




\section{Star formation}

\subsection{Standard models}

A considerable amount of literature exists on star formation and the various physical processes involved in forming stars. In this section, we review some of the most standard views that describe how stars are born and grow to acquire their final masses.

\subsubsection{Gravitational collapse}

A simple way to derive characteristic quantities related to the formation of stars is to consider a pre-stellar core as a spherical clump of uniform, isothermal gas in hydrostatic equilibrium. For such a system, the Virial theorem applies, which describes the balance between the gravitational potential and the kinetic thermal energy within the gas. In other words, in hydrostatic equilibrium, the core's self-gravity is compensated by the internal pressure caused by the temperature of the gas. For the same radius and temperature, a core with more mass will lead to a runaway collapse. While simplistic, this treatment leads to a handy derivation of critical timescales, sizes, and masses that form a good starting point for more elaborate theories. 

First, it is important to determine what are the characteristic timescales of star formation. In the core with a uniform density, the simplest timescale to define is called the free-fall time: this is the time it takes for the total gravitational collapse of a spherically-symmetric clump of uniform density $\rho$ if only the force of gravity is considered:
\begin{equation}
\tff \sim \left(\frac{3\pi}{32 G\rho}\right)^{-1/2}\sim \SI{2e5}{\year}\left(\frac{\rho}{\SI{e-19}{\gram\per\raiseto{3}\centi\meter}}\right)^{-1/2},
\end{equation}
where we have substituted a typical value for the gas density in clusters. The free-fall time is usually a lower limit on the collapse timescale, since there will always be some thermal pressure that will resist gravity and slow down the infall of gas into the potential well. 

The other relevant quantity that involves time is the sound speed in the cloud, $\cs = (kT/(\mu m_H))^{1/2}$, where $\mu$ is the mean molecular weight of the gas and $m_H$ the mass of hydrogen. For a given spatial scale $R$, the sound-crossing time is defined as $\ts = R/\cs = \SI{4.9e5}{\year}\left(\frac{R}{\SI{0.1}{\pc}}\right)\left(\frac{\cs}{\SI{0.2}{\kilo\meter\per\second}}\right)^{-1}$. This is the time it takes for a wave to cross the scale $R$ while traveling at the sound speed.
Intuitively, if the core has a size $R$ such that $\tff<\ts$, it will tend to collapse faster the gas in the cloud can compensate to maintain hydrostatic equilibrium. This corresponds to a characteristic sizescale that is called the Jeans' length, and corresponds to the characteristic sizescale of gravitational instability within a cloud \citep{McKee:2007bd}:
\begin{equation}
 \lambdaJ = \cs\times\tff = \SI{0.04}{\pc}\left(\frac{\cs}{\SI{0.2}{\kilo\meter\per\second}}\right)\left(\frac{\rho}{\SI{e-19}{\gram\per\raiseto{3}\centi\meter}}\right)^{-1/2}.
 \end{equation} 

The Jeans mass is the amount of mass within a sphere of diameter \lambdaJ, and corresponds intuitively to the minimum mass a core needs to gather in order to trigger an gravitational collapse:
\begin{eqnarray}
\MJ &=& \frac{4\pi}{3}\rho\left(\frac{\lambdaJ}{2}\right)^3 \\
&=& \SI{0.06}{\Msun} \left(\frac{\cs}{\SI{0.2}{\kilo\meter\per\second}}\right)^3 \left(\frac{\rho}{\SI{e-19}{\gram\per\raiseto{3}\centi\meter}}\right)^{-1/2}
\end{eqnarray}

Note that this formalism completely ignores the material that surrounds the core while it collapses. In practice, the cloud exerts an external pressure on the core that needs to be taken into account when calculating the critical masses. This more elaborate case of a clump of self-gravitating gas on the verge of collapse that is immersed in a medium of external pressure $\Pext$ is called a Bonnor-Ebert sphere. It can be shown \citep{McKee:2007bd} that the sizescale is similar to the Jeans' length, and the mass scale is a few times smaller than the Jeans' mass, which stays well within the accuracy limits of our simple model.

Once the gas starts its gravitational collapse, nothing stops it until the central pressure and density reach values that trigger the ignition of nuclear fusion. This is the birth of the star. This new mechanism creates a large amount of radiation pressure that balances out the collapse and forms a new hydrostatic equilibrium. 

In practice, it is likely that a single core fragments into multiple centers of collapse, each of them exceeding the local Jeans mass. This would create systems of binaries or small multiples instead of single stars, a scenario that is currently favored [LARSON?].

In the standard model, the collapse begins at the center of the core and propagates outward at the sound speed, so the density structure of the initial core will change as a function of time. Most models result in an infalling envelope with density profiles which follow power laws from $\renv^{-1.5}$ to $\renv^{-2}$, an important observable that can be useful to test these theories. Some models of slowly-rotating infalling clouds suggest more complex density profiles for the envelopes \citep[e.g.][]{Ulrich:1976ho,Terebey:1984hi} than simple power laws, but are observationally difficult to constrain due to the small differences with traditional power-law envelopes and the small scales at which those differences occur (a few 100's of AU).

Through conservation of angular momentum, some of the surrounding material naturally flattens into a centrifugally-supported flaring disk before it is fed to the star, and a bipolar cavity opens along the rotation axis of the system. The cavity opening can also be bolstered by mechanisms such as stellar winds and jets [REFERENCE].

The object now has three characteristic components: the star itself; the flattened disk; and a diffuse envelope with an open cavity, which constitutes a mass reservoir for future accretion onto the star.

The accretion rate represents the speed at which the mass is transferred between different objects, and are important to set relevant timescales and to relate observables to the physics. For relatively low-mass star formation, the usually adopted accretion mechanism is called Shu accretion \citep{Shu:1977ef}, and predicts a mass accretion rate of the envelope onto the disk $\Mdotenv \propto \cs^3/G$, where $\cs$ here represents the sound speed that includes turbulence, and $G$ is the gravitational constant. Typical accretion rates for $\cs\sim \SI{2.7}{\kilo\meter\per\second}$ are $\Mdotenv \sim  \SI{4.8e-6}{\Msun\per\year}$ \citep{Dunham:2010bx}.

Although most of the mass is contained in the $\textrm{H}_2$ gas, there is a small fraction of material in the form of dust grains of various sizes and populations. Despite their low mass, these grains play a very important role in determining the observable properties of YSOs, because of their tendency to absorb short wavelengths and radiate in the thermal infrared (see Section~\ref{subsec:dust}). 

%The accretion mechanism most commonly consists of Bondi accretion \citep{Bondi:1952fc}, and represents the transfer of mass from the envelope or the disk to the star. As the YSO moves through its parent cloud, it can also accrete more mass throughout its life that was not originally part of the original condensation that initiated the collapse. In these dense regions where multiple stars are forming, interactions between YSOs is more than likely and can significantly affect the final masses of stars.

%Various models exist for accretion, from a quasi-static slow process \citep{Shu:1977ef} to a very rapid and brutal process \citep[e.g.][]{Larson:1967ee}. \cite{McKee:2007bd} suggest that the infall process is most likely slowing down as a function of time, which complicates its observational validation. 


% , the speed of sound is $c_s = (kT/m)^{1/2}$.
%Perhaps the most elementary way to model the birth of a star consists of considering an isothermal sphere of self-gravitating gas (a core) on the verge of collapse, in hydrostatic equilibrium with its surrounding cloud medium of gas pressure \Pext. This is also called a Bonnor-Ebert sphere \citep{Ebert:1955wf,Bonnor:1956cy}, and the state of this sphere is said to be \textit{critically stable}, with a critical mass defined by $\Mcrit = 1.18\frac{c^4_s}{G^{3/2}}\Pext^{-1/2}$ \citep{Shu:1977ef}, where $c_s = (kT/m)^{1/2}$ is the speed of sound in the cloud. To put this in perspective using physical quantities, we have:
%\begin{equation}
%M_{BE} = 0.66 \left(\frac{T}{\SI{10}{\kelvin}}\right)^2 \left(\frac{\Pext/k_B}{\SI{3e-5}{\raiseto{-3}\centi\meter\kelvin}}\right)^{-1/2}\si{\Msun},
%\end{equation}
%where \Pext was normalized to mean kinetic pressure at the surface of molecular clouds \citep{McKee:2007bd}. The Bonnor-Ebert radius is $R_{BE} = 0.49 c_s/(G\rho_0)^{1/2}$,  which is comparable to the characteristic fragmentation size scale called the Jeans' length. Here $\rho_0$ is the uniform density of the core. 

%For a core of mass $M_{BE}$, the thermal pressure exactly compensates the gravitational forces. Collapse occurs for cores with mass $>M_{BE}$ when gravity overcomes the thermal pressure of the gas inside the sphere, which triggers the rapid mass growth of a central embryo towards infinitely high densities. In practice, the process stops when enough density is reached to ignite nuclear fusion and stabilize the embryo, which has just become a young stellar object. Even after the star is born, mass continues to fall into the potential well, emitting radiation and increasing the mass of the central object. Various scenarios exist for this phase of the accretion within individual cores, 




%Since the accretion rate can be linked to the observed luminosities as gas falls into the gravitational potential well, it is also an important outcome of a given collapse model.
%For low-mass star formation from cores that are marginally super-critical, the infall phase can be treated in the quasi-static limit \cite{Shu:1977ef} and follows this normalized form \citep[Eq.44,][]{McKee:2007bd}:
%\begin{equation}
%\dot{m}_{infall} = \num{1.54e-6} \left(\frac{T}{\SI{10}{\kelvin}}\right)^{3/2}\si{\Msun\per\year}.
%\end{equation}

%More massive stars follow an accretion behavior more similar to Bondi-Hoyle accretion
%Conserving its original angular momentum

%[Accretion rates are likely varying as a function of time {McKee:2007bd}, and can complicate the interpretation ]

%[Magnetic fields, turbulence, asymmetries and non-isothermal considerations all likely play role in this balance, but dramatically complicate the mathematical treatment of the problem. In order to introduce the basic concepts of star formation, these complex processes are ignored.]

%\citep{Myers:2014ct}; includes why accretion stops. Shu accretion for low mass, Bondi accretion at high mass?

\subsubsection{YSO classification and characteristics}

We have determined that YSOs are composed of a star, a disk, and an envelope. The star is believed to be fairly well understood as a young object in hydrostatic equilibrium on its way to the main sequence. Depending on many parameters, the spatial distribution of gas in the disk and the envelope can be predicted by simple models, but in all likelihood is very complex, inhomogeneous, and asymmetric. 
%Often, the envelope is thought to open up a cavity around the poles of the rotating system, where material can escape the system (through outflows, for example [REFERENCE?]). 
For clarity, we will discuss here the simple models that can be used to describe the YSOs in the multiple stages of their evolution.

In the most common model of the evolution of young stars, there are four stages in the lifetime of a YSO. The first stage consists of a dense core right after the YSO is born. The disk is almost inexistent, the envelope still is dense and circularly symmetric. This is called Stage 0. As the system evolves, the cavity opening angle grows, the density of the envelope decreases, and the size of the disk increases. When a YSO is Stage III, both the disk and the envelope are almost entirely depleted.

The various stages of YSO (from 0 to III) have very distinct observational signatures, although are highly dependent on the viewing angle. The most commonly used tool to classify YSOs based on their SEDs is to use the spectral index, which corresponds to the mid-IR slope $\alpha$ in the log-log plots, with $\alpha = \textrm{d}(\lambda F_\lambda)/\textrm{d}\lambda$ between 2 to \SI{20}{\micro\meter} \citep{McKee:2007bd}. The four classes of YSOs are:


\begin{figure}[!h]
\begin{center}
\subcaptionbox{Stage 0\label{subfig:Stage0}}{
\includegraphics[width=\textwidth]{Figures/test_whitney_class0_nice.png} 
} \par\medskip
\subcaptionbox{Stage I.\label{subfig:StageI}}{
\includegraphics[width=\textwidth]{Figures/test_whitney_classI_nice.png} 
} 
\caption[Early evolution of YSOs]{Early evolution of YSOs.}
\label{fig:EarlyStages}
\end{center}
\end{figure}

\begin{figure}
\begin{center}

\subcaptionbox{Stage II.\label{subfig:StageII}}{
\includegraphics[width=\textwidth]{Figures/test_whitney_classII_nice.png} 
} \par\medskip
\subcaptionbox{Stage III.\label{subfig:StageIII}}{
\includegraphics[width=\textwidth]{Figures/test_whitney_classIII_nice.png} 
}
\caption[Late evolution of YSOs]{Late evolution of YSOs.}
\label{fig:LateStages}
\end{center}
\end{figure}


\begin{itemize}
\item Class 0: Most the of short-wavelength ($<\SI{10}{\micro\meter}$) light is highly obscured by the dust in the massive envelope. Most of the emission is around \SI{100}{\micro\meter} and into the submillimeter/radio regimes. If there is a disk, it is very small. Some authors \citep{Dunham:2010bx} classify a source as Class 0 as long as the amount of the mass in the envelope is at least half the total mass.
\item Class I: Light scatters at short wavelength off the dust grains to give us a hint at the embedded object, but it still very obscured. The envelope's mass is lower, and the disk extends to larger distances. The typical spectral index $\alpha$ is positive.
\item Class II: The YSO is now a pre-main sequence star, with a spectral index $-1.5 < \alpha < 0$ a significant circumstellar disk. This is traditionally referred to as a classical T-Tauri star.
\item Class III: Still a pre-main sequence star, but most of the accretion has stopped, and $\alpha < -1.5$. The envelope has almost completely disappeared, and so has most of the disk.
\end{itemize}

An illustration of canonical spectral energy distributions (SED) and density structure is shown in Figs.~\ref{fig:EarlyStages,fig:LateStages} for the four main stages, with parameters taken in \citet{Whitney:2003kc}. On the left of each picture, the SED is the measurable quantity when the YSO is unresolved at all wavelengths. The challenge is to estimate the density structure (to the right) by measuring the SED. The different lines plotted in the SEDs are different inclination angles, highlighting the enormous impact of the viewing angle on the potential interpretation of these SEDs. The dashed line corresponds to the Planck function from the central source. These models were run using the Hyperion software \citep{Robitaille:2011fc} with "OH5" dust \citep{Ossenkopf:1994tq}, as discussed in more details in Section~\ref{subsec:dust}.

These SEDs are often characterized and classified with standard observational metrics, such as the bolometric temperature and luminosities \citep{Myers:1993en,Dunham:2010bx}:

\begin{align}
\Lbol &= 4\pi d^2\int_0^\infty\Snu d\nu,\\
\Tbol &= \num{1.25e-11} \frac{\int_0^\infty \nu\Snu d\nu}{\int_0^\infty \Snu d\nu}~\si{\kelvin},
\end{align}
%
where \Snu is the flux density in \si{\watt\per\raiseto{2}\meter\per\hertz}. 

\subsection{Mass accretion in clusters}


The discussion in the previous section represents a canonical view of how a single core collapses and forms a star. While it is convenient to assume that the original core forms a fixed reservoir of gas that will determine the star's final mass, it is likely too simplistic, since these YSOs are preferentially forming inside of clusters close to multiple other YSOs and sharing a dense, often turbulent environment \citep{Porras:2003kxa,Allen:2007wqa,Gutermuth:2009gca}. 

%Approximately 60\% of all stars are thought to form in clusters with 100 or more stars (\cite{Porras:2003p1395}; \cite{Allen:2007p1439}). These $>$100 star clusters have characteristic sizes of 2-4 parsecs with peak surface densities of 100-1,000 stars per square parsec and a typical median distance between nearest neighbor young stellar objects (YSOs) of 0.03 to 0.06 parsecs \citep{Gutermuth:2009p1325}.

The question of how stars acquire their final mass is key in studying star formation. Does dense gas fragment into isolated centers of collapse? Do young stars competitively accrete material from a surrounding common reservoir? Do gravitational interactions between forming young objects play a significant role in setting the final stellar mass function? Better observational understanding of these clusters is necessary to address these questions and to discriminate between the different models, as noted by \cite{Bonnell:2006ee}, \cite{Offner:2011ex} and \cite{Myers:2011fy}.

Given the typical stellar separations in clusters with fully formed young stellar objects and the typical densities of gas in these cores, \num{1000}'s of astronomical units (\si{\au}, \SI{1}{\parsec} = \SI{206265}{\au} - are the size scales over which forming stars must draw material to become 0.5-10 solar masses. Once the material is inside \SI{100}{\au}, it is strongly bound to the forming stellar system (which may be one or more stars) and its fate is determined. To give an idea of the possibilities for accreting material, Fig. \ref{fig:SFscenarios} sketches three scenarios for how stars could capture mass in the cluster environment: core collapse, competitive accretion, and collisional merging. In core collapse (CC) \citep[Fig.~\ref{scenarios:a},][]{McKee:2003gxa, Myers:2011fy}, the cluster's gas fragments into cores which collapse individually to form single, binary, or small multiple star systems; the available mass is defined by the original fragment. In competitive accretion (CA) \citep[Fig.~\ref{scenarios:b},][]{Bonnell:1997vta}, the initial core collapses but contains a small fraction of the star's final mass; additional mass is captured competitively with other forming stars from the surrounding dense core gas. In collisional merging (CM) \citep[Fig.~\ref{scenarios:c},][]{Bonnell:2002et}, the initial fragments interact gravitationally and form larger mass cores before and during the formation process. 

Are all these processes observed at once in star forming clusters? What conditions favor one versus the other, and why? Are these processes observed at different stages in the cluster's history?

Recent studies by \cite{Offner:2011ex} and \cite{Myers:2011p1338} compared protostar luminosity distributions with predictions of models based on these ideas. \cite{Offner:2011ex} suggest that both CC and CA could work if the star formation rate in the cluster increases with time; \citep{Myers:2011fy} finds that a CA-type model with additional Bondi accretion to produce massive stars works best. As highlighted at the end of the \cite{Offner:2011ex} paper, larger cluster samples and better data on massive stars are needed to improve the observational constraints on models.


%\paragraph{The Initial and Core Mass Functions}
%Initial mass function across multiple locations \citep{Myers:2014ct}: The IMF has similar properties of shape, mass scale, and high-mass slope among field stars, open clusters, and young clusters (Kroupa 2002, Chabrier 2005, Bastian et al. 2010).
%
%In the most widely-discussed explanation, an IMF distribution of masses arises as a cluster-forming clump fragments into condensations, or cores, due to self-gravity and turbulence. In such turbulent fragmentation models, the mass distribution of cores (CMF) is a mass-shifted version of the IMF (Padoan \& Nordlund 2002, Hennebelle \& Chabrier 2008, Hennebelle \& Chabrier 2009, Hopkins 2012), matching observational studies of cores in nearby star-forming regions (Motte et al. 1998, Alves et al. 2007, Konyves et al. 2010)
%
%\citep{Myers:2014ct} has all I need here to describe the IMF and CMF
%
%Describe the various theories of SF in clusters. 



\begin{figure}[ht!]
\begin{center}
\begin{subfigure}[b]{0.3\textwidth}
\centering
\includegraphics[width=0.98\textwidth]{Figures/CC.png} 
\caption{}
\label{subfig:scenarios:a}
\end{subfigure}
\begin{subfigure}[b]{0.3\textwidth}
\centering
\includegraphics[width=0.98\textwidth]{Figures/CA.png} 
\caption{}
\label{subfig:scenarios:b}
\end{subfigure}
\begin{subfigure}[b]{0.3\textwidth}
\centering
\includegraphics[width=0.98\textwidth]{Figures/Coalescence.png} 
\caption{}
\label{subfig:scenarios:c}
\end{subfigure}
\caption[Scenarios of clustered star formation]{Three scenarios of clustered star formation. Darker colors indicate higher densities.}
\label{fig:SFscenarios}
\end{center}
\end{figure}



%\subsubsection{Outstanding scientific questions}
%Examples: \citep{Hennebelle:2012dk} \citep{Kennicutt:2012ey}
%Ex: WHERE DO THE INEFFICIENCIES COME FROM?
%
%Luminosity problem \citep{McKee:2007bd}
%Angular momentum problem and magnetic flux problem  \citep{McKee:2007bd}
%How the mass-to-flux ratio increases so dramatically during star formation is one of the classic problems of star formation (Mestel \& Spitzer 1956). 

\section{Dust as a tracer of star formation}
\label{subsec:dust}

Despite being a small component by mass, interstellar dust is an important component of galaxies. Dust grains are heated up by absorbing the short wavelength emission from stars and re-radiate in the thermal infrared, accounting for $\sim 30\%$ of the total luminosity of the galaxy \citep{Mathis:1990jk}. 

Observationally, dust plays perhaps the most important role when it comes to studying star formation. It usually is assumed that dust is well mixed with the gas, which makes it an excellent tracer of the gravitational well and mass distribution in YSOs. Because H$_2$ and He molecules have very few spectral signatures, they are difficult to observe and study directly. Dust grains block UV and visible star light and emit continuum far-IR radiation, opening a large region of the electromagnetic spectrum for astronomers to study the properties of star formation. Alternative tools to study star formation are dedicated to observing spectral lines of the molecular compounds of the ISM such as CO and other dense gas tracers, a prospect that limits the study to the most dense regions since these compounds typically freeze out onto the surface of dust grains for sufficiently high densities [REF?]%and requires more assumptions to link the abundance of these compounds to the more massive neutral H$_2$ and He gas.


\subsection{Dust populations and properties}


Perhaps the first understanding of the composition of dust grains in the ISM was described by \cite{Mathis:1977hp}, where they studied the absorption spectrum of the diffuse ISM, and found that the measurements were appropriately fitted with a dust grain composition of silicates and small graphite particles \citep{Stecher:1965eq}. They were able to fit the observed extinction curve with canonical grain-size distribution, typically $n(a) \propto a^{-3.5}$, where $a$ is the grain size (assuming spherical grains) and $n(a)$ corresponds to the number of grains of sizes $<a$. This assumes low and high cutoffs for the grain sizes, typically \SI{50}{\angstrom} and \SI{0.25}{\micro\meter}, respectively.

This grain-size distribution model was later on enhanced by \cite{Cardelli:1989dp} to account for the difference in interstellar extinctions (hence size distributions) across different galactic lines of sight. These authors were able to successfully parameterize this size distribution using a single parameter, $R_V$, which is the ratio of the total extinction $A(V)$ to selective extinction\footnote{Extinction and colors are expressed in magnitudes} (or color) $E(B-V) = A(B) - A(V)$. Smooth distributions of sizes of graphite and silicate grains between the less dense regions of the ISM, where $R_V = 3.1$, and the dense clusters, where $R_V = 5.3$ \citep{Kim:1994iu}. 

Observations in the thermal infrared from space telescopes have detected strong absorption lines at \SI{9.7}{\micro\meter} and \SI{18}{\micro\meter} which are attributed to stretching mode of Si-O and bending mode of O-Si-O, confirming the presence of silicates in dust compositions \citep{Weingartner:2001du}. Other emission features at 3.3, 6.2, 7.7, 8.6, and \SI{11.3}{\micro\meter} \citep{Sellgren:1994vz} were attributed to bending and stretching modes of polycyclic aromatic hydrocarbons \citep[PAH, see][]{Gillett:1973bh,Allamandola:1985cf}, which are complex, planar organic molecules.
 
A consolidated model matching all-sky measurements by instruments on the COBE space observatory confirms the composition of amorphous silicates and carbonaceous grains with sizes ranging from large grains ($\approx\SI{1}{\micro\meter}$) down to tens of atoms \citep{Li:2001gk}, where the larger carbonaceous grains have graphitic properties and the smaller population have PAH-like properties.


[Molecular hydrogen is believed to form by recombination on the surface of dust grains [hollenbach and salpeter 1971, and are only able to survive from UV photodissociation within these obscured clouds.]




Knowing the dust composition and size distribution of grains is important to properly predict its observational behavior and relate it to the physical quantities of interest, since the goal of the exercise is to use dust as a tracer of star-forming mechanisms. A given dust model needs to provide several key quantities that can be used in radiative transfer modeling (see Section~\ref{subsubsec:radiative}), such as the albedo, the scattering function, and the opacity.

In the very cold regions surrounding a YSO, where the dust temperature typically never exceeds a few tens of \si{\kelvin}, it is expected that these dust grains are covered by a mantel of ices which can dramatically change their radiative properties, especially at short wavelengths. 

\subsection{Basics of dust extinction}

Dust grains are responsible for the extinction within molecular clouds, inside of clusters, and also within each YSO; although these various extinctions could originate in different types of grain populations. The typical representation of this extinction uses the ratio of observed over expected flux, measured in V-band: $\Av \equiv A(V) = 2.5\log(\Fnu^\textrm{obs}/\Fnu)$. The extinction, $A(\lambda)$, is a function of wavelength and is expressed in magnitudes. An alternative representation is to consider the extinction as being caused by an optical depth $\tau_\ext$ such as $\exp\tau_\ext = \Fnu^\textrm{obs}/\Fnu$. We have the equivalence $A(\lambda) = \num{1.086}\tau_\ext(\lambda)$.

At sufficiently long wavelength, dust opacity models can usually be represented by a simple power-law, $\Knu = \kappa_0(\nu/\nu_0)^\beta$, with the index $\beta$ depending on the specifics of the dust model. The opacity $\Knu$ is expressed in \si{\raiseto{2}\centi\meter\per\gram}, and can be interpreted as a extinction cross-section per unit mass. Most dust models assume a 1:100 dust-to-gas ratio, and derive opacities per unit {gas+dust} mass, instead of just dust mass. From a radiative transfer perspective, the observed specific intensity from a thermal source $\Bnu(T)$ in the optically thin regime is $I_\nu = \tau_\nu \Bnu(T)$, where the optical depth is $\tau_\nu = \Knu\int{\rho_\dust dl}$. $\rho$ is the density and the integral is calculated along the line of sight to the source. %Note that the optical depth can also be expressed in terms of the extinction cross-section \Cext~as $\Knu = \Cext(\nu) / m_\dust$, where $m_\dust$ corresponds to the mass of a dust grain.

A measure of the intensity from a source can thus lead to an approximation of the total mass within a primary beam, for a given dust grain model. For a source with a measured sub-millimeter flux density $S_\nu$, in the optically thin regime we can write $S_\nu=\tau_\nu\Bnu(T)\Omega$, where $\Omega$ is the solid angle of the source, $\Omega = A/d^2$, with $A$ the area of the source and $d$ its distance. We obtain a measure of the mass by writing $M\approx A\int{\rho dl}$, to obtain \citep{Shirley:2000gh}:
\begin{equation}
M = \frac{S_\nu d^2}{B_\nu(T_\dust)\Knu},
\end{equation}
with a dust temperature is usually taken to be between 10 to \SI{20}{\kelvin}. 

With only near- to far-IR wavelengths observations, however, it is more difficult to estimate the dust mass, because the system is usually not in the optically thin regime and very dependent on the local geometry and viewing angle. To use these observations, which are interesting because they naturally are at higher resolution than single-dish submillimeter data, detailed radiative transfer models are usually required (Section~\ref{subsubsec:radiative}).


Dust grains can either scatter or absorb photons, and both of these processes have their own frequency-dependent efficiency. Large grains are usually considered in local thermal equilibrium (LTE), in which case the thermal emission balances out the absorption. However, small grains ($<\SI{50}{\angstrom}$) can be subject to stochastic heating, where single photons can heat up the grains to much higher temperatures for very short amounts ot time. Scattering mechanisms can be much more complicated to represent, as they usually involve a scattering phase function, describing the deflection angle of incident photons (which also depends on wavelength). Most models show that dust grains are preferentially forward-scattering [CITE Draine?]. The scattering properties of the dust model exclusively influence the short-wavelength emission, while the absorption properties influence all wavelengths. 

%Talk about dust properties in the galaxy: \citep{Collaboration:2014dz} Using
%Also maybe mention papers by Dale etc DIRBE/COBE
%
%A large number of dust models exist (see Section~\ref{subsubsec:radiative})
%
%In clusters and cores, it is likely that grains are covered with a layer of ice which could dramatically influence their radiative transfer properties

\subsection{Radiative transfer modeling}
\label{subsubsec:radiative}

Several radiative transfer codes exist in the literature, and we have explored a few of them. We opted for a recently-developed package called Hyperion \citep{Robitaille:2011fc}, which is a Python interface to a 3D Monte-Carlo code by \citet{Whitney:2013cw}. The code is versatile, parallelized, can accept different dust models and can generate various types of geometries and density grids.

Hyperion functions in two steps. After choosing a discrete grid to represent a density model and adding energy sources, the temperature structure of the dust is calculated by propagating photon packets and determining the dust LTE temperature in each cell. Multiple iterations of this process are usually required to converge to a decent thermal structure.

Once the dust temperature is known, the dust becomes a source of thermal radiation. This type of radiation is modeled using ray tracing, which provides a very good signal-to-noise ratio (\SNR). The light from the central source which was not absorbed, however, needs to be propagated and scattered off the dust grains, for example using a method called peeling-off \citep{YusefZadeh:1984ff}. For non-isotropic scattering, this process has relatively low \SNR, hence requires a lot of photons packets to function properly. While there are future plans to implement raytracing for scattering \citep{Robitaille:2011fc}, we are currently forced to wait long times for simulating YSOs with massive envelopes because of this problem.

[Put models of YSOs with different masses here.]

These models usually present a large amount of degeneracies, especially when the entire range of wavelengths is not covered, as it is the case for most astronomical sources. For example, an SED will look very different depending on the viewing angle. If we see down the throat of the cavity, the short-wavelength light from the central source will not exhibit a lot of extinction. If we observe this same source through the disk and envelope, these same wavelengths will show a lot of extinction. The short wavelength, up to the peak of the SED, are very geometry-dependent and highly degenerate parameters.

This realization helped us target our work using this code. Others \citep[e.g][]{Robitaille:2006cb} have produced standardized grids of pre-computed models which randomly sample a very large number of source geometry parameters. These grids are routinely used by the community to fit a set of unresolved SED measurements at discrete wavelengths. However, most often the scatter in the parameters for the few best fit models prevents from drawing meaningful conclusions on the observations. 

[Example?]
%For example, in a Class 0 or I YSO, the disk geometry and mass has a very insignificant impact on the resulting SED. [Put a model of the SED with and without disk?]

One of the key challenges of using this code is to determine which dust models to use. For this work, we choose to use exclusively OH5 dust \citep{Ossenkopf:1994tq}, which represents grains with an ice mantle which are the result of a coagulation phase of an initial distribution of grain sizes following $n\propto a^{-3/2}$. This model was found to accurately represent some grain distribution in the ISM [NEED CITATION, CHECK OUT TRACY'S PAPER].

\subsection{Observing star formation}

In the past decade, space-based infrared observatories such as \Spitzer\ and \Herschel\ have really allowed the beginning of the detailed study of dust around forming stars, by sampling the SEDs in key spectral regions, such as the PAH region (with the IRAC instrument on \Spitzer), the mid-infrared (with the MIPS instrument, especially its \SI{24}{\micro\meter} channel), and the far-IR (with the PACS instrument on \Herschel). 

However, these observatories lack the required angular resolution to observe the key physics of star formation in dense clusters in the key wavelength region between \SI{30}{\micro\meter} and \SI{200}{\micro\meter}. For a diffraction-limited single aperture telescope, the angular resolution and spatial resolutions $R_\theta$ and $R_\textrm{linear}$ are:
\begin{align}
R_\theta &= \ang{;;17.6}\left(\frac{\lambda}{\SI{70}{\micro\meter}}\right)\left(\frac{D}{\SI{1}{\meter}}\right)^{-1}, \\
R_\textrm{linear} &= \SI{0.04}{\parsec}\left(\frac{d}{\SI{500}{\parsec}}\right)\left(\frac{\lambda}{\SI{70}{\micro\meter}}\right)\left(\frac{D}{\SI{1}{\meter}}\right)^{-1},
\end{align}
which shows that even with \Herschel and its \SI{3.5}{\meter} primary mirror and its \SI{70}{\micro\meter} channel, we can barely resolve clustered YSOs (typical separations of a few hundredths of \si{\parsec}), let alone study their structure in detail. 

To further complicate the problem, most space observatories are tailored for very sensitive observations, so the brightest regions of clusters often cause saturation issues due to a lack of dynamic range. These two issues have continually prevented us from gathering a good picture of the physics in these dense and important regions of stellar birth.

[Image that shows the saturation/lack of resolution]

[Talk about SOFIA]
%\subsubsection{Observing facilities}
%
%Introduction notes: [DELETE THIS]
%\begin{itemize}
%\item \citep{Kennicutt:2012ey} Maybe start with a description of the ISM
%\item \citep{Terebey:1984hi} Basics of infall; 
%It is well understood how self-gravity can concentrate gas in the presence of magnetic fields, turbulence, rotation, and thermal pressure, leading to protostar formation and accretion ( McKee \& Ostriker 2007; Adams \& Shu 2007; Ballesteros-Paredes et al. 2007)
%\item \citep{Adams:1987gy} Spectral evolution of young stellar objects; describe classes of stellar objects, using models and make sure to include size scales; YSO vs protostar
%\item Need to mention something about star formation efficiency
%\item Need to mention something about outflows and feedback \citep{Maury:2009co}
%\item \citep{Larson:1994cj} Molecular cloud characteristics; maybe a good start for text | Give definition of cloud (gravitationally bound in the Virial sense), in terms of number of candidates and stellar mass density (Lada 2003)
%\item \citep{Myers:2009fv} It is widely accepted that most stars form in concentrations of dense gas (Beichman et al. 1986), and that such young stars are most frequently found in groups and clusters in molecular clouds (Lada \& Lada 2003). It is less clear how the mass of a protostar is related to the mass of the dense core where it forms.
%In one view, a core is essentially a fixed-mass reservoir of gas, which contributes a significant fraction of its initial mass to its protostar. Then core and protostar masses are proportional, and their mass distributions have the same shape (Motte et al. 1998; Alves et al. 2007). Many authors have suggested ways to form a mass distribution of cores which resembles the IMF, particularly from processes of turbulent fragmentation (Hennebelle \& Chabrier 2008 and references therein).
%In another view, a core is the densest part of a more extended distribution of gas, with no physical barrier to accretion (Shu 1977; Myers \& Fuller 1992; Caselli \& Myers 1995; McKee \& Tan 2003). Protostars originate in cores, but their masses do not correlate (Bonnell et al. 1998; Bate \& Bonnell 2005), or their correlation depends on fragmentation and core definition (Swift \& Williams 2008), or on the range of gas dispersal times (Myers 2008, hereafter Paper 1). Alternately, their correlation may be coincidental rather than genetic (Hatchell \& Fuller 2008).
%\item \citep{McKee:2010iw} Protostellar mass function
%\item \citep{Bonnell:1997vta} Accretion in small clusters
%\item \citep{Evans:1999gz} Review of physical conditions of star formation, also a good place to start
%\item \citep{Myers:2011fy} These results suggest that a simple model is needed for the dense parts of clusters, where protostars start accreting in condensations resembling dense cores, where they can also gain mass from the core environment, where their accretion durations are specified, and where protostar mass is not tied to the gravitational collapse of an isolated initial condensation. | This comparison favors the core–clump model over the isolated core model for star formation in embedded clusters. It suggests that initial core structure need not set protostar mass, and that massive stars are clump-fed.
%\item Clusters are laboratories for studying a wide range of astrophysical phenomena; stars of large range of masses which are formed roughly simultaneously (Lada 2003), can understand stellar evolution theories; 
%\item \citep{Lada:2003il} Cluster formation within molecular clouds: Because a cluster is held together by the mutual gravitational attraction of its individual members, its evolution is determined by Newton's laws of motion and gravity; |
%Stars form in dense gas; therefore, it is not surprising that a high fraction of all stars form in highly localized rich clusters because most of a cloud’s dense gas is contained in its localized massive cores.; |
%This would suggest that there is a direct mapping of clump mass to stellar mass and that the substructure of cluster-forming cores reflects the initial conditions of the star-formation process in dense cores. |
%The structure of an embedded cluster is of great interest since it likely possesses the imprint of the physical process responsible for its creation. In particular, structure in the youngest embedded clusters reflects the underlying structure in the dense molecular gas from which they formed. | A fundamental consequence of the theory of stellar structure and evolution is that, once formed, the subsequent life history of a star is essentially predetermined by one parameter, its birth mass
%\item \citep{Bate:2003cv} Evolution of clusters (dynamics, showing simulations and such); same as \citep{Bonnell:2003iw}; realte to potentially having multiple generations of stars within the same cluster; \citep{Allen:2007wqa}
%\item \citep{Mathis:1990jk} Pioneering paper on the interstellar extinction by dust
%\item \citep{Weingartner:2001du} Paper on dust populations
%\item \citep{Li:2001gk} Very small grain population and non-LTE heating
%\item \citep{Compiegne:2010kk} Large surveys, PAHs, VSG, etc
%\item \citep{Bastian:2010ig} Big picture: importance to universal IMF
%\item \citep{Kennicutt:2012ey} The Challenge of Spatially Resolved Star-Formation Rates in Galaxies
%\item \citep{Hennebelle:2012dk} Observational issues: 
%
%\end{itemize}
%


%How do stars accrete their material in dense, clustered environments? Multiple theories have been put forth to explain the forces at play in these important regions of stellar birth \citep[e.g.][]{Bonnell:1997vta, McKee:2003gxa}. Pivotal differences in the theories center around what drives the parsec-scale dense cloud to form many stars and how the forming stars acquire their mass. Does dense gas fragment into isolated centers of collapse? Are turbulent motions in the gas driving creation of super-critical cores? Do young stars competitively accrete material from a surrounding common reservoir? Do gravitational interactions between forming young objects play a significant role in setting the final stellar mass function? Better observational understanding of these clusters is necessary to address these questions and to discriminate between the different models, as noted by \cite{Bonnell:2006ee}, \cite{Offner:2011ex} and \cite{Myers:2011fy}.
%
%\textbf{The primary goal of the proposed science plan is to understand how stars accrete material on the scales of 100's to 1,000's of AUs in forming cluster cores}. Given the typical stellar separations in clusters with fully formed young stellar objects (YSO) and the typical densities of gas in these cores, 1,000's of AUs are the size scales over which forming stars must draw material to become 0.5-10 solar masses. Once the material is inside 100 AU, it is strongly bound to the forming stellar system (which may be one or more stars) and its fate is determined. To give an idea of the possibilities for accreting material, Fig. \ref{fig:SFscenarios} sketches three scenarios for how stars could capture mass in the cluster environment: core collapse, competitive accretion, and collisional merging. In core collapse \citep[Fig.~\ref{scenarios:a},][]{McKee:2003gxa, Myers:2011fy}, the cluster's gas fragments into cores which collapse individually to form single, binary, or small multiple star systems; the available mass is defined by the original fragment. In competitive accretion \citep[Fig.~\ref{scenarios:b},][]{Bonnell:1997vta}, the initial core collapses but contains a small fraction of the star's final mass; additional mass is captured competitively with other forming stars from the surrounding dense core gas. In collisional merging \citep[Fig.~\ref{scenarios:c},][]{Bonnell:2002et}, the initial fragments interact gravitationally and form larger mass cores before and during the formation process. 
%
%Are all these processes observed at once in star forming clusters? What conditions favor one versus the other, and why? Are these processes observed at different stages in the cluster's history? These questions need more observational data to be answered. \textbf{We propose to gather information about the gravitational potential within the cluster, the distribution of the turbulence, and the gas densities} \citep{Bonnell:2006ee}. This will advance the problem to the next stage and help answer some of these questions.
%


%\subsection{The physics of star formation}
%
%In here, describe the basics of star formation; include state-of-the art dynamical modeling, radiative transfer modeling, SED fitting, etc.
%
%\subsubsection{Dust populations}
%
%About how to estimate the mass
%
%\subsubsection{Geometry and model degeneracies}
%
%\subsubsection{Foreground extinction}
%
%\subsection{Clustered environments}
%
%\subsubsection{Observational challenges}
%
%Surveys vs pointed observations; show why BETTII is important using IRAS20050's example;
%
%\subsubsection{Observing facilities}
%
%Spitzer, Herschel, SOFIA; radio interferometers;


\include{Chapters/intro-lgm}
\chapter{Star Formation in Clustered environments with SOFIA FORCAST}

\label{chap:SOFIA}

\section{Introduction}
Most stars in the Galaxy form in cluster environments of sizes 2-4 pc, often containing more than 100 young stellar objects (YSOs), with typical separations of $<$0.05~pc between stars near their centers \citep{Porras:2003kxa, Allen:2007wqa, Gutermuth:2009gca}.
Previous studies have been effective in elucidating the young stellar content and distribution in clouds on large scales (parsec down to 0.05~pc) \citep{Evans-ARAA2012}, but young cluster cores, born in dense portions of molecular clouds, are more difficult to observe. They are obscured at optical through near-IR wavelengths. At mid-IR through far-IR wavelengths, the material surrounding YSOs and involved in the stellar birth process emits due to heating by the young stars, but the resolution to date has not been sufficient to isolate individual stars in the cores of most nearby young clusters.


%\begin{enumerate}
%\item Description of the sample and the goals
%\item Description of the data we used: 2MASS, Spitzer, Herschel; SOFIA
%\item Instrument characterization
%\item Explain the data reduction process; include comparison with Megeath for Spitzer photometry; justify aperture correction with "total cluster" measurements
%\item Data products: maps, SEDs; spectral index distribution, Lbol, Tbol
%\item SED fitting method \& grid description (e.g. color-color diagram?)
%\item Focus on IRAS20050 and NGC2071 + Ophiuchus? SED fitting; our method, our results; analysis of the results similar to furlan?
%\item 
%\end{enumerate}
%SOFIA has a \SI{2.7}{\meter} primary mirror which is a significant size improvement over \Spitzer. The instrument we have used, FORCAST, provides unprecedented high angular resolution of \ang{;;2}-\ang{;;3.5} in multiple continuum bands from \SI{5.5}{\micro\meter} to \SI{37.1}{\micro\meter}, which allows us to probe a relatively unexplored region of phase space. We discuss our SOFIA project in Chapter~[].



\Spitzer has tremendously advanced our understanding of star formation, by providing sensitive observations in continuum bands from \SI{3.6}{\um} to \SI{160}{\micro\meter}. In particular, the IRAC instrument (with 4 bands from 3.6 to 8.4 $\mu$m) and MIPS \SI{24}{\micro\meter} channel provided a robust way to determine the spectral index of YSOs, hence leading to dramatic improvement of understanding of the YSO population in molecular clouds \citep[e.g.,][]{Gutermuth:2009gca,Gutermuth:2011he}.

However, the most dense regions of clusters presented a challenge for the MIPS instrument, as the YSOs are too bright and/or in too close proximity, which led to saturation and confusion, as exhibited in Fig.~\ref{fig:NGC2071saturated}. In this figure, we show the same region seen by the IRAC \SI{3.6}{\um} band, the MIPS \SI{24}{\um} band, and the \Herschel PACS \SI{70}{\um}, from left to right. While the IRAC instrument can clearly distinguish multiple objects within the region, the MIPS image is completely saturated, while the PACS image is confused and cannot properly resolve the individual objects due to the lower resolution of the \Herschel telescope at \SI{70}{\um}. Note that these YSOs are much closer to each other than it is typical in clusters (\SI{0.01}{\pc} instead of a typical value of \SI{0.04}{\pc}), however this scale of projected separation is not unusual at the centers of clusters.

\begin{figure}[!h]
\begin{center}
\includegraphics[width=\textwidth]{Figures/NGC2071_saturated_mosaic.png}
\vspace{-0.5cm}
\caption[Saturation and confusion]{Saturation and confusion in NGC2071.}
\label{fig:NGC2071saturated}
\end{center}
\end{figure}

Future instruments like BETTII will be able to tackle the confusion problem at wavelengths from 30 to \SI{100}{\micro\meter}, and be complementary to \Herschel observations of star-forming regions. In the meantime SOFIA, the Stratospheric Observatory For Infrared Astronomy, can already start studying these dense regions, providing 2-\ang{;;3.5} resolution between 10 and \SI{37}{\micro\meter}, without the saturation problems present in the \Spitzer data. This corresponds to a factor of 2-3 improvement in angular resolution over \Spitzer at \SI{24}{\um}. 

\section{Sample description and scientific goals}


This chapter reports the results of a survey of nearby star-forming cluster cores with the SOFIA FORCAST instrument \citep{Herter:2012hv}. The clusters were selected from a list of dense young clusters within \SI{1}{\kilo\pc} of the Sun derived from works by \citet{Porras:2003kxa} and \citet{Gutermuth:2009gca}. From their lists we selected clusters that were: (1) north of -\SI{25}{\degree} declination so that they could be observed from a northern hemisphere SOFIA flight; (2) included membership of $>$50 YSOs; and (3) included bright 8-\SI{24}{\micro\meter} sources within the dense cores based on \Spitzer and/or WISE data. 

We observed in four FORCAST science continuum bands: 11.1, 19.7, 31.5 and \SI{37.1}{\micron}, which covered the wavelength range available to the instrument at the time of proposal (2012). This wavelength coverage is complementary to archival data from \Spitzer and \Herschel. Our selection of bright regions spread all across the sky is convenient for SOFIA, as our project could be observed as a gap-filler between the primary science flight legs of other projects.

The main objectives of the survey are to gather statistics on the YSO content of the \Spitzer saturated regions, and fill the SED gap between {\Spitzer}'s bands and {\Herschel}'s bands, when the latter are available. While most of our targets have valid \Spitzer IRAC data, often the data from the MIPS instrument is unavailable due to saturation or confusion. \Herschel photometry usually is not published in the literature for our sources, but maps of our regions are sometimes available so we can retrieve the far-infrared fluxes for some sources. For the targets without MIPS or \Herschel data, these SOFIA observations are the best information available between the longest IRAC band at $\SI{8}{\um}$ and the shortest submillimeter bands from ground-based telescopes. Thus our data provide important constraints the SED of very clustered YSOs in these regions to infer their physical properties.


The data analysis and scientific interpretation are presented in the next few sections. First, we describe our observations, as well as the archival datasets that we use to complement them. Second, we characterize the systematics of the FORCAST instrument and their variations over multiple science flights spanning multiple years. Third, the data reduction process is explained, followed by a snapshot of the data products themselves. Fourth, we discuss our SED fitting strategy, and fit the SEDs of some of our clusters to derive the physical properties of their embedded YSOs. Finally, we focus on the case of the young stellar cluster IRAS~20050+2720, and discuss of our FORCAST data helps us understand the physics of such embedded regions.  


\section{Observations}
\label{subsec:SOFIAObservations}

The FORCAST camera \citep{Herter:2012hv} has two separate $256\times 256$ pixel infrared arrays that can image multiple bands in the wavelength range from 5.5-\SI{37}{\um} with $\ang{;;0.768}\times\ang{;;0.768}$ pixels. The two arrays can observe simultaneously through a dichroic beam splitter that divides the wavelength range shortward and longward of \SI{26}{\um}. Alternatively, the long wavelength array can be used by itself with the dichroic removed from the light path, gaining a sensitivity factor of $\sim 2.5$. We observe the 11.1 and \SI{37.1}{\um} together (hereafter "mode 1") and the 19.7 and \SI{31.5}{\um} together (hereafter "mode 2") . We set the 1$\sigma$ sensitivity threshold of the observations such that a a \SI{1.5}{\Lsun} source with a moderately rising SED would be detected at all wavelengths. The integration times were scaled appropriately for the distance to the cluster (see Table~\ref{tab:DesiredSensitivities}). This is allows is to probe the same luminosity limit at all distances and obtain a consistent sample of YSOs. 

\renewcommand{\arraystretch}{1.5}
\begin{table}[!h]
\scriptsize
\caption{List of desired sensitivities for different distances}
\vspace{-0.5cm}
\begin{longtable}{c|cccc|c}
\toprule
Distance & \multicolumn{4}{c|}{1$\sigma$ minimum detectable flux (Jy)} &  Corresponding minimum\\
(pc) & \SI{11}{\um}& \SI{19}{\um}& \SI{31}{\um}& \SI{37}{\um}& \si{\Lsun} \\
\hline
   200.0& 0.1& 0.1& 0.32& 0.7&$\sim$0.5\\
   400.0& 0.1& 0.1& 0.32& 0.6&$\sim$1.5\\
   600.0& 0.05& 0.04& 0.18& 0.25&$\sim$1.5\\
   800.0& 0.02& 0.02& 0.1& 0.12&$\sim$1.5\\
1,000.00& 0.01& 0.01& 0.06& 0.1&$\sim$1.5\\

\bottomrule																																		\end{longtable} 
%\caption*{List of desired sensitivities for different distances.}
\label{tab:DesiredSensitivities}
\end{table}


For the most nearby clusters ($<$ 300 pc), the required observing time was so short that the overhead from the observatory was very costly. Hence, we put a lower threshold to the integration time of \SI{30}{\second}. Similarly, the sensitivity of the \SI{37}{\um} band is such that in order to be consistent with our sensitivity target, this band was heavily driving the observing time in mode 1. Hence, we observe in this mode as long as is required to meet the sensitivity target for the \SI{11}{\um} band, and obtained additional observations in the \SI{37}{\um} band with the dichroic removed (hereafter "mode 3"). This allowed us to request less total observing time while achieving our sensitivity goals. A summary of our sensitivities for various distances is shown in Table~\ref{tab:DesiredSensitivities}.

Several observing strategies are available to the FORCAST user to deal with background subtraction. The most robust techniques are very costly in terms of time overhead for the observatory, so we decided to request the cheapest observing mode: the Chop-Nod mode (C2N), combined with 9 ditherings for each field, which dramatically helps when co-adding images together. Most of our data was processed by the SOFIA automated pipeline that provided calibrated Level 2 images, except for the data from the first few flights, for which we received the help of FORCAST's Principal Investigator, Dr.~Joe~Adams, who processed the raw data through his own instrument pipeline.

%\begin{figure}[!h]
%\begin{center}
%\includegraphics[width=\textwidth]{Figures/SOFIA_bands.pdf}
%\label{fig:SOFIAbands}
%\caption[SOFIA bands]{SOFIA FORCAST bands.}
%\end{center}
%\end{figure}

The data were acquired over 10 SOFIA flights spanning multiple years, with the last batch dating from February 2015. The actual observing times for each band and cluster is shown in Table~\ref{tab:times}. In that table, we have estimated the time for the \SI{37}{\um} band using a composite formula that levels the observing time from mode 3 to that of mode 1, considering their respective sensitivities. We obtained about \SI{10}{\hour} total of on-sky data, and 10 out of our 12 original target clusters were observed.

\renewcommand{\arraystretch}{1.5}
\begin{table}[!h]
\scriptsize
\caption{List of targets}
\vspace{-0.5cm}
\begin{longtable}{cP{4cm}P{2cm}P{1cm}P{0.5cm}P{0.5cm}P{0.5cm}P{0.5cm}P{0.5cm}}
\toprule																																			
Cluster 	&	 Coordinates 	&	 SOFIA 	&	 $N_\textrm{Fields}$	&	$d $	&	$T_{11} $  	&	$T_{19}  $&	$T_{31}  $&	$T_{37}  $\\
	&	(J2000)	&	Flight IDs	&		&	(pc)	&	(s)	&	(s)	&	(s)	&	(s)	\\
\midrule																	
Cepheus A	&	 22h56m10s +62d03m26s 	&	 F132 F109 	&	2	&	730	&	206	&	234	&	235	&	490	\\
Cepheus C	&	 23h05m45s +62d30m05s 	&	 F132 	&	1	&	730	&	150	&	121	&	121	&	286	\\
IRAS20050	&	 20h07m05s +27d28m51s 	&	 F166 F131 	&	2	&	700	&	321	&	224	&	256	&	266	\\
NGC1333 	&	 03h29m00s +31d17m20s 	&	 F129 F193 F190 	&	9	&	240	&	530	&	558	&	467	&	446	\\
NGC2071 	&	 05h47m06s +00d21m45s 	&	 F192 	&	2	&	420	&	36	&	25	&	33	&	42	\\
NGC2264 	&	 06h41m07s +09d33m35s 	&	 F156 	&	4	&	913	&	495	&	300	&	331	&	587	\\
NGC7129 	&	 21h43m07s +66d06m42s 	&	 F109 	&	1	&	1000	&	383	&	214	&	214	&	709	\\
Ophiuchus 	&	 16h27m05s -24d30m29s 	&	 F157 	&	11	&	150	&	396	&	468	&	501	&	365	\\
S140 	&	 22h19m23s +63d18m44s 	&	 F129 	&	1	&	900	&	322	&	393	&	393	&	568	\\
S171 	&	 00h04m01s +68d34m50s 	&	 F132 	&	1	&	850	&	253	&	219	&	219	&	476\\	\bottomrule																																		\end{longtable} 
\caption*{\textbf{Notes}: For each cluster, we list the SOFIA flights on which the data was taken, the number of individual fields within the cluster, the distance, and the total integration time for each of the 4 observation bands, including all fields. The \SI{37}{\micro\meter} time quoted is a composite time calculated by combining the exposure time of mode 1 with that of mode 3, as discussed in the text.}
%List of our 12 proposed targets, with approximate RA and Dec, distance $d$ in parsecs, peak number density in \# stars/pc$^{2}$ from \citep{Gutermuth:2009p1325}, whether the image saturates in Spitzer/in WISE, the number of different fields for each target, the number of YSOs above our threshold level derived from WISE photometry, and the requested time in minutes on source that we request. Note that the latter DOES NOT include overheads.}
\label{tab:times}
\end{table}

To complement our SOFIA observations, we obtained publicly available \Spitzer, and \Herschel images. Most of our targets have already published \Spitzer IRAC and/or MIPS photometry \citep[mostly from][]{Gutermuth:2009gca,Megeath:2012cn,Evans:2009bka}, which we use in the relevant cases. In the cases where no IRAC photometry was available, we applied our own photometry algorithms to publicly available archival images. We could not find published photometry for the targets with available \Herschel images, hence we also used our own photometry pipeline to derive fluxes from archival images. In some cases, we found published submillimeter continuum measurements to help constrain the long-wavelength behavior of the SEDs.


\section{SOFIA FORCAST characterization}

In addition to the science images, a number of calibrators were observed during each flight for different dichroic settings and wavelength bands. These calibrators are usually bright stars which are point sources for SOFIA's angular resolution, and have known mid-IR fluxes, so they can be used both for flux and PSF calibration. We use them for two purposes: the first is to obtain a robust metric to determine whether sources are extended or not; the second is to determine the aperture correction factor which will be used for aperture photometry of science sources. 

\subsection{PSF size}
The size of the PSF can be defined in multiple ways. We adopt the approach of characterizing the PSF using its encircled energy distribution. Fig~\ref{fig:averageEE} shows the average of the normalized encircled energy distribution of the PSF, measured on all the calibrators observed during our flights which use each filter settings. Each curve represents one of the five different combinations of bandpass filter and dichroic setting that we use for our observations. For each radius, the total energy is the sum of the pixels within the circular aperture of that radius, to which we subtract an estimate of the background in an annulus around the source (see Section~\ref{sec:sourceFluxExtraction} for details on the background subtraction methods). 

\begin{figure}[!h]
\begin{center}
\includegraphics[width=\textwidth]{Figures/average.png}
\vspace{-0.5cm}
\caption[PSF size]{Average PSF encircled energy distribution profile for all calibrator observations.}
\label{fig:averageEE}
\end{center}
\end{figure}


As expected, the PSF at \SI{37.1}{\micro\meter} is larger than the PSFs at shorter wavelengths, but by less than the traditional diffraction limit rule. This indicates that additional PSF smearing is occurring at short wavelengths, likely due to telescope jitter and pointing errors, which is consistent with what other authors have found \citep[e.g.][]{Herter:2013by}. Throughout all the flights, point source calibrators have the same encircled energy distribution shape within $\sim 4\%$ rms. 

To look at the behavior of the PSF in more detail, we can use the full width at half maximum of the encircled energy distribution, \Rfifty, as a proxy for PSF size. The variation of this quantity for the various flights, bandpass/dichroic setting, and calibrators used is showed in Fig.~\ref{fig:Rfiftydist}. This shows the flight-to-flight differences and, for some calibrators, the in-flight variability. We find that the latter is usually small, except for the SOFIA flight on 05-02-2014, for which the spread is quite considerable and could have been caused by instrumental malfunction or abnormal levels of water vapor in the atmosphere. The variation from flight to flight is larger than the variation within a given flight, which indicates variability in the observing conditions, systematics, or thermal radiation environment of the observatory between different flights. Even considering the flight-to-flight and calibrator-to-calibrator variations, the overall spread in \Rfifty for a given observation setting is almost always less then 10\%, making this metric a useful reference to compare with scientific data. In our analysis we will compute \Rfifty for our sources and compare it to the \Rfifty from the current flight for the same filter setting, if the calibration file exists. If no calibration observation exists for a given setting, we use the mean \Rfifty for that setting from calibrators observations in other flights. The ratio $\beta_{37}=\Rfifty_\textrm{source}/\Rfifty_\textrm{cal}$ helps quantify the extension of the source, to within $\sim 10\%$ confidence level. 

\begin{figure}[!h]
\begin{center}
\includegraphics[width=\textwidth]{Figures/R50.png}
\vspace{-0.5cm}
\caption[PSF size of calibrators]{Distribution of the \Rfifty for all calibrators observations within each bandpass. Lower wavelengths have lower \Rfifty. In red: \SI{11}{\um} band, with dichroic; in green: \SI{19}{\um} band, with dichroic; in blue: \SI{31}{\um} band, with dichroic; in yellow: \SI{37}{\um} band, with dichroic; in purple: \SI{37}{\um} band, no dichroic. Down triangles: $\alpha$ Boo; Pentagons: $\alpha$ Cet; Diamonds: $\alpha$ Tau;  Up triangles: $\beta$ And; Hexagons: $\beta$ Peg; Circles: $\beta$ UMi.}
\label{fig:Rfiftydist}
\end{center}
\end{figure}

\subsection{Aperture correction factor}
\label{subsec:apcorr}
In Fig.~\ref{fig:averageEE}, we observe that the encircled energy does not vary much by an aperture with radius of 12 pixels, so we consider this fiducial aperture as our "total flux" aperture. The goal of aperture photometry is to estimate the amount of flux in this large aperture, which we consider to be the total amount of flux from the source, by only measuring flux within a much smaller aperture. This has the advantage of reducing contamination from other sources, and increases the signal-to-noise ratio of the flux estimate since the pixels near the tail of the PSF usually contain more noise than signal. 
In Fig~\ref{fig:apercorr}, we plot the aperture correction factor that we compute from the ratio of the flux measured within an aperture of 3 pixels radius and this 12-pixel radius aperture.  Not surprisingly, this graph follows very closely the plot of $\Rfifty$ from Fig~\ref{fig:Rfiftydist}, showing the close link between the aperture correction factor and the shape of the calibrator's PSF. We match each observation in our data to the mean of the aperture correction factors for the same observation setting and flight.

\begin{figure}[!h]
\begin{center}
\includegraphics[width=\textwidth]{Figures/Aper_corr.png}
\vspace{-0.5cm}
\caption[aperture correction]{Instrumental response and aperture correction. The color code and marker shape is the same as in Fig.~\ref{fig:Rfiftydist}. Lower wavelengths usually have smaller aperture correction.}
\label{fig:apercorr}
\end{center}
\end{figure}

\subsection{Instrument response and overall uncertainty}
To validate our approach, we take a look at the calibrator fluxes after normalization by the calibration factor, which is provided directly by the FORCAST pipeline. This calibration factors converts the pixel digital value a physical flux density unit, and presumably is determined using the flux from calibrator stars as well. Here we re-measure the flux from each calibrator for each observation setting and each flight, using our standard aperture photometry method and background subtraction. Ideally, we would always obtain the same flux for each setting and calibrator, independently of the flight, an assertion we find true to within $\sim 5\%$ r.m.s (Fig~\ref{fig:response}). The in-flight errors are typically lower than this. This validates our aperture photometry method, and we can trust that the instrument's systematics are well-behaved to within these levels. 

This would suggest that we can adopt systematic $1\sigma$ uncertainties of $\sim 5\%$, a value which is consistent with the published uncertainties of $3\sigma \approx 20\%$ \citep{DeBuizer:2012ie}. In an effort to be conservative, we chose to follow those authors and adopt a $1\sigma\approx 7\%$ systematic measurement uncertainty.


\begin{figure}[!h]
\begin{center}
\includegraphics[width=\textwidth]{Figures/Phot_val.png}
\vspace{-0.5cm}
\caption[Instrumental response]{Instrumental response, showing decreased calibrator fluxes with longer wavelengths, which is expected since all calibration targets are evolved stars. The color code and marker shape is the same as in Fig.~\ref{fig:Rfiftydist}. In this plot, the variation across multiple flights for a given marker type of a given color is usually less than $\sim$5\%. Note that for the bottom green triangles ($\alpha$ Boo at \SI{19}{\um}), there seems to be a systematic change between flights occurring before and after 2013-09-19.}
\label{fig:response}
\end{center}
\end{figure}


\section{Data reduction and photometry}

The data were processed through various versions of the online pipeline to yield Level 2 data products available on the archive \citep{Herter:2013by}. We apply our own reduction procedure and photometry pipeline on those products to derive final images, source positions, fluxes and sensitivities. Our software makes extensive use of the Python \textit{astropy} package \citep{2013A&A...558A..33A} and its associated modules \textit{photutils} and \textit{APLpy}. 

\subsection{Pre-treatment}
Some manual treatment of each image was necessary before it could be analyzed by our software. We followed this procedure: a) visually align the WCS coordinate system, often 10-20" off, using point sources and archival data from other wavelengths and facilities such as IRAC \SI{8}{\micro\meter}; b) crop the images to clean off the nodded fields, and c) identify the coordinates of each source, both point-like and extended.

After these manual steps, the Level 2 images are multiplied by the calibration factor provided by the online pipeline, which converts them to Jy/pixel. We do not proceed to any systematic color correction, but the effects on the fluxes are very small \citep{Herter:2013by}.
%\begin{enumerate}
%\item Adjust WCS coordinates: use images at other wavelengths (2MASS, IRAC, MIPS, WISE) to re-align the (RA, DEC) position of the field. We estimate that this process is good to within one SOFIA pixel (\ang{;;0.768}) for the fields where one or more point sources can be identified. Extended fields are less trustworthy, since matching the extended emission to other wavelengths is harder. The rotation of the field produced by the SOFIA pipeline is correct for all of our data. 
%\item Crop each image, remove chopped fields, remove artifacts.
%\item Identify and categorize sources: isolated point sources, clustered point sources, and extended sources. For extended sources, a circular or elliptical aperture is used to try to encompass the entirety of the emission.
%\item Manually identify a location in the field that corresponds to a representative background.
%\end{enumerate}

\subsection{Source flux extraction}
\label{sec:sourceFluxExtraction}

We fed the adjusted files to our photometry pipeline. For each identified source, we determine its flux in all bands using aperture photometry with local background subtraction. The aperture correction factor we used is the one determined from the calibrators observed for the same observation setting during the same flight as the one when the data was taken. If a calibrator is not available during the flight, we use the average aperture correction factor taken over 9 of our 10 flights (we choose to exclude the flight on 05/02/2014 which seems to have abnormal behavior).

We distinguish between 3 types of sources after manual identification: \textit{isolated}, which are point sources with no nearby objects; \textit{clustered}, which are point sources with nearby objects; and \textit{extended}, which are not consistent with being point sources based on visual inspection. 

For point sources that are isolated, we use our standard aperture of 3 pixels at all wavelengths. We consider an annulus surrounding the source extending from 12 to 20 pixels radius (24 to 40 pixels for clustered sources): the local background is determined from the mode of the pixels in the annulus, while the sensitivity is calculated by measuring the standard deviation of the flux values within 3-pixel apertures spread over that annulus \citep{Shimizu:2016if}. We apply the aperture correction derived from the calibrator observations taken during that flight.

For extended sources, an elliptical aperture is determined manually from the \SI{37}{\micro\meter} images. The local background is determined from the mode of an elliptical annulus, with an inner boundary at the elliptical aperture and an outer boundary corresponding to an ellipse 20\% larger. The sensitivity quoted is the point source sensitivity, and is determined following the same method as for point sources, using the standard deviation of apertures spread across the elliptical annulus. 

The photometry from sources that were observed in different flights is then combined to increase the signal-to-noise ratio. This combination takes into account the sensitivity of each source by appropriately weighing each image.

The noise level calculated for the observation is added in quadrature to the systematic uncertainty of the instrument, for which we follow the recommendation from \citep{Herter:2012hv} and adopt a 7\%, $1\sigma$ uncertainty. 

\renewcommand{\arraystretch}{1.5}
\begin{table}[!h]
\scriptsize
\caption{SOFIA photometry comparison} \label{tab:SOFIAPhotometryHarvey}
\vspace{-0.5cm}
\begin{longtable}{l|P{1cm}P{1cm}|P{1cm}|P{1cm}P{1cm}|P{1cm}P{1cm}}
\toprule															
SOFIA name	&	F11	&	F11L	&	F19	&	F31	&	F31L	&	F37	&	F37L	\\
	&	Jy	&	Jy	&	Jy	&	Jy	&	Jy	&	Jy	&	Jy	\\
\midrule															
S140.3	&	10.28	&	9.70	&	101.49	&	419.41	&	401.00	&	525.90	&	669.00	\\
S140.4	&	3.80	&	4.00	&	88.95	&	337.22	&	368.00	&	352.07	&	485.00	\\
S140.5	&	110.57	&	110.00	&	830.97	&	2065.13	&	1585.00	&	2278.61	&	2176.00	\\
\midrule															
Sum of sources in cluster	&	124.65	&	123.70	&	1021.40	&	2821.76	&	2354.00	&	3156.58	&	3330.00	\\
Total cluster emission	&	135.20	&	145.00	&	1194.57	&	4449.46	&	3780.00	&	5840.64	&	6730.00	\\
Ratio	&	1.08	&	1.17	&	1.17	&	1.58	&	1.61	&	1.85	&	2.02	\\
\bottomrule					
	\end{longtable} 
\caption*{Comparison of SOFIA four-band photometry from \citet{Harvey:2012kw} on S140 (columns with 'L'). All fluxes are in Janskies. The authors' "total emission" actually represents the total emission in the entire field of view, whereas out measurement corresponds to a manually-selected source region encompassing only the dense core. The total emission in the entire field of view is less representative, as it could include contribution from other sources as well as areas of negative flux from the chopping and nodding steps. In this cluster, there is a large amount of emission which is not clearly associated to the three identified sources.}

\end{table}


To validate our flux extraction method, we compare our results with data from \citet{Harvey:2012kw} who observed one of the sources in our sample, S140. Their photometry (shown in their Table 1) of IRS 1, 2 and 3 (corresponding to our targets S140.5, S140.4, and S140.3, respectively) is compared to our photometry in Table~\ref{tab:SOFIAPhotometryHarvey}. We find reasonable agreement between our fluxes and theirs, although some differences are larger than expected in the longer wavelength bands. We attribute this to differences in the exact centroid location of the sources, which could be due to a different aperture size. Centroid errors have more impact at longer wavelengths, where the flux is larger and the PSF wings more extended.

\subsection{Image sensitivity}

\renewcommand{\arraystretch}{1.5}
\begin{table}[!h]
\scriptsize
\caption{FORCAST Sensitivities}
\vspace{-0.5cm}
\begin{longtable}{c|P{0.5cm}P{0.5cm}P{0.5cm}|P{0.5cm}P{0.5cm}P{0.5cm}|P{0.5cm}P{0.5cm}P{0.5cm}|P{0.5cm}P{0.5cm}P{0.5cm}|P{1cm}}
\toprule																			
Cluster 	&	\multicolumn{3}{c|}{F11}					&	\multicolumn{3}{c|}{F19}					&	\multicolumn{3}{c|}{F31}					&	\multicolumn{3}{c|}{F37}					&	 Sources 	\\
	&	$\sigman$ 	&	$\sigstd$ 	&	$\sigth$	&	$\sigman$ 	&	$\sigstd$ 	&	$\sigth$	&	$\sigman$ 	&	$\sigstd$ 	&	$\sigth$	&	$\sigman$ 	&	$\sigstd$ 	&	$\sigth$	&		\\
\midrule																											
CepA 	&	0.07	&	0.04	&	0.05	&	0.11	&	0.05	&	0.05	&	0.19	&	0.07	&	0.16	&	0.26	&	0.09	&	0.34	&	4	\\
CepC 	&	0.03	&	0.03	&	0.04	&	0.10	&	0.05	&	0.04	&	0.19	&	0.06	&	0.16	&	0.16	&	0.09	&	0.30	&	4	\\
IRAS20050 	&	0.04	&	0.03	&	0.04	&	0.08	&	0.04	&	0.05	&	0.13	&	0.05	&	0.16	&	0.30	&	0.11	&	0.32	&	7	\\
NGC1333 	&	0.12	&	0.04	&	0.07	&	0.07	&	0.07	&	0.07	&	0.22	&	0.08	&	0.25	&	0.48	&	0.13	&	0.52	&	11	\\
NGC2071 	&	0.19	&	0.10	&	0.12	&	0.32	&	0.15	&	0.15	&	0.21	&	0.22	&	0.49	&	0.45	&	0.28	&	0.81	&	6	\\
NGC2264 	&	0.07	&	0.03	&	0.05	&	0.19	&	0.05	&	0.06	&	0.28	&	0.07	&	0.20	&	0.21	&	0.09	&	0.43	&	21	\\
NGC7129 	&	0.07	&	0.03	&	0.03	&	0.10	&	0.04	&	0.03	&	0.26	&	0.09	&	0.12	&	0.17	&	0.08	&	0.19	&	5	\\
Ophiuchus 	&	0.11	&	0.05	&	0.08	&	0.16	&	0.07	&	0.08	&	0.31	&	0.09	&	0.27	&	0.41	&	0.18	&	0.65	&	19	\\
S140 	&	0.04	&	0.03	&	0.03	&	0.16	&	0.03	&	0.03	&	0.21	&	0.07	&	0.09	&	0.35	&	0.11	&	0.21	&	7	\\
S171 	&	0.04	&	0.03	&	0.03	&	0.07	&	0.04	&	0.03	&	0.07	&	0.05	&	0.12	&	0.16	&	0.06	&	0.23	&	2	\\
\bottomrule					
	\end{longtable} 
\caption*{\textbf{Notes}: For each band F11, F19, F31 and F37, we measure the 1$\sigma$ sensitivity \sigman and \sigstd in each field from the data using two different methods (see text), and present here the median of all fields. The theoretical sensitivity \sigth corresponds to the expected sensitivity for the actual integration time, using the SOFIA FORCAST observation planning tools and assuming moderate water vapor content. All sensitivity values are in Janskies.}
%List of our 12 proposed targets, with approximate RA and Dec, distance $d$ in parsecs, peak number density in \# stars/pc$^{2}$ from \citep{Gutermuth:2009p1325}, whether the image saturates in Spitzer/in WISE, the number of different fields for each target, the number of YSOs above our threshold level derived from WISE photometry, and the requested time in minutes on source that we request. Note that the latter DOES NOT include overheads.}
\label{tab:SOFIASensitivity}
\end{table}

In order to determine the absolute sensitivity in the image, we use two methods. First, we manually determine a region near each cluster that visually appears devoid of sources. We calculate the sensitivity as if this background region was a source, by patching apertures in an annulus around this background location and calculating the standard deviation of the obtained fluxes. We call this sensitivity measurement $\sigman$. The main downside of this method is that it requires a manual operation to select the appropriate background field, and hence could have more variation depending on which field we select. Second, we use a routine that iteratively isolates the pixel values above $2\sigma$ of the image, in order to remove the contamination from our actual sources. The standard deviation of the resulting image is then calculated, and is multiplied by the square root of the number of pixels in an aperture of 3 pixel radius. This corresponds to a floor sensitivity $\sigstd$. We present our results in Table~\ref{tab:SOFIASensitivity}, where we also compare this sensitivity with the expected sensitivity $\sigth$ obtained using the online calculator with the actual exposure time of our images. We note that usually, the theoretical values are more in agreement with our first method for F31 and F37, while more in agreement with our second method for F11 and F19. 



\subsection{Other photometry}

SOFIA provides mid-IR photometry. We looked in the literature for published fluxes on our targets in order to reconstruct more complete SEDs. In addition to our four SOFIA bands, We collected data from 2MASS, \Spitzer, and other instruments. Photometry from these sources is published in online catalogs, which we programmatically cross-reference with the positions of our targets. The closest target that corresponds to a Vizier location query is selected to be the correct catalog match. For the 2MASS data, the location of the target was required to be less than \ang{;;2} away from our coordinates for point sources, and \ang{;;5} for extended sources. For the \Spitzer data, the matching radius is \ang{;;3} for point sources and \ang{;;10} for extended sources. In addition to automated online catalog searches, we add values for sources in NGC2071 from \citet{vanKempen:2012fb}.

For our two most clustered regions, the cores of NGC~2071 and IRAS~20050+2720, the published catalogs do not have all available fluxes. We assume that the sources are so clustered that the source extraction software from the authors do not register them as point sources, due to confusion or saturation effects. Hence we adapt our own photometry routines for these clustered environments and obtain the fluxes directly from the calibrated Level 3 images, which are all available on the archive. In Table~\ref{tab:SpitzerPhotometry}, we compare our photometry results with published fluxes from \citet{Megeath:2012cn} and \citet{Gutermuth:2009gca} for isolated sources elsewhere in these same fields of view. We use the \Spitzer handbook recommendations for aperture photometry on \Spitzer archival images (\ang{;;2.4} aperture with and an annulus that extends from 12 to \ang{;;20}). We find that our results are within 10\% of the other authors' results for isolated sources, which can reflect a simple difference in exact aperture centroiding position. 

\renewcommand{\arraystretch}{1.5}
\begin{table}[!h]
\scriptsize
\caption{Spitzer photometry comparison}
\vspace{-0.5cm}
\begin{longtable}{l|P{1cm}P{1cm}P{1cm}P{1cm}}
\toprule																			
SOFIA name	&	i1	&	i2	&	i3	&	i4	\\
	&	Jy	&	Jy	&	Jy	&	Jy	\\
\midrule									
NGC2071.1	&	0.060	&	0.056	&	0.004	&	-0.021	\\
NGC2071.3	&	0.018	&	-0.010	&	-0.004	&	-0.047	\\
NGC2071.4	&	0.090	&	-0.054	&	0.036	&	-0.066	\\
NGC2071.5	&	-0.130	&	-0.109	&	-0.144	&	-0.139	\\
\midrule									
IRAS20050.1	&	0.020	&	0.039	&	0.017	&	0.131	\\
IRAS20050.3	&	0.181	&	0.122	&	0.082	&	0.121	\\
IRAS20050.6	&	-0.044	&	-0.046	&	-0.092	&	-0.056	\\
\bottomrule					
	\end{longtable} 
\caption*{\textbf{Note:} Fractional difference between our own aperture photometry on \Spitzer archival images and published \Spitzer photometry from \citet{Megeath:2012cn} for NGC2071, and \citet{Gutermuth:2009gca} for IRAS20050+2720. When values are negative, it means that their photometry is lower than ours.}
\label{tab:SpitzerPhotometry}
\end{table}

In some cases, we also found archival Herschel images, although no published photometry was available for most our sources. We then apply our same aperture photometry routines for those calibrated Herschel images, using aperture and background subtraction parameters from \citet{Shimizu:2016if} for the PACS and SPIRE. We find also very good agreement between our photometry results for the PACS~\SI{70}{\um} band and the published \Spitzer MIPS~\SI{70}{\um} for some of these sources. %Because of the very large beams of Herschel compared to FORCAST, the SPIRE bands are considered upper limit fluxes for sources that are further than \SI{300}{\pc}.

\section{Data products}

We identify 70 point sources and 14 extended sources in our sample. We produced three types of data products: the mosaic images in each band for all the clusters we observed; the photometry catalogs which can be used to make SEDs; and the fitted physical parameters for the point sources, which are determined from our radiative transfer model, explained in Section~\ref{sec:SEDFitting}.

\subsection{Mosaics}
The SOFIA FORCAST images consist of $\sim 200$ individual images, each representing a field at a given wavelength. Some fields are revisited multiple times when the entire observation could not be completed in a single flight leg. These individual fields are processed and mosaiced together to form one single map for each wavelength and each cluster. 

Before mosaicing the fields, we did a 2D background subtraction. This method divides the images into sections of $50\times 50$ pixels, estimates the median in each cell, and fits a 2D function to these median values. This function is used to construct a smooth background, which is then removed from the image. Each background-subtracted image is calibrated (using the calibration factor that is supplied by the FORCAST pipeline), and weighed by its exposure time before it is co-added into a mosaic in the WCS coordinate reference frame. Although these maps are useful for view the source distribution and spot artifacts, the actual photometry described in the previous sections uses each individual raw field, before the mosaicing and without our background subtraction (some level of background subtraction is already done by the SOFIA pipeline on the archive). If a source is present in multiple fields, the photometry from each of these fields is combined to provide a better flux estimate.

In Fig.~\ref{fig:varietySources} we present four maps from our cluster sample. Each map is a three-color image (red: \SI{37}{\micron}, green: \SI{31}{\micron} and blue: \SI{19}{\micron}), and the scale and stretch of each color is adjusted to balance each color. The three bands have resolutions of
6.4~pixels (\ang{;;4.9}), 6~pixels (\ang{;;4.6}) and 5~pixel (\ang{;;3.8}) FWHM for 37, 31 and \SI{19}{\um} respectively.

\begin{figure}[!h]
\begin{center}
\includegraphics[width=\textwidth]{Figures/RGBmosaic.png}
\vspace{-1cm}
\caption[RGB images of select sample of sources]{RGB images of selected sample of sources (red: \SI{37}{\micron}, green: \SI{31}{\micron} and blue: \SI{19}{\micron}). In these images, the three bands have 6.4, 6 and 5 pixels FWHM for 37, 31 and \SI{19}{\um} respectively.}
\label{fig:varietySources}
\end{center}
\end{figure}

\subsection{Photometry catalog}
%We also produce enhanced SEDs for all of our sources, where we combine archival data with our new SOFIA photometry. The SED plots are complemented with snapshots of the source seen with IRAC and SOFIA. In addition, a snapshot of the corresponding FORCAST \SI{37}{\um} calibrator is shown, which allows to quickly determine the degree of spatial extension of the sources.

\begin{landscape}
\begin{table}
\tiny
\caption[NGC1333 photometry]{Extract of NGC1333 photometry used for SED modeling.}
\label{tab:OphiuchusNGC1333}
\vspace{-0.5cm}
\begin{longtable}{llrrrrrrrrrrrrrrrrrrrrrrrrrrrrrrrrrrrrrrrrrrrrrrr}																\toprule																														
SOFIA name	&	Coordinates		&	R37	&	Lbol	&	j	&	e\_j	&	h	&	e\_h	&	ks	&	e\_ks	&	i1	&	e\_i1	&	i2	&	e\_i2			\\
\midrule																														
NGC1333.1	&	03h29m07.7s	+31d21m57.0s	&	0.746	&	8.385	&	0.0012	&	0.0001	&	0.0031	&	0.0003	&	0.0450	&	0.004	&	0.696	&	0.070	&	1.800	&	0.180			\\
NGC1333.2	&	03h29m10.3s	+31d21m55.5s	&	2.232	&	27.832	&	0.2853	&	0.0285	&	0.6539	&	0.0654	&	0.9010	&	0.090	&	0.637	&	0.064	&	0.446	&	0.045			\\
NGC1333.3	&	03h29m01.5s	+31d20m20.5s	&	0.904	&	8.104	&	0.0008	&	0.0001	&	0.0029	&	0.0003	&	0.0296	&	0.003	&	0.544	&	0.054	&	1.090	&	0.109			\\
NGC1333.4	&	03h29m11.1s	+31d18m30.8s	&	1.103	&	3.056	&	0.0007	&	0.0007	&	0.0009	&	0.0009	&	0.0015	&	0.002	&	0.001	&	0.0007	&	0.004	&	0.0004			\\
NGC1333.5	&	03h29m10.6s	+31d18m19.6s	&	1.623	&	2.786	&	0.0007	&	0.0007	&	0.0009	&	0.0009	&	0.0015	&	0.002	&	0.002	&	0.0002	&	0.007	&	0.001			\\
NGC1333.6	&	03h29m13.0s	+31d18m13.8s	&	0.951	&	1.155	&	0.0007	&	0.0007	&	0.0009	&	0.0009	&	0.0015	&	0.0004	&	0.046	&	0.005	&	0.180	&	0.018			\\
\midrule																														
	&			&	i3	&	e\_i3	&	i4	&	e\_i4	&	F11	&	e\_F11	&	F19	&	e\_F19	&	M24	&	e\_M24	&	F31	&	e\_F31	&	F37	\\
\midrule																														
NGC1333.1	&	03h29m07.7s	+31d21m57.0s	&	3.060	&	0.306	&	2.550	&	0.255	&	0.225	&	0.169	&	1.502	&	0.208	&	--	&	0.260	&	6.886	&	0.640	&	10.994	\\
NGC1333.2	&	03h29m10.3s	+31d21m55.5s	&	0.448	&	0.080	&	0.913	&	0.128	&	8.414	&	0.596	&	36.517	&	2.562	&	--	&	--	&	106.490	&	7.457	&	135.723	\\
NGC1333.3	&	03h29m01.5s	+31d20m20.5s	&	1.690	&	0.211	&	3.060	&	0.306	&	1.681	&	0.131	&	6.902	&	0.493	&	--	&	0.069	&	9.256	&	0.656	&	9.406	\\
NGC1333.4	&	03h29m11.1s	+31d18m30.8s	&	0.005	&	0.001	&	0.004	&	0.0004	&	0.097	&	0.060	&	0.076	&	0.115	&	0.607	&	0.061	&	1.785	&	0.209	&	3.040	\\
NGC1333.5	&	03h29m10.6s	+31d18m19.6s	&	0.010	&	0.001	&	0.011	&	0.001	&	0.114	&	0.093	&	0.150	&	0.119	&	0.771	&	0.077	&	1.946	&	0.234	&	2.166	\\
NGC1333.6	&	03h29m13.0s	+31d18m13.8s	&	0.274	&	0.027	&	0.320	&	0.032	&	0.160	&	0.035	&	0.570	&	0.093	&	0.735	&	0.074	&	1.446	&	0.180	&	1.806	\\
\midrule																														
	&			&	e\_F37	&	M70	&	e\_M70	&	H70	&	e\_H70	&	H160	&	e\_H160	&	H70	&	e\_H70	&	H160	&	e\_H160	&	H250	&	e\_H250	\\
\midrule																														
NGC1333.1	&	03h29m07.7s	+31d21m57.0s	&	0.948	&	49.300	&	4.930	&	52.724	&	5.272	&	66.529	&	35.197	&	52.724	&	5.272	&	66.529	&	35.197	&	71.541	&	14.258	\\
NGC1333.2	&	03h29m10.3s	+31d21m55.5s	&	9.507	&	--	&	--	&	70.039	&	7.004	&	77.574	&	20.036	&	70.039	&	7.004	&	77.574	&	20.036	&	87.661	&	15.014	\\
NGC1333.3	&	03h29m01.5s	+31d20m20.5s	&	0.695	&	23.400	&	2.340	&	20.218	&	2.022	&	78.316	&	7.832	&	20.218	&	2.022	&	78.316	&	7.832	&	101.472	&	18.943	\\
NGC1333.4	&	03h29m11.1s	+31d18m30.8s	&	0.341	&	--	&	--	&	16.609	&	1.661	&	53.689	&	5.369	&	16.609	&	1.661	&	53.689	&	5.369	&	57.215	&	6.293	\\
NGC1333.5	&	03h29m10.6s	+31d18m19.6s	&	0.377	&	20.600	&	2.060	&	14.627	&	1.463	&	49.868	&	4.987	&	14.627	&	1.463	&	49.868	&	4.987	&	52.536	&	6.166	\\
NGC1333.6	&	03h29m13.0s	+31d18m13.8s	&	0.345	&	4.290	&	0.429	&	1.527	&	3.883	&	4.702	&	13.332	&	1.527	&	3.883	&	4.702	&	13.332	&	29.105	&	6.272	\\
\midrule																														
	&			&	H350	&	e\_H350	&	H500	&	e\_H500	&	S850	&	e\_S850	&	F1100	&	e\_F1100	&	S1300	&	e\_S1300	&	$\alpha$	&	e\_$\alpha$	&		\\
\midrule																														
NGC1333.1	&	03h29m07.7s	+31d21m57.0s	&	45.559	&	17.857	&	24.264	&	16.301	&	--	&	--	&	1.300	&	0.130	&	--	&	--	&	0.280	&	0.564	&		\\
NGC1333.2	&	03h29m10.3s	+31d21m55.5s	&	51.506	&	16.114	&	24.742	&	13.062	&	--	&	--	&	--	&	--	&	--	&	--	&	1.243	&	--	&		\\
NGC1333.3	&	03h29m01.5s	+31d20m20.5s	&	70.907	&	17.371	&	40.867	&	11.474	&	--	&	--	&	1.500	&	0.150	&	--	&	--	&	0.714	&	0.385	&		\\
NGC1333.4	&	03h29m11.1s	+31d18m30.8s	&	38.449	&	6.033	&	18.594	&	4.666	&	--	&	--	&	2.000	&	0.200	&	--	&	--	&	1.864	&	0.458	&		\\
NGC1333.5	&	03h29m10.6s	+31d18m19.6s	&	36.232	&	6.189	&	18.007	&	4.878	&	--	&	--	&	2.000	&	0.200	&	--	&	--	&	1.705	&	0.273	&		\\
NGC1333.6	&	03h29m13.0s	+31d18m13.8s	&	34.781	&	8.007	&	21.255	&	6.628	&	--	&	--	&	0.630	&	0.063	&	--	&	--	&	1.001	&	0.501	&		\\
\bottomrule																														
\end{longtable}																																	\caption*{\textbf{Note:} The table contains the source name, coordinates in J2000, the ratio of R37$=\Rfifty_\textrm{source}/\Rfifty_\textrm{cal}$, the bolometric luminosity determined by integrating the data points in log-log space, followed by the photometry and its 1$\sigma$ error in the 2MASS bands (j, h, k$_s$ at 1.3, 1.6 and \SI{2.2}{\um} respectively), IRAC bands (i1, i2, i3, i4 at 3.6, 4.5, 5.8 and \SI{8}{\um} respectively), the FORCAST bands (F11, F19, F31, F37), the \Spitzer MIPS bands (M24 and M70), the Herschel PACS and SPIRE bands (H70, H160, H250, H350), the SCUBA band (S850), the BOLOCAM band (F1100) and the SMA continuum band (S1300). The number following capital letters in the band denomination indicates the band's central wavelength. Flags are used to designate whether or not a source is considered an upper limit, and are not shown in this table for clarity. The fluxes that are upper limit can be seen in the SED images, Fig.~\ref{fig:NGC1333_SEDs} and Fig~\ref{fig:Oph_SEDs}. The complete version of this table is made available electronically.}
\end{table}																														\end{landscape}

We produced a consolidated list of fluxes for our clusters, where we gather 2MASS, \Spitzer, FORCAST, \Herschel, SCUBA, and SMA data, when available, for $\sim 90$ sources. Most sources are point sources for the SOFIA FORCAST \SI{37}{\um} band, but some sources present a certain spatial extension which was not known before. 


A few other parameters are determined from the FORCAST data and shown in the catalog: the $R_{37}$ ratio, which is the ratio of \Rfifty for the source and \Rfifty for the last observed calibrator; the spectral index and its uncertainty, computed from the \SI{2.2}{\um} - \SI{37}{\um} fluxes; and the bolometric luminosity for each source. Note that the bolometric luminosity is the integration of the observed emission across the observed wavelength; as such, it is an observed quantity but it is generally not the true luminosity of the source due to extinction (which is not corrected) and viewing angle corrections. An excerpt of the final table containing just the results for a few NGC~1333 sources is shown in Table~\ref{tab:OphiuchusNGC1333}.

\begin{figure}[!h]
\begin{center}
\includegraphics[width=\textwidth]{Figures/NGC1333_SEDs.png}
\caption[NGC1333 SEDs]{SEDs of the point sources in NGC1333. The red curve represents the best fit. The grey curves represent all the fits with $R$ within 0.5 of the best fit (see Section~\ref{subsec:fittingmethod} for details about the fitting process). The white arrows show which data point is considered an upper limit for the fitting routine. Note that 2MASS J- and H-band measurements, as well as \Spitzer MIPS \SI{24}{\um} and \SI{70}{\um} are plotted, but never used for fitting. Red triangles: 2MASS. Green diamonds: \Spitzer (our photometry or data from existing catalogs). Dark blue triangles: FORCAST (our data). Purple stars: \Herschel (our photometry). Green triangles: SCUBA \SI{850}{\um} and SMA \SI{1.3}{\milli\meter} data from \citep{vanKempen:2009ku} and \citep{vanKempen:2012fb}. Left-pointing blue triangles: \SI{1.1}{\milli\meter} data from \citet{Enoch:2009ch}.}
\label{fig:NGC1333_SEDs}
\end{center}
\end{figure}

The SEDs for our most complete clusters, NGC1333 and Ophiuchus, are shown in Fig~\ref{fig:NGC1333_SEDs} and Fig~\ref{fig:Oph_SEDs}. These show the data points in various color codes and marker types, as well as the best fit models for each source, as determined using our custom fitting routine, described in Section~\ref{sec:SEDFitting}. The R-value, indicated for each fit, is a metric that indicates how well the software was able to find a match between the data points and a pre-computed grid of models. Lower values of R generally indicate better fits.

Many SEDs are well fitted.
Some of the sources show poor fits (R>3), where it seems difficult to find a model that both satisfies the long-wavelength measurements and the IRAC measurements. This indicated that none of the models in the grid fit well. We think this could be explained either by a mismatch of the spatial resolution for the different measurement bands, in which case the long-wavelength bands sample flux that is not necessarily associated with the protostar, but rather is associated with another, nearby source or extended dense cloud emission; or by an excess flux from the IRAC bands that could be explained by the proximity to an outflow \citep{NoriegaCrespo:2004fe,Hudgins:2004wa}. Fig.~\ref{fig:NGC1333_Confusion} shows an example of a poor fit which could be attributed to excess IRAC emission due to a nearby outflow. 

%\begin{figure}[!h]
%\begin{center}
%\includegraphics[width=\textwidth]{Figures/NGC1333_6_saturated_mosaic.png}
%\caption[Confusion in NGC1333]{Example of NGC1333.6, a poorly-fitting source that exhibits confusion. The Herschel is done at the same location as the sources identified using \Spitzer and FORCAST, hence could lead to situation where the long-wavelength comes from emission that is either extended, or too close from another source. [NOW THAT WE KNOW THAT, SHOULDN' I JUST REDO THE FIT WITH THE HERSCHEL AS AN UPPER LIMIT?]}
%\label{fig:NGC1333_Confusion}
%\end{center}
%\end{figure}

\begin{figure}[!h]
\begin{center}
\includegraphics[width=\textwidth]{Figures/NGC1333_1_saturated_mosaic.png}
\caption[Confusion in NGC1333]{Fields centered on NGC1333.1, a poorly-fitting source that shows a mismatch between the IRAC fluxes and the longer-wavelength fluxes.  The \Herschel SPIRE \SI{250}{\um} (right) is taken at the same location as the sources identified using \Spitzer IRAC \SI{3.6}{\um} (left) and FORCAST \SI{37}{\um} (middle). The resolution of the SPIRE beam is not shown on the figure, because it has a radius of \ang{;;22}. In this particular case, it appears that the long-wavelength emission is associated with the source. However, this source is in close proximity to NGC1333.2, an extended source of our sample, which shows a bow shock structure in the southwest of its center seen in IRAC \SI{3.6}{\um}, as well as diffuse emission that extends all the way to NGC1333.1.}
\label{fig:NGC1333_Confusion}
\end{center}
\end{figure}


\begin{figure}[!h]
\begin{center}
\includegraphics[width=\textwidth]{Figures/Oph_SEDs.png}
\caption[Ophiuchus SEDs]{SEDs of the point sources in the Ophiuchus cluster. Same legend as Fig.~\ref{fig:NGC1333_SEDs}}
\label{fig:Oph_SEDs}
\end{center}
\end{figure}


\subsection{Fitted physical parameters}

The spectral index ($\alpha \equiv (d\log(\lambda F_\lambda)/d\log\lambda$) distribution for the point sources in our sample is shown on the left of Fig.~\ref{fig:SpectralIndex}. Most sources have strongly positive spectral indices, indicative of a rise in the SED with increasing wavelength and a significant contribution to the total luminosity by long-wavelength emission. These objects are more embedded, and thought to be younger than objects with negative spectral index. Note however that the emission generally
peaks a little shortward of 100 $\mu$m. A closer inspection of the distribution of our sources reveals that the targets with negative index mostly lie in the Ophiuchus cluster, and are consistent with late Class I objects which have already cleared a significant fraction of their envelopes \citep{Jorgensen:2008gz}. 

\begin{figure}[!h]
\begin{center}
\includegraphics[width=\textwidth]{Figures/SpectralIndex.pdf}
\vspace{-1cm}
\caption[Spectral Index distribution of point sources]{Spectral index distribution of all point sources in our sample. \textit{Left}: standard determination of the spectral index, using 2MASS and \Spitzer from \SI{2}{\micron} to \SI{24}{\micron}, when data is available. \textit{Right}: Determination of the spectral index using data from 2MASS, \Spitzer and our FORCAST data up to \SI{37}{\micron}. The distribution changes significantly when you account for the longer fluxes in these clustered regions.}
\label{fig:SpectralIndex}
\end{center}
\end{figure}

The data tables also include all of the physical parameters derived using the technique from Section~\ref{sec:SEDFitting}, as well as their uncertainties.

\section{SED fitting}
\label{sec:SEDFitting}

This section looks more closely at the SED fitting process: examining its value and its common shortcomings. First, we need to recognize that SED fitting is prone to many degeneracies \citep[see e.g.][for an introduction on the degeneracies of SED fitting]{Robitaille:2007dl} unless one has a great deal of spatial and spectral information about the source, which is usually not the case. In order to make physically plausible models, there are
usually many geometrical and physical parameters in detailed radiative transfer models, but only a handful of measurement points are available to fit, leading to a dramatically under-constrained problem. The goals of our fitting procedures are then to reduce the number of parameters to those which have a significant quantitative impact on the SED, to identify the families of model parameters that fit the SED, and to define the "best fit" model and its "uncertainty" which represents the range in the model parameters with "reasonable" fits.

As our starting point of our investigation of fitting SEDs to our sources, we used the \textit{sedfitter} tool from \citep{Robitaille:2006cb}. These authors computed a large grid of tens of thousands of SED models using a radiative transfer code by \citep{Whitney:2003ke}, by varying 14 geometrical and physical parameters in the dust density grid such as the size of the disk, the accretion rates, the radius and mass of the envelope, etc. The models are then evaluated in the bands corresponding to our data, and a $\chi^2$ metric is evaluated for each model. By exploring the distribution of $\chi^2$, we noticed, as expected, the very large correlations between the parameters which is indicative of many local minimas in the 14-dimensional grid. Hence, inferring geometrical and physical parameters from such a grid can be misleading.

\subsection{A custom grid of models}

We use Hyperion \citep[][see also in Section~\ref{subsubsec:radiative}]{Robitaille:2011fc} to develop our own capability of calculating SEDs and understand the sensitivity of these parameters on the SED shape of our sources. Based on our investigation, the degeneracy between viewing angle and multiple geometrical parameters is considerable. The sensitivity of our SED to disk properties is small, as most of our YSOs are younger objects with significant envelopes. Since no central star is visible, parameters describing the central source such as the mass, radius and temperature are primarily important when they are combined into one single term, which is the central luminosity. Similarly, the luminosity created by accretion onto the central object can not be distinguished from a more luminous central object and a non-accreting disk. Finally, we find that there is very little difference between Ulrich envelope models \citep{Ulrich:1976ho} and standard power-law envelopes (see for example Fig.~14 from \citet{Whitney:2013cw}), except that the latter can more easily be related to physical parameters such as the envelope mass. 

From these findings, we created a simplified grid of models by significantly reducing the number of parameters in Hyperion. Table~\ref{tab:SEDModelGrid} describes most of the geometric and physical parameters that are available in Hyperion: divided into the central source, the disk, the envelope and the bipolar cavity. We set most parameters to constants which we determined as average values using literature examples as well as our own investigations for the objects we try to study, which are primarily class 0 and I YSOs. The parameters which we varied in the fits are shown at the bottom of the table: the inclination angle, the central luminosity (irrespective of whether it is caused by the central star or by accretion), the envelope mass, the external extinction and a scaling factor (explained below). The only two physical parameters that we vary are the luminosity and the envelope mass. While others \citep[e.g.][]{Furlan:2016df} have also attempted to reduce the number of parameters for their fitting, they still include more parameters such as the disk radius, but generally conclude that they are not able to properly constrain all of their parameters. As will be discussed later, there are a few YSOs which are
not well fitted with are heavily reduced set of fitting parameters.

It is an important point to emphasize that we are not know the values of these "fixed" model parameters but rather that fixing them to a typical literature value does not have major impact on the fitted parameters and hence the SED fit. Disk mass are radius are two examples; in the presence of
an envelope of comparable or greater mass, the disk emission is a weak function of mass in the
wavelengths (<20 $\mu$m) where it contributes significantly to the SED because it is optically thick at those wavelengths. Similarly, the outer radius of the disk controls its contributions are longer
wavelengths (>50 $\mu$m) where the envelope usually emits effectively; significantly reducing the
disk emission at longer wavelengths requires small disk outer radii (10-30 AU) but has little impact on its shorter wavelength emission.

%LGM Add reference for use of multiple dust types.
Unlike most authors, who use multiple kinds of dust models for different regions of the SED which add complexity and number of parameters \citep{Whitney:2003kc,Robitaille:2006cb,Whitney:2013cw}, we choose to use the same dust model (OH5) for both the envelope and the disk, and assume a 1:100 dust-to-gas ratio. By doing so, we tend to overestimate the short-wavelength emission from SEDs, because the OH5 model assumes isotropic scattering whereas most dust grains appear to be forward-scattering \citep{Draine:2011tr}.


\renewcommand{\arraystretch}{1.5}
\begin{table}[!h]
\scriptsize
\caption[SED model grid]{SED model grid.}
\label{tab:SEDModelGrid}
\vspace{-0.5cm}
\begin{longtable}{lP{5cm}P{3cm}P{2cm}}
\toprule																			
Parameter	&	Description	&	Values	&	Units	\\
\midrule							
\midrule							
\multicolumn{4}{c}{\textbf{Constant parameters}}							\\
\midrule							
\multicolumn{4}{c}{Central source}							\\
\Mstar	&	Stellar mass	&	1	&	\si{\Msun}	\\
\Tstar	&	Stellar temperature	&	4000	&	K	\\
\midrule							
\multicolumn{4}{c}{Disk}							\\
Type	&	Flared or alpha disk	&	Flared	&		\\
\Mdisk	&	Disk mass	&	0.001	&	\si{\Msun}	\\
\Rdiskmax	&	Disk outer radius	&	100	&	\si{\au}	\\
\Rdiskmin	&	Disk inner radius	&	 sublimation radius	&	\si{\au}	\\
$\beta$	&	Flaring parameter	&	1.25	&		\\
$p$	&	Disk surface density exponent	&	-1	&		\\
$r_0$	&	Reference distance for scale height	&	\Rdiskmin	&	\si{\au}	\\
$h_0$	&	Disk scale height at $r_0$	&	0.01$\times$\Rdiskmin	&	\si{\au}	\\
$d$	&	Dust	&	OH5	&		\\
\midrule							
\multicolumn{4}{c}{Envelope}							\\
Type	&	Power-law or Ulrich	&	Power-law	&		\\
\Renvmin	&	Envelope inner radius	&	\Rdiskmin	&	\si{\au}	\\
\Renvmax	&	Envelope outer radius	&	5000	&	\si{\au}	\\
$\alpha$	&	Power	&	-1.5	&		\\
$r^\textrm{env}_0$	&	Reference radius	&	\Renvmin	&	\si{\au}	\\
$d$	&	Dust	&	OH5	&		\\
\midrule							
\multicolumn{4}{c}{Cavity}							\\
$r^\textrm{cav}_0$	&	Cavity outer radius	& 	\Renvmax	&	\si{\au}	\\
$\theta_0$	&	Opening angle at $r^\textrm{cav}_0$	&	10	&	degrees	\\
	&	Flaring exponent	&	1.5	&		\\
$\rho_0$	&	Density at $r^\textrm{cav}_0$	&	0	&	\si{\gram\per\centi\meter}	\\
$\alpha_e$	&	Density profile exponent	&	0	&		\\
\midrule							
\midrule							
\multicolumn{4}{c}{\textbf{Fitted parameters}}							\\
\midrule							
$i$	&	Inclination angle	&	0 to 90 in 10 constant increments of $\cos i$	&	degrees	\\
\Lstar	&	Central luminosity	&	$5\times 1.5^p$ for $p=-4, -3, \dots, 10$ (from 0.99 to 288)	&	\si{\Lsun}	\\
\Menv	&	Envelope mass	&	$0.01\times 1.5^p$ for $p=-2, -1, \dots, 19$ (from 0.001 to 22.17)	&	\si{\Msun}	\\
\Av	&	External extinction	&	$0, 1, \dots, 14$	&	mag	\\
$s$	&	Scaling	&	0.7, 0.85, 1, 1.5, 1.3	&		\\
\bottomrule					
	\end{longtable} 
\end{table}

To facilitate the calculation of models, we constructed a wrapper program that can run the Hyperion software for the range of parameters given in Table~\ref{tab:SEDModelGrid} to create our model grid.

 Because of time and resource limitations, a moderate number of photons was chosen in the Monte Carlo calculation, which can increase the noise at short wavelengths. The details of our modeling parameters, which will be familiar to the Hyperion user, are described in Table~\ref{tab:HyperionParams}. Note that models of more than \SI{1}{\Msun} are actually run with more photons (\num{1e6} instead of \num{2e5}) for imaging, in order to obtain acceptable \SNR at short wavelengths.

\renewcommand{\arraystretch}{1.5}
\begin{table}[!h]
\scriptsize
\caption[Hyperion simulation parameters]{Hyperion simulation parameters.}
\label{tab:HyperionParams}
\vspace{-0.5cm}
\begin{longtable}{lP{4cm}}
\toprule																			
Number of photons (initial)	&	\num{2e5}	\\
Number of photons (imaging)	&	\num{2e5}	\\
Number of photons (raytracing sources)	&	\num{1e6}	\\
Number of photons (raytracing dust)	&	\num{1e6}	\\
Lucy max iterations	&	6	\\
Max photon interactions	&	\num{1e5}	\\
Geometrical grid parameters (radial, theta and azimuthal)	&	400, 199, 2	\\
MRW	&	True	\\
\bottomrule					
	\end{longtable} 
\end{table}

The grid is composed of $\sim 418$ models which are calcuated with Hyperion. For models with $\Menv>\SI{0.5}{\Msun}$, we interpolate the grid in mass by increments of 20\%, which allows for a finer sampling at higher masses, but increases the number of individual models to $958$. The interpolation is done at constant luminosity. Each model is sampled at 10 inclinations, 15 values for external extinction, and five different scaling factors, for a total of \num{718500} grid models. Each model is evaluated at all relevant observing bands, from the 2MASS bands all the way to the \SI{1.3}{\milli\meter} SMA bands. Given the sparsity of the grid, and the relatively simple model used, we do not apply color correction to the fluxes, nor do we convolve the model fluxes with the band transmission function: the resulting corrections fall within our approximations, and do not affect significantly the outcome of the fitting.

The scaling factor in our fits is used to represent the uncertainty in the distance determination \citep[e.g.][]{Robitaille:2006cb}, but it can also be considered as a scaling to represent modestly different luminosities from the model value \citep{Furlan:2016df}. Indeed, \citet{Furlan:2016df} show that, to first order, changing luminosity by a small amount is approximately equivalent to scaling the SED in flux. In their grid, they use a scaling factor that ranges from 0.5 to 2.0, which allows them to have factors of 2.0 in the luminosity of their calculated models. We choose a more conservative approach  by actually running the grid at closer luminosity steps (factor of 1.5) and hence have a smaller range of scaling factors. 

The extinction parameter is used to represent extinction by material along the line of sight that is \textit{outside} of the core: foreground material which may be extended material within the cluster core or may be additional cloud along the line of sight. This parameter is essential to providing good fits, as most other authors have also found \citep[e.g.][]{Robitaille:2006cb,Furlan:2016df}. A discussion of the meaning and importance of this parameter is given in the following sections.

\subsection{Fitting method}
\label{subsec:fittingmethod}

In order to determine which model fits the data best, we adopt a metric defined by \citet{Fischer:2012dj,Furlan:2016df}:

\begin{equation}
R = \frac{1}{N}\sum_i w_i|\log[\Fobs(\lambda_i)] - \log[\Fmod(\lambda_i)]|,
\end{equation}
where $i$ are the indices of the valid data points, the weights $w_i$ correspond to the inverse of the fractional uncertainty of each measurement, \Fobs and \Fmod are the observed and model fluxes respectively, and $N$ is the number of valid measurements. For our models, we set the fractional uncertainty to a minimum of 10\%, to avoid having a few points completely drive the fit. Early versions of the fitting routines, which used the published $1\sigma$ uncertainties would completely skew the results by putting all the weight into a few flux measurements. This was most notable for the \Spitzer IRAC points, for which published uncertainties are sometimes only have a few percent. We chose to override these uncertainties, in large part because the assumptions of geometry and dust properties that go into a model calculation do not justify that level of confidence in the model output.

\citet{Furlan:2016df} discuss in more detail the meaning of this $R$ metric, which differs from a standard $\chi^2$ metric such as the one used by \citet{Robitaille:2007dl}. $R$ represents a weighted average of the logarithmic deviations between the observations and the model. It is important to note that, although it is normalized, it does not have a statistical interpretation like the standard $\chi^2$ metric. In particular, models with fewer data points or large measurement uncertainties will tend to have smaller values of $R$, even if the fit is poor. $R$ is only useful as a relative measure of the goodness of fit to the specific observations.

For each source, we calculate $R$ for each model in our grid. The model with the smallest value for $R$ is the best-fitting model by this metric, but given our sparse sampling and the errors of our observations, this is not necessarily the most likely model to best fit the data. We can consider two extremes: in the first case, the best fit has a value of $R$ which is much lower than other models. Then, it is clearly the best fit. In the second case, let's suppose that the \num{1000} best-fitting models lie very close to the best $R$. In this case, concluding that the model that best fits our observations (and from which will interpret physical quantities) is the one with the minimum $R$ is too strict and does not account for the uncertainties that are present. 

In practice, most of our models fall in the second case. After visual inspection of the fits, we conclude there is very little significant difference between values of $R$ which are separated by $\sim 0.5$. They all can be considered equally good (or bad) fits. Hence, for a robust measure of the best-fitting model parameters, we choose the mode (the most likely value) of the parameters from models which are within $R_\textrm{min}$ and $R_\textrm{min}+0.2$, in order to really pick the best possible fits. The error on the parameter is then estimated using the models within $R_\textrm{min}$ and $R_\textrm{min}+0.5$, since these models all similarly fit, and is described in the next section.

Because we use exclusively the OH5 dust model, which we know overestimates the short-wavelength fluxes, we expect to overestimate the extinction required to match the observations. For this reason, we choose to ignore the 2MASS J and H band data points, which drive the extinction values up dramatically and sometimes leads the fit towards non realistic solutions. However, we choose to keep the \SI{2.2}{\um} Ks Band data point to give some weight to the shorter wavelength data.

\subsection{Overview of derived parameters}

The distribution of the best fit solutions of the envelope mass and central luminosity is shown in Fig.~\ref{fig:MassLumHist} for the clusters for which we have long-wavelength data (Ophiuchus and NGC1333). Our sample covers a broad range of masses, but is naturally biased towards high luminosities given SOFIA's instrumental sensitivity and our cluster selection.


\begin{figure}[!h]
\begin{center}
\includegraphics[width=\textwidth]{Figures/MassLumHist.pdf}
\vspace{-1cm}
\caption[Fitted envelope mass and luminosity distribution]{Fitted envelope mass and luminosity distribution for all observed point sources in Ophiuchus and NGC~1333, where long-wavelength data is available.}
\label{fig:MassLumHist}
\end{center}
\end{figure}

From visual inspection, data with $R$ less or close to 1 appear to fit the data well. Larger $R$ show less good fits. The distribution of $R$ for all the isolated point sources is shown in Fig.~\ref{fig:Rdistr}. Note that targets where less data points are available, or where data points are more noisy, usually have lower $R$ than targets with a lot of available data points, even if the fits are not necessarily as good. This has also been observed by \citep{Furlan:2016df} and is one of the drawbacks of using the $R$ metric.

\begin{figure}[!h]
\begin{center}
\includegraphics[width=\textwidth]{Figures/Rdistr.pdf}
\vspace{-1cm}
\caption[Distribution of $R$]{$R$ distribution across all observed point sources in Ophiuchus and NGC1333, where long-wavelength data is available.}
\label{fig:Rdistr}
\end{center}
\end{figure}

%For our sample, we can compare the fitted central luminosity, \Ltot, with the integrated luminosity from the datapoints, \Lbol for our entire sample of point sources (see Fig.~\ref{fig:LbolVsLest}). This shows relatively good agreement, although a systematic excess in fitted central luminosity can be observed, which we attribute to the widespread choice of using an external extinction coefficient. By using this external extinction as a model parameter, we artificially reduce the emission at short wavelengths, which would tend to decrease the bolometric luminosity.
%\begin{figure}[!h]
%\begin{center}
%\includegraphics[width=\textwidth]{Figures/LbolVsLest.pdf}
%\vspace{-1cm}
%\caption[Estimated luminosity vs bolometric luminosity]{Estimated luminosity vs bolometric luminosity. The best fit line is shown in red, along with 95\% confidence intervals. The grey dashed line represents $\Ltot = \Lbol$. The excess modeled luminosity for smaller luminosities is caused by the external extinction, which absorbs a large fraction of the luminosity emitted by the central object but does not re-radiate it at longer wavelengths - this is one of the limitations of this exercise.}
%\label{fig:LbolVsLest}
%\end{center}
%\end{figure}
%
%The luminosity excess is more pronounced for lower masses, as the short wavelength emission represents a larger portion of the total emission from the source (Fig.~\ref{fig:LbolMinusLestVSMass}). 
%
%\begin{figure}[!h]
%\begin{center}
%\includegraphics[width=\textwidth]{Figures/LbolMinusLestVSMass.pdf}
%\vspace{-1cm}
%\caption[Luminosity excess as a function of envelope mass]{Luminosity excess as a function of envelope mass.}
%\label{fig:LbolMinusLestVSMass}
%\end{center}
%\end{figure}

%LGM is the rewrite below what you meant?
One major limitation and inconsistency in all broad fitting works to date is the inclusion of a foreground material which provide extinction without emission. For example, \citet{Furlan:2016df} fit for external extinction up to $\Av = 40$ for some of their sources, and use all of the 2MASS bands in their fitting. It is not consistent to assume that so much material is present along the line of sight, while not also being observed at longer wavelengths. Since the dust is optically thin at longer wavelengths, the far-infrared and submillimeter observations should see emission from this material which is obscuring the shortest wavelengths.

Our exploration with the fitting routine shows that limiting the external extinction forces more inclined geometries, where the light from the central star passes through the disk before reaching us. However, we were not able to account for the entirety of the short wavelength extinction by doing this, as the mid-infrared wavelength (IRAC and FORCAST bands) are also affected by more inclined geometries, which can compromise the fits. This could indicate a fundamental limitation in our geometrical representation of YSOs or assumed dust properties, since there is no possible way to account for both the low amount of far-IR emission seen, e.g. by \Herschel and high amount of extinction seen from the 2MASS and IRAC bands. The highly cited publications, that we have referred to, adopt an external extinction factor, the rigor of which we now strongly question. However, we have not been able to find an appropriate solution to circumvent this issue.  Hence, we choose to follow the examples of previous authors and adopt an external extinction factor. Unlike \citet{Furlan:2016df}, which consider \Av as high as 40~mag, we choose to limit it to \Av = 14~mag, a moderate value of the diffuse extinction in our clusters of interest. 

An external extinction of \Av = 14~mag means that fluxes at \SI{2.2}{\um} are reduced by a factor of $\sim 17$, while \SI{8}{\um} fluxes are reduced only by a factor of $\sim 2$. For most Class 0 and I sources, most of the emission has already been reprocessed by dust out to longer wavelengths, and the 2MASS \SI{2.2}{\um} data points are usually extremely low, so a difference of a factor of 17 in this small region of the spectrum will not lead to significantly different luminosity estimates when compared to the contribution from other parts of the spectrum. In fact, a small exploration of our fits reveals that for these sources usually the extinction from within the envelope itself is already much larger than this factor. However, at $\Av = 40$, the flux reduction at \SI{2.2}{\um} is $>3000$, at which point we argue this could become a problem. Which such a large ratio, it is more difficult to claim that the fitted luminosity from the model is not overestimating the actual luminosity, since the external extinction reduces by many orders of magnitude the short-wavelength emission in order to fit the data. 

%LGM Why is this every true since you are using external extinction to mask some of the luminosity.
%   Think more about this paragraph.
For all sources we calculate the bolometric luminosity as the integral of all the available data points, even those which correspond to upper limits. This makes \Lbol an upper limit as well on the observed luminosity. Since most of the upper limits are from long-wavelength data points, this impacts sources with a larger envelope more. We note that the fitted luminosity \Lmod tends to be lower than the bolometric luminosity \Lbol for low inclination angles, but \Lmod tends to be higher than \Lbol for more inclined geometries. This is expected since for high inclinations a large fraction of the emission is not directed towards the observer \citep[see, e.g.][for a discussion]{Furlan:2016df}. On the contrary, when seen almost face-on, the observer sees both the emission from the source as well as the light scattered on the walls of the cavity and the disk.
%LGM see above comment. The last sentence is true if the external extinction is low. Is that systematically true?


%\begin{figure}[!h]
%\begin{center}
%\includegraphics[width=\textwidth]{Figures/massEstvsCalc.pdf}
%\vspace{-1cm}
%\caption[Estimated mass vs calculated mass]{Estimated mass vs calculated mass for the point sources which have sub-millimeter data.}
%\label{fig:massEstvsCalc}
%\end{center}
%\end{figure}

%For the clusters which do have submillimeter data points, we can use the traditional mass estimate described in Section~\ref{subsec:DustExtinction} with the \SI{1.1}{\milli\meter} or \SI{1.3}{\milli\meter} fluxes. For this calculation, we use an effective dust temperature of \SI{20}{\kelvin}, assuming an opacity of \SI{0.0114}{\raiseto{2}\centi\meter\per\gram} and \SI{0.009}{\raiseto{2}\centi\meter\per\gram} for \SI{1.1}{\milli\meter} and \SI{1.3}{\milli\meter}, respectively. Note that this measurement is very sensitive on these assumptions; for example, lowering the dust temperature estimate to \SI{10}{\kelvin} increases the mass estimate by a factor of 3.

%We find that our fitted mass estimates agree with this traditional method of deriving masses, as shown in Fig~\ref{fig:massEstvsCalc}. For lower masses, however, our fits tend to underestimate the mass compared to the derived quantities. This could be explained since most of the lower-mass envelopes belong to more evolved objects, which can thus have higher dust temperatures due to less opacity: the derived mass, assuming a temperature of \SI{20}{\kelvin}, would then overestimate the amount of material in the envelope.



%There are few caveats in determining the luminosity of these sources. First,  Second, the best-fitted extinction \Av is large for most of our fits. In fact, it is usually right at the maximum value we sample of 14. This means that the luminosity of the source, \Lmod, has to be dramatically extincted, which can lead to high model luminosities. Whether these model luminosities are correct or not depends on exactly where the shorter wavelengths are being absorbed and whether the heating from the absorption of this energy is being properly accounted for in the model.

Finally, for the clusters which have submillimeter data points, we calculate the traditional mass estimate described in Section~\ref{subsec:DustExtinction} using the \SI{1.1}{\milli\meter} or \SI{1.3}{\milli\meter} fluxes. For this calculation, we use an effective dust temperature of \SI{20}{\kelvin}, assuming an opacity of \SI{0.0114}{\raiseto{2}\centi\meter\per\gram} and \SI{0.009}{\raiseto{2}\centi\meter\per\gram} for \SI{1.1}{\milli\meter} and \SI{1.3}{\milli\meter}, respectively, based on the
expected emissivity for OH5 dust. These values for \Menv are shown in Table~\ref{tab:FittedParameters}.
Note that this measurement is very sensitive to these assumptions; for example, lowering the dust temperature estimate to \SI{10}{\kelvin} increases the mass estimate by a factor of 3. In addition, these measurements can be overestimates by a large amount if the 1.1 and \SI{1.3}{\milli\meter} fluxes are
measured with single-dishes, and hence upper limits on the flux from the YSO environment.

A summary of our fit results for Ophiuchus and NGC1333 is shown in Table~\ref{tab:FittedParameters}. All fits results for all our sources are shown in Appendix~\ref{ap:data}. Note that the luminosity that is used in this analysis is always the luminosity multiplied by the scaling factor $s$, under the assumption that the SED scales for small changes in luminosity. This scaling factor improved the fit but it must be remembered that the true luminosity is dependent on the distance. For example, a distance error of 10\% would cause a luminosity estimate that would differ by 20\%.


\begin{landscape}
\begin{table}[!h]
\scriptsize
\caption[Fitted parameters]{Fitted parameters for the point sources in Ophiuchus and NGC1333 where long-wavelength photometry is available.}
\label{tab:FittedParameters}
\vspace{-0.5cm}
\hspace*{-1.5cm}
\begin{center}
\begin{longtable}{lcccccccccc}

\toprule																									
SOFIA Name	&	Coordinates	&	R37	&	$\alpha$	&	$R$	&	\Menv			&	Calc. \Menv	&	\Ltot			&	$\Lbol$	&	i	&	$\Av$	\\
	&	(J2000)	&		&		&		&	(\si{\Msun})			&	(\si{\Msun})	&	(\si{\Lsun})			&	(\si{\Lsun})	&	(\si{\degree})	&	(mag)	\\
\midrule																									
NGC1333.1	&	03h29m08s +31d21m57s	&	0.75	&	0.28	&	3.40	&	0.004	$\pm$	0.005	&	0.97	&	32.5	$\pm$	7.8	&	8.4	&	51	&	14	\\
NGC1333.3	&	03h29m02s +31d20m21s	&	0.90	&	0.71	&	3.39	&	0.004	$\pm$	0.03	&	1.12	&	3.5	$\pm$	2.1	&	8.1	&	0	&	14	\\
NGC1333.4	&	03h29m11s +31d18m31s	&	1.10	&	1.86	&	0.83	&	2.919	$\pm$	0.45	&	1.50	&	2.3	$\pm$	0.4	&	3.1	&	19	&	11	\\
NGC1333.5	&	03h29m11s +31d18m20s	&	1.62	&	1.70	&	0.77	&	1.297	$\pm$	0.33	&	1.50	&	1.3	$\pm$	0.3	&	2.8	&	19	&	14	\\
NGC1333.6	&	03h29m13s +31d18m14s	&	0.95	&	1.00	&	1.21	&	0.001	$\pm$	0.0007	&	0.47	&	7.5	$\pm$	1.2	&	1.5	&	27	&	14	\\
NGC1333.7	&	03h28m43s +31d17m35s	&	1.19	&	1.08	&	1.83	&	0.001	$\pm$	0.001	&	--	&	9.6	$\pm$	1.8	&	1.4	&	58	&	0	\\
NGC1333.8	&	03h29m04s +31d16m04s	&	0.77	&	1.14	&	1.06	&	1.946	$\pm$	0.75	&	2.02	&	17.0	$\pm$	2.4	&	35.1	&	0	&	13	\\
NGC1333.9	&	03h28m56s +31d14m37s	&	0.80	&	2.79	&	2.62	&	2.919	$\pm$	0.35	&	1.72	&	17.0	$\pm$	2.4	&	24.3	&	19	&	14	\\
NGC1333.10	&	03h28m57s +31d14m15s	&	0.80	&	1.83	&	1.16	&	0.256	$\pm$	0.18	&	0.45	&	5.6	$\pm$	0.9	&	4.8	&	19	&	14	\\
NGC1333.11	&	03h28m37s +31d13m30s	&	1.02	&	1.69	&	0.99	&	0.38	$\pm$	0.18	&	0.27	&	7.7	$\pm$	0.8	&	7.5	&	19	&	14	\\
Oph.1	&	16h27m10s -24d19m13s	&	0.92	&	0.27	&	0.67	&	0.01	$\pm$	0.002	&	0.04	&	7.9	$\pm$	1.3	&	3.6	&	78	&	3	\\
Oph.2	&	16h26m44s -24d34m48s	&	0.93	&	0.83	&	2.08	&	0.001	$\pm$	0.002	&	0.05	&	32.3	$\pm$	21.1	&	1.2	&	84	&	14	\\
Oph.3	&	16h27m09s -24d37m18s	&	0.99	&	0.57	&	1.54	&	0.004	$\pm$	0.002	&	0.04	&	85.0	$\pm$	19.7	&	13.4	&	0	&	14	\\
Oph.5	&	16h27m07s -24d38m15s	&	1.31	&	0.31	&	1.36	&	0.001	$\pm$	0	&	0.03	&	4.3	$\pm$	0.5	&	0.5	&	81	&	14	\\
Oph.6	&	16h27m16s -24d38m46s	&	1.29	&	2.54	&	0.93	&	0.001	$\pm$	0.001	&	0.02	&	26.6	$\pm$	6.4	&	0.8	&	90	&	13	\\
Oph.7	&	16h27m28s -24d39m34s	&	0.97	&	1.35	&	1.39	&	0.015	$\pm$	0.002	&	0.03	&	26.6	$\pm$	3.5	&	6.5	&	72	&	14	\\
Oph.8	&	16h27m37s -24d30m35s	&	1.02	&	0.55	&	1.13	&	0.007	$\pm$	0.002	&	0.03	&	17.7	$\pm$	3.4	&	5.0	&	78	&	12	\\
Oph.9	&	16h27m22s -24d29m54s	&	--	&	0.51	&	2.08	&	0.001	$\pm$	0	&	0.01	&	11.8	$\pm$	1.2	&	1.0	&	81	&	14	\\
Oph.10	&	16h27m18s -24d28m55s	&	1.26	&	0.50	&	1.38	&	0.003	$\pm$	0.0006	&	0.004	&	5.0	$\pm$	1.6	&	0.6	&	81	&	14	\\
Oph.13	&	16h27m30s -24d27m43s	&	0.00	&	-0.37	&	2.23	&	0.001	$\pm$	0	&	0.01	&	17.7	$\pm$	5.5	&	1.5	&	81	&	14	\\
Oph.14	&	16h27m28s -24d27m21s	&	1.89	&	-0.13	&	1.00	&	0.001	$\pm$	0.0009	&	0.02	&	4.3	$\pm$	0.6	&	1.0	&	81	&	14	\\
Oph.15	&	16h27m29s -24d39m17s	&	1.25	&	0.10	&	1.12	&	0.004	$\pm$	0.0007	&	0.02	&	3.3	$\pm$	0.4	&	0.6	&	27	&	14	\\
Oph.16	&	16h26m24s -24d24m48s	&	1.80	&	-0.76	&	1.87	&	0.001	$\pm$	0	&	--	&	17.7	$\pm$	2.9	&	2.2	&	78	&	10	\\
Oph.17	&	16h26m24s -24d24m39s	&	0.96	&	-0.09	&	1.21	&	0.001	$\pm$	0	&	--	&	5.3	$\pm$	0.6	&	1.3	&	81	&	14	\\
Oph.18	&	16h26m17s -24d23m45s	&	1.18	&	0.63	&	1.34	&	0.003	$\pm$	0.004	&	0.04	&	2.8	$\pm$	0.9	&	0.3	&	81	&	14	\\
Oph.19	&	16h26m30s -24d23m00s	&	2.51	&	0.57	&	0.89	&	0.001	$\pm$	0.001	&	0.01	&	5.3	$\pm$	1.0	&	1.2	&	75	&	14	\\
\bottomrule																									
\end{longtable}																																	
%\caption*{\textbf{Note:} The complete version of this table is made available electronically}			
\end{center}																						
\end{table}	
\end{landscape}			


%\begin{figure}[!h]
%\begin{center}
%\includegraphics[width=\textwidth]{Figures/massVSalpha.pdf}
%\vspace{-1cm}
%\caption[Mass versus spectral index]{Mass versus spectral index for the point sources which have sub-millimeter data, and for which the fits have $R<2$.}
%\label{fig:massVSalpha}
%\end{center}
%\end{figure}
%
%By examining the envelope mass fitted values with respect to the measured spectral index $\alpha$, we note that there is a correlation (Fig.~\ref{fig:massVSalpha}). This correlation appears to be good for $\alpha <1$, but much more loose for $\alpha>1$.
%Of the notable relationship that can be seen among our parameters, we can see for example the envelope mass being correlated with the spectral index (see Fig.~\ref{fig:massVSalpha}) - this indicates a possibility that the SOFIA data points might become a tool to predict the envelope mass, once a sufficient amount of statistics have been gathered (to lower the scatter in that figure). 


\subsection{Estimating parameter uncertainty}

It is important to estimate the uncertainty in fitted parameters
 to quantify the confidence in a given fit. Without uncertainties, no meaningful conclusion can be drawn about the physical meaning of the fits. This estimation is also one of the most difficult aspect of the fitting process, since it really depends on the method used and the modeling strategy. It is also difficult to compare results with the findings of other authors who used a different approach to their fitting. 

In this work, we propose a novel methodology to derive the uncertainty on the best fit. First, we determine the best fit for a given parameter as the mode of the parameter values from the models that fit within $[\Rmin,\Rmin+0.2]$, where \Rmin is the minimum value of $R$ in the entire grid. This is statistically more robust than picking simply the model with the lower $R$, since, given our uncertainties and approximations, there is no statistically-significant difference between models that fit within that range.

Once this best fit value is determined for all parameters, the uncertainty is determined using all models that fit within $[\Rmin,\Rmin+0.5]$. We determine three quantities from these models: the standard deviation from the best fit; the median absolute deviation from the best fit; and the skewness of the distribution. %All of these values are given the data table.
%LGM Are they in the data table in the thesis. They should be. In a number of places, you have made reference to releasing infomation in some data or fit tables to be releases but the thesis should be self contained unless referring to already published work. I have deleted these references in places where I saw them.

The choice of the $R$ intervals are empirical based on our fitting experience. Since the metric $R$ is not model-dependent but instead is a distance of \textit{distance} between models and observations, we think that similar values will still lead to reasonable parameter and uncertainty estimates in other future works. One limitation could occur from the density of the grid: if the models are so sparse that there are only a handful of model within each interval used in the uncertainty estimation, this could lead to poor estimate of uncertainties. 


%Another important consideration is the cross-correlation between parameters. This is a phenomenon that we encountered heavily using the \textit{sedfitter} function from \citet{Robitaille:2006cb}, which attempts to fit 14-parameter models (plus extinction and scaling factor). This cross-correlation is apparent when the same observations can be fitted with models with wildly different sets of parameters, indicative of multiple local minimas in the grid.

%By limiting the number of parameters in the model, we dramatically reduce this effect. However, some cross-correlations...



\subsection{Discussion}

Several factors have been omitted for simplicity our model fitting. First, the models we use have an axisymmetric geometry which is unlikely to account for realistic mass distributions in the envelope and the disk. Second, we ignore the surrounding medium and consider it devoid of emission. In reality, the transition to the surrounding medium is likely smooth and its emission relevant at the longest wavelengths. Third, we assume that the only heating source is located at the center of the YSO. The heating source consists of both the light from the star, and from the accretion luminosity, which can not be distinguished from our point of view. It is important to realize that external heating can also play a role in raising the dust temperature and changing the SED signature in the cluster environment. The impact of the interstellar heating is explored in \citet{Furlan:2016df}, who show that it can have a substantial effect on the SED - but they nevertheless do not include this parameter in their grid, since it is too case-specific. The Hyperion radiative transfer code that is used to model our grid could accommodate external radiation fields as well, and this could become a future addition to our modeling. Finally, the observations that form our SEDs were not taken simultaneously, so it is possible for the YSO flux to change over the period of years. This YSO variability has been shown to be fairly common at the
10 to 20\% level at near infrared wavelengths \citep{Rebull:2014iw} and larger optical outbursts in luminosity are known to occur in a small sub-class of T Tauri stars called FU Ori stars \citep{Hartmann:1996gd}. However, we do not anticipate that YSO variability would drastically change the fit results, given typical variability amplitudes modest \citep[e.g.][]{Poppenhaeger:2015hm}.

Given the relative simplicity of our model grid, most of observations are fit reasonably well and the fitted parameters have acceptable uncertainties for a large fraction of sources. Our range of $R$ values is similar to that of \citet{Furlan:2016df} in their analysis, although they used more free parameters than just the luminosity and the envelope mass. 
%LGM The mention of 330 Oph source here will naturally lead to the question of whether he did your sources and how his answer compared to yours....
This further confirms the degeneracies that exist when trying to fit for too much physics into very elaborate models. The difference in the resolutions, sensitivity, and photometric techniques for each wavelength in the SED limits the value of a more thorough analysis, especially when located in very clustered environment when extended emission and nearby sources can contaminate the measurements. 

We argue that more complex models would not help in estimating the physical parameters of YSOs - but instead, this work highlights the need for higher angular resolution at wavelengths longward of \SI{37}{\um}. Such data can be obtained in the future at arcsecond and sub-arcsecond resolution at millimeter and submillimeter wavelengths with ALMA at $>\SI{350}{\um}$. The new continuum cameras for large radio telescopes like the Green  Bank 100-meter telescope and the 50-meter Large Millimeter Telescope can produce $\sim$5 arcsecond images of the extended envelope and surrounding cloud material to provide strong constraints on the external extinction issue. It is also essential to improve the resolution of observations from 30 to 200 $\mu$m to constrain the disk and envelope masses, and refine our knowledge of dust properties in the different regions of the circumstellar environment.
%Sources with only data up to \SI{37}{\um} are likely to have poorly constrained masses, suggesting that SEDs could have the same near- to mid-IR response while having substantially different long wavelength response (see Section~\ref{sec:IRAS}). 

In our models, we have used OH5 dust as it was recommended by various authors \citep[e.g.][]{Dunham:2010bx}. However, the models for OH5 do not include scattering properties, which might jeopardize the accuracy of the models at short wavelengths, which are dominated by scattered light. However, it seems to have a opacity that fits best the clustered environments (Huard et al., 2016, in prep.). By ignoring the fact that grains are preferentially forward-scattering,  we could cause the model to fit for larger extinction values or higher inclination angles than desired. We are considering using other types of dust, such as the one used by \citet{Furlan:2016df}, which has fully detailed scattering properties, but less long-wavelength opacity. Using this type of dust would require running a new model grid with considerably more photons for each model, which we estimate would take $\sim 4$ weeks on the UMD 8-core computer we have been using. 

Finally, the issue of external extinction needs much more investigation. To date, we have not found in the literature a proper treatment of this problem. In order to re-establish self-consistency, we suggest exploring ways to add constraints on the long-wavelength flux when adding more extinction. Since the material that causes truely external extinction is presumably far away from the source, we can assume that is it at the temperature of the surrounding molecular cloud. The extinction \Av can be converted to a column density of material by assuming a dust composition and knowing its opacity at short wavelengths. By knowing the amount of material along the line of sight and assuming a temperature, we could infer how much emission is expected at long wavelengths and test if it is consistent with \Herschel mesurements along the line of sight. A second approach is to add the flux from the extinction material to the long wavelength emission from the model as part of the fitting process. This suggestion has not yet been tested or implemented.


%\begin{figure}[!h]
%\begin{center}
%\includegraphics[width=\textwidth]{Figures/incVSmass.pdf}
%\vspace{-1cm}
%\caption[Inclination versus mass]{The correlation between inclination angle and envelope mass might indicate a degeneracy of the modeling software. [I AM GOING TO DELETE THIS PICTURE]}
%\label{fig:incVSmass}
%\end{center}
%\end{figure}
%
%
%
%[I AM GOING TO DELETE THESE TWO PARAGRAPHS, AS THEY DON'T TELL MUCH STORY] Similarly to \citet{Furlan:2016df}, we find that the distribution of inclination angles for the best fits is not uniform, which is not intuitive. There is no reason why protostars should have a selection effect in their inclination angle with respect to us. This is indicative of an artifact of the fitting process, and possibly a degeneracy between inclination angle and envelope mass (see Fig.~\ref{fig:incVSmass}), which is much more prominent when no long-wavelength data is available.
%
%Finally, we observe that there is no statistically significant relationship between the spectral index and the total luminosity of the object. This is perhaps not too surprising, as the total luminosity of the object is not expected to significantly change with its evolutionary stage. In addition, this also points out that there are no significant cross-correlation between  the luminosity and envelope mass in our model. 







%\begin{figure}[!h]
%\begin{center}
%\includegraphics[width=\textwidth]{Figures/slumVSalpha.pdf}
%\label{fig:slumVSalpha}
%\vspace{-1cm}
%\caption[Luminosity versus spectral index.]{There is no apparent correlation between fitted luminosity and spectral index.}
%\end{center}
%\end{figure}

\section{A close look at IRAS~20050+2720}
\label{sec:IRAS}

In this section, we focus our attention on IRAS~20050+2720 which shows very clustered sources that are resolved for the first time in the mid-IR with our SOFIA FORCAST observations. The fields that were observed are shown in Fig.~\ref{fig:NGC2071_IRAS20050_RGB}, superimposed with IRAC 3-color images to provide some context.

\begin{figure}
\begin{center}
\includegraphics[width=0.7\textwidth]{Figures/IRAS20050_RGB.png}

\caption[IRAS~20050+2720]{IRAC 3-color images of IRAS~20050+2720. The two SOFIA fields corresponds approximately to the two white squares in the image.}
\label{fig:NGC2071_IRAS20050_RGB}
\end{center}
\end{figure}




\subsection{Overview}
IRAS~20050+2720 is part of an active site of intermediate-mass star formation in the Cygnus Rift located at \SI{700}{\pc} \citep{Wilking:1989el}, with the particularity that it doesn't seem to contain any massive stars \citep{Gunther:2012dq}. The main cluster core is associated with water and methanol masers \citep{Palla:1991up,Fontani:2010cf} and multipolar molecular outflows observed at millimeter wavelengths \citep{Bachiller:1995cy,Anglada:1998uu,Beltran:2008gu}, suggesting that the region might have experienced a episode of star formation in the past 0.1 Myr which contrasts with the average age of the cluster of 1 Myr \citep{Chen:1997tb,Gutermuth:2005hx}. \cite{Gutermuth:2009gca} have identified $>170$ YSOs surrounding the core and measured their continuum fluxes up to \SI{8}{\micro\meter} with the \Spitzer IRAC instrument. While measurements at longer wavelengths are able to provide estimates of the total luminosity of the cluster \citep[e.g. using IRAS,][\SI{388}{\Lsun}]{Molinari:1996td}, the measurements are confused in the densest region and it has not been possible to properly associate the far-IR emission with its short wavelength counterpart because of the small separation between IRAC-detected protostars. 
%LGM has it been spelled out what IRAS was? Should be somewhere
The IRAS point source was classified as a luminous class 0 protostar \citep{Bachiller:1996ja}, and its emission associated with the bright millimeter source MMS1 to the northwest of the core \citep{Chini:2001fa}, also called OVRO1 in \citet{vanKempen:2012fb}. \citet{Beltran:2008gu} show strong evidence that this region has multiple generations of stars, and suggest that a group of low-mass stars first completed their main accretion phase, before the birth of new intermediate-mass stars at the core of this cluster. A recent study by \citet{Poppenhaeger:2015hm} investigated the YSO variability in the IRAC 3.6 and \SI{4.5}{\um} fluxes in the region, and found that a large fraction exhibit variability, some as large much as 0.55~mag over periods ranging from a few days to $\sim 30$~days. 


\subsection{Observations and discussion}

We observed two fields within the cluster (see Fig.~\ref{fig:NGC2071_IRAS20050_RGB}), including the brightest core at $20^h 07^m 06.70^s +\ang{27;28;54.5}$. Multiple sources in the core can be distinguished in the IRAC maps, but the core appears extended in \Spitzer MIPS at \SI{24}{\micro\meter}, and is identified as a single source with WISE. No high resolution far-infrared continuum data longward of \SI{24}{\micro\meter} was available for this source. To our knowledge, our observations are the only mid-IR observations available that can properly resolve the YSOs in the dense region. 

\subsubsection{A clustered region with an outflow}

\begin{figure}
\begin{center}
\includegraphics[width=\textwidth]{Figures/IRAS20050_core.png}
\caption[IRAS~20050+2720 core]{\SI{37}{\um} observations of the IRAS~20050+2720 core, with the 5 identified objects. The blue contours are from a \SI{2.7}{\milli\meter} continuum emission observed by the OVRO array \citep{Beltran:2008gu} at levels from 10 to \SI{46}{\milli\Jy\per\beam} by increments of \SI{4}{\milli\Jy\per\beam}. The resolution of the \SI{2.7}{\milli\meter} beam is $\sim\ang{;;4.8}$, while the r.m.s noise is \SI{1.5}{\milli\Jy\per\beam}. The dashed line is the axis of a bipolar outflow identified by \citet{Bachiller:1995cy}. The beam shown at the bottom left represents the resolution of the FORCAST instrument.}
\label{fig:IRAS20050_core}
\end{center}
\end{figure}
We find 5 separate sources in the core which appear to share an envelope of extended emission at \SI{37}{\um}. These sources are labeled in Fig.~\ref{fig:IRAS20050_core}, and their IRAC and FORCAST photometry is summarized in Table~\ref{tab:IRAS20050fluxes}. IRAS20050.4 is coincident with the source at the northwestern end of the cluster, which is named OVRO1 in \citet{Beltran:2008gu}. Two more sources are identified with the blue millimeter wavelength continuum contours from \citet{Beltran:2008gu}, to the south and east of OVRO1, but they do not appear to correlate with our SOFIA sources. The CO outflow axis \citep[Outflow "A",][]{Bachiller:1995cy}, which is associated with IRAS20050.4, appears to be aligned with extended emission that is visible to the east of the 5 sources. This extended emission is visible in both IRAC and FORCAST, and coincides with blue-shifted CO emission in the velocity maps from \citet{Beltran:2008gu}. The emission, totalling $\sim\SI{6}{\Jy}$ at \SI{37}{\um}, appears diffuse and not connected to any particular YSO. A multi-wavelength view of the region is shown in Fig.\ref{fig:IRAS20050_mosaic}.

\renewcommand{\arraystretch}{1.5}
\begin{table}
\scriptsize
\caption[Source fluxes in IRAS~20050+2720's dense core]{Sources fluxes in IRAS~20050+2720.}
\label{tab:IRAS20050fluxes}
\vspace{-0.5cm}
\begin{longtable}{lP{2cm}P{0.7cm}P{0.7cm}P{0.7cm}P{0.7cm}P{0.7cm}P{0.7cm}P{0.7cm}P{0.7cm}P{0.7cm}}
\toprule																			
SOFIA name	&	Coordinates	&	ks			&	i1			&	i2			&	i3			&	i4			&	F11			&	F19			&	F31			&	F37			\\
	&	J2000	&	Jy			&	Jy			&	Jy			&	Jy			&	Jy			&	Jy			&	Jy			&	Jy			&	Jy			\\
\midrule																																		
IRAS20050.1	&	20h07m06.6s +27d28m48.0s	&	0.214	$\pm$	0.021	&	0.489	$\pm$	0.049	&	0.57	$\pm$	0.057	&	0.731	$\pm$	0.073	&	0.858	$\pm$	0.086	&	0.64	$\pm$	0.07	&	1.93	$\pm$	0.20	&	4.50	$\pm$	0.35	&	6.32	$\pm$	0.59	\\
IRAS20050.2	&	20h07m06.2s +27d28m49.1s	&	0.002	$\pm$	0.002	&	0.041	$\pm$	0.004	&	0.142	$\pm$	0.014	&	0.264	$\pm$	0.026	&	0.308	$\pm$	0.031	&	0.06	$\pm$	0.06	&	1.45	$\pm$	0.19	&	9.31	$\pm$	0.72	&	11.96	$\pm$	1.19	\\
IRAS20050.3	&	20h07m06.3s +27d28m56.6s	&	0.028	$\pm$	0.003	&	0.09	$\pm$	0.009	&	0.218	$\pm$	0.022	&	0.339	$\pm$	0.034	&	0.429	$\pm$	0.043	&	0.18	$\pm$	0.06	&	2.58	$\pm$	0.27	&	12.53	$\pm$	0.94	&	19.34	$\pm$	1.41	\\
IRAS20050.4	&	20h07m05.9s +27d28m59.2s	&	0.002	$\pm$	0.002	&	0.023	$\pm$	0.003	&	0.039	$\pm$	0.004	&	0.053	$\pm$	0.008	&	0.055	$\pm$	0.008	&	0.06	$\pm$	0.05	&	0.25	$\pm$	0.20	&	8.54	$\pm$	0.80	&	12.85	$\pm$	1.25	\\
IRAS20050.5	&	20h07m06.6s +27d28m53.1s	&	0.042	$\pm$	0.004	&	0.118	$\pm$	0.012	&	0.176	$\pm$	0.018	&	0.235	$\pm$	0.024	&	0.32	$\pm$	0.032	&	0.19	$\pm$	0.05	&	1.03	$\pm$	0.21	&	2.97	$\pm$	0.33	&	5.65	$\pm$	0.65	\\
IRAS20050.6	&	20h07m02.2s +27d30m26.0s	&	0.155	$\pm$	0.016	&	0.537	$\pm$	0.054	&	0.771	$\pm$	0.077	&	1.113	$\pm$	0.111	&	1.805	$\pm$	0.181	&	1.81	$\pm$	0.13	&	2.29	$\pm$	0.17	&	1.64	$\pm$	0.14	&	1.22	$\pm$	0.38	\\
IRAS20050.7	&	20h07m07.9s +27d27m15.8s	&	0.002	$\pm$	0.002	&	0.004	$\pm$	0.004	&	0.024	$\pm$	0.002	&	0.06	$\pm$	0.006	&	0.072	$\pm$	0.007	&	0.06	$\pm$	0.05	&	0.11	$\pm$	0.06	&	1.15	$\pm$	0.14	&	2.09	$\pm$	0.31	\\			
\bottomrule	
	\end{longtable} 
\end{table}

% : this requires a mechanism to keep the dust emitting at these wavelengths, since no viable heating source is available to heat this material at these distances (many thousands of au from the nearest YSO).

%Since the emission appears associated with the outflow, one possible scenario is that  the material was recently ejected from the central clump of YSOs by this powerful outflow. This could be material from the diffuse envelope which seem to surround the 5 sources, or material from one given YSO's gravitationally bound envelope. The gas and dust being ejected at high velocities \citep{Bachiller:1995cy}, it might not yet have time to completely thermalize with the surrounding medium (at which point it would not emit at these wavelengths). This scenario could be confirmed with high sensitivity sub-millimeter maps of the region, with a focus on dense gas tracers that would follow the mass in these regions. The existing maps from \citet{Beltran:2008gu} do not have sufficient sensitivity or resolution to properly identify the velocity field from the gas associated with this continuum emission.
\begin{landscape}
\begin{figure}
\begin{center}
%\hspace*{-1.2in}
\includegraphics[width=1.3\textwidth]{Figures/IRAS20050.png}
\caption[Multi-wavelength view of the IRAS~20050+2720 core]{The core of IRAS20050+2720 is seen in the four bands of the \textit{Spitzer} IRAS instrument, as well as with the four FORCAST bands. The increased resolution of FORCAST compared to previous instruments allows to match the long-wavelength emission with its short wavelength counterpart. The stretch in each image is adjusted for optimal readability. The red contours correspond to the FORCAST \SI{37}{\micro\meter} emission at 0.03, 0.07, 0.13, 0.2, 0.3 and 0.4 Jy.}
\label{fig:IRAS20050_mosaic}
\end{center}
\end{figure}
\end{landscape}



A likely explanation for this emission is that the outflow from IRAS20050.4 (MMS1/OVRO1) is colliding with cold material in the surrounding cloud, creating a shock layer and heating the dust a few hundreds of K. This could explain the arc shape of the emission. The hypothesis is supported by the SED for the extended emission (Fig.~\ref{fig:IRAS20050_ext_SED}) which exhibits strong features at 5.8 and \SI{8}{\um} (IRAC bands 3 and 4 respectively). These bands are known to have multiple PAH broad emission lines, and significant PAH emission has been found to be associated with other outflows \citep{NoriegaCrespo:2004fe,Hudgins:2004wa} as the shock energy and the direct starlight hitting the dust through the outflow cavity excite these emission features. PAH emission is significantly weaker in IRAC band 1, while it is expected to be non-existent in band 2 \citep{NoriegaCrespo:2004fe}. The presence of excess emission in bands 3 and 4 indicates that the emission is not purely thermal.


\begin{figure}[!h]
\begin{center}
\includegraphics[width=0.6\textwidth]{Figures/IRAS20050_extended.jpg}
\caption[IRAS20050+2720 extended emission SED]{SED from extended emission to the east of the cluster, using photometry from IRAC and FORCAST in a \ang{;;2.4} radius aperture. Excess emission at 5.8 and \SI{8}{\um} (IRAC bands 3 and 4 respectively) can be attributed to PAHs excited by the shock and/or by the radiation from the outflow.}
\label{fig:IRAS20050_ext_SED}
\end{center}
\end{figure}




\subsubsection{SEDs and fitted parameters}
The 5 sources in the densest part of the cluster shown in Fig.~\ref{fig:IRAS20050_mosaic} are all highly extincted based on the slopes of the emission in the 2MASS bands and the depth of the \SI{10}{\um} silicate absorption feature (see Fig.~\ref{fig:IRAS20050_SEDs}). IRAS20050.1 has a flat spectrum out to \SI{37}{\um}, unlike the four other sources which are rising. IRAS20050.4 is the most steeply rising source, and is weak in the IRAC bands, suggesting that it is the most embedded source, which is corroborated by the fact that it is coincident with the strongest millimeter continuum source in the region. 

The fitted parameters for the 7 identified sources in our two fieldsare shown in Table~\ref{tab:IRAS20050params}. When the mass error is 0, it means that there is only one mass parameter for all the fits which are within 0.5 of \Rmin. Since no long-wavelength data is available, the envelope masses are not very well constrained. Sources 6 and 7 are far away from the main core which was discussed previously, and do not appear to be associated with the first 5 sources. We interpret these results in several ways:
\renewcommand{\arraystretch}{1.5}
\begin{table}[!h]
\scriptsize
\caption[Fitted parameters in IRAS~20050+2720]{Fitted parameters of sources in IRAS~20050+2720.}
\label{tab:IRAS20050params}
\vspace{-0.5cm}
\begin{longtable}{lccccccccc}
\toprule																			
SOFIA Name	&	Coordinates	&	$\alpha$	&	R	&	\Menv			&	\Ltot			&	$i$	&	$\Av$	\\
	&	(J2000)	&		&		&	(\si{\Msun})			&	(\si{\Lsun})			&	(\si{\degree})	&	(mag	)\\
\midrule																			
IRAS20050.1	&	20h07m06.6s +27d28m48.0s	&	0.07	&	0.74	&	0.004	$\pm$	0	&	128	$\pm$	15.3	&	65	&	9	\\
IRAS20050.2	&	20h07m06.2s +27d28m49.1s	&	1.65	&	0.77	&	0.58	$\pm$	0.22	&	26.6	$\pm$	6.0	&	19	&	14	\\
IRAS20050.3	&	20h07m06.3s +27d28m56.6s	&	1.13	&	0.73	&	0.26	$\pm$	0.11	&	48.5	$\pm$	6.3	&	27	&	5	\\
IRAS20050.4	&	20h07m05.9s +27d28m59.2s	&	1.71	&	0.27	&	0.38	$\pm$	0.32	&	48.5	$\pm$	8.8	&	43	&	5	\\
IRAS20050.5	&	20h07m06.6s +27d28m53.1s	&	0.54	&	0.78	&	0.01	$\pm$	0	&	49.4	$\pm$	6.2	&	43	&	14	\\
IRAS20050.6	&	20h07m02.2s +27d30m26.0s	&	-0.34	&	2.22	&	0.004	$\pm$	0	&	201.6	$\pm$	32.1	&	81	&	14	\\
IRAS20050.7	&	20h07m07.9s +27d27m15.8s	&	1.29	&	1.50	&	0.015	$\pm$	0.09	&	3.5	$\pm$	3.6	&	0	&	14	\\
\bottomrule	
\end{longtable} 
\caption*{\textbf{Notes}: The envelope masses are not well constrained due to the lack of long-wavelength emission. When the mass error is 0, it means that there is only one mass parameter for all the fits which are within 0.5 of \Rmin.}
\end{table}

\begin{itemize}
\item Generally, the best-fitting luminosity is better constrained than the masses ($10-25\%$ uncertainties in luminosity for $R<1$);
\item The sum of the protostellar luminosities is $\sim \SI{300}{\Lsun}$, which is consistent with IRAS luminosity measurements of the entire region of \SI{388}{\Lsun}.
\item Sources 1 and 5 appear to be at a later stage of their evolution, with a lower spectral index and much lower envelope mass. However, less models in our grid fit these models well, as all well-fitting models have the same mass at the very edge of our range of grid parameters;
\item Sources 2, 3 and 4 are more embedded, with steeply rising SOFIA fluxes. They are consistent with having sub-solar mass envelopes, but the uncertainties on the envelope mass are large due to the lack of long-wavelength measurements;
\item Source 6 fits less well and appears to have a very low envelope mass, as the SOFIA fluxes are decreasing with increasing wavelength.
\item Source 7 has the most fractional scatter in terms of envelope mass, as well as in luminosity. The fits from this source do not allow to draw meaning conclusions. It would greatly benefit from having a high-resolution data point at long wavelength.
\end{itemize}

IRAS20050.6 is the poorest fit in this cluster, and it shows the limit of our grid of models. Judging by the shape of the SED and its negative spectral index, we can conclude that this object is not a Class 0 or Class I protostar with a large envelope. In fact, it is classified as a Class II YSO in \citet{Gutermuth:2009gca}, and our grid is not particularly well-suited to fit sources of this type. By doing some exploration of the parameter space around the best fit, it appears that the emission could potentially be fitted by a much smaller disk outer radius with no envelope. The smaller disk size is required to reduce long-wavelength emission, as all the dust stays warmer and emits only at shorter wavelengths. In order to by thorough, we might decide to run a larger grid to fit this type of objects as well. 

\begin{figure}
\begin{center}
\includegraphics[width=\textwidth]{Figures/IRAS20050_SEDs.png}
\caption[IRAS20050+2720 SEDs]{SEDs of the 7 sources in the two fields. }
\label{fig:IRAS20050_SEDs}
\end{center}
\end{figure}

\subsubsection{Diffuse emission}
In testing the various scenarios of star formation, it is useful to obtain a measure of how much mass is available for the YSOs to grow after their original collapse. For this, clustered regions such as this one are an ideal laboratory since the YSOs appear to share an envelope. In this cluster, the typical separations between the sources are \ang{;;6}-\ang{;;8}, which correspond to projected distances of \num{4200}-\SI{5600}{\au}. This strongly indicates that the envelopes of individual YSOs are interacting with each other.

\renewcommand{\arraystretch}{1.5}
\begin{table}[!h]
\scriptsize
\caption[Clustered sources in IRAS~20050+2720's dense core]{Clustered sources in the densest region of IRAS~20050+2720.}
\label{tab:IRAS20050sum}
\vspace{-0.5cm}
\begin{longtable}{lP{2cm}P{2cm}P{2cm}P{2cm}}
\toprule																			
SOFIA name	&	F11	&	F19	&	F31	&	F37	\\
	&	Jy	&	Jy	&	Jy	&	Jy\\
\midrule									
IRAS20050.1	&	0.64	&	1.93	&	4.50	&	6.32	\\
IRAS20050.2	&	0.06	&	1.45	&	9.31	&	11.96	\\
IRAS20050.3	&	0.18	&	2.58	&	12.53	&	19.34	\\
IRAS20050.4	&	0.06	&	0.25	&	8.54	&	12.85	\\
IRAS20050.5	&	0.19	&	1.03	&	2.97	&	5.65	\\
\midrule									
Sum of point sources in cluster	&	1.13	&	7.24	&	37.84	&	56.11	\\
Total cluster emission	&	1.79	&	7.07	&	37.36	&	49.33	\\
Ratio	&	1.58	&	0.98	&	0.99	&	0.88	\\
\bottomrule					
	\end{longtable} 
	\caption*{\textbf{Notes}: The "total cluster" emission corresponds to the entirety of the region shown in Fig.~\ref{fig:IRAS20050_mosaic}, which is then background-subtracted.}
\end{table}

However, measuring the flux from each individual source in these clustered regions is challenging, since the sources are so close together. With an aperture of \ang{;;2.4} (3-pixel radius on FORCAST), we managed to put non-overlapping apertures for all the 5 sources in IRAS~20050+2720. However, since the aperture correction was derived considering a "total flux" aperture to be $\sim$12 pixel radius, we are accounting for the same flux multiple times, even if the apertures are not overlapping. We estimate the \SI{37}{\um} flux from the eastern extended emission to total $\sim\SI{6}{\Jy}$, we obtain about 22\% of excess \SI{37}{\um} flux when comparing the sum of the point sources and the total emission from the cluster (see Table~\ref{tab:IRAS20050sum}). At \SI{31}{\um}, the flux excess is only about 10\%. At \SI{19}{\um} and below, the extended emission is within the noise uncertainty of the map. 

This excess flux can only partially be explained by the tails of the PSF extending well below the aperture size (see Fig.~\ref{fig:averageEE}), with 10-15\% of the total energy still existing in the annulus outward of 8~pixels (\ang{;;6}) from the aperture center. However, the contribution of a source to any given other source is only a fraction of this since it would only correspond to the amount of flux within a 3-pixel aperture. We conclude that the PSF shape is not responsible for the bulk of the observed excess flux at both wavelengths.

One possible explanation would be that diffuse thermal emission occurs across the entire region. This could be caused by heating internal to the cluster (powered by the outflow, for example, like the eastern extended emission) or by a population of stochastically heated very small grains, which are not in LTE. The high outflow activity in this region could carve out multiple cavities which facilitate short-wavelength photons from the individual stars to reach out to larger distances within the envelopes and the shared mass reservoir. At \SI{37}{\um}, the level of diffuse emission required to account for the excess flux is about \SI{0.05}{\Jy\per\pixel}, which is the same as the average diffuse emission in the eastern region. Such an explanation would also help account for the high amount of external extinction that is needed to fit most of the SEDs in this region.

This tends to favor a scenario where protostars are fragmenting from a cloud and continue accreting material within that original envelope. The envelopes of neighboring YSOs interact, and possibly can exchange material as some YSOs become more massive (competitive accretion). 


\subsubsection{Conclusions on IRAS~20050+2720}

We have determined the photometry for 7 objects in IRAS~20050+2720. 5 of these objects are highly clustered and our FORCAST data is the first mid- to far-IR photometry for these sources. Our findings can be summarized as follows:
\begin{itemize}
\item Fitted luminosities for the 5 clustered sources are between 26 and \SI{128}{\Lsun}, with estimated scatter ranging from 10 to 25\%. 
\item IRAS20050.1 and IRAS20050.5 have smaller envelope mass estimates compared to the other 3 sources, which is consistent with the different in their spectral index. Fitted masses are less robust and show more scatter for the most embedded sources 2, 3 and 4, which are limited by the lack of far-IR, high-resolution data points.
\item We detect extended emission to the east of the main core, which is strongest in the 31 and \SI{37}{\um} images. It appears to be associated with the blue lobe of the outflow coming from IRAS20050.1 (MMS1/OVRO1). We argue that the emission is arising from shock-heated material where the outflow is impacting the cloud, and we suggest that the enhanced IRAC band 3 and 4 fluxes are a signature of PAHs emission, which can be characteristic of such outflow regions.
\item The grid might need to be expanded to lower masses and/or to smaller disk and envelope sizes to manage to fit Class II sources such as IRAS20050.6.
\item Finally, the inconsistency between the sum of point source fluxes and the total cluster emission at 31 and \SI{37}{\um} could be explained by the presence of an extended, diffuse component the 5 clustered sources appear to share. This is consistent with competitive accretion theory of clustered star formation in which multiple cores will attempt to accrete mass from a same, shared envelope.
%\item Our findings support the work by \citet{Beltran:2008gu} who suggest that there are multiple generations of star formation coexisting in the same cluster, judging by the spectral indices and the fitted masses.
\end{itemize}


%This highlights the complexity of the physical processes in these regions, which 
%LGM Your luminosities fits do not support that 1 and 5 are lower luminosity that 2,3,4 so why are you repeating Beltran story?

%LGM STOPPED READING HERE
%
%\subsection{NGC 2071}
%
%\subsubsection{Context}
%The NGC~2071 star-forming region is one of several active areas of star formation in the northern part of L1630 giant molecular cloud which is located at a distance of \SI{422}{\pc} \citep{vanDishoeck:2011em}. NGC~2071 itself is a reflection nebula.
%The NGC~2071 infrared cluster, located about 4' north of the reflection nebula, is a region of intermediate mass star formation \citep{Strom1976, Persson1981, Butner1990}. Maps of the cloud in CO and its isotopomers \citep{Buckle2010} show a large scale clump with $\sim\SI{1000}{\Msun}$ associated with the cluster. Dust continuum emission at $\lambda$=0.85 and \SI{1.3}{\milli\meter} peaks on center of the cluster extending ~1' in diameter containing ~\SI{30}{\Msun} of gas and dust \citep{Johnstone2001,Mitchell2001,Launhardt1996}. Emission from CS in the J=2-1 through J=7-6 indicate that the gas in this region is centrally condensed with a density of $sim\SI{1e6}{\raiseto{-3}\centi\meter}$ \citep{Zhou1990}. 
%
%There are a number of near infrared surveys of the young cluster \citep[e.g.,][]{Strom1976,Lada1991,Megeath2012,Spezzi2015}. \citet{Spezzi2015} identify 52 YSOs associated with the NGC~2071 cluster, with the majority Class II sources. \citet{Flaherty2008} estimate an age of $\sim$\SI{2}{\mega\year} for the cluster, consistent with the large fraction of Class II sources \citep{Evans2009}. The brightest far infrared emission from the cluster is associated with the IRS1 region \citep{Harvey1979,Butner1990}, which has an estimated total luminosity of \SI{520}{\Lsun}. The immediate region of IRS 1 is, in fact, home to a number of YSOs that are infrared, X-ray, and radio sources \citep{Skinner2009,C-G2012,Kempen2012}. The radio \citep{C-G2012} and H$_2$ emission line imaging indicate that IRS~1, IRS~2, IRS~3, and, perhaps, VLA~1 are YSOs with outflows. The larger scale molecular outflow associated with this region is well studied in a number of molecules \citep{Bally1982,Chernin1993,Stoji2008}.
%
%Figure \ref{fig:NGC2071_Lee} shows the Spitzer 3.1~$\mu$m image of the IRS~1 region on the left \citep[image from Spitzer Archive:][]{Megeath2012} and the Herschel 70~$\mu$m image on the right (image from Herschel Archive: Gould Belt Project, P.I. Andr\'e). The plus marks in both panels indicate the position of the brighter YSOs: IRS~1, IRS~2, IRS~3, IRS~4, and VLA~1. The inner red circle with a diameter of \ang{;;26} indicates the extend of the saturated region in the Spitzer MIPS \SI{24}{\um} image; the outer red circle, diameter \ang{;;60}, encompasses the region with strong imaging artifacts in the MIPS \SI{24}{\um} image.
%The right panel shows Herschel 70~$\mu$m image which does not resolve the emission from IRS~1, IRS~2, IRS~3, and VLA~1. The centroid of the \SI{24}{\um} and \SI{70}{\um} emission is between IRS~1 and VLA~1 indicating that several of the sources are contributing to the total observed emission. Interferometric observations show that the millimeter wavelength dust emission is dominated by envelopes associated with IRS~1 and IRS~3, with estimated masses of 8.2 and \SI{12.3}{\Msun} material, respectively \citep{Kempen2012}. The millimeter emission also reveals the presence of disks with radii $\le$\SI{100}{\au} associated with IRS~1 and IRS~3 \citep{Kempen2012}.
%
%The luminosities and masses of the individual source, IRS~1, IRS~2, IRS~3, and VLA~1, are not known. The Spectral Energy Distributions (SEDs) shortward of 10~$\mu$m support their identification as embedded YSOS \citep{Skinner2009}. \citet{Skinner2009} gives a clear discussion of the possibilities for IRS~1 and concludes that it is likely a mid-to late B~star. \cite{Kempen2012} find luminosities of 10, 3.4, and $\le$\SI{27}{\Lsun} for IRS~1, 2, and 3, respectively, and stellar masses of $\le$\SI{1}{\Msun} for each, based on SED fitting. These masses and luminosities are not consistent with estimate of the total luminosity of the region of \SI{520}{\Lsun} \citep{Butner1990}. The far infrared images from Herschel reveal that IRS~1 alone does not totally dominate, as seen in Fig.~\ref{fig:NGC2071_Lee}; IRS~1, VLA~1, and IRS~3 likely make substantial contributions to the emission with lesser emission from IRS~2 and IRS~4.
%
%\subsubsection{Observations and discussion}
%
%\begin{figure}
%\begin{center}
%\includegraphics[width=\textwidth]{Figures/Lee_NGC2071.png}
%
%
%\caption[Multi-wavelength view of the NGC~2071 core]{The core of NGC2071 is seen in two bands of the \textit{Spitzer} IRAC instrument ("I1" and "I4"), as well as with the four FORCAST bands. The increased resolution of FORCAST compared to previous instruments allows to match the long-wavelength emission with its short wavelength counterpart. The stretch in each image is adjusted for optimal readability. The red contours correspond to the FORCAST \SI{37}{\micro\meter} emission, between 0.1 and 2.4~Jy by increment of 0.25~Jy. }
%\label{fig:NGC2071_Lee}
%\end{center}
%\end{figure}
%
%
%\begin{landscape}
%\begin{figure}
%\begin{center}
%\includegraphics[width=1.4\textwidth]{Figures/NGC2071_mosaic.png}
%
%
%\caption[Multi-wavelength view of the NGC~2071 core]{The core of NGC2071 is seen in two bands of the \textit{Spitzer} IRAC instrument ("I1" and "I4"), as well as with the four FORCAST bands. The increased resolution of FORCAST compared to previous instruments allows to match the long-wavelength emission with its short wavelength counterpart. The stretch in each image is adjusted for optimal readability. The red contours correspond to the FORCAST \SI{37}{\micro\meter} emission, between 0.1 and 2.4~Jy by increment of 0.25~Jy. }
%\label{fig:NGC2071_mosaic}
%\end{center}
%\end{figure}
%\end{landscape}
%
%
%\renewcommand{\arraystretch}{1.5}
%\begin{table}[!h]
%\scriptsize
%\caption[Sources in NGC2071's dense core]{Sources in the densest region of NGC2071.}
%\label{tab:NGC2071sum}
%\vspace{-0.5cm}
%\begin{longtable}{lP{2cm}P{2cm}P{2cm}P{2cm}}
%\toprule																			
%SOFIA name	&	F11	&	F19	&	F31	&	F37	\\
%	&	Jy	&	Jy	&	Jy	&	Jy	\\
%\midrule									
%NGC2071.1	&	10.07	&	72.041	&	167.93	&	234.93	\\
%NGC2071.2	&	0.38	&	11.207	&	56.70	&	89.55	\\
%NGC2071.3	&	0.19	&	3.027	&	19.97	&	37.56	\\
%\midrule									
%Sum of point sources in cluster	&	10.65	&	86.28	&	244.61	&	362.03	\\
%Total cluster emission	&	13.523	&	94.16	&	280.14	&	362.99	\\
%Ratio	&	1.27	&	1.09	&	1.15	&	1.00	\\
%\bottomrule					
%	\end{longtable} 
%\end{table}
%
%
%\renewcommand{\arraystretch}{1.5}
%\begin{table}[!h]
%\scriptsize
%\caption[Fitted parameters in NGC2071]{Fitted parameters of sources in NGC2071.}
%\label{tab:NGC2071params}
%\vspace{-0.5cm}
%\begin{longtable}{lcccccccccc}
%\toprule																					
%SOFIA Name	&	Coordinates	&	$\alpha$	&	R	&	\Menv			&	\Ltot			&	\Lbol	&	$i$	&	$\Av$	\\
%	&	J2000	&		&		&	\si{\Msun}			&	\si{\Lsun}			&	\si{\Lsun}	&	\si{\degree}	&	mag	\\
%\midrule																					
%NGC2071.1	&	05h47m04.8s +00d21m43.1s	&	2.31	&	2.83	&	22.17	$\pm$	2.597	&	43.7	$\pm$	3.3	&	297.2	&	0	&	14	\\
%NGC2071.2	&	05h47m04.7s +00d21m48.2s	&	2.22	&	1.31	&	1.30	$\pm$	0.294	&	74.1	$\pm$	12.7	&	199.8	&	19	&	13	\\
%NGC2071.3	&	05h47m05.4s +00d21m50.3s	&	1.01	&	1.51	&	0.17	$\pm$	0.315	&	28.8	$\pm$	7.8	&	113.7	&	19	&	14	\\
%NGC2071.4	&	05h47m04.0s +00d22m10.5s	&	1.08	&	2.57	&	0.001	$\pm$	0.002	&	111.1	$\pm$	62.2	&	21.4	&	84	&	13	\\
%NGC2071.5	&	05h47m10.7s +00d21m14.0s	&	0.32	&	0.96	&	0.002	$\pm$	0.001	&	39.9	$\pm$	5.1	&	14.9	&	27	&	14	\\
%\bottomrule																																										
%	\end{longtable} 
%\end{table}
%
%Show sum of sources compared to cluster total
%
%
%\begin{figure}
%\begin{center}
%\includegraphics[width=\textwidth]{Figures/NGC2071_SEDs.png}
%\caption[NGC2071 SEDs]{SEDs of the 5 sources in the two fields. }
%\label{fig:NGC2071_SEDs}
%\end{center}
%\end{figure}



\section{Conclusion and future work}

We have used SOFIA FORCAST to image 42 fields in bright, nearby stellar clusters. We derive aperture photometry in 4 bands: \SI{11.1}{\um}, \SI{19.7}{\um}, \SI{31.5}{\um}, \SI{37.1}{\um}, for a total of 70 point sources and 14 extended sources. In many cases, our photometry is the only mid- to far-IR photometry available for these sources, since archival \Spitzer observations were either saturated or confused.

In most cases, we complete our SOFIA photometry using \Spitzer IRAC and 2MASS data to produce SEDs from \SI{1.2}{\um} to \SI{37}{\um}. In a limited number of cases, we also obtained \Herschel data. When the photometry catalogs cannot be found, we use the same photometry pipeline that we developed for SOFIA on the \Spitzer and \Herschel calibrated images.

%In our sample there were 15 cases of extended emission at \SI{37}{\um}. This spatial extension is not simple to model: with a FORCAST FWHM of $\sim\ang{;;3.5}$, an object with spatial extension has a size on the order of a few thousands of \si{\au} at \SI{500}{\pc}: we haven't been able to show that the central object can heat dust grains at this distance sufficiently for them to emit thermally at \SI{37}{\um}. Hence, another heating mechanism is responsible for this emission: we suggest that the emission could be due to a population of non-LTE, very small dust grains, or a proximity to outflows. Answering this question will require further analysis and more study.

We proceed to SED fitting of a subset of our sources, based on a radiative transfer code called Hyperion. Starting from a standard model, we argue that there are 4 primary parameters that are needed to model the SEDs for Class 0 and I YSOs: the central luminosity, the envelope mass, the inclination angle, and an external extinction component. A scaling factor can be used as a proxy for finer luminosity sampling in the grid. We argue that as a system approaches the boundary between Class I and Class II, a fifth parameter might need to be varied in the model: the disk/envelope outer radius. Reducing this radius is necessary to appropriately reducing the emission around \SI{100}{\um} while maintaining the fluxes at shorter wavelength, as exhibited for example in our source IRAS20050.6. 

Fits for most our sources are reasonable. The luminosity of the family of best fits is usually constrained well, with scatter usually less than 25\%. We attribute this good accuracy to the FORCAST 31 and \SI{37}{\um} bands which really sample the envelope. This typical scatter does not change when long-wavelength data points are not available, for example in IRAS~20050+2720. Envelope masses, however, are constrained much better when long-wavelength data (such as \Herschel) is available. Unfortunately, for most of our sources the long-wavelength data come from single-dish telescopes, which do not have sufficient angular resolution to guarantee that the measured flux is associated with the source; the measured emission could belong to an extended component, or to another nearby source. 

We find that the fitted luminosity is substantially different from the observationally-defined bolometric luminosity of our sources (which corresponds to the integral of the observed data points), with a high dependence on the inclination angle of the fit. We argue that the fitted luminosity is a more accurate measurement of the central luminosity (which is composed of the source's luminosity and the accretion luminosity). Indeed, the bolometric luminosity is highly geometry-dependent, as a source seen edge-on will exhibit a dramatically lower bolometric luminosity as opposed to sources seen through the throat of the cavity, because of the line of sight passes through the disk which has a lot of opacity. SED fitting allows to lift this degeneracy and provide a more robust estimate of the actual luminosity.

Finally, we discuss several caveats with the fitting methods that are traditionally employed, such as the use of an external extinction factor which is modeled purely as extinction with emission counterpart. While this could matter substantially for estimating masses, we argue that its impact on the luminosity determination is small, provided that the allowed range of extinction remains reasonable. 


%\subsection{NGC 2071}
%
%\subsubsection{Context}
%The NGC~2071 star-forming region is one of several active areas of star formation in the northern part of L1630 giant molecular cloud which is located at a distance of \SI{422}{\pc} \citep{vanDishoeck:2011em}. NGC~2071 itself is a reflection nebula.
%The NGC~2071 infrared cluster, located about 4' north of the reflection nebula, is a region of intermediate mass star formation \citep{Strom1976, Persson1981, Butner1990}. Maps of the cloud in CO and its isotopomers \citep{Buckle2010} show a large scale clump with $\sim\SI{1000}{\Msun}$ associated with the cluster. Dust continuum emission at $\lambda$=0.85 and \SI{1.3}{\milli\meter} peaks on center of the cluster extending ~1' in diameter containing ~\SI{30}{\Msun} of gas and dust \citep{Johnstone2001,Mitchell2001,Launhardt1996}. Emission from CS in the J=2-1 through J=7-6 indicate that the gas in this region is centrally condensed with a density of $sim\SI{1e6}{\raiseto{-3}\centi\meter}$ \citep{Zhou1990}. 
%
%There are a number of near infrared surveys of the young cluster \citep[e.g.,][]{Strom1976,Lada1991,Megeath2012,Spezzi2015}. \citet{Spezzi2015} identify 52 YSOs associated with the NGC~2071 cluster, with the majority Class II sources. \citet{Flaherty2008} estimate an age of $\sim$\SI{2}{\mega\year} for the cluster, consistent with the large fraction of Class II sources \citep{Evans2009}. The brightest far infrared emission from the cluster is associated with the IRS1 region \citep{Harvey1979,Butner1990}, which has an estimated total luminosity of \SI{520}{\Lsun}. The immediate region of IRS 1 is, in fact, home to a number of YSOs that are infrared, X-ray, and radio sources \citep{Skinner2009,C-G2012,Kempen2012}. The radio \citep{C-G2012} and H$_2$ emission line imaging indicate that IRS~1, IRS~2, IRS~3, and, perhaps, VLA~1 are YSOs with outflows. The larger scale molecular outflow associated with this region is well studied in a number of molecules \citep{Bally1982,Chernin1993,Stoji2008}.
%
%Figure \ref{fig:NGC2071_Lee} shows the Spitzer 3.1~$\mu$m image of the IRS~1 region on the left \citep[image from Spitzer Archive:][]{Megeath2012} and the Herschel 70~$\mu$m image on the right (image from Herschel Archive: Gould Belt Project, P.I. Andr\'e). The plus marks in both panels indicate the position of the brighter YSOs: IRS~1, IRS~2, IRS~3, IRS~4, and VLA~1. The inner red circle with a diameter of \ang{;;26} indicates the extend of the saturated region in the Spitzer MIPS \SI{24}{\um} image; the outer red circle, diameter \ang{;;60}, encompasses the region with strong imaging artifacts in the MIPS \SI{24}{\um} image.
%The right panel shows Herschel 70~$\mu$m image which does not resolve the emission from IRS~1, IRS~2, IRS~3, and VLA~1. The centroid of the \SI{24}{\um} and \SI{70}{\um} emission is between IRS~1 and VLA~1 indicating that several of the sources are contributing to the total observed emission. Interferometric observations show that the millimeter wavelength dust emission is dominated by envelopes associated with IRS~1 and IRS~3, with estimated masses of 8.2 and \SI{12.3}{\Msun} material, respectively \citep{Kempen2012}. The millimeter emission also reveals the presence of disks with radii $\le$\SI{100}{\au} associated with IRS~1 and IRS~3 \citep{Kempen2012}.
%
%The luminosities and masses of the individual source, IRS~1, IRS~2, IRS~3, and VLA~1, are not known. The Spectral Energy Distributions (SEDs) shortward of 10~$\mu$m support their identification as embedded YSOS \citep{Skinner2009}. \citet{Skinner2009} gives a clear discussion of the possibilities for IRS~1 and concludes that it is likely a mid-to late B~star. \cite{Kempen2012} find luminosities of 10, 3.4, and $\le$\SI{27}{\Lsun} for IRS~1, 2, and 3, respectively, and stellar masses of $\le$\SI{1}{\Msun} for each, based on SED fitting. These masses and luminosities are not consistent with estimate of the total luminosity of the region of \SI{520}{\Lsun} \citep{Butner1990}. The far infrared images from Herschel reveal that IRS~1 alone does not totally dominate, as seen in Fig.~\ref{fig:NGC2071_Lee}; IRS~1, VLA~1, and IRS~3 likely make substantial contributions to the emission with lesser emission from IRS~2 and IRS~4.
%
%\subsubsection{Observations and discussion}
%
%\begin{figure}
%\begin{center}
%\includegraphics[width=\textwidth]{Figures/Lee_NGC2071.png}
%
%
%\caption[Multi-wavelength view of the NGC~2071 core]{The core of NGC2071 is seen in two bands of the \textit{Spitzer} IRAC instrument ("I1" and "I4"), as well as with the four FORCAST bands. The increased resolution of FORCAST compared to previous instruments allows to match the long-wavelength emission with its short wavelength counterpart. The stretch in each image is adjusted for optimal readability. The red contours correspond to the FORCAST \SI{37}{\micro\meter} emission, between 0.1 and 2.4~Jy by increment of 0.25~Jy. }
%\label{fig:NGC2071_Lee}
%\end{center}
%\end{figure}
%
%
%\begin{landscape}
%\begin{figure}
%\begin{center}
%\includegraphics[width=1.4\textwidth]{Figures/NGC2071_mosaic.png}
%
%
%\caption[Multi-wavelength view of the NGC~2071 core]{The core of NGC2071 is seen in two bands of the \textit{Spitzer} IRAC instrument ("I1" and "I4"), as well as with the four FORCAST bands. The increased resolution of FORCAST compared to previous instruments allows to match the long-wavelength emission with its short wavelength counterpart. The stretch in each image is adjusted for optimal readability. The red contours correspond to the FORCAST \SI{37}{\micro\meter} emission, between 0.1 and 2.4~Jy by increment of 0.25~Jy. }
%\label{fig:NGC2071_mosaic}
%\end{center}
%\end{figure}
%\end{landscape}
%
%
%\renewcommand{\arraystretch}{1.5}
%\begin{table}[!h]
%\scriptsize
%\caption[Sources in NGC2071's dense core]{Sources in the densest region of NGC2071.}
%\label{tab:NGC2071sum}
%\vspace{-0.5cm}
%\begin{longtable}{lP{2cm}P{2cm}P{2cm}P{2cm}}
%\toprule																			
%SOFIA name	&	F11	&	F19	&	F31	&	F37	\\
%	&	Jy	&	Jy	&	Jy	&	Jy	\\
%\midrule									
%NGC2071.1	&	10.07	&	72.041	&	167.93	&	234.93	\\
%NGC2071.2	&	0.38	&	11.207	&	56.70	&	89.55	\\
%NGC2071.3	&	0.19	&	3.027	&	19.97	&	37.56	\\
%\midrule									
%Sum of point sources in cluster	&	10.65	&	86.28	&	244.61	&	362.03	\\
%Total cluster emission	&	13.523	&	94.16	&	280.14	&	362.99	\\
%Ratio	&	1.27	&	1.09	&	1.15	&	1.00	\\
%\bottomrule					
%	\end{longtable} 
%\end{table}
%
%
%\renewcommand{\arraystretch}{1.5}
%\begin{table}[!h]
%\scriptsize
%\caption[Fitted parameters in NGC2071]{Fitted parameters of sources in NGC2071.}
%\label{tab:NGC2071params}
%\vspace{-0.5cm}
%\begin{longtable}{lcccccccccc}
%\toprule																					
%SOFIA Name	&	Coordinates	&	$\alpha$	&	R	&	\Menv			&	\Ltot			&	\Lbol	&	$i$	&	$\Av$	\\
%	&	J2000	&		&		&	\si{\Msun}			&	\si{\Lsun}			&	\si{\Lsun}	&	\si{\degree}	&	mag	\\
%\midrule																					
%NGC2071.1	&	05h47m04.8s +00d21m43.1s	&	2.31	&	2.83	&	22.17	$\pm$	2.597	&	43.7	$\pm$	3.3	&	297.2	&	0	&	14	\\
%NGC2071.2	&	05h47m04.7s +00d21m48.2s	&	2.22	&	1.31	&	1.30	$\pm$	0.294	&	74.1	$\pm$	12.7	&	199.8	&	19	&	13	\\
%NGC2071.3	&	05h47m05.4s +00d21m50.3s	&	1.01	&	1.51	&	0.17	$\pm$	0.315	&	28.8	$\pm$	7.8	&	113.7	&	19	&	14	\\
%NGC2071.4	&	05h47m04.0s +00d22m10.5s	&	1.08	&	2.57	&	0.001	$\pm$	0.002	&	111.1	$\pm$	62.2	&	21.4	&	84	&	13	\\
%NGC2071.5	&	05h47m10.7s +00d21m14.0s	&	0.32	&	0.96	&	0.002	$\pm$	0.001	&	39.9	$\pm$	5.1	&	14.9	&	27	&	14	\\
%\bottomrule																																										
%	\end{longtable} 
%\end{table}
%
%Show sum of sources compared to cluster total
%
%
%\begin{figure}
%\begin{center}
%\includegraphics[width=\textwidth]{Figures/NGC2071_SEDs.png}
%\caption[NGC2071 SEDs]{SEDs of the 5 sources in the two fields. }
%\label{fig:NGC2071_SEDs}
%\end{center}
%\end{figure}



%\section{Conclusion and future work}
%
%We have used SOFIA FORCAST to image 42 fields in bright, nearby stellar clusters. We derive aperture photometry in 4 bands: \SI{11.1}{\um}, \SI{19.7}{\um}, \SI{31.5}{\um}, \SI{37.1}{\um}, for a total of 90 sources. In many cases, our photometry is the only mid- to far-IR photometry available for these sources, since archival \Spitzer observations were either saturated or confused.
%
%In multiple cases, we complete our SOFIA photometry using \Spitzer IRAC as well as \Herschel data. When the catalogs cannot be found, we use the same photometry pipeline that we developed for SOFIA on the \Spitzer and \Herschel calibrated images.
%
%We also proceed to SED fitting of our sources, based on a radiative transfer code called Hyperion. Using a simple grid, we produce estimates for physical parameters of these YSOs, and carefully approximate the error in the parameter estimate.
%
%We take a closer look at two special clusters: IRAS~20050+2720, which contains five close-by YSOs sharing what appears to be an extended envelope, favoring a competitive accretion scenario; and NGC2071 [insert conclusions here]
%
%In our sample there were 15 cases of extended emission at \SI{37}{\um}. This spatial extension is not simple to model: with a FORCAST FWHM of $\sim\ang{;;3.5}$, an object with spatial extension has a size on the order of a few thousands of \si{\au} at \SI{500}{\pc}: we haven't been able to show that the central object can heat dust grains at this distance sufficiently for them to emit thermally at \SI{37}{\um}. Hence, another heating mechanism is responsible for this emission: we suggest that the emission could be due to a population of non-LTE, very small dust grains. Answering this question will require further analysis and more study.
%\section{Introduction}
%Most stars in the Galaxy form in cluster environments of sizes 2-4 pc, often containing more than 100 young stellar objects (YSOs), with typical separations of $<$0.05~pc between stars near their centers \citep{Porras:2003kxa, Allen:2007wqa, Gutermuth:2009gca}.
%Previous studies have been effective in elucidating the young stellar content and distribution in clouds on large scales (parsec down to 0.05~pc) \citep{Evans-ARAA2012}, but young cluster cores, born in dense portions of molecular clouds, are more difficult to observe. They are obscured at optical through near-IR wavelengths. At mid-IR through far-IR wavelengths, the material surrounding YSOs and involved in the stellar birth process emits due to heating by the young stars, but the resolution to date has not been sufficient to isolate individual stars in the cores of most nearby young clusters.
%
%Space telescopes such as \textit{Spitzer} and WISE have tremendous sensitivity, which made them so scientifically productive, but it limits their utility in the densest regions of star-forming clusters because of detector saturated and imaging artifacts (See \citep{2008ApJ...672.1013P} for examples). This is particularly problematic at wavelengths of \SI{24}{\micro\meter} and beyond, where a bright cluster star can dominate a region 3-5 nominal resolution elements out from the star. In fact, it is often difficult for \textit{Spitzer} and WISE to provide good flux estimates for even the brightest YSOs in the cores of clusters.
%
%The FORCAST instrument on SOFIA provides the opportunity to study cluster cores at 10 to \SI{37}{\micro\meter} with better angular resolution than \textit{Spitzer} and WISE, without saturation even on the brightest sources. Although it is less sensitive than space-based instrument at comparable wavelengths due to the large thermal background noise at 13~km altitude, FORCAST images can lift degeneracies in assigning flux by separating sources that were previously unresolved or hidden by saturation artifacts. The mid-infrared fluxes of very clustered objects are essential contraints on their YSO's spectral energy distribution (SED) which is used to determine luminosity and evolutionary state.
%
%This paper presents the results for two clusters, IRAS~200050
%and NGC~2071, which were observed as part of a FORCAST survey program to observe bright, nearby star-forming cluster cores for which the \textit{Spitzer} and WISE archival data show extensive amounts of saturation and source confusion based on near infrared images. 
%
%Section 2 provides an overview of what is know about the two clusters. In section 3, we describe our data and reduction methods in detail, and discuss systematic of the FORCAST instrument. Section 4 presents our SOFIA images and discusses them in the context of other observations. Section 5 discusses flux measurements for cluster sources at other wavelengths and outlines our procedures for deriving improved fluxes where applicable. Section 6 presents Spectral Energy Distributions and fits for the SOIFA sources, and section 7 discusses our findings.
%
%\section{Target Clusters}
%IRAS~20050+2720 and NGC~2071 are embedded young stellar clusters with total luminosities in the cores that are characteristic of intermediate mass YSOs. The following two subsections provide overviews of each cluster and its environment.
%
%\subsection{IRAS20050+2720}
%
%IRAS~20050+2720 is part of an active site of intermediate-mass star formation in the Cygnus Rift located at 700~pc \citep{Wilking:1989el}, with the particularity that it doesn't seem to contain any massive stars \citep{Gunther:2012dq}. The main cluster core is associated with water and methanol masers \citep{Palla:1991up,Fontani:2010cf} and multipolar molecular outflows observed at millimeter wavelengths \citep{Bachiller:1995cy,Anglada:1998uu,Beltran:2008gu}, suggesting that the region might have experienced a recent episode of star formation in the past 0.1 Myr which contrasts with the average age of the cluster of 1 Myr \citep{Chen:1997tb,Gutermuth:2005hx}. \cite{Gutermuth:2009gca} have identified $>170$ YSOs surrounding the core and measured their continuum fluxes up to \SI{8}{\micro\meter} with IRAC. While measurements at longer wavelengths were able to provide estimates of the total mass of the cluster \citep[e.g. using IRAS,][388~$L_\odot$]{Molinari:1996td}, the measurements are confused in the densest region and it has not been possible to properly associate the far-IR emission with its short wavelength counterpart because of the small separation between IRAC-detected protostars. The IRAS point source was classified as a luminous class 0 protostar \citep{Bachiller:1996ja}, and its emission associated with the bright millimeter source MMS1 to the northwest of the core \citep{Chini:2001fa}. \cite{Beltran:2008gu} show strong evidence that this region has multiple generations of stars, and suggest that a group of low-mass stars first completed its main accretion phase, before setting the stage for the birth of new intermediate-mass stars at the core of this cluster.
%
%We have observed two fields within the cluster (see Fig.~\ref{fig:IRAS20050_RGB}), including the brightest core at $20^h 07^m 06.70^s +\ang{27;28;54.5}$. Multiple sources in the core can be distinguished in the IRAC maps, but the core appears extended in Spitzer MIPS at \SI{24}{\micro\meter}, and is identified as a single source with WISE. No good high resolution far-infrared continuum data longward of \SI{24}{\micro\meter} was available for this source.
%
%%Cite also: \citep{Kumar:2006jo} if we want to talk about multiple generations of star formation.
%
%\begin{figure*}
%\begin{center}
%\includegraphics[width=\textwidth]{Figures/RGB.png}
%\label{fig:RGB}
%\caption{
%%\textit{Left:} The two white squares correspond to the FORCAST fields that we observed in around the core of IRAS~20050+2720. The background RGB image is a composition of \textit{Spitzer} IRAC 8~$\um$ (red), \textit{Spitzer} IRAC 5.7 $\um$ (green) , and IRAC 3.6~$\um$ (blue). The dashed red square at the center of the image correspond to the core of the cluster, displayed in greater detail in  the picture to the right. \textit{Right:} This RGB picture shows the IRAC 1, 3, and 4 bands of the core at the center of Fig.~\ref{fig:IRAS20050_RGB}. The white contours represent the contours of the FORCAST 37~$\um$ maps, and the red circles show the FORCAST-identified point sources. An infrared nebulosity can be seen to the East of the core with physical projected size of about 0.015~pc, surrounding a cavity with slightly smaller size. The nebulosity and its cavity can be seen all the way up to 37~$\um$. The far-IR emission is mapped well onto the IRAC sources, except for SOF4, for which almost no emission can be seen at shorter wavelengths. SOF4 matches the location of the bright millimeter source MMS1 \citep{Chini:2001fa}.}
%}
%\end{center}
%\end{figure*}
%
%% \begin{figure*}
%% \begin{center}
%% \includegraphics[width=5in]{}
%% \label{fig:IRAS20050_RGB_core}
%% \caption{}
%% \end{center}
%% \end{figure*}
%
%
%[Include a discussion about de-reddening towards that region?]
%
%\subsection{NGC 2071}
%The NGC~2071 star-forming region is one of several active areas of star formation in the northern part of L1630 giant molecular cloud which is located at a distance of 390 pc \citep{A-T1982}. 
%NGC~2071 itself is a reflection nebula.
%The NGC~2071 infrared cluster, located about 4' north of the reflection nebula, is a region of intermediate mass star formation \citep{Strom1976, Persson1981, Butner1990}. Maps of the cloud in CO and its isotopomers \citep{Buckle2010} show a large scale clump with $\sim$1,000 M$_\odot$ associated with the cluster. Dust continuum emission at $\lambda$=0.85 and 1.3 mm peaks on center of the cluster extending ~1' in diameter containing ~30 M$_\odot$ of gas and dust \citep{Johnstone2001,Mitchell2001,Launhardt1996}. Emission from CS in the J=2-1 through J=7-6 indicate that the gas in this region is centrally condensed with a density of ~10$^6$ cm$^{-3}$ \citep{Zhou1990}. 
%
%There are a number of near infrared surveys of the young cluster \citep[e.g.,][]{Strom1976,Lada1991,Megeath2012,Spezzi2015}. \cite{Spezzi2015} identify 52 YSOs associated with the NGC~2071 cluster, with the majority Class II sources. \cite{Flaherty2008} estimate an age of $\sim$2 Myr for the cluster, consistent with the large fraction of Class II sources (\cite{Evans2009}. The brightest far infrared emission from the cluster is associated with the IRS1 region \citep{Harvey1979,Butner1990}, which has an estimated total luminosity of 520~L$_\odot$. The immediate region of IRS 1 is, in fact, home to a number of YSOs that are infrared, X-ray, and radio sources \citep{Skinner2009,C-G2012,Kempen2012}. The radio \citep{C-G2012} and H$_2$ emission line imaging indicate that IRS~1, IRS~2, IRS~3, and, perhaps, VLA~1 are YSOs with outflows. The larger scale molecular outflow associated with this region is well studied in a number of molecules \citep{Bally1982,Chernin1993,Stoji2008}.
%\begin{figure*}
%\begin{center}
%\includegraphics[width=\textwidth]{Figures/NGC2071_saturated_mosaic.png}
%\label{fig:n2071saturated}
%\caption{%NGC 2071 IRS~1 Region: The left panel show the Spitzer IRAC 3.1~$\um$ image. The right panel is the Herschel 70~$\um$ image. The green plus marks indicate the positions of IRS~1, IRS~2, IRS~3, and VLA~1. The inner red circle show the extend of the saturated region in the Spitzer MIPS 24~$\um$ image and the outer red circle encompasses the region strong affected by imaging artifacts. The gray circle on the lower right of the right panel is the resolution the 70~$\um$ image.
%NGC2071 seen with IRAC, MIPS and Herschel.}
%\end{center}
%\end{figure*}
%
%Figure \ref{fig:n2071overview} shows the Spitzer 3.1~$\mu$m image of the IRS~1 region on the left \citep[image from Spitzer Archive:][]{Megeath2012} and the Herschel 70~$\mu$m image on the right (image from Herschel Archive: Gould Belt Project, P.I. Andr\'e). The plus marks in both panels indicate the position of the brighter YSOs: IRS~1, IRS~2, IRS~3, IRS~4, and VLA~1. The inner red circle with a diameter of 26" indicates the extend of the saturated region in the Spitzer MIPS 24~$\mu$m image; the outer red circle, diameter 60", encompasses the region with strong imaging artifacts in the MIPS 24~$\mu$m image.
%The right panel shows Herschel 70~$\mu$m image which does not resolve the emission from IRS~1, IRS~2, IRS~3, and VLA~1. The centroid of the 24~$\mu$m and 70~$\mu$m emission is between IRS~1 and VLA~1 indicating that several of the sources are contributing to the total observed emission. Interferometric observations show that the millimeter wavelength dust emission is dominated by envelopes associated with IRS~1 and IRS~3, with estimated masses of 8.2 and 12.3~M$_\odot$ material, respectively \citep{Kempen2012}. The millimeter emission also reveals the presence of disks with radii $\le$100~AU associated with IRS~1 and IRS~3 \citep{Kempen2012}.
%
%The luminosities and masses of the individual source, IRS~1, IRS~2, IRS~3, and VLA~1, are not known. The Spectral Energy Distributions (SEDs) shortward of 10~$\mu$m support their identification as embedded YSOS \citep{Skinner2009}. \cite{Skinner2009} gives a clear discussion of the possibilities for IRS~1 and concludes that it is likely a mid-to late B~star. \cite{Kempen2012} find luminosities of 10, 3.4, and $\le$27~L$_\odot$ for IRS~1, 2, and 3, respectively, and stellar masses of $\le$1~M$_\odot$ for each, based on SED fitting. These masses and luminosities are not consistent with estimate of the total luminosity of the region of 520~L$_\odot$ \citep{Butner1990}. The far infrared images from Herschel reveal that IRS~1 alone does not totally dominate, as seen in Figure N; IRS~1, VLA~1, and IRS~3 likely make substantial contributions to the emission with lesser emission from IRS~2 and IRS~4.
%
%\section{SOFIA Observations}
%NGC 2071 and IRAS 200050+2720 were observed with the FORCAST instrument on SOFIA as part of a survey of selected nearby ($\le$1~kpc) bright star-forming clusters which show high protostellar density \citep{Gutermuth:2009gca} and are located in the northern hemisphere. The survey focussed on the clusters that contain one or more saturated or confused region in the archival \textit{Spitzer} and WISE data.
%
%\subsection{Data Acquisition}
%The IRAS 200050 and NGC~2071 observations were collected over three flights out of the 10-flight survey campaign. A summary is shown in Table~\ref{tab:obssummary}. Because the clusters are dominated by bright sources, the observations fit into small flight segments which filled gaps in the flight schedule. The source coordinate in Table~\ref{tab:obssummary} correspond to the centers of the green areas in Fig.~\ref{fig:IRAS20050_RGB} and Fig.~\ref{fig:n2071overview}. In each cluster, two fields separated by $\sim$ 3 arcminutes were observed to focus on two saturated regions. All fields were observed using the chop-and-node C2N mode from FORCAST, with 9-point dithering to reduce the flat field issues.
%
%%\capstartfalse
%%\begin{deluxetable*}{ccccccccccc}
%%\tablecaption{target list}
%%\tablenum{1}
%%\tablehead{\colhead{Cluster} & \colhead{RA} & \colhead{DEC} &  \colhead{Flight} & \colhead{Fields} & \colhead{Dist.} & \colhead{Time/Field} & \colhead{Sen{\_}11} & \colhead{Sen${\_}$19} & \colhead{Sen${\_}$31} & \colhead{Sen${\_}$37} \\
%%\colhead{} & \colhead{(deg)} & \colhead{(deg)} & \colhead{} & \colhead{}  & \colhead{(pc)} & \colhead{(s)} & \colhead{(Jy)} & \colhead{(Jy)} & \colhead{(Jy)} & \colhead{(Jy)}} 
%%\startdata
%%IRAS20050+2720 & 301.771 & 27.481 & F166, F131 & 2 & 700$^{1}$ & 253 & 0.026 & 0.039 & 0.068 & 0.127 \\
%%NGC2071 & 86.775 & 0.363 & F192 & 2 & 420$^{2}$ & 35 & 0.118 & 0.119 & 0.196 & 0.474
%%\enddata
%%\label{tab:obssummary}
%%\end{deluxetable*}
%%\capstarttrue 
%
%
%We observed the clusters in 4 bands: 11.1, 19.7, 31.5 and \SI{37.1}{\micro\meter}. The requested observation time in each band was calculated to detect YSOs with rising spectral energy distribution (SED) of same luminosity at the two distances of the clusters. We estimated integration time based on a rising-SED protostar model for $\sim 1.5\Lsun$. [HOW SHOULD I JUSTIFY THE FLUXES THAT WE SET AS OUR LIMITS?]
%
%Average observing time per field and average measured 1-sigma sensitivity levels are shown in Table~\ref{tab:obssummary} for the 4 bands. The measured background levels indicate the smallest detectable point source flux density at each wavelength, based on the noise measurements in the image itself. Bands 11.1 and \SI{37.1}{\micro\meter} were observed simultaneously using a dichroic filter, as were 19.7 and 31.5. However, in order to reach the required flux sensitivity for the \SI{37.1}{\micro\meter} band, we completed some of our observations with single-band observations at \SI{37.1}{\micro\meter}. 
%
%% \begin{longtable*}{ccccccc}
%% \hline
%% \hline
%% Cluster Name & Field & Field Coordinates & Band & Time (s) & Flight & Date \\
%% \hline
%
%% IRAS20050+2720 & 2 & 20h07m03s +27d30m38s & 11.1d & 135 & F131 & 2013-09-17 \\
%% IRAS20050+2720 & 2 & 20h07m03s +27d30m30s & 19.7d & 140 & F166 & 2014-05-02 \\
%% IRAS20050+2720 & 2 & 20h07m03s +27d30m47s & 31.5d & 160 & F166 & 2014-05-02 \\
%% IRAS20050+2720 & 2 & 20h07m04s +27d31m02s & 37.1d & 135 & F131 & 2013-09-17 \\
%% IRAS20050+2720 & 3 & 20h07m06s +27d28m12s & 11.1d & 135 & F166 & 2014-05-02 \\
%% IRAS20050+2720 & 3 & 20h07m06s +27d28m05s & 19.7d & 84 & F166 & 2014-05-02 \\
%% IRAS20050+2720 & 3 & 20h07m06s +27d28m13s & 31.5d & 96 & F166 & 2014-05-02 \\
%% IRAS20050+2720 & 3 & 20h07m06s +27d28m06s & 37.1d & 126 & F166 & 2014-05-02 \\
%% \hline
%
%% NGC2071 & 1 & 05h47m07s +00d17m49s & 11.1d & 18 & F192 & 2015-02-05 \\
%% NGC2071 & 1 & 05h47m07s +00d18m02s & 19.7d & 11 & F192 & 2015-02-05 \\
%% NGC2071 & 1 & 05h47m06s +00d17m55s & 31.5d & 11 & F192 & 2015-02-05 \\
%% NGC2071 & 1 & 05h47m07s +00d18m03s & 37.1d & 21 & F192 & 2015-02-05 \\
%% NGC2071 & 2 & 05h47m07s +00d21m30s & 11.1d & 18 & F192 & 2015-02-05 \\
%% NGC2071 & 2 & 05h47m07s +00d21m31s & 19.7d & 18 & F192 & 2015-02-05 \\
%% NGC2071 & 2 & 05h47m07s +00d21m34s & 31.5d & 24 & F192 & 2015-02-05 \\
%% NGC2071 & 2 & 05h47m07s +00d21m34s & 37.1d & 21 & F192 & 2015-02-05 \\
%
%% \caption{List of observations}
%% \end{longtable*}
%
%%Make sure to mention the distances picked and the references for it, as well as the cluster's age estimate
%%Need to find references to cite for details about each region
%\subsection{Data reduction}
%The data are processed through various versions of the online pipeline to yield Level 2 data products available on the archive \citep{Herter:2013by}. We apply our own reduction procedure and photometry pipeline on those products to derive final images, source positions, fluxes and sensitivities. The software utilizes the Python \textit{astropy} package \citep{2013A&A...558A..33A} and its associated modules \textit{photutils} and \textit{APLpy}. 
%
%\subsubsection{Pre-treatment}
%Some manual treatment of each image is necessary before it can be analyzed by our software, which follows this procedure: a) visually aligning the WCS coordinate system, often 10-20" off, using point sources and archival data from other wavelengths and facilities such as IRAC \SI{8}{\micro\meter}; b) cropping the images to clean off the nodded fields, and c) identify the coordinates of each source, both point-like and extended.
%
%After these manual steps, the Level 2 images are multiplied by the calibration factor provided by the online pipeline, which converts them to Jy/pixel. We do not proceed to any systematic color correction, but the effects on the fluxes are very small \citep{Herter:2013by}.
%\begin{comment}
%\begin{enumerate}
%\item Adjust WCS coordinates: use images at other wavelengths (2MASS, IRAC, MIPS, WISE) to re-align the (RA, DEC) position of the field. We estimate that this process is good to within one SOFIA pixel (0.768") for the fields where one or more point sources can be identified. Extended fields are less trustworthy, since matching the extended emission to other wavelengths is harder. The rotation of the field produced by the SOFIA pipeline is correct for all of our data. 
%\item Crop each image, remove chopped fields, remove artifacts.
%\item Identify and categorize sources: isolated point sources, clustered point sources, and extended sources. For extended sources, a circular or elliptical aperture is used to try to encompass the entirety of the emission.
%\item Manually identify a location in the field that corresponds to a representative background.
%\end{enumerate}
%\end{comment}
%\subsubsection{Source flux extraction and calibration}
%\begin{comment}
%We feed the adjusted FITS and associated metadata files to a photometry pipeline. The pipeline first processes all the calibrator stars that are observed during each flight, with each filter setting, and derives a new aperture correction factor for each image, based on an aperture of 4 pixels radius (3.072") and an aperture of 12 pixels radius. We consider that the latter aperture contains all the flux from a given point source.
%
%We distinguish between 3 types of sources after manual identification: \textit{isolated}, which are point sources with no nearby objects; \textit{clustered}, which are point sources with nearby objects; and \textit{extended}, which are not consistent with being point sources. [THIS PARAGRAPH MAY GO AWAY IF WE DON'T WANT TO TALK ABOUT GENERALITIES ABOUT THE PIPELINE]
%
%
%For point sources, we use our standard aperture of 4 pixels at all wavelengths. We consider an annulus surrounding the source extending from 12 to 20 pixels radius (24 to 40 for clustered sources): the local background is determined from the mode of the pixels in the annulus, while the sensitivity is calculated by measuring the standard deviation of 4-pixel apertures within that annulus [Cite Taro's paper and the Herschel photometry paper that Tracy gave us]. We apply the aperture correction derived from the calibrator observations taking during that flight.
%
%For extended sources, an elliptical aperture is determined manually from the \SI{37}{\micro\meter} images. The local background is determined from the mode of an elliptical annulus, with an inner boundary at the elliptical aperture and an outer boundary corresponding to an ellipse 20\% larger. The sensitivity quoted is the point source sensitivity, and is determined following the same method as for point sources, using the standard deviation of apertures spread across the elliptical annulus. [NO NEED TO MENTION THIS SINCE WE MIGHT NOT TREAT EXTENDED SOURCES]
%
%The photometry from sources that were observed in different flights is then combined to increase the signal-to-noise ratio. This combination takes into account the sensitivity of each source by appropriately weighing each image.
%
%Although we can compute source sensitivities based on the local noise, and we use the calibrators each flight to determine the aperture correction, observations with SOFIA have additional noise from the water vapor overburden and air mass, as well as from the flat field variations. These noise contributions usually amount to much higher than the sensitivities estimated from the local background, and we follow the recommendation from \citep{Herter:2012hv} to adopt a 20\% uncertainty for our flux estimates. 
%
%\end{comment}
%
%\subsection{Photometry at other wavelengths}
%
%\subsection{Instrumental Characterization}
%[THIS WHOLE DISCUSSION MIGHT FIT BETTER IN THE PAPER ABOUT THE WHOLE SURVEY, RATHER THAN THIS PAPER WHICH IS JUST ABOUT TWO CLUSTERS...]
%The three flights discussed here were part of the total of 10 data flights for the entire survey. The larger context of the entire survey allowed us to follow the evolution of the instrument properties throughout the two years of science observations. We discuss three metrics: the evolution of the aperture correction factor, the evolution of the instrument's residuals, and the evolution of the PSF size, through half width at half max of the encircled energy distribution, that we call $\Rfifty$. This is done in an attempt to assess the uncertainties in our flux determination, and our confidence in determining that sources are point-like or extended. %This study is based  on the large number of calibrator observations during our 10 science flights.
%
%\subsubsection{PSF size}
%
%Fig~\ref{fig:average_EE} shows the average of the normalized encircled energy distribution of the PSF, measured on all the calibrators of our sample. Each curve represents one of the five different combinations of bandpass filter and dichroic setting that we use for our observations: \SI{11.1}{\micro\meter}, \SI{19.7}{\micro\meter}, \SI{31.5}{\micro\meter}, \SI{37.1}{\micro\meter} with dichroic and \SI{37.1}{\micro\meter} without a dichroic (open). 
%
%As expected, the PSF at \SI{37.1}{\micro\meter} is larger than the PSFs at shorter wavelengths, but less that the traditional diffraction limit rule. This indicates that additional PSF smearing is occurring at short wavelengths, likely due to plane jitter and pointing errors. This was predicted and mentioned in the SOFIA Observer's Handbook. Throughout all the flights, the largest 1-sigma error occurs for the \SI{37}{\micro\meter} observations at about XX\%. CONCLUDE ON OUR ABILITY TO DETERMINE WHETHER A SOURCE IS EXTENDED OR NOT.
%
%To look at the behavior of the PSF in more detail, we can use the half width at half maximum of the encircled energy distribution, $\Rfifty$ as a proxy for PSF size. The variation of this quantity for the various flights, bandpass/dichroic setting, and calibrators used is showed in Fig~\ref{fig:Rfifty_dist}. This shows the flight-to-flight differences and, for some calibrators, the in-flight variability. We find that the latter is usually NN\% or less, except for the SOFIA flight on 05-02-2014, for which the spread is quite considerable. The variation from flight to flight is larger than the variation within a given flight, which indicates variability in the observing conditions, systematics, or thermal radiation environment of the observatory between different flights. Hence, we conclude that the extension of a source can be best determined by comparing $\Rfifty$ for that source with $\Rfifty$ for the corresponding calibrator measurement for that flight and dichroic setting, to within NN\%, 1-sigma confidence.
%\begin{figure}
%\begin{center}
%
%\includegraphics[width=0.45\textwidth]{Figures/average.png}
%\label{fig:average_EE}
%\caption{PSF size distribution}
%\end{center}
%\end{figure}
%
%\begin{figure}
%\begin{center}
%
%\includegraphics[width=0.45\textwidth]{Figures/fwhs.png}
%\label{fig:Rfifty_dist}
%
%\caption{PSF size distribution}
%\end{center}
%\end{figure}
%
%\subsubsection{Aperture correction}
%In Fig~\ref{fig:response}, we plot the aperture correction factor that we compute from the ratio of the flux measured within an aperture of 4 pixels, divided by the flux measured in an aperture of 12 pixel radius, which we consider to be encompassing the total flux in the source. Not surprisingly, this graph follows very closely the plot of $\Rfifty$ shown in Fig~\ref{fig:Rfifty_dist}, showing the close link between the aperture correction factor and the shape of the calibrator's PSF. For the aperture correction variability, we adopt a XX, 1-sigma uncertainty value on the flux estimate.
%
%
%
%\subsubsection{Instrument response}
%To further study the variability of the observing, we can look at the detector response and the aperture correction evolution after applying the calibration factor. In the ideal conditions, the calibration factor always leads to the same flux measurement of the calibrator source, within the pipeline's systematic errors and residual noise. The detector response is measured here by simply applying aperture photometry on the calibrators to measure their fluxes, using the same local background subtraction as the one described in the previous sections. Calibrator stars are good ways to correct for telescope and atmospheric variability, as their far-infrared fluxes are not expected to vary significantly over any relevant timescale. %We adopt a value of XX, 1-sigma value for the response variability, effectively representing the uncertainties in the observatory and the atmosphere.
%In Fig~\ref{fig:response}, we plot all the measured fluxes of the calibrators. The flux variability from flight to flight for a given calibrator is small (typically [CALCULATE THIS]), and the variability within the same flight is even smaller [QUANTIFY]. We adopt a XX, 1-sigma uncertainty value for the absolute flux measurement. This is the residual uncertainty after applying the systematic correction produced by the SOFIA FORCAST online pipeline.
%
%
%
%\begin{figure}
%\begin{center}
%
%\includegraphics[width=0.45\textwidth]{Figures/Phot_val.png}
%\label{fig:response}
%\caption{Instrumental response and aperture correction}
%
%\end{center}
%\end{figure}
%
%\begin{figure}
%\begin{center}
%\includegraphics[width=0.45\textwidth]{Figures/Aper_corr.png}
%\label{fig:aper_corr}
%
%\caption{Instrumental response and aperture correction}
%\end{center}
%\end{figure}
%
%%WHAT IS THE BOTTOM LINE FROM THIS DISCUSSION? IS A FLUX MEASUUREMENT LIMITED BY VARIATIONS IN THE PSF IF IT IS BRIGHT? SOMETHING ELSE? IT SEEMS LIKE THE CONCLUSION FROM THIS SECTION SHOULD BE A STATEMENT ABOUT THE SYSTEMATIC ERRORS ON ANY QUOTED FLUX MEASUREMENT AND A STATEMENT ABOUT LIMITATIONS ON KNOWING WHETHER A SOURCE IS EXTENDED RELATIVE TO A POINT SOURCE.
%\section{Observational results}
%\subsection{IRAS 200050}
%
%SOFIA photometry results, IRAC photometry issues and results; looking at the sources that are fit by guthermuth, we find a 10\% agreement between our photometry results and theirs.
%
%\subsubsection{Photometry results and maps}
%
%\subsubsection{SEDs}
%
%
%\subsubsection{Upper limits on other sources in the field}
%
%
%\begin{figure}
%\begin{center}
%\includegraphics[width=1\textwidth]{Figures/IRAS20050_SEDs.png}
%\label{fig:IRAS20050_SEDs}
%\caption{}
%\end{center}
%\end{figure}
%
%
%
%
%\subsection{NGC2071}
%
%
%\subsubsection{Photometry results and maps}
%
%\begin{figure}
%\begin{center}
%\includegraphics[width=1\textwidth]{Figures/NGC2071_mosaic.png}
%\label{fig:NGC2071_mosaic}
%\caption{The core of IRAS20050+2720 is seen in the four bands of the \textit{Spitzer} IRAS instrument, as well as with the four FORCAST bands. The increased resolution of FORCAST compared to previous instruments allows to match the long-wavelength emission with its short wavelength counterpart. The stretch in each image is adjusted for optimal readability. The white contours correspond to the FORCAST \SI{37}{\micro\meter} emission [mention the contour levels]. The red circles indicate the location of the five FORCAST point sources in the core.}
%\end{center}
%\end{figure}
%
%\begin{figure}
%\begin{center}
%\includegraphics[width=1\textwidth]{Figures/NGC2071_SEDs.png}
%\label{fig:NGC2071_SEDs}
%\caption{}
%\end{center}
%\end{figure}
%
%\subsubsection{SEDs}
%
%\subsubsection{Upper limits on other sources in the field}
%
%
%\subsection{Summary}
%
%Put here the table of photometry + flags
%
%
%
%\section{SED and dust modeling}
%
%YE05 assumed the dust opacities of Ossenkopf \& Henning
%(1994) appropriate for thin ice mantles after 105 year of coagulation at a gas density of 106 cm-3 (OH5 dust), which several recent studies have shown to be appropriate for cold, dense cores (e.g., Evans et al. 2001; Shirley et al. 2002; Young et al. 2003; Shirley et al. 2005) [LOOK AT REST OF DISCUSSION IN DUNHAM et al. 2010]
%
%\subsection{Radiative transfer models}
%
%We use radiative transfer models to fit the SEDs we observe and extract physical parameters. We explored the tool by \cite{Robitaille:2006cb} as a starting point for this process. While the tool provides results that fit the data well, the large number of parameters makes it difficult to draw meaningful conclusions on the real physics behind the observations. We have observed a large amount of cross-correlations between the model parameters, as well as many local minimas in the $\chi^2$ minimization scheme that is used.
%
%In an effort to understand the dependence of the observations with the physical parameters used in the models, we use the HYPERION radiative transfer code \citep{Robitaille:2011fc} and create our own grid of models by varying a small amount of meaningful parameters. HYPERION is a very versatile code with lots of options for various geometries, but we simplify the problem to its most essential components: a circularly symmetric disk with a power-law envelope.
%
%We explore the various geometries and parameters that are available in the code, and come to following conclusions:
%\begin{enumerate}
%\item Modeling accretion through an $\alpha$-disk instead of a flared disk is equivalent to increasing the central luminosity by an appropriate amount. Hence, we use only standard flared disks and quote a total central luminosity
%\item It is good enough to only vary the total luminosity of the central object, instead of varying its mass, radius and temperature
%\item Models are very insensitive to disk mass, when the envelope's mass is larger
%\item The model is sensitive to the envelope's mass distribution within a given radius, but not sensitive to the envelope's inner or outer radius
%\item The outer radius of the disk has no effect on the models
%\item The inner radius of the envelope 
%\end{enumerate}
%
%Based on our exploration of HYPERION, we proceed with the following simulation set up. We use a density structure composed of a standard flared disk of \SI{e-3}{\Msun} that extends from the dust sublimation radius out to 50~AU. The scale height at the dust sublimation radius is set to be 10\% of that radius. The disk's flaring power is a constant set to $\beta=1.1$. 
%
%We add an envelope with a central cavity. The envelope extends from 30~AU out to a variable radius, and has a variable mass and power law exponent. The inner cavity has a constant 25 degree opening angle. Setting this latter parameter has some effect that is correlated with the viewing angle.
%
%All elements in our model are using the same dust model by \cite{Draine:2003di}. We have experimented with various other types of dust models, notably with the OH5 dust [REFERENCE], as it was suggested in [CITE TRACY]. We found that the fits with the OH5 dust were much worse. In most cases, it was particularly difficult to fit the \SI{10}{\micro\meter} silicate absorption feature. 
%
%We use a variable foreground extinction also with the same dust model, as we anticipate that most of the extinction along the line of sight will occur in the cluster itself. We chose to not use any ambient medium, as they complicated the interpretation of the results.
%
%We run our simulations using $10^5$ photons, and spherical grids with 199 radial cells, 49 meridional cells, and 1 single azimuthal cell. We have tested these various parameters and sought an optimum of fast computing times and low statistical noise. With this, the calculated uncertainties are a few percent at long wavelengths and can be on the order of 10 to 15 percent at short wavelengths. This is acceptable since the short wavelength range is largely used in the fit to determine the overall external extinction magnitude. At the wavelenghts relevant for the IRAC fluxes, the uncertainties due to the simulation are on the order of 5\% - less than our estimated measurement error. With these parameters, a typical model takes about 5-10 minutes to run on a standard desktop computer. Our wrapper software allows us to run multiple different grids in a row and merge them into one single entity that we can feed to a minimization routine.
%
%In order to fit the data, we use the $\chi^2$ method described in \cite{Robitaille:2007dl}, with the exception of the overall optimal scaling step. We calculate the $\chi^2$ for our targets with all calculated models, inclinations, and a range of values of foreground extinction magnitudes.
%
%
%%Table of set parameters]
%
%%variable parameters: envelope mass, envelope density power law, source luminosity, inclination, extinction
%
%
%
%
%%This section discusses how we set up a grid of models to fit, and why we moved away from Robitaille's sedfitter. Maybe show some results from the Robitaille's fits?
%
%\subsection{Extracting physical parameters from the fits}
%This section discusses the results from the fits: best fits, parameter estimation and variation about the best fits, color-color diagrams, etc.
%\begin{itemize}
%\item \citep{Shirley:2000gh}: typical masses of protostars are a few tenths to \SI{2}{\Msun}. 
%
%\end{itemize}
%
%
%\section{Discussion}
%
%\section{Conclusion}
%blabla
%
%\begin{landscape}
%\renewcommand{\arraystretch}{1}
%
%\tiny
%
%\begin{longtable}{lllrrrrrrrrrrrrrrrrrrrrrrrrrrrrrrrrrrrrrrrrrrrrrrrrrrrrrrr}
%\toprule
%{} &               Coordinates &   Property &         j &       e\_j &         h &       e\_h &        ks &      e\_ks &        i1 &      e\_i1 &        i2 &      e\_i2 &         i3 &      e\_i3 &         i4 &      e\_i4 &        F11 &     e\_F11 &        F19 &     e\_F19 &        m1 &      e\_m1 &         F31 &      e\_F31 &         F37 &      e\_F37 &        m2 &      e\_m2 &         H70 &      e\_H70 &        H160 &     e\_H160 &        H250 &      e\_H250 &        H350 &      e\_H350 &       H500 &     e\_H500 &   S850 &  e\_S850 &  F1100 &  e\_F1100 &  S1300 &  e\_S1300 &       R37 &       Lbol &     alpha &         R &   env\_mass &  env\_mass\_std &  calc\_mass &       sLsun &  sLsun\_std &       Lbol &        inc &  ext &     s \\
%SOFIA\_name  &                           &            &           &           &           &           &           &           &           &           &           &           &            &           &            &           &            &           &            &           &           &           &             &            &             &            &           &           &             &            &             &            &             &             &             &             &            &            &        &         &        &          &        &          &           &            &           &           &            &               &            &             &            &            &            &      &       \\
%\midrule
%\endhead
%\midrule
%\multicolumn{3}{r}{{Continued on next page}} \\
%\midrule
%\endfoot
%
%\bottomrule
%\endlastfoot
%Oph.12      &  16h26m34.0s -24d23m40.7s &   Extended &  0.000700 &  0.000700 &  0.000900 &  0.000900 &  0.001500 &  0.002000 &        -- &        -- &        -- &        -- &         -- &        -- &         -- &        -- &   1.622251 &  0.234589 &  49.015377 &  3.439944 &        -- &        -- &  185.704775 &  13.021490 &  244.997024 &  17.197742 &        -- &        -- &   38.022248 &  16.367280 &   31.874880 &  31.874880 &  160.940056 &  160.940056 &  119.104440 &  119.104440 &  63.397558 &  63.397558 &     -- &      -- &     -- &       -- &     -- &       -- &  2.969460 &  10.280410 &  3.369796 &  1.086848 &   0.076000 &      0.025482 &         -- &   39.899998 &   5.133392 &  10.280410 &  18.671719 &   14 &  0.70 \\
%NGC1333.2   &  03h29m10.3s +31d21m55.5s &   Extended &  0.285291 &  0.028529 &  0.653883 &  0.065388 &  0.901012 &  0.090101 &  0.637000 &  0.063700 &  0.446000 &  0.044600 &   0.448000 &  0.079800 &   0.912525 &  0.127753 &   8.414433 &  0.595773 &  36.516910 &  2.562018 &        -- &        -- &  106.489879 &   7.456661 &  135.723202 &   9.507282 &        -- &        -- &   70.038748 &   7.003875 &   77.573890 &  20.036312 &   87.660813 &   15.014373 &   51.505937 &   16.114088 &  24.741970 &  13.061772 &     -- &      -- &     -- &       -- &     -- &       -- &  2.232105 &  27.831752 &  1.242967 &  1.773727 &  22.167999 &      9.901331 &         -- &    7.700000 &   1.112940 &  27.831752 &   0.000000 &    2 &  0.70 \\
%NGC1333.8   &  03h29m03.7s +31d16m03.9s &   Isolated &  0.034236 &  0.003424 &  0.140832 &  0.014083 &  0.360023 &  0.036002 &  0.904000 &  0.090400 &  0.359000 &  0.068700 &   2.750000 &  0.331000 &   5.690000 &  0.569000 &   4.125824 &  0.292108 &  25.042361 &  1.757174 &        -- &  0.434000 &   99.119629 &   6.940325 &  106.511012 &   7.463216 &    125.00 &    12.500 &  187.822184 &  18.782218 &  228.025009 &  22.802501 &  156.558418 &   15.655842 &   90.190510 &   14.084944 &  42.435647 &  13.650262 &     -- &      -- &  2.700 &   0.2700 &     -- &       -- &  0.770085 &  35.106406 &  1.144899 &  1.061540 &   1.946000 &      0.745726 &   2.019725 &   17.000000 &   2.389399 &  35.106406 &   0.000000 &   13 &  1.30 \\
%Oph.3       &  16h27m09.4s -24d37m18.3s &   Isolated &  0.000700 &  0.000700 &  0.038967 &  0.003897 &  0.928818 &  0.092882 &        -- &  0.021700 &        -- &        -- &  12.800000 &  1.350000 &         -- &  0.031200 &  20.696882 &  1.450637 &  57.194587 &  4.005008 &        -- &  0.480000 &   83.549816 &   5.856017 &   87.378670 &   6.163625 &     49.10 &     4.910 &   84.839941 &   8.483994 &   65.551102 &   6.804876 &   36.490880 &    9.759635 &   16.302506 &    9.272575 &   7.565188 &   7.565188 &  0.410 &  0.0410 &  0.057 &   0.0057 &  0.095 &   0.0095 &  0.992850 &  13.394719 &  0.574003 &  1.543482 &   0.004000 &      0.001816 &   0.040478 &   85.000000 &  19.741953 &  13.394719 &   0.000000 &   14 &  1.00 \\
%Oph.7       &  16h27m28.0s -24d39m33.8s &   Isolated &  0.000700 &  0.000700 &  0.003460 &  0.000346 &  0.047025 &  0.004703 &  0.731000 &  0.097100 &  1.830000 &  0.183000 &   2.940000 &  0.294000 &   2.320000 &  0.252000 &   2.927179 &  0.224177 &  30.721915 &  2.162581 &        -- &        -- &   70.940218 &   4.972795 &   70.480142 &   4.956635 &     34.70 &     3.470 &   65.516629 &   6.551663 &   44.630179 &   4.463018 &   37.363205 &   24.113419 &   23.657531 &   22.485618 &  15.313136 &  15.313136 &  0.360 &  0.0360 &  0.360 &   0.0360 &  0.060 &   0.0060 &  0.970612 &   6.474091 &  1.353049 &  1.392024 &   0.015000 &      0.002340 &   0.025565 &   26.579000 &   3.493491 &   6.474091 &  71.591522 &   14 &  0.70 \\
%Oph.8       &  16h27m37.2s -24d30m34.8s &   Isolated &  0.094643 &  0.009464 &  0.305000 &  0.030500 &  0.618202 &  0.061820 &  1.410000 &  0.141000 &  1.600000 &  0.160000 &   4.060000 &  0.406000 &   6.000000 &  0.600000 &   3.535605 &  0.258554 &  23.107697 &  1.620863 &        -- &        -- &   39.950535 &   2.798894 &   51.640811 &   3.625942 &     20.80 &     2.080 &   27.932258 &   2.793226 &   13.925723 &   2.408777 &    5.603603 &    5.603603 &    5.408533 &    5.408533 &         -- &         -- &  0.180 &  0.0180 &     -- &       -- &  0.060 &   0.0060 &  1.015303 &   5.017779 &  0.545431 &  1.133202 &   0.007000 &      0.001843 &   0.025565 &   17.716999 &   3.387279 &   5.017779 &  77.846802 &   12 &  0.70 \\
%Oph.11      &  16h26m59.2s -24d35m00.2s &   Extended &  0.000700 &  0.000700 &  0.000900 &  0.000900 &  0.001500 &  0.002000 &  0.143000 &  0.017000 &  0.487000 &  0.060400 &   0.564000 &  0.433000 &   0.553000 &  0.134000 &   0.839171 &  0.088616 &  12.550480 &  0.891950 &  5.640000 &  0.790000 &   40.712799 &   2.852379 &   47.672216 &   3.345748 &     47.60 &     4.760 &  101.124889 &  10.112489 &   34.731077 &   7.937487 &   19.808539 &    4.897385 &   19.421774 &    7.187933 &  17.293798 &   6.872309 &     -- &      -- &     -- &       -- &     -- &       -- &  2.600471 &   4.034299 &  2.039836 &  0.776784 &   0.034000 &      0.010207 &         -- &   11.900000 &   1.886536 &   4.034299 &  77.846802 &   14 &  0.70 \\
%NGC1333.9   &  03h28m55.6s +31d14m36.6s &   Isolated &  0.000700 &  0.000700 &  0.000900 &  0.000900 &  0.001500 &  0.002000 &  0.001050 &  0.000105 &  0.012600 &  0.001260 &   0.022700 &  0.002270 &   0.030217 &  0.003022 &   0.122895 &  0.070532 &   1.213009 &  0.134313 &  5.194206 &  0.519421 &   32.151835 &   2.262214 &   45.738104 &   3.238440 &        -- &        -- &  210.068777 &  21.006878 &  336.489802 &  33.648980 &  180.769192 &   18.076919 &   91.496328 &   18.612656 &  38.205550 &  17.115160 &     -- &      -- &  2.300 &   0.2300 &     -- &       -- &  0.799509 &  24.282985 &  2.793514 &  2.624300 &   2.919000 &      0.353591 &   1.720506 &   17.000000 &   2.412459 &  24.282985 &  18.671719 &   14 &  0.85 \\
%Oph.1       &  16h27m10.3s -24d19m12.9s &   Isolated &  0.506394 &  0.050639 &  1.017419 &  0.101742 &  1.368770 &  0.136877 &  1.300000 &  0.130000 &  1.170000 &  0.117000 &   1.280000 &  0.128000 &   1.660000 &  0.166000 &   1.780982 &  0.156661 &  12.230152 &  0.869434 &        -- &  0.113000 &   25.509892 &   1.788463 &   29.200074 &   2.093071 &     16.60 &     1.660 &   23.562913 &   2.356291 &   15.325085 &   1.532508 &    6.082145 &    4.223516 &    4.014879 &    4.014879 &   3.230003 &   3.230003 &  0.400 &  0.0400 &     -- &       -- &  0.095 &   0.0095 &  0.916298 &   3.633096 &  0.268865 &  0.666547 &   0.010000 &      0.001820 &   0.040478 &    7.875000 &   1.273325 &   3.633096 &  77.846802 &    3 &  0.70 \\
%NGC1333.11  &  03h28m37.1s +31d13m30.0s &   Isolated &  0.000700 &  0.000700 &  0.001183 &  0.000153 &  0.010128 &  0.002000 &  0.030100 &  0.003010 &  0.089200 &  0.008920 &   0.267000 &  0.026700 &   0.006630 &  0.028300 &   0.143330 &  0.156646 &   4.398746 &  0.323481 &  2.180000 &  0.739000 &   20.890854 &   1.483130 &   23.313519 &   1.649508 &        -- &        -- &   58.802758 &   5.880276 &   63.081860 &   6.308186 &   38.422454 &    3.842245 &   22.023228 &    2.202323 &   9.683953 &   1.694426 &     -- &      -- &  0.360 &   0.0360 &     -- &       -- &  1.016810 &   7.468708 &  1.693932 &  0.987321 &   0.384000 &      0.181414 &   0.269297 &    7.700000 &   0.802750 &   7.468708 &  18.671719 &   14 &  0.70 \\
%IRAS20050.3 &  20h07m06.3s +27d28m56.6s &  Clustered &  0.002014 &  0.000201 &  0.005833 &  0.000583 &  0.027947 &  0.002795 &  0.090188 &  0.009019 &  0.218415 &  0.021842 &   0.338759 &  0.033876 &   0.428667 &  0.042867 &   0.180523 &  0.059050 &   2.575909 &  0.265602 &        -- &        -- &   12.532365 &   0.940287 &   19.334657 &   1.412497 &        -- &        -- &          -- &         -- &          -- &         -- &          -- &          -- &          -- &          -- &         -- &         -- &     -- &      -- &     -- &       -- &     -- &       -- &  1.998293 &  12.813751 &  1.134668 &  0.732251 &   0.256000 &      0.113978 &         -- &   48.450001 &   6.340800 &  12.813751 &  26.525352 &    5 &  1.00 \\
%Oph.4       &  16h27m02.5s -24d37m27.6s &   Extended &  0.003443 &  0.000344 &  0.065933 &  0.006593 &  0.396580 &  0.039658 &  1.400000 &  0.140000 &  1.970000 &  0.197000 &   5.030000 &  0.503000 &  10.700000 &  1.460000 &   5.894930 &  0.417953 &   8.756463 &  0.618932 &        -- &        -- &   16.278797 &   1.143412 &   17.928454 &   1.308147 &     29.10 &     2.910 &   60.526791 &   7.333157 &   28.022688 &   2.802269 &          -- &          -- &          -- &          -- &         -- &         -- &     -- &      -- &  0.260 &   0.0260 &     -- &       -- &  1.801189 &   4.503086 &  0.185629 &  2.224143 &   0.004000 &      0.000439 &   0.066181 &   14.299999 &   2.702848 &   4.503086 &  37.863647 &   14 &  1.30 \\
%NGC1333.10  &  03h28m57.4s +31d14m15.0s &   Isolated &  0.000700 &  0.000700 &  0.000900 &  0.000900 &  0.001500 &  0.002000 &  0.031600 &  0.003160 &  0.104000 &  0.010400 &   0.262000 &  0.026200 &   0.370036 &  0.037004 &   0.116106 &  0.100473 &   1.594437 &  0.170151 &  4.650709 &  0.465071 &   10.144191 &   0.751700 &   15.630996 &   1.139252 &        -- &        -- &   27.238466 &   2.723847 &   30.473024 &   6.811199 &   53.872778 &   16.332824 &   54.379025 &   20.295716 &  32.772184 &  20.160709 &     -- &      -- &  0.600 &   0.0600 &     -- &       -- &  0.799389 &   4.822243 &  1.831622 &  1.155856 &   0.256000 &      0.178401 &   0.448828 &    5.600000 &   0.929834 &   4.822243 &  18.671719 &   14 &  0.70 \\
%IRAS20050.4 &  20h07m05.9s +27d28m59.2s &  Clustered &  0.000700 &  0.000700 &  0.000900 &  0.000900 &  0.001500 &  0.002000 &  0.023476 &  0.002743 &  0.039114 &  0.003911 &   0.052865 &  0.007526 &   0.054601 &  0.007638 &   0.059474 &  0.053307 &   0.251180 &  0.196063 &        -- &        -- &    8.537096 &   0.795371 &   12.845401 &   1.251667 &        -- &        -- &          -- &         -- &          -- &         -- &          -- &          -- &          -- &          -- &         -- &         -- &     -- &      -- &     -- &       -- &  1.800 &   0.2000 &  2.087205 &  14.943041 &  1.709257 &  0.266312 &   0.384000 &      0.324725 &  19.173673 &   48.450001 &   8.820868 &  14.943041 &  42.536900 &    5 &  1.30 \\
%IRAS20050.2 &  20h07m06.2s +27d28m49.1s &  Clustered &  0.000700 &  0.000700 &  0.000900 &  0.000900 &  0.001500 &  0.002000 &  0.040636 &  0.004064 &  0.142323 &  0.014232 &   0.263840 &  0.026384 &   0.308258 &  0.030826 &   0.059474 &  0.055310 &   1.449680 &  0.191045 &        -- &        -- &    9.309298 &   0.715988 &   11.959660 &   1.193237 &        -- &        -- &          -- &         -- &          -- &         -- &          -- &          -- &          -- &          -- &         -- &         -- &     -- &      -- &     -- &       -- &     -- &       -- &  2.280466 &   8.026514 &  1.643030 &  0.771839 &   0.577000 &      0.216318 &         -- &   26.600000 &   6.046117 &   8.026514 &  18.671719 &   14 &  0.70 \\
%NGC1333.1   &  03h29m07.7s +31d21m57.0s &   Isolated &  0.001240 &  0.000124 &  0.003087 &  0.000309 &  0.044950 &  0.004495 &  0.696000 &  0.069600 &  1.800000 &  0.180000 &   3.060000 &  0.306000 &   2.550000 &  0.255000 &   0.225415 &  0.169458 &   1.501840 &  0.207759 &        -- &  0.260000 &    6.886420 &   0.639683 &   10.993506 &   0.947935 &     49.30 &     4.930 &   52.723662 &   5.272366 &   66.528595 &  35.196775 &   71.541448 &   14.257565 &   45.559105 &   17.857228 &  24.263509 &  16.300584 &     -- &      -- &  1.300 &   0.1300 &     -- &       -- &  0.746145 &   8.384582 &  0.279856 &  3.398432 &   0.004000 &      0.004675 &   0.972460 &   32.500000 &   7.795440 &   8.384582 &  50.833290 &   14 &  1.30 \\
%NGC1333.3   &  03h29m01.5s +31d20m20.5s &   Isolated &  0.000801 &  0.000080 &  0.002865 &  0.000286 &  0.029562 &  0.002956 &  0.544000 &  0.054400 &  1.090000 &  0.109000 &   1.690000 &  0.211000 &   3.060000 &  0.306000 &   1.680629 &  0.130992 &   6.901880 &  0.493400 &        -- &  0.069000 &    9.255659 &   0.655862 &    9.406444 &   0.694632 &     23.40 &     2.340 &   20.217963 &   2.021796 &   78.315869 &   7.831587 &  101.472321 &   18.943177 &   70.906602 &   17.370831 &  40.866774 &  11.474091 &     -- &      -- &  1.500 &   0.1500 &     -- &       -- &  0.904212 &   8.104207 &  0.713901 &  3.393613 &   0.004000 &      0.027190 &   1.122069 &    3.500000 &   2.055202 &   8.104207 &   0.000000 &   14 &  1.15 \\
%Oph.6       &  16h27m15.7s -24d38m45.8s &   Isolated &  0.000700 &  0.000700 &  0.000900 &  0.000900 &  0.001500 &  0.002000 &  0.000653 &  0.001670 &  0.029700 &  0.002970 &   0.037400 &  0.011300 &   0.059200 &  0.022800 &   0.430812 &  0.140062 &   4.305059 &  0.331922 &  0.993000 &  0.355000 &    8.538799 &   0.616541 &    8.179484 &   0.625100 &  14000.00 &  1400.000 &   11.849532 &   1.184953 &    9.622496 &   4.264276 &   10.224295 &   10.224295 &   10.464819 &   10.464819 &   8.968175 &   8.968175 &  0.180 &  0.0180 &     -- &       -- &  0.047 &   0.0047 &  1.294131 &   0.767300 &  2.540989 &  0.934818 &   0.001000 &      0.001320 &   0.020026 &   26.579000 &   6.414977 &   0.767300 &  90.000000 &   13 &  0.70 \\
%Oph.2       &  16h26m44.2s -24d34m48.2s &   Isolated &  0.000700 &  0.000700 &  0.003208 &  0.000321 &  0.015231 &  0.002000 &  0.239000 &  0.023900 &  0.744000 &  0.074400 &   1.610000 &  0.161000 &   2.240000 &  0.224000 &   0.785942 &  0.102364 &   3.190894 &  0.294129 &        -- &  0.065300 &    7.548146 &   0.601857 &    7.941138 &   0.766174 &      8.12 &     0.812 &    8.830506 &   0.883051 &   10.640690 &   1.882673 &    8.527568 &    5.283352 &   10.097091 &    7.482095 &   8.455704 &   8.455704 &  0.220 &  0.0220 &     -- &       -- &  0.120 &   0.0120 &  0.925523 &   1.190757 &  0.828853 &  2.083560 &   0.001000 &      0.002054 &   0.051130 &   32.274502 &  21.100348 &   1.190757 &  83.957672 &   14 &  1.00 \\
%IRAS20050.1 &  20h07m06.6s +27d28m48.0s &  Clustered &  0.002113 &  0.000211 &  0.042102 &  0.004210 &  0.213959 &  0.021396 &  0.489159 &  0.048916 &  0.570047 &  0.057005 &   0.731244 &  0.073124 &   0.857892 &  0.085789 &   0.643347 &  0.067657 &   1.932588 &  0.202346 &        -- &        -- &    4.497465 &   0.352530 &    6.323872 &   0.593341 &        -- &        -- &          -- &         -- &          -- &         -- &          -- &          -- &          -- &          -- &         -- &         -- &     -- &      -- &     -- &       -- &     -- &       -- &  0.000000 &  14.907880 &  0.071313 &  0.741234 &   0.004000 &      0.000000 &         -- &  128.000000 &  15.305788 &  14.907880 &  65.098938 &    9 &  0.85 \\
%Oph.9       &  16h27m21.8s -24d29m53.7s &   Isolated &  0.000700 &  0.000700 &  0.000900 &  0.000900 &  0.031127 &  0.003113 &  0.467000 &  0.046700 &  0.925000 &  0.092500 &   1.440000 &  0.144000 &   1.730000 &  0.173000 &   1.167176 &  0.118828 &   3.424540 &  0.304219 &  4.360000 &  0.436000 &    6.027531 &   0.494175 &    5.806116 &   0.729619 &      5.11 &     0.521 &    4.564878 &   0.456488 &    3.191735 &   1.066859 &          -- &          -- &          -- &          -- &         -- &         -- &  0.030 &  0.0030 &     -- &       -- &  0.020 &   0.0020 &  0.000000 &   0.990687 &  0.505343 &  2.076775 &   0.001000 &      0.000000 &   0.008522 &   11.815999 &   1.159392 &   0.990687 &  80.915283 &   14 &  0.70 \\
%IRAS20050.5 &  20h07m06.6s +27d28m53.1s &  Clustered &  0.001892 &  0.000189 &  0.010053 &  0.001005 &  0.042455 &  0.004246 &  0.118129 &  0.011813 &  0.176229 &  0.017623 &   0.235331 &  0.023533 &   0.319871 &  0.031987 &   0.186287 &  0.051583 &   1.030396 &  0.212319 &        -- &        -- &    2.969365 &   0.325861 &    5.644845 &   0.651528 &        -- &        -- &          -- &         -- &          -- &         -- &          -- &          -- &          -- &          -- &         -- &         -- &     -- &      -- &     -- &       -- &     -- &       -- &  2.066712 &   5.844922 &  0.536668 &  0.781840 &   0.010000 &      0.002597 &         -- &   49.361000 &   6.241276 &   5.844922 &  42.536900 &   14 &  1.00 \\
%Oph.10      &  16h27m17.5s -24d28m55.0s &   Isolated &  0.000700 &  0.000700 &  0.001752 &  0.000175 &  0.015500 &  0.002000 &  0.127000 &  0.012700 &  0.206000 &  0.020600 &   0.286000 &  0.028600 &   0.268000 &  0.026800 &   0.164212 &  0.069534 &   0.668142 &  0.203877 &  0.780000 &  0.078000 &    1.718547 &   0.203886 &    3.740913 &   0.488908 &     12.20 &     1.220 &    6.927097 &   0.692710 &   18.971498 &   2.276556 &   15.879686 &    8.555977 &   11.114462 &   11.114462 &         -- &         -- &  0.093 &  0.0093 &  0.230 &   0.0230 &  0.010 &   0.0010 &  1.260146 &   0.563149 &  0.503234 &  1.378907 &   0.003000 &      0.000601 &   0.004261 &    5.000000 &   1.609805 &   0.563149 &  80.915283 &   14 &  1.00 \\
%Oph.5       &  16h27m06.8s -24d38m15.4s &   Isolated &  0.000700 &  0.000700 &  0.001958 &  0.000196 &  0.027260 &  0.002726 &  0.240000 &  0.024000 &  0.416000 &  0.041600 &   0.553000 &  0.055300 &   0.695000 &  0.069500 &   0.335569 &  0.065278 &   1.179738 &  0.156206 &  2.790000 &  0.279000 &    2.357097 &   0.231135 &    3.054466 &   0.429695 &      6.07 &     0.607 &    4.098763 &   1.070658 &    7.596331 &   3.955141 &    5.783070 &    5.783070 &    3.851165 &    3.851165 &   2.548819 &   2.548819 &  0.100 &  0.0100 &  0.160 &   0.0160 &  0.075 &   0.0075 &  1.307895 &   0.544752 &  0.311926 &  1.355876 &   0.001000 &      0.000000 &   0.031956 &    4.250000 &   0.463816 &   0.544752 &  80.915283 &   14 &  0.85 \\
%NGC1333.4   &  03h29m11.1s +31d18m30.8s &   Isolated &  0.000700 &  0.000700 &  0.000900 &  0.000900 &  0.001500 &  0.002000 &  0.000718 &  0.000072 &  0.003940 &  0.000394 &   0.005000 &  0.000500 &   0.004328 &  0.000433 &   0.097076 &  0.060302 &   0.076183 &  0.115364 &  0.607462 &  0.060746 &    1.784878 &   0.208835 &    3.039983 &   0.341298 &        -- &        -- &   16.609251 &   1.660925 &   53.689046 &   5.368905 &   57.215194 &    6.292760 &   38.448845 &    6.032870 &  18.593682 &   4.666282 &     -- &      -- &  2.000 &   0.2000 &     -- &       -- &  1.102551 &   3.055830 &  1.864111 &  0.830834 &   2.919000 &      0.446509 &   1.496092 &    2.331000 &   0.350655 &   3.055830 &  18.671719 &   11 &  0.70 \\
%NGC1333.7   &  03h28m43.4s +31d17m34.8s &   Isolated &  0.000700 &  0.000700 &  0.000900 &  0.000900 &  0.001500 &  0.002000 &  0.178000 &  0.017800 &  0.351000 &  0.035100 &   0.488000 &  0.048800 &   0.771000 &  0.077100 &   0.702686 &  0.072989 &   1.608865 &  0.177482 &  1.880000 &  0.188000 &    2.083729 &   0.234452 &    2.613522 &   0.346819 &      2.08 &     0.208 &    1.674428 &   0.167443 &    2.587022 &   0.590856 &    7.939325 &    2.613854 &    7.999988 &    3.217143 &   6.634023 &   2.677956 &     -- &      -- &     -- &       -- &     -- &       -- &  1.189289 &   1.356686 &  1.076204 &  1.832949 &   0.001000 &      0.000995 &         -- &    9.562500 &   1.758680 &   1.356686 &  58.243137 &    0 &  0.70 \\
%Oph.16      &  16h26m24.1s -24d24m48.3s &   Isolated &  0.057081 &  0.005708 &  0.333503 &  0.033350 &  0.784026 &  0.078403 &  0.755000 &  0.202000 &  2.100000 &  0.210000 &   2.440000 &  0.244000 &   2.690000 &  0.269000 &   1.780024 &  0.196095 &   1.821645 &  0.268900 &  2.390000 &  0.239000 &    1.602580 &   0.298194 &    2.370660 &   0.566745 &        -- &        -- &    3.481016 &   0.711042 &   18.760602 &  18.760602 &  150.022181 &  150.022181 &  103.094646 &  103.094646 &  51.331864 &  51.331864 &     -- &      -- &     -- &       -- &     -- &       -- &  1.797880 &   2.175843 & -0.755982 &  1.869389 &   0.001000 &      0.000000 &         -- &   17.716999 &   2.923584 &   2.175843 &  77.846802 &   10 &  0.70 \\
%NGC1333.5   &  03h29m10.6s +31d18m19.6s &   Isolated &  0.000700 &  0.000700 &  0.000900 &  0.000900 &  0.001500 &  0.002000 &  0.001840 &  0.000184 &  0.006690 &  0.000669 &   0.010300 &  0.001030 &   0.010900 &  0.001090 &   0.114497 &  0.092701 &   0.150284 &  0.118763 &  0.771000 &  0.077100 &    1.946230 &   0.233640 &    2.166457 &   0.377374 &     20.60 &     2.060 &   14.626733 &   1.462673 &   49.867882 &   4.986788 &   52.535950 &    6.166298 &   36.231726 &    6.188723 &  18.007443 &   4.878490 &     -- &      -- &  2.000 &   0.2000 &     -- &       -- &  1.623470 &   2.786059 &  1.704848 &  0.765793 &   1.297000 &      0.326535 &   1.496092 &    1.258000 &   0.261528 &   2.786059 &  18.671719 &   14 &  0.85 \\
%IRAS20050.7 &  20h07m07.9s +27d27m15.8s &   Isolated &  0.000700 &  0.000700 &  0.000900 &  0.000900 &  0.001500 &  0.002000 &  0.003867 &  0.000387 &  0.024170 &  0.002417 &   0.059890 &  0.005989 &   0.072165 &  0.007216 &   0.059474 &  0.052650 &   0.111262 &  0.057550 &  0.659962 &  0.065996 &    1.146484 &   0.143348 &    2.087927 &   0.311746 &        -- &        -- &          -- &         -- &          -- &         -- &          -- &          -- &          -- &          -- &         -- &         -- &     -- &      -- &     -- &       -- &  0.300 &   0.0600 &  0.000000 &   3.057660 &  1.273086 &  1.499519 &   0.015000 &      0.085220 &   3.195612 &    3.500000 &   3.609952 &   3.057660 &   0.000000 &   14 &  0.70 \\
%Oph.13      &  16h27m30.1s -24d27m43.3s &   Isolated &  0.001186 &  0.000119 &  0.025369 &  0.002537 &  0.163806 &  0.016381 &  0.740000 &  0.074000 &  1.190000 &  0.119000 &   1.580000 &  0.158000 &   2.040000 &  0.204000 &   1.382168 &  0.139436 &   1.806742 &  0.186310 &  1.720000 &  0.172000 &    2.014207 &   0.226905 &    1.984322 &   0.550785 &      7.04 &     0.704 &    4.928438 &   0.492844 &   37.384324 &  13.912175 &   62.016241 &   11.593029 &   48.591685 &   16.514025 &  36.513717 &  13.273109 &  1.200 &  0.1200 &  0.620 &   0.0620 &  0.020 &   0.0020 &  0.000000 &   1.491269 & -0.373531 &  2.225736 &   0.001000 &      0.000000 &   0.008522 &   17.716999 &   5.501830 &   1.491269 &  80.915283 &   14 &  0.70 \\
%NGC1333.6   &  03h29m13.0s +31d18m13.8s &   Isolated &  0.000700 &  0.000700 &  0.000900 &  0.000900 &  0.001506 &  0.002000 &  0.045600 &  0.004560 &  0.180000 &  0.018000 &   0.274000 &  0.027400 &   0.320000 &  0.032000 &   0.160283 &  0.034904 &   0.570316 &  0.092861 &  0.735000 &  0.073500 &    1.445675 &   0.180093 &    1.805938 &   0.344682 &      4.29 &     0.429 &    3.882886 &   3.882886 &   13.332256 &  13.332256 &   29.105069 &    6.271950 &   34.780780 &    8.007092 &  21.255056 &   6.628205 &     -- &      -- &  0.630 &   0.0630 &     -- &       -- &  0.950532 &   1.495050 &  1.001035 &  1.205061 &   0.001000 &      0.000698 &   0.471269 &    7.500000 &   1.188587 &   1.495050 &  26.525352 &   14 &  1.30 \\
%Oph.15      &  16h27m29.4s -24d39m16.6s &   Isolated &  0.000700 &  0.000700 &  0.009026 &  0.000903 &  0.077541 &  0.007754 &  0.172000 &  0.017200 &  0.271000 &  0.027100 &   0.402000 &  0.040200 &   0.411000 &  0.041100 &   0.681783 &  0.112742 &   1.527776 &  0.367109 &  0.639000 &  0.264000 &    2.012071 &   0.344345 &    1.631132 &   0.599339 &        -- &        -- &    3.615509 &   0.361551 &    9.898856 &   4.256740 &   20.072302 &   15.115748 &   18.185662 &   18.185662 &  15.017925 &  15.017925 &  0.180 &  0.0180 &     -- &       -- &  0.045 &   0.0045 &  1.250547 &   0.565569 &  0.098897 &  1.122236 &   0.004000 &      0.000699 &   0.019174 &    3.330000 &   0.410340 &   0.565569 &  26.525352 &   14 &  1.00 \\
%Oph.17      &  16h26m23.6s -24d24m39.4s &   Isolated &  0.001519 &  0.000152 &  0.012856 &  0.001286 &  0.054241 &  0.005424 &  0.297000 &  0.029700 &  0.431000 &  0.043100 &   0.561000 &  0.056100 &   0.619000 &  0.061900 &   0.452223 &  0.231609 &   0.968562 &  0.471099 &  1.010000 &  0.101000 &    1.410939 &   0.394304 &    1.457149 &   0.449171 &        -- &        -- &    3.772271 &   0.478031 &   46.248139 &  46.248139 &  144.052854 &  144.052854 &   95.975629 &   95.975629 &  52.648627 &  52.648627 &     -- &      -- &     -- &       -- &     -- &       -- &  0.955404 &   1.286765 & -0.087406 &  1.208607 &   0.001000 &      0.000000 &         -- &    5.250000 &   0.646469 &   1.286765 &  80.915283 &   14 &  0.70 \\
%Oph.19      &  16h26m30.5s -24d22m59.9s &   Isolated &  0.000700 &  0.000700 &  0.000900 &  0.000900 &  0.001500 &  0.002000 &  0.286000 &  0.028600 &  0.371000 &  0.037100 &   0.438000 &  0.043800 &   0.466000 &  0.046600 &   0.168926 &  0.157077 &   0.637756 &  0.124966 &  0.639000 &  0.161000 &    0.597403 &   0.399042 &    1.297142 &   0.884549 &        -- &        -- &          -- &         -- &   59.165540 &  59.165540 &  101.675272 &   48.875106 &   91.340358 &   91.340358 &  67.219849 &  50.880915 &  0.330 &  0.0330 &     -- &       -- &  0.020 &   0.0020 &  2.510181 &   1.241395 &  0.571762 &  0.893497 &   0.001000 &      0.000633 &   0.008522 &    5.250000 &   0.997850 &   1.241395 &  74.742477 &   14 &  0.70 \\
%Oph.18      &  16h26m17.2s -24d23m45.1s &   Isolated &  0.000700 &  0.000700 &  0.000900 &  0.000900 &  0.008386 &  0.002000 &  0.045300 &  0.004530 &  0.087700 &  0.008770 &   0.161000 &  0.016100 &   0.237000 &  0.023700 &   0.300678 &  0.148358 &   0.672159 &  0.191628 &  0.516000 &  0.051600 &    1.201269 &   0.274837 &    1.286545 &   0.539335 &        -- &        -- &    1.559830 &   1.559830 &          -- &         -- &   12.703012 &   12.703012 &   15.149902 &   15.149902 &         -- &         -- &  0.210 &  0.0210 &     -- &       -- &  0.085 &   0.0085 &  1.180654 &   0.281605 &  0.628201 &  1.341163 &   0.003000 &      0.003573 &   0.036217 &    2.830500 &   0.919428 &   0.281605 &  80.915283 &   14 &  0.85 \\
%IRAS20050.6 &  20h07m02.2s +27d30m26.0s &   Isolated &  0.077284 &  0.007728 &  0.094689 &  0.009469 &  0.155428 &  0.015543 &  0.537046 &  0.053705 &  0.770690 &  0.077069 &   1.112551 &  0.111255 &   1.805481 &  0.180548 &   1.809239 &  0.132991 &   2.292125 &  0.169155 &  2.288333 &  0.228833 &    1.635956 &   0.136628 &    1.215817 &   0.380688 &        -- &        -- &          -- &         -- &          -- &         -- &          -- &          -- &          -- &          -- &         -- &         -- &     -- &      -- &     -- &       -- &     -- &       -- &  1.404580 &  19.327547 & -0.372075 &  2.221856 &   0.004000 &      0.000000 &         -- &  201.599991 &  32.128288 &  19.327547 &  80.915283 &   14 &  0.70 \\
%Oph.14      &  16h27m28.4s -24d27m21.1s &   Isolated &  0.000807 &  0.000096 &  0.012198 &  0.001220 &  0.060748 &  0.006075 &  0.187000 &  0.018700 &  0.272000 &  0.027200 &   0.382000 &  0.038200 &   0.444882 &  0.044488 &   0.520242 &  0.115900 &   0.872317 &  0.234499 &  0.717000 &  0.071700 &    0.966333 &   0.298191 &    1.003625 &   0.487493 &  11000.00 &  1100.000 &    5.991281 &   0.757259 &   35.124328 &   8.697466 &   69.349126 &    8.553092 &   59.827719 &   10.566376 &  34.294671 &   9.572711 &  1.700 &  0.1700 &  0.950 &   0.0950 &  0.050 &   0.0050 &  1.894093 &   0.950774 & -0.131346 &  1.002063 &   0.001000 &      0.000943 &   0.021304 &    4.250000 &   0.559817 &   0.950774 &  80.915283 &   14 &  0.70 \\
%\end{longtable}
%\end{landscape} 
\chapter{The Balloon Experimental Twin Telescope for Infrared Interferometry}
\label{chap:BETTII}

\section{Towards higher angular resolution in the far-IR}
Observations at mid- to far-infrared wavelengths from the Earth's surface are extremely 
limited by the large atmospheric opacity in this region of the spectrum. Space-based telescopes 
like IRAS \citep[12-100 \um;][]{1984ApJ...278L...1N}, ISO \citep[2.5-240 $\um$;][]{1996A&A...315L..27K}, \textit{Spitzer} \citep[3.6-160 $\um$;][]{2004ApJS..154....1W}, AKARI  \citep[1.7-180 $\um$;][]{2007PASJ...59S.369M}, WISE \citep[3.4-22 $\um$;][]{2010AJ....140.1868W} and \textit{Herschel} \citep[55-672 $\um$;][]{2010A&A...518L...1P} have demonstrated the scientific value of observations at 
these wavelengths; but the spatial resolution of space-based observatories is limited by the cost 
and complexity of building and flying progressively larger aperture telescopes. 

High-altitude platforms are a good compromise between ground and space observatories: while less sensitive because of the surrounding thermal emission from the atmosphere and the instrument components which are at ambient temperature, they can still feature larger optics and payloads, more experimental setups, and instrumentation that can be changed on a more frequent and significantly less costly basis.

BETTII is an experiment that aims at breaking from the single-aperture paradigm by using interferometry between 30 and \SI{110}{\micro\meter} from a balloon platform. Interferometry is commonly used on the ground at other wavelengths such as optical and radio, and is a viable path forward to obtain much higher resolution than what single apertures can reasonably provide.

BETTII is founded on a particular technique called \textit{spatio-spectral interferometry} \citep{Mariotti:1988vea}, which is a way to achieve 
high angular and moderate spectral resolutions at far-IR wavelengths, without the cost and limitations of large single apertures. 


%\subsection{The stratospheric balloon environment}
%
%Mention briefly the conditions at balloon altitude: temperature, pressure, cosmic rays, pendulum modes; required shock resistance; picture of the launch; 


\section{BETTII description}

%The Ballon Experimental Twin Telescope for Infrared Interferometr (BETTII) project is pioneering a new technique that could lead to dramatically increased spatial resolution in the far-infrared: spatio-spectral interferometry. 
As a cryogenic payload flying at an altitude of \SI{37}{\kilo\meter} (\num{120000}~ft), BETTII is the first flying "direct detection" interferometer: it will attempt to coherently combine light from two different telescopes to provide increased angular resolution. Because it is operating from above most the atmosphere, it can see the far-infrared universe between 30 and 110 \si{\micro\meter}, and provide \ang{;;0.5}-\ang{;;1} spatial resolution at these wavelengths - a key region of parameter space well-suited to study protostars evolving in dense clustered environments.

To provide this resolution (which matches that of \JWST  at \SI{25}{\micro\meter}), BETTII needs to be have two collectors separated by $\sim$~\SI{8}{\meter}; because of its operating wavelength, it needs to have a cryogenic instrument; because it is an interferometer, it needs optics with exquisite surface quality; and because it flies on a balloon platform, it needs to be robust to large changes in temperature, large pointing errors, and severe shock resistance for the landing phase.

This chapter will first discuss the basics of double-Fourier interferometers, then present the general design of BETTII payload and most of its subsystems.


\section{Basics of interferometry}

Since the end of the 19th century, scientists have learned how to use the wave properties of light to learn about new astrophysical phenomena. It did not take long for what first started as a laboratory experiment by \citet{Michelson:1887wc} to be applied to astronomy, with the Michelson Stellar Interferometer experiment. 

The principle of interferometry is simple. Because light behaves like a wave, two beams of light coming from the same source can be combined \textit{coherently}, provided that their amplitudes and phases are controlled. The intensity of the combined signal is a function of the brightness of the light beam, and the relative phase and wavefront of each beam. Changes in the relative phase create a modulation of that brightness.

\citet{Michelson:1887wc} created what became the standard Michelson interferometer (Fig.~\ref{fig:michelson}). It uses one single source of light and a 50/50 beam splitter that creates two coherent light beams from that one source. The two light beams go through two separate \textit{arms} before being recombined. While adjusting the length of one arm with respect to the other, we modulate the phase difference between the two arms, leaving everything else the same. This creates a modulation called an \textit{interferogram}, which describes the measured intensity variation as a function of the phase difference between the two arms.

\begin{figure}[!h]
\centering
\includestandalone[width=\textwidth]{Figures/michelson}
\caption[Michelson interferometer]{\textit{Left}: Schematics of a Michelson interferometer. \textit{Right}: Intensity modulation resulting from the mirror linear motion. The center of the modulation, called "ZPD" for zero path difference, is the precise location of the mirror where the distance is equal in each arm.}
\label{fig:michelson}
\end{figure}


The phase difference is expressed in radians and depends on the wavelength of the light that is used. In this work, we will usually refer to this difference in terms of an actual physical distance: the optical path difference (\OPD). This has the advantage of being wavelength-independent and relate more easily to opto-mechanical considerations.




\subsection{Fourier transform spectroscopy}

One immediate consequence of the original Michelson experiment is to realize that the interferogram actually contains spectral information. For an ideal monochromatic source, the interferogram depends on the \OPD only modulo a wavelength. This means that the modulation is identical whether we introduce an $\OPD = \lambda$, or $\OPD = n\lambda$, where $n$ is an integer. This is because the monochromatic wave can essentially be represented by an amplitude times a cosine function of phase (a cosine function of $2\pi\OPD/\lambda$).

\begin{figure}[!ht]
	\centering
	\includestandalone[width=\textwidth]{Figures/interferogram}
	\caption[Simple interferogram]{An ideal interferogram here is shown as a sum of cosine waves of different frequencies.}
	\label{fig:interferogram}
    \end{figure}


The interferogram for a given wavelength is a cosine wave, with an amplitude related to the intensity of the signal, and a period equal to the wavelength of the incident light.

If we consider a polychromatic signal as a sum of monochromatic wavelengths, this phenomenon happens for each single wavelength, and the resulting intensity modulations add \textit{coherently}: the total intensity is the coherent sum of the intensity modulations created by each individual wavelength (see Fig.~\ref{fig:interferogram}). This has the effect of smearing the resulting modulation in most places except around the precise location where the \OPD is zero (which is called \textit{ZPD}). Around this location, a modulation is always seen. This is commonly referred to as \textit{white light fringes}, where a \textit{fringe} represents one wavelength of the interferogram. The range of OPD in which fringes can be seen is called the \textit{coherence length} \Lc. When all wavelengths are weighted equally in a bandpass $\Delta\lambda$, the coherence length can be expressed as:
\begin{equation}
\Lc = \frac{\lambda^2}{\Delta\lambda},
\end{equation}
and the interferogram can be represented by a carrier frequency modulated by an envelope function.

%[add equation for the integral of interferograms]
%\begin{equation}
%\sum_{\lambda_i}\I_i = \frac{\lambda^2}{\Delta\lambda},
%\end{equation}

Since the modulation is a coherent superposition of cosine waves, it contains spectral information. A cosine transform of the interferogram will decompose the contribution of each individual wavelength, hence reproducing the spectrum of the polychromatic source. This technique, called "Fourier Transform Spectroscopy", has led to many scientific discoveries in astronomy, chemistry and other fields over the last 100 years.

\subsection{Aperture synthesis}


\begin{figure}[!h]
	\centering
	\includestandalone[width=\textwidth]{Figures/interferometer}
	\caption[Michelson Stellar interferometer]{Schematics of a Michelson Stellar interferometer. Two sources are shown at the top of the picture. The red source is off-axis by an angle $\theta$. Since it is infinitely far away, its wavefront is essentially planar as it reaches us. The two siderostats sample the identical wavefront at different points, but because of the incidence angle, the light in the left arm travels slightly more path than the light in the right arm. As a result, the interferogram from that source will be shifted, since the position of ZPD is now offset by this extra distance the light has to cross. On the other hand, light from the blue source, which is perfectly on axis, produces an interferogram which has a ZPD at the nominal position. Those two intensity modulations co-add in the detector plane, and the sum is shown in black. By observing these summed interferograms over multiple baseline angles and distances, one can reconstruct entirely the spatio-spectral scene.}
	\label{fig:interferometer}
    \end{figure}


An interferogram is produced by coherently combining photons from one single source of light. This can be applied for example for an infinitely distant astronomical source: as the light propagates from the source, by the time it reaches our instrument the radius of curvature of its wavefront is extremely large, and the latter can be approximated as being flat. The photons from this source nominally enter each arm of the interferometer with the same phase, when the alignment is perfect. When combined, these photons interfere and create an interferogram.

However, let's suppose that a second source is sufficiently far away from the first source that its wavefront enters the interferometer at an angle (see the red source in Fig.~\ref{fig:interferometer}). This means the photons from the second source enter one arm slightly later than the other - photons need to cross over more optical path in one arm than in the other. These photons would also create an interferogram, but the latter will be centered about a different position in \OPD  space than the interferogram created by the photons from the first source. Now let's suppose that the second source is exactly as bright as the first one, and that it is apart from the first by an angle $\theta$ such that $\baseline\cdot\hat{\vectors{s}} = |\B|\sin\theta = \lambda/2$, where $\hat{\vectors{s}}$ is a unity vector representing the line of sight of the telescope, and $\B$ is the baseline vector projected on the plane of the sky. In this case, the interferogram created by the photons from the second source has the same amplitude as the first interferogram, but is shifted by half a wavelength in \OPD. As a result, the two (monochromatic) interferograms would exactly cancel each other, and we would say that the visibility (sometimes referred to as the \textit{complex degree of coherence} \citep[e.g.][]{Mariotti:1988vea}) between the two sources is zero. Although the sources are not coherent in the strict sense because they are completely independent sources, the interferograms caused by each source would, in this case, cancel out. If the angular separation was such that $|\B|\sin\theta = \lambda$, then the modulations would add up and the resulting modulation would have twice the amplitude of that with just one single source. We would say that the visibility between the two sources is unity. In the bottom right of Fig.~\ref{fig:interferometer}, we show the addition of two polychromatic interferograms (in blue and red), adding up to the measured curve in black. By measuring the curve in black, we know that there are two sources along our baseline vector. The spatial resolution of the interferometer is its ability to resolve nearby sources directly in the interferogram space - in other words, it is its ability to resolve fringe packets. Usually, this spatial resolution is equal to $\theta\sim\lambda/(2B)$. A summary of the relevant planes used in aperture synthesis is shown in Fig~\ref{fig:aperturesynthesis}.

\begin{figure}[!h]
	\centering
	\includestandalone[width=\textwidth]{Figures/aperturesynthesis}
	\caption[Aperture synthesis]{Relevant planes in the optical train for aperture synthesis, inspired by Fig. 3.14 in \cite{Glindemann:2011hk}. Three sources are shown on the sky on the left. The second relevant plane is the entrance aperture or pupil plane, which is the 2D Fourier transform of the source plane, which is also called the u,v-plane. The interferometer samples two apertures in this complex plane at each given baseline length and orientation with respect to the sky. In the third plane, we show that an optical delay in introduced between each sub-aperture in the pupil plane, and the pupils are overlapped. Finally, an image is formed out of the overlapped pupils, which is shown in the detector plane.}
	\label{fig:aperturesynthesis}
    \end{figure}



One way to formalize the concept of spatial coherence is to consider an interferometer with a given baseline length and angle as a filter of the source's spatial distribution on the sky. For a given baseline length and angle with respect to the sky, the interferometer is only sensitive to a single angular frequency in a single direction on the sky (as well as the total power). Various sources observed simultaneously by the interferometer will all contribute to a single measured interferogram, which can be characterized in terms of the complex visibility between the sources for a given baseline angle and length. 

The generalization of this property is called the Van Cittert-Zernike theorem \citep{Zernike:1938kq}: the 2D Fourier transform of the intensity distribution on the sky is its complex visibility function. In other words, by mapping the complex visibility (through measuring interferograms) for all baseline angles and lengths, we can reconstruct the original image through an inverse Fourier transform. The plane of complex visibilities is commonly referred to as the (u,v)-plane \citep{Thompson:2008ww}.

Interferometry and aperture synthesis is used commonly at radio wavelengths, where coherent detectors can retain the direct phase of the incoming light by mixing the signal with a local oscillator. Both the amplitude and the phase of the signal can be recorded for each antenna, and can be combined with all the other antennas at a later time.

Aperture synthesis has also been achieved at optical and near-infrared wavelengths, where a nearby guide star is used to determine a reference phase of the incoming beam \citep[e.g.][]{Monnier:2004fd,Gillessen:2010fo}. The interferograms measured for the science sources can then be non-ambiguously aligned with each other. This process requires very rapid imaging capabilities (on the order of \SI{10}{\milli\second}, a typical atmospheric coherence timescale, see discussion in \cite{Mariotti:1988vea}) to freeze the atmospheric variations across the synthetic aperture. This requires bright guide stars. In addition, because of the large baselines, the field of view is very limited, so the targets accessible by optical interferometers are limited to scientific sources which are a few arcseconds of a bright guide star \citep{Glindemann:2000bf}: this dramatically limits the capabilities of ground-based interferometry at these wavelengths.


\subsection{Double-Fourier interferometry}


In this work, we introduce the concept of Double-Fourier interferometry, which uses a standard Fourier Transform Spectrometer at the back-end of a Michelson stellar interferometer (see Fig.~\ref{fig:FTSvsDoubleFourier} and \citet{Mariotti:1988vea}). 

%In the far-infrared, coherent-detection interferometers are a possibility [ESPRIT], but they presently lack sensitivity and may be fundamentally less efficient than direct-detection arrays. This takes advantage of recently-developed Transition Edge Sensor bolometer arrays, which are direct-detection, power sensors in the far-infrared (which means that we do not have access to the phase information). Hence phase referencing during flight will have to be achieved very carefully.

We adopt a Michelson interferometer configuration with pupil-plane combination. Unlike image-plane combination, where fringes are seen across a single Airy disk in the image plane, no fringes are visible across the field of view at a given \OPD. Instead, the intensity of the entire field of view is modulated as a function of \OPD. 

By scanning the \OPD, we obtain a modulation of each pixel on the detector, which contains information on both the spectral (through the Fourier transform of the scan) and the spatial (through the amplitude and phase of the fringe packet) characteristic of the source, at that baseline orientation and length. By repeating the measurement over a full range of baseline angles and lengths, one can unambiguously retrieve both the spatial and spectral content of the astronomical scene by filling the synthetic aperture. 

Pupil-plane combination allows for an interferometric response of the entire field of view. The price we pay is that the \OPD scans need to be longer in order to cover enough range, going through ZPD for each pixel in the field of view. For a single-pixel detector, the \OPD scan would only need to cover enough stroke to obtain the desired spectral resolution for the one single pixel.



\begin{figure}[!ht]
	\centering
	\includestandalone[width=\textwidth]{Figures/FTSvsDoubleFourier}
	\caption[FTS vs Double-Fourier]{Standard FTS telescope layout (left) versus double-Fourier telescope layout (right) \citep{Mariotti:1988vea}.}
	\label{fig:FTSvsDoubleFourier}
    \end{figure}

A detailed derivation of the equations of Double-Fourier interferometry starting from first principles is presented, in Appendix~\ref{ap:interfero} and Chapter~\ref{chap:phasenoisepaper}.

\section{BETTII Instrument design}

%This section summarizes the design and architecture of BETTII, including discussion of the expected sensitivity of its different channels. All key subsystems are discussed here, except the control system, which is discussed in greater detail in Chapters~\ref{chap:controls}~and~\ref{chap:implementation}.


The BETTII payload is an \SI{8}{\meter} fixed-baseline interferometer, equipped with two \SI{50}{\centi\meter} siderostats. It operates in two wavelength bands, 30-55~\si{\micro\meter} and 55-110~\si{\micro\meter}. In these two bands, its theoretical angular resolution is $\sim$\ang{;;0.5} and $\sim$\ang{;;1}, respectively. This is significantly better than all existing or previous facilities that operate in the far-infrared, which are limited by the mirror size. In addition, this matches the resolution of JWST at \SI{25}{\micro\meter}, hence providing the ability to probe astrophysical phenomena at longer wavelength with the same angular resolution.

There are four major components to BETTII: the mechanical structure and design; the optics and their mounts; the cryostat and the detectors; and the control system. The latter will be discussed extensively in Chapter \ref{chap:controls}. In this section, we first describe the balloon environment and its constraints, before discussing these four major BETTII components. 

\subsection{Stratospheric balloon environment}

High-altitude balloons have for many years served as a test platforms for future space instruments, such as the FIRAS instrument on COBE \citep{Fixsen:2002jv}. These balloon platforms fly between 30 and \SI{40}{\kilo\meter}, above more than 99\% of the atmosphere, which make them particularly well suited for studying the universe at infrared, far-infrared and sub-millimeter wavelengths. Balloon launches occur year-round across multiple continents, including Antarctica. NASA and other agencies organize these campaigns for various areas of science.

For a typical launch, the scientific payload is attached on the bottom of a train of about \SI{100}{\meter} that includes a parachute and a ladder. The top of the ladder attaches to the bottom of the large helium-filled balloon. 
\begin{figure}[!ht]
	\centering
	\includegraphics[width=\textwidth]{Figures/balloonLaunch.jpg} 
	\caption[Balloon launch]{Picture of a balloon launch. The payload is captured by the launch vehicle (in yellow) until the balloon is inflated and released. The parachute assembly, which is a part of the long train from the top of the payload to the bottom of the balloon, can be seen in red. Credit: NASA.}
	\label{fig:BalloonLaunch}
    \end{figure}





At float altitude, the air temperature is between \SI{230}{\kelvin} and \SI{250}{\kelvin}, while the air pressure is 0.5\% of the sea level pressure (about 5 mbar). Upper altitude winds are large-scale laminar flows that move the balloon and the payload as one. This can excite pendulum motions about the pivots underneath the balloon and at the top of the payload, which are typically of the order of a few arcminutes and have periods of a few to many tens of seconds \citep{Fixsen:1996kha}.

The payload's temperature distribution is influenced by the air temperature, infrared radiation of the Earth, and sunlight, which can result in complex temperature gradients across the instrument. A better temperature uniformity is expected for night flights, which is what BETTII is expecting.

Balloon experiments can also be affected by cosmic rays which can damage the electronics, lead to data corruption and or failures of the software/control system. However, this becomes more of an issue for long-duration balloon flights around Antarctica, during which the payloads are exposed for many weeks to the cosmic ray environment.

BETTII is expected to launch from Fort Sumner, NM, for its first engineering flight. After a morning launch, we expect to wait until nightfall to achieve proper thermal stabilization and achieve our science goals. We expect the flight to last about \SI{16}{\hour}, although this is highly dependent on the weather and wind patterns.


\subsection{Mechanical design}

BETTII has two main structures. The first is a carbon fiber and steel truss that is used as our optical bench. This was the first item that was designed in the project. The elements of this structure are built by bonding \SI{7.5}{\centi\meter} diameter hollow carbon fiber tubes to custom-made steel conical ends, that we call \textit{nose cones}. The steel nose cones are lightweight and strong, and have a threaded hole on the axis: they attach to multi-faceted steel nodes like tinker toys. There are three lengths of tubes on the truss. At the interface between the nosecone and the nodes, spherical washers or polypropylene washers are used, depending on the location on the payload. The difference of material compensates for differential thermal contraction on the beams that form the long side of triangles.

The structure is about \SI{9}{\meter} long. It is designed to be lightweight, strong, and have a first resonant mode above \SI{20}{\hertz} to ensure fast damping of residual mechanical oscillations. We measured the first resonance peaks to be within \SI{1}{\hertz} of their expected frequency, at \SI{25}{\hertz} (see Chapter~\ref{chap:implementation}).

The entire balloon payload needs to be robust to handle 10~g vertical force and 5~g force at \SI{45}{\deg}, which are the safety guidelines from the launch facility. With an expected total mass of \SI{1000}{\kilo\gram}, we need yield strength sufficient to hold \SI{100000}{\newton} of force. 

An annotated rendering of BETTII is shown in Fig.~\ref{fig:BETTIICAD}. The gondola is what holds the truss and attaches to the balloon train. It also holds the electronics, reaction wheels, batteries, and communications to the ground. The frame is made out of 80/20 T-slotted aluminum bars that are attached together using T-inserts, and reinforced by screwed-on corner plates. The precision of this frame is of no importance to the optical alignment. 


\begin{figure}[!ht]
	\centering
	\includegraphics[width=\textwidth]{Figures/BETTII-annotated.jpg} 
	\caption[BETTII Rendering]{CAD rendering of the BETTII payload in its final state.}
	\label{fig:BETTIICAD}
    \end{figure}



The various electronic components of the system are attached to the gondola using aluminum or honeycomb aluminum plates, which are painted with white appliance paint for better thermal behavior. These plates act as radiator panels which allow us to dissipate the heat out to space.

The most critical portion of the gondola is the assembly that connects to the balloon train. This contains a single pin that needs to have the highest yield strength, since it is the only point of the payload that needs to support the entire weight. A more detailed description of the pin is presented in Section~\ref{chap:controls}.\ref{subsec:chap3momdumpmotor}.

The entire payload is designed, assembled and tested in the building 20 high bay at NASA GSFC (Fig.~\ref{fig:HighBayOpen}).
\begin{figure}[!h]
		\centering
		\includegraphics[width=\textwidth]{Figures/HighBayOpen.jpg} 
		\caption[Payload in high bay]{Payload in the high bay before a controls test.}
		\label{fig:HighBayOpen}
\end{figure}


\subsection{Warm optical system}

The optical system was one of the most challenging design aspects of the project. It is beyond the scope of this work to go into details about all the considerations that went into the design, but we will review some of the main aspects: the overall optics layout, and the fabrication of the telescope assemblies.

\subsubsection{Optics layout}
Because the nature of balloon payloads, there can be extensive damage to the structure during parachute opening and landing. In order to minimize the repair costs from one flight to the next, it was decided to place the telescope assemblies - which are expensive, long lead-time items - away from the edges of the truss. 

Instead, flat mirrors (that we call \textit{siderostats}) are used to redirect the light towards the telescope assemblies, which are kept close to the center of the truss where damage is expected to be minimal. 

The telescope assemblies (Fig.~\ref{fig:TelescopeAssemblyLayout}) consist of 3 powered mirrors and a folding flat. They provide a 20:1 compression ratio of the beams with reasonable tolerance on the mirror positioning. As an all-aluminum assembly, they shrink homologously as the temperature varies during the different phases of the flight, hence maintaining optical prescriptions. 

\begin{figure}[!h]
		\centering
		\includegraphics[width=0.7\textwidth]{Figures/TelescopeAssembly.jpg} 
		\caption[Telescope assembly layout]{Telescope assembly model and layout.}
		\label{fig:TelescopeAssemblyLayout}
\end{figure}


In order to perform double-Fourier interferometry, an extra reflection needs to be introduced in the system in order to properly combine the polarizations of the light at the beam combiner (see Fig.~\ref{fig:FTSvsDoubleFourier}). This asymmetry occurs after the telescope assemblies and before entering the cryostat. In one arm, a 3-mirror assembly (called the K-mirror assembly, or KMA) is used on a rotating stage to match the field of view rotations as the two siderostats change elevation. On the other side, a 4-mirror delay line assembly (called the Warm Delay Line, or WDL) is set at a fixed orientation. Its role is to compensate for the optical delays caused by the residual pointing errors. 

On both the KMA and the WDL (Fig.~\ref{fig:SmallAssemblies}), one of the mirrors is actuated in tip and tilt, which provides the fine control required to properly overlap the two beams at the detectors. There is an extensive discussion of the control system in Chapter~\ref{chap:controls}.

\begin{figure}[!h]
		\centering
		\includegraphics[width=\textwidth]{Figures/Assemblies.png} 
		\caption[Small optical assemblies]{K-Mirror Assembly, Warm Delay Line, and Cold Delay Line.}
		\label{fig:SmallAssemblies}
\end{figure}

The beams from each side enter the cryostat through thin polypropylene windows. We tested different window thicknesses and selected the \SI{15}{\micro\meter} thickness as our baseline design. %Test pieces with this window have been shown to comfortably resist about 1.5 times the atmospheric pressure, even after 50 cycles of pressurization. The maximum resistance is a little smaller when the pressurization occurs very rapidly and does not leave time to the window to deform elastically. A number of tests were done on identical batches of windows to ensure the repeatability of our test method. 
Once the beams are inside the cryostat, they are split into a NIR tracking channel, and into the FIR optics train where they are delay-modulated by the Cold Delay Line (Fig.~\ref{fig:SmallAssemblies}), combined, and image onto the detectors. A complete layout of the optics train is shown in Fig.~\ref{fig:OpticsLayout} (Dhabal \textit{et al.}, 2016, in press).


\begin{figure}[!h]
		\centering
		\includegraphics[width=\textwidth]{Figures/OpticsLayout.png} 
		\caption[Optics layout]{Optics layout for BETTII (Dhabal \textit{et al.}, 2016, in press).}
		\label{fig:OpticsLayout}
\end{figure}


\subsubsection{Optics manufacturing}

Despite working at relatively long wavelengths, the tolerance in the surface figure of all the mirrors is an important consideration. %Traditionally, figure errors are specified in terms of the required beam quality at the focal plane, which starts to degrade when the wavefront errors in the optical train are comparable with the wavelength of the light beam, or introduce specific aberrations. In interferometers, the fidelity of the final image is not a priority. However,
Differential wavefront errors between the two optics trains before combination will result in decreased contrast of the interferograms, which reduces our signal-to-noise ratio. As a result, the surface quality of the mirrors pre-combination needs to be much lower than a wavelength of light, since errors will stack after hitting many mirrors from both sides. We allocate \SI{2}{\um} of total wavefront error at combination, which translates to $\sim$\SI{0.23}{\um} of surface error per mirror. Given that known processes exist to manufacture small mirrors below this requirement, we relax the requirement for the primary mirrors and the siderostats to a \SI{300}{\nano\meter} r.m.s surface figure error over the entire aperture.



The company Nu-Tek, in Aberdeen, MD manufactured all of our small optics out of aluminum. The procedure includes an initial milling process, heat treatment using a method called \textit{uphill quenching}, followed by diamond turning and gold coating to avoid oxidation. 

However, very few manufacturers in the United States were able to diamond-turn the siderostats and the primary mirror assemblies, while ensuring the level of surface figure we needed. The diamond-turning process uses a slowly moving diamond blade that is controlled in 3 axes to carve out the required shape. This process requires extreme temperature stability, which is often not available in traditional machine shops. Companies which are familiar working with NASA on space missions were not affordable for a small project like us.

\begin{figure}[!h]
		\centering
		\includegraphics[width=\textwidth]{Figures/TelescopeAssemblies.jpg} 
		\caption[Telescope assemblies]{Telescope assemblies in the optics lab.}
		\label{fig:TelescopeAssemblies}
\end{figure}



The Department of Advanced Manufacturing at North Carolina State University proposed to manufacture our mirrors on a 'best effort' basis for a reasonable cost.  The results are published in Furst \textit{et al.} (2016, in press), although the surface quality has not yet been measured, due to difficulty with the equipment  and setup in the optics lab at NASA Goddard. Each telescope assembly (see Fig.~\ref{fig:TelescopeAssemblies}) has a stacked r.m.s surface figure error of \SI{300}{\nano\meter}, while the siderostats have a surface error of \SI{100}{\nano\meter} r.m.s. The siderostats are more complicated because they did not exactly fit in their diamond-turning spindle. We decided to proceed with a two-step diamond turning, where they turned two sections of the ellipse consecutively. This does not guarantee that the two areas will be at the same height since they have to unmount the mirror off the spindle. However, our models show that even if different sections of the mirrors are at different heights, the beam combination can still be successful, as the parts of the pupil that are shifted in one arm are also shifted in the other. 




\subsection{Cryogenic instrument}

The cryostat was designed by our team. Items were sent out for manufacturing to different companies and assembled in our lab. The cold volume is passively cooled by liquid nitrogen and helium and does not require any mechanical cryo-cooler. It is designed to operate for a duration of \SI{40}{\hour}, which should give us enough margin considering the typical lengths of balloon flights from the U.S. of about \SI{16}{\hour}. 

\begin{figure}[!h]
		\centering
		\includegraphics[width=0.6\textwidth]{Figures/Dewar.jpg} 
		\caption[Cryostat crossection]{Cryostat crossection.}
		\label{fig:CryostatCrosssection}
\end{figure}

The optics inside the cryostat are split into two sections: a near-infrared fine guidance system, and the far-infrared channels with the science detector (Fig.~\ref{fig:CryostatCrosssection}). The incoming light beam is split right after entering the cryostat with a NIR/FIR dichroic beam splitter. This custom-made filter reflects the far-IR and transmits the near-IR. At the bottom of the cryostat, in the \SI{77}{\kelvin} volume, the fine guidance sensor is composed of 12 optics and one HAWAII-1RG detector from Teledyne. 


At the top of the cryostat and attached to the \SI{4}{\kelvin} cold plate, there is a cold optics bench that holds all of the far-IR optics, filters, and the Cold Delay Line. All filters were manufactured by Cardiff University in the U.K. The layout of the optical system is shown in Fig.~\ref{fig:OpticsLayout}, and more details can be found in (Dhabal \textit{et al.}, 2016, in press). A picture of the cold optical bench with populated and aligned optics is shown in Fig.~\ref{fig:ColdBench}.

\begin{figure}[!h]
		\centering
		\includegraphics[width=\textwidth]{Figures/ColdBench-anottated.jpg} 
		\caption[Cold bench]{Optics on the cold bench.}
		\label{fig:ColdBench}
\end{figure}

The cold plate of the dewar is cooled down to \SI{4}{\kelvin} with liquid Helium. A ($^3$He+$^4$He) sorption refridgerator from Chase Research is used to obtain an intermediate cold finger at \SI{1}{\kelvin} and a final stage that brings down the detector temperature to $\sim$\SI{400}{\milli\kelvin}. Fig.~\ref{fig:CryostatTopPlate} shows a a picture of the top plate of the cryostat while cold. 

\begin{figure}[!h]
		\centering
		\includegraphics[width=\textwidth]{Figures/Cryostat.jpg} 
		\caption[Cryostat top plate]{Cryostat top plate during cool down.}
		\label{fig:CryostatTopPlate}
\end{figure}


At the heart of the instrument are four $9\times 9$ close-packed linear arrays of multiplexed superconducting transition edge sensor (TES) bolometers \citep{Benford:2008wk} incorporating the Backshort Under Grid (BUG) architecture \citep{Allen:2006jn}. These arrays are scaled versions of similar arrays already built for ground-based instruments \citep[e.g., GISMO,][]{Staguhn:2014jg}. Detectors are read out using advanced linear SQUID multiplexer and amplifiers. A $4\times 22$ multiplexed readout is used for each array; the extra seven channels are used for calibration signals (unilluminated pixels, “dark SQUID” channels, and an “always on” channel), allowing monitoring of all potential noise contributors \citep{deKorte:2003km}.  






\subsection{Data products \& analysis}


Once in flight the payload operations consist of pointing at a target, stabilizing the attitude motions, and scanning the delay while recording detector data. A number of operational modes are required to ensure we reach this stable observing stage, and are described in more details in Chapter~\ref{chap:controls}. 

Individual scans will last for a nominal duration of \SI{2.5}{\second}, and consist of \si{1024} individual detector frames, which are matched to a given \OPD. To increase the signal-to-noise ratio (\SNR), we expect to stack \SI{10}{\minute} worth of data, which corresponds to 200 scans. For this duration, we expect that the change in the baseline angle due to the rotation of the Earth is negligible. It is critical to correctly stack the interferograms, as \OPD errors from scan to scan can significantly reduce the fringe contrast (see Chap.~\ref{chap:phasenoisepaper}).

To describe post-processing, let's consider a \SI{10}{\minute} cube which is the {\OPD}-corrected stack of images from the 200 individual scans. The cube has a crossection of $9\times 9$ (which is the size of an individual detector frame), and a depth of 1024 frames. For each frame, the intensity of each source in the detector is determined for each \OPD, and combined into interferograms. We repeat the process for the same field observed at different baseline angles. 

Using starting points involving our existing SOFIA multi-wavelength observations, as well as the IRAC images, these interferograms will help determine the multiplicity of bright sources, their individual SED, and their position relative to the large-scale extended emission. 

If BETTII had more baseline lengths and the reconstructed cubes filled the aperture more densely, it would be possible to feed them directly to an inversion software that was developed in Dr. Juanola-Parramon's Ph.D. thesis \citep{Juanola:2016}, which provides a final datacube corresponding to the images as a function of wavelength, with the spectral resolution that the user chooses. 

\begin{figure}[!h]
	\centering
	\includegraphics[width=\textwidth]{Figures/DataProcessing.png}
	\caption[Data processing]{BETTII data processing steps.}
	\label{fig:dataProcessing}
    \end{figure}



\newpage
\section{Sensitivity analysis}

Early in my involvement with BETTII, I led the effort in trying to estimate the sensitivity of our instrument, in order to select relevant scientific targets, but also find astronomical calibrator objects which would help us understand the systematics of our payload.

In this section, we summarize our findings and give details on the methods and equations we used. Since only very few authors have approached the problem of double-Fourier interferometers, we were able to derive a new formalism to estimate the spectral sensitivity of double-Fourier interferometers for point sources. Our method uses propagation of gaussian errors through Fourier transforms, and is described in detail in Chapter~\ref{chap:phasenoisepaper}. This can be useful to determine the sensitivity of other types of instruments, such as a space-based follow-up of BETTII, which we briefly discuss in the conclusion of this work.

\subsection{Instrument and observing parameters}

\renewcommand{\arraystretch}{1.5}
\begin{table}
\small
\caption[Instrument parameters]{Instrument design parameters for BETTII.}
\vspace{-0.5cm}
\label{tab:instrumentParameters}
\begin{longtable}{P{6cm}|c|c|P{1.5cm}|P{3cm}}
\toprule													
Parameter	&	\multicolumn{2}{c|}{          		 Value		} 			&	Units	&	Science driver/impact	\\
\midrule													
\multicolumn{5}{c}{		Top-level parameters	}										\\
Input aperture	&	\multicolumn{2}{c|}{		0.196		}			&	\si{\raiseto{2}\meter}	&	Sensitivity	\\
Baseline length	&	\multicolumn{2}{c|}{		8		}			&	\si{\meter}	&	Angular resolution	\\
Detector pixels	&	\multicolumn{2}{c|}{	$	9\times 9	$	}			&	pixels	&	Wide field of view	\\
Detector quantum efficiency	&	\multicolumn{2}{c|}{		70	\%	}			&		&	Sensitivity	\\
Integration time per full frame	&	\multicolumn{2}{c|}{		2.5		}			&	\si{\milli\second}	&	Sensitivity	\\
Time per baseline orientation	&	\multicolumn{2}{c|}{		10		}			&	\si{\minute}	&	Sensitivity	\\
Number of data points in one scan	&	\multicolumn{2}{c|}{		1024		}			&	points	&	Wide FOV	\\
OPD range required	&	\multicolumn{2}{c|}{		8.2		}			&	\si{\milli\meter}	&	Wide FOV \& spectral resolution	\\
\midrule													
\multicolumn{5}{c}{		Optical system	}										\\
\midrule													
	&		Band 1		&		Band 2		&		&		\\
Central wavelength	&		40		&		82		&	\si{\micro\meter}	&	Study YSOs	\\
Fractional bandwidth	&		62.5	\%	&		54.9	\%	&		&	SNR at ZPD	\\
Field of view	&		2		&		3		&	\si{\arcmin}	&	YSO regions	\\
Etendue per pixel	&	\num{	8.2E-10	}	&	\num{	1.8E-09	}	&	\si{\raiseto{2}\meter\steradian}	&	Sensitivity	\\
Estimated efficiency	&		20	\%	&		24	\%	&		&	Sensitivity	\\
Pixel angular size	&		13.32		&		19.72		&	\si{\arcsec}	&	Wide field of view	\\
Primary full width half max	&		17.31		&		35.49		&	\si{\arcsec}	&	Sensitivity	\\
\bottomrule																					
\end{longtable}
\caption*{Instrument parameters that flow from the science requirement of \ang{;;0.5} and \ang{;;1} spatial resolution in bands 1 and 2 respectively, and spectral resolution $\R = 10$ in both bands.}
\end{table}


Table~\ref{tab:instrumentParameters} represents the key instrument parameters that are relevant for the sensitivity estimation of the two science channels of BETTII. The main impact of each parameter on some aspects of the science is shown. A detailed, custom calculator tool that we developed compiles most of the instrument parameters that flow down from these requirements, which in turn serve as design baseline for various subsystems. For example, the "OPD range required" is a derived output, depending on the baseline length, the field of view and the required spectral resolution.


\subsection{Far-IR background noise estimation}



We proceed to an estimation of the known far-IR background noise contributions from sources in thermal equilibrium. We assume that each source of noise emits like a Planck function \Bnu with a certain emissivity $\epsilon$. In Table~\ref{tab:noiseparams}, we list the number of photons generated per second for the amount of solid angle seen by a single pixel (with the exception of the atmospheric contribution, which is treated separately). The thermal emission is weighted by the normalized transmission function, which was measured in the laboratory (Fig~\ref{fig:BETTIITransmission}). By far the strongest contributors from our system are the warm optics and the cryostat's polypropylene window.
 

In addition to the noise of our own system and the astronomical background, we need to take into account the noise generated by the atmosphere, which results in a more complex calculation. For best accuracy, we use quantities from \cite{Harries:1980cva}, who measured the actual sky radiance in a large range of wavelengths from balloon altitudes. We obtain a radiance of \SI{0.16}{\watt\per\raiseto{2}\meter\per\steradian} and \SI{0.07}{\watt\per\raiseto{2}\meter\per\steradian} for band 1 and 2 respectively. This corresponds to \num{2.6e10}~photons~\si{\per\second} and \num{5.2e10}~photons~\si{\per\second}, respectively. 

\renewcommand{\arraystretch}{1.5}
\begin{table}
\small
\caption{Thermal noise contributors}
\label{tab:noiseparams}
\vspace{-0.5cm}
\begin{longtable}{P{2.5cm}ccP{2cm}P{2cm}p{3cm}}
\toprule
Noise source	&		T (K)		&		Emissivity		&		Photons~\si{\per\second} Band 1		&		Photons~\si{\per\second} Band 2		&	Reference	 \\
\midrule
Warm optics	&	\num{	240	}	&	\num{	0.1	}	&	\num{	1.38E+11	}	&	\num{	9.97E+10	}	&	Assumes 99\% per mirror	\\
Window	&	\num{	240	}	&	\num{	0.02	}	&	\num{	2.76E+10	}	&	\num{	1.99E+10	}	&	Lab measurements	\\
Zodi dust	&	\num{	245	}	&	\num{	3.00E-07	}	&	\num{	2.92E+05	}	&	\num{	3.41E+05	}	&	\cite{Fixsen:2002da}	\\
Galactic Cirrus	&	\num{	20	}	&	\num{	1.23E-04	}	&	\num{	1.79E+01	}	&	\num{	7.67E+04	}	&	\cite{Bracco:2011gw}	\\
Zodi scattering	&	\num{	5800	}	&	\num{	1.00E-13	}	&	\num{	1.47E+01	}	&	\num{	1.31E+01	}	&	\cite{Fixsen:2002da}	\\
CIB	&	\num{	18.5	}	&	\num{	1.30E-05	}	&	\num{	2.19E+00	}	&	\num{	9.68E+03	}	&	\cite{Fixsen:1998br}	\\
Instrument	&	\num{	4	}	&	\num{	0.7	}	&	\num{	5.80E-27	}	&	\num{	2.50E-07	}	&	Estimate	\\
CMB	&	\num{	2.728	}	&	\num{	1	}	&	\num{	1.02E-45	}	&	\num{	9.35E-17	}	&	\cite{Fixsen:1996di}	\\
\bottomrule
\end{longtable}
\caption*{\textbf{Notes}: The calculator was designed to be scalable to designing a space mission, which is why we kept track of terms which are negligible compared to the main contributors. In space, the warm optics and window contributions would be significantly reduced and more comparable to the other terms. These quantities do not yet include the losses from the instrument's throughput}
\end{table}

\begin{figure}[!h]
	\centering
	\includegraphics[width=\textwidth]{Figures/BETTII_transmission.pdf}
	\vspace{-0.5cm}
	\caption[BETTII Transmission curves]{BETTII total transmission curves $\Tbp(\lambda)$ from all cold filters, excluding the beam combiner, cryostat window, and NIR/FIR dichroic. Bands are shown in different colors.}
	\label{fig:BETTIITransmission}
    \end{figure}



To know how much power is actually reaching the detectors, we need a measurement of our optical throughput. The throughput is the product of the efficiencies of the various elements along the optical train: the mirrors, the cryostat window, the NIR/FIR dichroic, and all the cold filters. The latter multiply to give the transmission profile shown in Fig.~\ref{fig:BETTIITransmission}, which we call $\Tbp$. We write $f_\textrm{arm1->detN}$ (resp. $f_\textrm{arm2->detN}$) the throughput of light from arm M (resp. arm 2) falling on the detector N, where N = {1, 2}:
\begin{align}
f_\textrm{arm{1,2}->det{1,2}}(\lambda) &= \tau_\textrm{combiner}\tau_\textrm{window}\tau_\textrm{dichroic}r^{N_\textrm{mirrors}}\Tbp(\lambda) \\
& \approx 0.38\times \Tbp(\lambda),
\end{align}
where we have used lab measurements to estimate $\tau_\textrm{window}\approx 0.98$, $\tau_\textrm{dichroic}\approx 0.95$, $\tau_\textrm{combiner}\approx 0.5$ and $r\approx 0.99 $ is the far-IR reflection of each warm mirror, in both bands. There are $N_\textrm{mirrors} = 9$ within the warm optics train on the left side, and 8 on the right side. Until we obtain precise measurement of the throughput of each element as a function of wavelength, we consider that this extra factor is wavelength-independent and represents an average over the band. This is valid since most of these materials do not have steep dependence at such a long wavelength. The transmission $\Tbp(\lambda)$ has an average of 27\% (resp. 31\%) for band 1 (resp. 2) respectively, so the throughput amounts to about $\sim 10\%$ (resp. $\sim 12\%$) efficiency for the light coming from one arm falling onto one detector. 

\renewcommand{\arraystretch}{1.5}
\begin{table}
\small
\caption[Power and NEP contributors]{Estimated power and NEP contributors for a single detector pixel.}
\vspace{-0.5cm}
\begin{longtable}{P{3cm}|P{2cm}|P{2cm}|P{2cm}|P{2cm}}
\toprule																	
Noise source	 &		\multicolumn{2}{c}{		Power reaching the detector (pW)			}	 &		\multicolumn{2}{c}{		NEP (\SI{e-16}{\watt\per\raiseto{0.5}\hertz})			}	\\
	&		Band 1		&		Band2		&		Band 1		&		Band2		\\
\midrule																	
Warm optics	 &	\num{	92	}	&	\num{	45	}	 &	\num{	9.6	}	&	\num{	6.7	}	\\
Atmosphere	 &	\num{	18	}	&	\num{	24	}	 &	\num{	4.6	}	&	\num{	4.9	}	\\
Window	 &	\num{	21	}	&	\num{	9	}	 &	\num{	4.3	}	&	\num{	3.0	}	\\
Detectors	 &		-		&				 &	\num{	5	}	&	\num{	5	}	\\
\midrule							-										
Total	&		131		&		77		&		12		&		10		\\
\bottomrule																						\end{longtable}
\caption*{\textbf{Notes}: These values are lower than the ones cites in \citet{Rinehart:2014gk} and \citet{Rizzo:2015gf} since we now have more precise measurements of the transmission as a function of wavelength.}
\label{tab:powerNEP}
\end{table}


After accounting for all losses, we approximate the total noise power per pixel as:

\begin{equation}
P_\textrm{pix} = (f_\textrm{arm1->detN}+ f_\textrm{arm2->detN})N_\textrm{Photons~\si{\per\second}}E_\textrm{ph}\QE,
\end{equation}
where $N_\textrm{Photons~\si{\per\second}}$ is the total number of photons per second per pixel from the warm optics, the window, and the atmosphere, which are the three main contributors of noise (see Table~\ref{tab:noiseparams}). We also use the photon energy $E_\textrm{ph}$ and detector efficiency of the detector, $\QE\approx 0.7$. Throughout most the design phase of BETTII, this equation was used for the band-averaged quantities, for lack of better knowledge of the exact wavelength dependence of the various optical components. However, this is also valid on a finer scale and can be integrated over wavelength to provide more accurate estimates. In Table~\ref{tab:powerNEP}, we used our knowledge of the bandpass transmission and integrate over the band. The Noise Equivalent Power (NEP), a common measure of noise in the far-IR, is calculated as $\NEP = \sqrt{2P_\textrm{pix}E_\textrm{ph}}$. Note that the detectors are designed to contribute less than 30\% of the total estimated photon NEP, so that their noise contribution is negligible.


\subsection{Interferometric visibility budget}

Estimating the noise from each arm separately can help us determine important quantities such as the photon loading and NEP, which can be used to design the detectors. However, the scientific signal from an interferometer also depends on how well the two arms combine. This is roughly a measure of how symmetric the optical system is. In table~\ref{tab:visbudget}, we identify two kinds of error contributions: the static contributors, which are caused by differential wavefront errors (WFE), amplitude mismatch, polarization errors and pupil area overlap. These are caused mostly by misalignments of the optics along each train, or by errors in the manufacturing of the mirror surfaces. Second, we have the dynamic contributors, which are caused by \OPD errors and differential tip/tilt. These are errors which need to hold over the timescale corresponding to a single data point, so about \SI{2.5}{\milli\second}. The OPD errors correspond to fast uncorrected motion of the delay lines, while the differential tip/tilt corresponds to an error is co-aligning the two beams at the detector. Note that in Chap.~\ref{chap:phasenoisepaper}, we discuss the various timescales involved with the \OPD motions. In this table and for the calculation of the visibility, we only take into account the instantaneous, un-recoverable error in \OPD. The error in \OPD over longer timescales, resulting in a decrease in \SNR as we co-add consecutive interferograms, is not taken into account here. For reference, the equations are explicitly stated here, as we have found it handy to gather them all in one single place. The derivation for most equations can be found in \citet{Lawson:2000vf}.

\renewcommand{\arraystretch}{1.5}
\begin{table}[!h]
\small
\caption[Interferometric visiblity budget]{Interferometric visiblity budget.}
\label{tab:visbudget}
\begin{longtable}{P{3cm}|P{1cm}|P{1cm}|P{4cm}|P{1.5cm}|P{1.5cm}}
\toprule													
Term	& 	Symbol	& 		Alloc.		& 	Effect on visibility	& 	\multicolumn{2}{c}{\Vloss}			\\
	&		&				&		&	Band 1	&	Band 2	\\
\midrule													
\multicolumn{6}{c}{Static contributors}													\\
\midrule													
Total WFE in mirror surfaces	& 	\sigWFE	& 	\SI{	2	}{\micro\meter}	& 	$\exp(-[2\pi\sigWFE/\lambda]^2)$	& 	0.906	& 	0.977	\\
Amplitude mismatch	& 	$R$	& 		95	\%	& 	$2/(R^{1/2}+R^{-1/2})$	& 	0.999	& 	0.999	\\
Polarization effects	& 	$\theta$	& 	\SI{	12	}{\degree}	& 	$\cos(\pi\theta/180/2)$	& 	0.995	& 	0.995	\\
Pupil area overlap	& 	\foverlap	& 		90	\%	& 	\foverlap	& 	0.900	& 	0.900	\\
\midrule													
\multicolumn{6}{c}{Dynamic contributors}													\\
\midrule													
Error in OPD knowledge	& 	\sigOPD	& 	\SI{	2	}{\micro\meter}	& 	$\exp(-[2\pi\sigOPD/\lambda]^2)$	& 	0.906	& 	0.977	\\
Differential tip/tilt	& 	\sigtt	& 	\ang{;;	1.5	}	& 	$2J_1(\pi D\sigtt/\lambda)/(\pi D\sigtt/\lambda)$	& 	0.990	& 	0.998	\\
\midrule													
\multicolumn{3}{r}{Total visibility}							&	$\Pi(\Vloss)$	&	0.726	&	0.851	\\
\bottomrule											
\caption*{\textbf{Notes}: The dynamic contributors need to hold true for \SI{2.5}{\milli\second}, and consist of the residual amount that cannot be corrected in post-processing.}
\end{longtable}
\end{table}

\subsection{Science channel estimated sensitivity}

Now that we know the noise per pixel and the efficiency of the interferometric beam combination, we can determine the \SNR for a single source of known flux. For this, we use the formalism by \citet{Mighell:2005fwa} who derive the proper equation for a matched filter representing a point-spread function (PSF) discretized on a noisy detector array. The efficiency $\etamf$ of the matched filter is the inverse of the square root of the effective background area of the PSF, $\beta = 4\pi \mathcal{S}^2$, where $\mathcal{S}$ is the standard deviation of the PSF in pixels, $\mathcal{S} = \frac{0.42\lambda/D}{\theta_\textrm{pix}}$. We obtain $\etamf \approx 0.55$ and $0.39$ for band 1 and 2 respectively.

This matched filter efficiency is due to the uneven spread of the light from a PSF onto multiple pixels, and corresponds to the error in fitting the detector to the PSF assuming an even noise floor among all pixels. Pixels with more photons will have more \SNR, hence should be weighted more when attempting to extract the flux from the PSF. In this sense, using a matched filter is a best-case scenario. Another approach would consist of simply dividing the PSF area by the area of one single pixel, which is a worst-case alternative that would lead to efficiencies of 0.13 and 0.07 in band 1 and band 2 respectively. In what follows, we are using the optimistic approach and assume we can recover the flux from the PSF using matched filtering. 

We define the Minimum Detectable Line Flux (\MDLF) as the flux per pixel which corresponds to a $\SNR = 1$:
\begin{align}
\MDLF = \frac{\NEP}{(f_\textrm{arm1->detN}+ f_\textrm{arm2->detN})\Area\sqrt{2\Tint}},
\end{align}
where $\Tint = \SI{2.5}{\milli\second}$ corresponds to the integration time per pixel (or detector frame). The \MDLF is expressed in \si{\watt\per\raiseto{2}\meter}.

The Minimum Detectable Flux Density (\MDFD) is the \MDLF divided by the bandwidth. This is expressed in \si{\watt\per\raiseto{2}\meter\per\hertz} and can be converted to \si{\jansky}.

The faintest detectable interferometric point source with $\SNR = 1$ is then given by $\Smin= \MDFD/\Vi/\etamf$, where the \MDFD is increased due to the interferometric visibility losses and the spreading of the photons onto multiple pixels of the detector. \Smin represents the smallest flux density that leads to an $\SNR = 1$ within a single scan.

Co-adding consecutive scans will improve the \SNR considerably, but it will also introduces errors and inefficiencies. We quickly realized the impact of systematic errors in co-adding scans, so a significant amount of effort went into understanding the behavior of the various error contributions, and analyzing mitigation strategies. The result of this investigation was published in \citet{Rizzo:2015gf}, and is shown here in Chap.~\ref{chap:phasenoisepaper}. In that chapter, we discuss the meaning and importance of the phase noise or \OPD noise, and quantify the impact on the sensitivity. The \OPD noise arises when residual uncertainties in the knowledge and control of the \OPD result in errors while co-aligning and co-adding consecutive interferograms. For the rest of this discussion, we will assume that the \OPD noise amounts to \SI{5}{\micro\meter} r.m.s over 200 consecutive scans.

Using the formulas derived in Chap.~\ref{chap:phasenoisepaper}, we can now correctly determine the \SNR in the co-added interferograms. However, co-added interferograms are not the only goal of BETTII. Although interferograms allow for the distinction between multiple, nearby point sources, most of the scientific information is retrieved by analyzing the spectrum of each source in the field by taking the Fourier transform of the interferogram. Hence, we want to characterize the spectral sensitivity of the instrument, and establish this metric as the default observing metric for our science.

A summary of the results is presented in Table~\ref{tab:BETTIIsensitivity}.

\begin{table}[ht!]
\begin{center}
\caption{BETTII sensitivity estimates}
\label{tab:BETTIIsensitivity}
\vspace{-0.5cm}
\begin{longtable}{cccc}
\toprule
  Quantity   & Band 1 &  Band 2 & SNR Target \\
     \midrule 
 \multicolumn{4}{c}{\textbf{Single scan (\SI{3}{\second})}} \\
MDFD & 91 Jy  & 113 Jy & $\SNR_\mI = 1$\\ 
\midrule
\multicolumn{4}{c}{\textbf{Normal observing (200 scans, 10 min)}} \\
MDFD & 6 Jy  & 8 Jy & $\SNR_\mI = 1$\\ 
\bottomrule 
\end{longtable} 
\caption*{\textbf{Notes}: $\SNR_\mI$ represents the \SNR in the interferogram, while $\SNR_k$ represents the spectral \SNR, or the \SNR for each wavenumber bin (see Chapter~\ref{chap:phasenoisepaper}).}
\end{center}
\end{table} 


\subsection{Tracking channel estimated sensitivity}
A similar sensitivity analysis is done for the tracking channel. This is simplified somewhat since the tracking channels consists only of two cameras, and does not involve beam combination. The levels of background noise are less obvious to estimate. We primarily use the findings of \citet{Matsumoto:1994io}, which measured \SI{2}{\micro\meter} emission line strengths from balloon altitude. This emission is thought to arise from a thin layer of OH radicals at $\sim\SI{100}{\kilo\meter}$ altitude, and is sometimes referred to as \textit{airglow}. Using the measurements by these authors, who span multiple balloon flights in the 60s and 70s, we obtain an average radiance in the NIR bands of $\Rnir~\approx~\SI{1e-4}{\watt\per\raiseto{2}\meter\per\steradian}$. According to our estimates, this is two orders of magnitudes lower than the brightest astronomical noise source in the NIR, which is the zodi scattering.

Balloon altitudes provide significantly better atmosphere transmission in the NIR wavelength region, compared to ground observatories. Fig. \ref{fig:trans} illustrates this difference using a modelling software called MODTRAN. The transmission from an altitude of 4~km shows transmission windows (J, H, K bands) that would limit the design of a ground-based interferometer. At float, the bands are not limited by the atmospheric transmission and thus we can use larger bands than the traditional J, H and K in order to optimize our photon signal.

\begin{figure}[ht!]
\begin{center}
\includegraphics[width=\textwidth]{Figures/BETTII_atmo_transmission.pdf}
\vspace{-0.5cm}
\caption{Model atmospheric transmission, from \cite{Rizzo:2012jp}.}
\label{fig:trans}
\end{center}
\end{figure}

Due to the prioritization of the science channels, the NIR tracking channel is less advanced at the time of writing. Hence, we use estimates for the transmission and reflection efficiencies of the various components along the optical train. We estimate the efficiency of the major components: the mirrors (95\% reflective), the cryostat window (90\% transmissive), and the NIR/FIR dichroic (90\% transmissive), which transmits the NIR light. There is an additional filter just in front of the detector, which limits the bands from 1 to \SI{2.5}{\micro\meter}. The detectors are not responsive for longer wavelengths. The total amount of efficiency for this channel is expected to be on the order of $\epsilon = 20\%$. Using this, a \SI{1}{\jansky} source will correspond to a number of photons \Nph within a PSF at the detector:
\begin{align}
\Nph = \frac{\num{1e-26}}{h}\Area\times\FBW\times\epsilon\Tint \approx \num{8100},
\end{align}
where $\FBW \approx 0.67$ corresponds to the fractional bandwidth, and $h$ is the Planck constant. 

The detector is expected to have a read noise of $\sigRON = 18$~electrons r.m.s in up-the-ramp sampling, according the manufacturer specifications. Its frame rate changes throughout the acquire mode (see Chapter~\ref{chap:controls}), but the fastest mode will have a frame rate of $\sim\SI{50}{\hertz}$. Since the detector does not read destructively, saturation is an issue and needs to be addressed carefully - having to reset the pedestals to avoid saturation can complicate the software and might require a lot of tuning. Our calculations take into account a \SI{20}{\milli\second} integration time, a quantum efficiency of 70\%, and a \SI{0.6}{\arcsec\per\pixel} plate scale, which provides an effective background area $\beta = 0.43$ for a diffraction-limited PSF of diameter \ang{;;1.5} at \SI{1.5}{\micro\meter}. We consider that most of the photons will be spread on $1/\beta\approx~2.35$~pixels, so we expect about $\Nel\approx\num{3440}$~electrons per pixel from a \SI{1}{\jansky} source. A much more rigorous analysis is required once the efficiencies are measured.

For convenience, we express the \SNR of a source using its flux $S$ in electron per second:
\begin{align}
\SNR = \frac{S}{\sqrt{S + \beta(B + \sigRON^2)}},
\end{align}
where $B$ is the number of electrons per pixel from the background. In our case, we calculate $B\approx 2.4$~electrons, which is negligible compared to the read noise, so we will ignore this term in the future.

The required flux density for a given \SNR is then found by solving the previous equation for $S$:
\begin{align}
S\units{\si{\jansky}} = \frac{\SNR^2 + \sqrt{\SNR^4 + 4\SNR^2\beta\sigRON^2}}{2\Nel}.
\end{align}

For a $\SNR=10$, this corresponds to $\sim\SI{0.13}{\jansky}$, or $\sim$~9.66~H~magnitude. 

\section{Targets}

The primary science targets for BETTII have fluxes that are above the spectral sensitivities from Table~\ref{tab:BETTIIsensitivity}, with a bright NIR guide star nearby. In addition, in order to correctly know the \OPD, we need sets of bright calibrator targets which provide high-\SNR fringes in one single scan of the delay line.

The science targets need to be available during our launch window, and preferably cover a large range of projected angles (so we can study the source at multiple angles to retrieve more of the spatial distribution). For this reason, we favor circumpolar sources, since they are the ones which change orientation at the fastest pace.


\subsection{Calibrators}

Calibrators ideally need to be point sources $\gg\SI{100}\Smin$~\si{\jansky} in our FIR bands, and it is not straightforward to identify which astronomical sources exist that would provide this kind of flux density. The planets of the solar system and their moons are usually bright enough, but they are often resolved by our instrument, which dramatically reduce their interferometric contrast. For example, we estimate the Uranus is $>\SI{1000}{\jansky}$, but because it is so resolved, the actual fringe contrast if very small, hence drastically reducing the \SNR. Nearby, bright A stars such as Alpha Boo are most likely point sources, but are usually not as bright as we would want, especially not in Band 2 since they are essentially thermal sources with temperatures of thousands of Kelvin. It is possible to use actual science sources as calibrators, but of course it is unknown whether or not they actually are extended (this is the purpose of a mission like BETTII!). 

We find that bright asteroids such as Ceres, Pallas and Vesta are the best candidates for bright calibrators (respectively $> 320, 150 and 120$~Jy, respectively. In addition, because of their albedo, they also reflect the sunlight so they would also be suitable for the tracking channels (e.g. Ceres has Hmag$\sim$3. Their only disadvantage is that they are not inertial targets - this complicates the pointing control system as their expected position moves across the sky, which requires the payload to have accurate timing capabilities to know where the object is at a given time.

\begin{figure}[!h]
	\centering
	\includegraphics[width=\textwidth]{Figures/Visibilities.pdf}
	\caption[Visibilities of calibrators]{Visibilities of calibrators.}
	\label{fig:Visibilities}
    \end{figure}


\subsection{Science targets}
For our first flight, our science target list will be primarily composed of sources we have already observed with SOFIA FORCAST. In our source list, our best BETTII candidates are the sources which are bright at \SI{37}{\um}, have a large spectral index, appear point-like, are up in the sky at night during our flight, and are preferably circumpolar. 

\begin{table}[ht!]
\begin{center}
\caption{BETTII Targets}
\label{tab:BETTIITargets}
\vspace{-0.5cm}
\begin{longtable}{ccP{4cm}}
\toprule										
	Cluster	&	Coordinates	&	Fraction of night time between 10-\SI{75}{\degree} elevation		\\	
\midrule										
	S140	&	 22h19m23s +63d18m44s 	&	100.0	\%	\\		
	Cepheus~A	&	 22h56m10s +62d03m26s 	&	100.0	\%	\\	
	NGC~7129	&	 06h41m07s +09d33m35s 	&	100.0	\%	\\
	IRAS~20050+2720	&	 20h07m05s +27d28m51s 	&	50.0	\%	\\
\bottomrule										
\end{longtable}
\end{center}
\end{table}			

Table~\ref{tab:BETTIITargets} gives a list of such sources, and includes the fraction of time that the target spends above 10 degrees elevation and below \SI{75}{\degree} during the planned observing night of September 15, 2016. In addition, Fig.~\ref{fig:Targets} shows the tracks in the sky. The circumpolar targets S140, Cepheus~A and NGC~7129 are available for the most time. All are located well in the East at the beginning of the night, which means we can point towards them as the Sun sets in the West. Note that when the source is at low elevations, we could experience a substantial amount of additional atmospheric noise since the line of sight sees more airmass.

\begin{figure}[!h]
	\centering
	\includegraphics[width=\textwidth]{Figures/TargetPlot.pdf}
	\caption[Targets]{Polar plot showing the tracks of our targets in the night sky, between 8pm on Sept 15th and 6am on Sept 16th. The coordinates represent the local azimuth (with respect to North) and elevation, which is \SI{0}{\degree} at the horizon. Note that NGC~2071 and NGC~2264 cannot be observed at night in this period of the year.}
	\label{fig:Targets}
    \end{figure}
 
% Chapter 2

\chapter[Far-IR double-Fourier interferometers and their spectral sensitivity]{Far-infrared double-Fourier interferometers and their spectral sensitivity} % Main chapter title

\label{chap:phasenoisepaper} % For referencing the chapter elsewhere, use \ref{Chapter1} 

%----------------------------------------------------------------------------------------

\section{Introduction}
Observations at mid- to far-infrared wavelengths from the Earth's surface are extremely 
limited by the large atmospheric opacity in this region of the spectrum. Space-based telescopes 
like IRAS \cite[12-100 $\um$;][]{1984ApJ...278L...1N}, ISO \cite[2.5-240 $\um$;][]{1996A&A...315L..27K}, \textit{Spitzer} \cite[3.6-160 $\um$;][]{2004ApJS..154....1W}, AKARI  \cite[1.7-180 $\um$;][]{2007PASJ...59S.369M}, WISE \cite[3.4-22 $\um$;][]{2010AJ....140.1868W} and \textit{Herschel} \cite[55-672 $\um$;][]{2010A&A...518L...1P} have demonstrated the scientific value of observations at 
these wavelengths; but the spatial resolution of space-based observatories is limited by the cost 
and complexity of building and flying progressively larger aperture telescopes. 
Interferometry is a common solution to this problem on the ground, and is a viable path forward to obtain much
higher resolution than what single apertures can provide. 
In particular, spatio-spectral interferometry \citep{Mariotti:1988vea} is a way to achieve 
high angular and spectral resolutions at far-IR wavelengths from above the atmosphere, without the cost and limitations of large single apertures. 

Several space-based interferometer concepts, the Far Infrared Interferometer \citep[FIRI;][]{2009ExA....23..245H}, the Space Infrared Interferometer Telescope
\citep[SPIRIT;][]{Leisawitz:2007if}, and the Submillimeter Probe of the Evolution of Cosmic Structure \citep[SPECS;][]{Harwit:2006hl}, have been proposed and use spatio-spectral interferometry to achieve the much needed angular resolution to 
study astronomical processes such as the birth of stars and planetary systems, the activity in 
galactic nuclei and the formation of galaxies in the distant universe. The FIRI and SPIRIT concepts have
two mirrors which are movable on one axis along a monolithic truss to provide a range of baseline lengths.
%the angle of the baselines is changed by rolling the spacecraft about the line of sight to the desired target. 
SPECS consists of three spacecraft connected via tether to achieve baselines of order 1~km. 

There are numerous engineering challenges to be addressed before such missions can become reality. A number of them can be tackled with testbeds \citep[e.g.][]{Leisawitz:2012ik, 2012ApOpt..51.2202G} and small-scale pathfinder missions. These missions 
will likely be two-element, single baseline interferometers in space or on balloon platforms,
such as the Balloon Experimental Twin Telescopes for Infrared Interferometry \cite[BETTII;][]{2014PASP..126..660R} and to a certain extent the Far-Infrared Interferometric Telescope Experiment \cite[FITE;][]{2010TrSpT...7.Tm47K}.  
These pathfinders will have very limited baseline coverage and
rather than producing full images, they will focus on reconstructing 
spectral information from closely-spaced sources. This paper explores
aspects of the noise in spectral measurements specific to these instruments.

\subsection{Spatio-spectral interferometry}

In their pioneering paper, \cite{Mariotti:1988vea} lay out the principles of spatio-spectral 
(or double-Fourier) interferometry. A spatio-spectral interferometer consists of a Fourier transform 
spectrometer (FTS), where a delay line mechanism modulates the optical path difference (OPD) between two independent light beams before combining them in the pupil plane. The instrument produces interferograms, which are arrays of power measurements as a function of the OPD. Unlike traditional FTS, 
where a single incoming beam is split, delay-modulated, and recombined, a double-Fourier 
interferometer utilizes multiple light collectors pointing to the same astronomical source and 
combines the incoming light from the collectors pairwise in the pupil plane. The orientation 
and magnitude of the baselines - the vectors between each pair of light collectors - determines 
which  spatial frequency of the astronomical image the instrument measures. 
Longer baselines correspond to higher angular resolutions. The ``double-Fourier" aspect comes from 
the fact that the interferogram measured on a given baseline is related to the Fourier Transform (FT)
of the spatial and spectral distribution of the source emission.
Two FTs are used to reconstruct the full spatio-spectral datacube representing the 
astronomical scene: the spectra which are more directly related to the power as a function of time
delay difference between the two incoming beams (equivalent to the OPD) and
the source 2D spatial structure on the sky which is more directly related to measurements
accumulated from many different baseline vectors. The length of the baseline vectors can be changed by modifying the distance between the light collectors. The orientation of the vectors can be changed by rotating the baseline with respect to the source on the sky.
The plane representing the source visibilities
as a function of baseline vector is referred to as the ($u, v$)-plane and is a common notion in ground-based submillimeter and radio interferometry. This paper focuses on the reconstruction of the spectrum from closely-spaced point sources using single-baseline measurements, and does not address the techniques and sensitivities involved in using multiple baseline lengths to produce an image of the scene; a mathematical formalism that covers imaging is already proposed in \cite{Elias:2007jsa}.

%LGM: ZPD dependence within columm for large offset not done

Proposed double-Fourier instruments at far-IR wavelengths distinguish themselves from operating interferometers at sub-millimeter and radio wavelengths in several ways. First, they do not directly measure the phase information. The fundamental measurement is a time series of real-valued power as a delay line modulates the OPD in a controlled sequence (for example a linear ramp). The OPD from the delay line, as well as other OPD contributors in each arm of the instrument, and the external OPD created when the line of sight to a source is not perpendicular to the baseline vector, add up to the total OPD.
In double-Fourier instruments, the OPD can be determined by measuring or estimating the various contributors to the total OPD. For a given detector location along the projected baseline vector, there exists a value of the OPD in the delay line that exactly compensates all other OPD contributors. This delay line position results in a zero net total OPD, and is called the Zero Path Difference (ZPD). At this value of OPD, an incoming plane wave traverses the two beam paths reaching the detector exactly with the same phase, for all wavelengths. ZPD corresponds to the center of an interferogram for that detector location. In the
context of this paper, the phase for a given wavelength $
\phi_\lambda$ is related to the OPD between the beams from each arm when they combine, at the time of a data point measurement: $\phi_\lambda = 2\pi\textrm{OPD}/ \lambda$.
%LGM: rearraged above paragraph to define phase later and keep focus on path difference

% In the context of this paper, the "phase" refers to the total optical delay between the beams from both arms when they are combined, for a given location on the detector array. The total OPD is influenced by the OPD between the arms of the instrument, the OPD introduced by the delay line, and  In double-Fourier instruments, the phase for each point of the interferogram can only be known indirectly by measuring or estimating these three OPD contributors: while the measured signal is an power as a function of delay line OPD, the physical information lie in the measured power as a function of total OPD (or phase). For each detector location along the baseline vector, there exists a value of delay line OPD that will exactly compensate the other OPD contributors, hence making the net OPD zero. This point, called the Zero Path Difference (ZPD), is the center of an interferogram at that detector location. Hence, measuring the location of the center of an interferogram can be used to retrieve the phase for each of the points in the interferogram, provided that other factors stay constant

%It is possible, for example, to accurately estimate the phase for each point of the interferogram if the Phase information is derived from acquired knowledge of the true Zero Path Difference (ZPD, equal optical path length for a target position on the sky).

A second important difference for balloon and space interferometers is that collectors are not fixed
to the Earth. In the case of BETTII and SPIRIT, the collectors are fixed to a truss structure which
is part of the mechanical system for pointing the collectors. Consequently, baseline length
and external OPD, as relevant to an astronomical source, are not independent of pointing errors. The impact of errors in baseline length is modest because the relevant measure is in terms of fractions of the collector diameter. Errors in pointing translate into external OPD as the sine of the error angle times the baseline length, while the relevant measure is the wavelength. This can easily become significant;
for example, a 1" pointing error for an 8~m long baseline corresponds to a $38~\um$ shift in OPD.
%LGM: Added last sentence

Third, bolometer-type detectors, such as being built for BETTII and envisioned for SPIRIT, are
easily, and indeed typically, configured as two-dimensional arrays. With pupil plane combination,
the entire field of view has an interferometric response; hence wide-field interferometry over
multi-pixel arrays is straightforward. Fig.~\ref{fig:widefield} shows this concept and sketches the instrumental
response. For the configuration shown with the detector array columns aligned perpendicular
to the baseline vector, ZPD is the same along lines perpendicular to the baseline vector projected on the detector. As the OPD is swept, it moves across ZPD for the different columns in the array, yielding interferograms with shifted centers corresponding to the changes in external OPD for each source location in the field. 

By sweeping the OPD, the double-Fourier instrument measures interferograms which contain both spectral and spatial information over the detector array. The full spatial and spectral source information can be unambiguously recovered by repeating the delay line sweep over a range of baseline angles and lengths, which correspond to different spatial frequencies on the sky \citep{Mariotti:1988vea}.


\subsection{The case study: BETTII}

The BETTII project \citep{2014PASP..126..660R}, is a motivation for this paper and a near-term application of spatio-spectral interferometry. BETTII consists of two 50~cm siderostats on a fixed 8~m baseline, with a far-IR beam-combining instrument at the center. It will observe the far-IR universe in two 
wavelength bands, 30-50~$\um$ and 60-90~$\um$. The instrument is currently under construction at NASA Goddard Space Flight Center and is scheduled to launch in the Fall of 2016 on a stratospheric balloon from Fort Sumner, New Mexico, to an altitude of 35~km in order to be above most of the atmosphere. For its first flight, BETTII will focus on the study of dense star formation in nearby clusters. While a complete image reconstruction is not possible due to the static baseline length, BETTII will help resolve point source objects that are 0.5-1" apart in the short and long band, respectively, more than ten times the spatial resolution of \textit{Spitzer} at 24~$\um$ and six times the resolution of SOFIA at 37~$\um$.  Combined with a modest spectral resolution of $\R=10 - 50$, BETTII will measure the spectral energy distributions (SEDs) of clustered young stars to determine their evolutionary stage, locate the origin of the far-IR emission, and improve our understanding of how stars accrete their mass in these very dense regions of stellar birth \cite[e.g. see][and references therein]{2014prpl.conf..149T}. For resolved sources, the fixed baseline will not completely lift degeneracies between the spectral and spatial information; however detailed source modeling can put constraints on the distribution of the far-IR emission
\citep[e.g][]{2013ApJS..207...30W}.
%LGM: modified last sentence and added recent reference for modeling.

In this paper, we study how various types of noise propagate to the derived spectrum in an
instrument like BETTII or SPIRIT. In section~\ref{sec:formalism}, we establish a mathematical formalism that can be used to represent interferograms. In section~3, we look at the dominant types of noise in the interferogram and define the relevant timescales associated with spatio-spectral interferometers. In section~4, we derive the spectral signal-to-noise ratio ($\SNR$). In section~5, we apply these results to the special case of BETTII to derive its point source spectral sensitivity.

\section{Mathematical formalism}
\label{sec:formalism}
The general optics layout for a double-Fourier system is shown in Fig.~\ref{fig:optics} for a single baseline. 
The combination of the siderostat and beam compressor acts as an afocal telescope 
which outputs a  parallel beam with a diameter convenient for the rest of the optical train.
The K-mirror in one beam path corrects for the pupil rotation so that the
images of the sky from the two collectors are matched over the field of view.
At the center of the instrument, there are optics for pupil re-imaging, filtering, and beam folding, as required by the specific implementation.
The key components for our purpose are the delay line, beam combiner and detectors. 
The delay line introduces a controlled OPD between both arms.
The two incoming beams are combined in the outputs from the beam combiner. 
We arbitrarily define one output as the ``+" and the other as the \mbox{``-"}. 
To conserve photon energy, the two outputs must be complimentary such that the summed power of the two is
independent of the OPD. In an ideal double-Fourier system, the two beam paths are symmetric about ZPD; 
hence, the power 
from the ``+" and ``-" outputs are equal at ZPD, and have odd symmetry about ZPD. In a traditional FTS at ZPD, one output has fully constructive interference while the other has fully destructive interference, with even symmetry about ZPD.

\subsection{Interferograms for a single baseline}

The interferogram for a single frequency of light measured at the outputs of the ideal double-Fourier instrument can be 
described in terms of the normalized intensity:
\begin{equation}
\hat{I}_\pm(x,\s) = \real(1 \pm i \; \Vb(\s) e^{-2i\pi \s x}),
\label{eq:basicinterferogram}
\end{equation}
where $\s \equiv {1 \over \lambda}$ is the wavenumber of the light in $\cm1$ as per the convention for the FTS literature, $x$ is the instrumental OPD created by the delay line with $x=0$ corresponding to ZPD, and $\Vb(\s)$ is the complex spatial visibility
%LGM: added x=0 statement above
 of the astronomical source for the baseline vector $\baseline$. ``$\real(f)$" indicates the real part of the complex-valued function $f$. The~$\pm$ indicates values for the two output beams: ``+" and ``-" in Fig.~\ref{fig:optics}.
%The "\textit{i}" in the second term of Equation 1 arises because the beam splitter puts a $\pi/2$ phase shift in the reflected beam.  The expression is the real part only because the measured interferogram is real valued. 
The derivation of this expression is given in Appendix~\ref{ap:interfero}.

The normalized complex spatial visibility $\Vb$ has a magnitude of 1 for all baselines for which the source is completely unresolved. For extended sources, the spatial visibility depends on the source geometry, intensity distribution, and the instrument baseline vector as described in Chapter~2 of \cite{2000plbs.conf.....L} 
and Chapter~3 of \cite{Thompson:2008ww}.  
For a normalized source brightness distribution $\hat{\F}$, the spatial visibility with respect to a phase reference position on the sky can be written as:
\begin{equation}
\Vb(\s)  =  \int_\textrm{source} d\Omega \Ahat(\Ds) \hat{\F}(\Ds ) e^{-2i\pi\s\Ds\cdot \baseline},
\label{eq:viseq}
\end{equation}
where $\Ahat$ is the normalized reception pattern of the collecting area; $\baseline$ is the baseline vector between the two collectors and $\Ds$ is the vector on the plane of the sky from the phase reference position to the infinitesimal solid angle $d\Omega$. The resulting visibility as a function of baseline vector is the 2-dimensional FT of the source's sky distribution. 
Since $\hat{\F}$ does not have to be symmetric with respect to the chosen phase center, $\Vb$ is in general complex and can be expressed as an amplitude and a phase, $\Phib(\s)$: $\Vb(\s)  =  |\Vb(\s)|e^{i\Phib(\s)} $.


%, where we are implicitly expressing that
%$\Phib$ is also a function of $\baseline$.

Real instruments have asymmetries, imperfections, and measurement errors which can create phase-shifts between the two optical paths and across the pupils.
% , and differences
%reflection, and transmission properties of optical elements in . 
Fixed instrumental effects 
can be represented by a normalized instrumental visibility loss term, 
$\Vi(\s)$ where the complex quantity $\Vi(\s) = |\Vi(\s)|e^{i\Phii(\s)}$, as described in details in Chapter~3 of \cite{2000plbs.conf.....L}, represents both amplitude losses and phase shifts (see Appendix~\ref{ap:interfero}). Additional phase errors can arise from
imperfect knowledge of the real-time optical path lengths which we will represent as
$e^{i\Phir(\s, x)}$, where $\Phir(\s, x)$ is the ``phase noise"; this term depends on the OPD $x$ through time-dependent phenomena such as mechanical jitters, temperature variations in
the optics support, or pointing errors. In the rest of this paper, we will mostly talk about this ``OPD noise", which is the physical source of the noise, whereas phase noise represents its effects on the interferogram.
The total complex visibility sampled at a single $\s$ by the system is $\Vb(\s)\Vi(\s) e^{i\Phir(\s, x)}$, and it is normalized such that, for an ideal instrument observing a point source, this quantity is equal to 1 at ZPD.

Using Eq. \ref{eq:basicinterferogram} for the monochromatic source, the polychromatic interferogram is the integral over $\s$ of this dimensionless response at each wavenumber. The total amount of power coming into the 2-aperture interferometer within a small wavenumber range $d\s$ is $2\A \B(\s) c d\s$ where $2\A$ is the total aperture area in m$^2$, $\B(\s)$ is the spectral flux density in W$\cdot$m$^{-2}\cdot$Hz$^{-1}$ and $c$ is the speed of light in cm$\cdot$s$^{-1}$. 
%LGM: The wavenumber $\s$ has units of cm$^{-1}$ to follow the convention in the FTS literature.
%LGM: added this to the  definition of wavenumbers after Eq 1  
Filters and optics in an instrument cause a wavenumber-dependent transmission profile $\Tbp(\s)$. The quantum efficiency of the detector can depend on wavenumber, $\etaD(\s)$. For multi-pixel detectors the interferogram is measured by matched filtering a point-spread function on a pixel array, which has some efficiency $\etamf$.% Optical elements such as filters and beam splitters/combiners can cause $\s$-dependent phase shifts. 

%Since optical elements along the two light paths are not perfectly
%identical, the source flux density, as seen at the detector, is modified by an instrument transmission function which
%can be complex:
%\begin{equation}
%T_{\inst}(\s) \equiv \etamf\etaD\Tbp \Vi = |T_{\inst}(\s)|e^{i\Phi_{\inst}(\s)},
%\end{equation}
%where all of the terms can be functions of $\s$.

The actual power measured by the instrument can be represented as:
%I_\pm(x) = \real\left(\A \int_0^{+\infty} T_{\inst} \B \left(1\pm i\; \Vb e^{i\Phir(\s, x)} e^{-2i\pi \s x}\right)cd\s \right),
\begin{equation}
I_\pm(x) = \A c\int_0^{+\infty} \etamf\etaD\Tbp \B \times
 \quad  \real \left[\left(1\pm i \Vi\Vb e^{i\Phir} e^{-2i\pi \s x}\right) \right]d\s,
\end{equation}
where the factor of 2 for the two apertures is dropped because it is implicit in Eq. \ref{eq:basicinterferogram}. All quantities within the integral can be functions of wavenumber, and all the instrumental phase and interferometric loss terms are in $\Vi$ and $e^{i\Phir}$.

Instead of considering each separate output, we use $\I = \I_+ - \I_-$ as our interferogram expression, which cancels out the constant term. We also introduce an interferometric instrument transmission function, which can be complex, which represents the normalized amplitude and phase of the interferogram for a point source of uniform spectrum and no phase noise:
\begin{equation}
T_{\inst}(\s) \equiv \A c\etamf\etaD\Tbp \Vi = |T_{\inst}(\s)|e^{i\Phi_{\inst}(\s)},
\end{equation}

 We can then write the modulated signal as:
%I(x) = \real\left( 2 \A\int_{0}^{+\infty} i |T_{\inst}| \B \Vb e^{i\Phir(\s, x) + i\Phi_{\inst}(\s)} e^{-2i\pi \s x}cd\s \right).
\begin{equation}
I(x) = \real\left( 2\int_{0}^{+\infty} i |T_{\inst}| \B \Vb e^{i\Phir + i\Phi_{\inst}} e^{-2i\pi \s x}d\s \right),
\label{eq:modsignal}
\end{equation}
where $\B$ is real and $\Vb$ can be complex.

Eq.~\ref{eq:modsignal} can be turned into a Fourier transform by mirroring all quantities to negative wavenumbers. This
convention is explained in detail in \citet{Davis:2001tr} for FTS instruments; the odd symmetry of the interferogram for a system with one beam combiner
and the complex instrumental transfer function means that the incident spectrum on the detectors
must be mirrored to -$\s$ as the negative of the complex conjugate of +$\s$: 
$\S_e(\s) \equiv [T_{\inst} \B \Vb]_e(\s) = {1 \over 2}\left[T_{\inst}(\s) \B(\s) \Vb(\s) - T^*_{\inst}(-\s) \B(-\s) \Vb^*(-\s)\right]$. 
We use the subscript $e$ to denote the reflected function, and will apply this convention in the rest of this paper; this reflection ensures that the integrals keep the same value when are expressed from $-\infty$ to $+\infty$, and does not affect the $\SNR$ estimates: although the signal appears to be divided by a factor of two, so is the noise, as it is spread between positive and negative frequencies.
The interferogram expression is then:
%\begin{equation}
%I(x) = \real\left( 2 \A\int_{-\infty}^{+\infty} i [T_{\inst} \B \Vb]_e  e^{-2i\pi \s x + i\Phir(\s, x)} c d\s \right).
%\end{equation}
\begin{equation}
I(x) = \real\left( \int_{-\infty}^{+\infty} i \S_e  e^{-2i\pi \s x + i\Phir} d\s \right).%\end{equation}
\label{eq:interfero2}
\end{equation}
%[CHANGE THIS TO: INCLUDE THE SINC FUNCTION WITHIN THE INTEGRAL AND DIVIDE IT UP LATER IN THE BANDPASS PROFILE]
\subsection{Measured interferograms}

In practice, the interferogram data are discrete measurements of a real-valued signal on the detectors. Like for most FTS instruments, each data point on the interferogram corresponds to an integration of the detector while the delay line is continually in motion. This decreases the amplitude of the interferogram due to the local smearing of the fringes, but it can be kept to low values by increasing the fringe sampling.
% For example, for constant delay line velocity $v$, a delay distance $\Dx=v\Dt$ is swept during an integration time $\Dt$. 
At each delay $\xn$, the interferogram has a measured value $\Ixn = \frac{1}{\Dx}\int_{\xn-\Dx/2}^{\xn+\Dx/2}\I(x)dx$. To first order, this has the effect of multiplying the power at each wavenumber by $\sinc(\pi\s\Dx)$. For the purpose of this paper, we consider this term to be included as part of the instrumental transmission $T_{\inst}$. Note that the value of the optical delay $\xn$ is the path difference from ZPD, not the physical location of the delay line, since there could be a multiplying factor between the two due to beam folding (e.g., for BETTII, a motion of 1~mm of the delay line creates 4~mm of OPD).
%LGM: Added "the path difference from ZPD" above
%The final, real-valued interferogram can be represented as:
%\begin{equation}
%\Ixn = Real(\A\times \intinf \eta \Tbp\Be\left|\Vb\Vi\right|\sin(2\pi\s\xn + \Phi)d\s,
%\label{eq:finvis}
%\end{equation}
%where $\Phi = \Phib+\Phii+\Phir$ is the sum of the phases corresponding to the source visibility, instrumental, and phase-delay  terms. The measured quantity $\Ixn$ has units of power. 

%\subsection{Fourier transform spectroscopy}
%\label{sec:FTS}

%For a source that is spatially unresolved for all wavenumbers in the bandpass ($\Vb(\s)=1$), the true source spectrum $\Be$ is recovered by Fourier transform of the interferogram with respect to the delay $x$. This is mathematically valid since we made the spectrum of the source symmetric using $\Be$ instead of $\B$, as described in \cite{Davis:2001tr}; this allows us to transform back and forth between the interferogram and the spectrum.

A discrete Fourier transform (DFT) is used to transform a discrete interferogram of $N$ measurements into a complex discrete spectrum with $N$ points. The resolving power of the instrument, $\R = \lambda / \D \lambda$, is dependent on the physical length scanned by the delay line $L$: $\R = L \s / 2 $ for a scan with symmetric length on both sides of ZPD. For these instruments where we scan through the whole interferogram, the data should be
sampled at least at the Nyquist rate for the interferogram response frequency of $\Dx = \lambda/2$. For a sampling exactly equal to Nyquist, we have the relationship: $N = 4 \R$.

For a double-Fourier instrument, as shown in Fig.~\ref{fig:widefield}, the ZPD for different columns on the array occurs at different delay positions $x_{\col}$, related to
the projected baseline length. The simplest way to express this is in terms of the angular offset on the sky of each column, $\xi$, along the
direction of the baseline, $\baseline$:
\begin{equation}
x_{\col} = | \baseline | \sin\xi \approx | \baseline | \xi = 48.7 \um \left({| \baseline | \over 10~\textrm{m}}\right) \ \left( {\xi \over 1~\textrm{arcsec}}\right) ,
\label{eq:delay}
\end{equation}
where we have filled in
practical units for an infrared instrument. For a far-IR interferometer working at $50~\um$, with
1-2~m diameter collectors, the delay shift across the collector point spread function (collector angular resolution) is several to ten wavelengths.
%LGM: Added PSF above to clarify the this is refering to the resolution of the collector
Hence the scan length to cover a wide-field array detector is comparable to the scan length required to achieve
$\R$'s of 100's to 1000's. This property is an important consideration for observation and data analysis strategies.
%With $x_0 = 0$ defined as the center of the interferogram, we can write:
%\begin{equation}
%\DFT(\Ixn) =\sum_{n=-N/2}^{N/2-1}\Ixn e^{2i\pi n k/N}.
%\end{equation}

The ideal interferogram for a point source from a perfect instrument is an odd function of the OPD $x$, so its DFT is purely imaginary. The noise in the interferogram will be converted into spectral noise in both the real and imaginary axes so the real axis is a proportional measure of the noise. 
Referring back to Eq.~\ref{eq:interfero2}, phase shifts caused by the instrumental transfer function and source spatial visibility will
break the anti-symmetry; in practice, the DFT of a measured interferogram is complex and the real and imaginary parts are of interest.
The scientifically interesting quantities are the source spectrum and source spatial visibility: $\B$ and $\Vb$; the fixed
instrumental terms have to be calibrated or properly modeled
by observing a bright point source of known spectrum. The techniques for calibrating FTS systems are well developed
\citep[e.g.][]{Davis:2001tr}, and there are many methods proposed to correct some phase and amplitude errors \citep[e.g.][]{Forman:1966wx, Sromovsky:2003in}. 

The phase noise term $\Phir(x,\s)$ in Eq.~\ref{eq:interfero2}, and the $\SNR$ in the measured interferogram can have significant
impact on the ability to recover the source spectrum with a real instrument. The upper panel in Fig.~\ref{fig:interfero} shows an example of an interferogram (left), and the transformed $\S_e(\s_k)$ (right) for a source with flat power spectrum, multiplied by a flat bandpass function with smoothed edges.
 The middle panel of Fig.~\ref{fig:interfero} shows
the same source and instrument parameters as the upper panel, now with an assumed Gaussian OPD noise of standard deviation equal to 10\% of the central wavelength of the band $\lambda_0\equiv {1\over\s_0}$ (\textit{i.e.}, there is a $\lambda_0/10$ OPD uncertainty for each data point in the interferogram). The lower panel is the top panel observed with a incoherent background noise corresponding to $\SNR=10$ at the peak of the interferogram, and no phase noise. 
The next sections of this paper will analyze these noise contributions and quantify their impact on the derived spectrum.

\section{Noise sources}
The two primary types of noise in a double-Fourier instrument are intensity and OPD noise. The intensity
noise consists of the astronomical and thermal background noise, the photon noise from the source, and the detector noise. The OPD noise arises primarily from uncertainties and changes in OPD, which would prevent us from accurately knowing the $x$-values of measurements in the interferogram before the FT. For convenience, we usually refer to the OPD noise as a percentage of the carrier wavelength. In the rest of this paper, a ``10\% OPD noise" signifies that the OPD for each measurement in the interferogram is known to within an error of 10\% of the carrier wavelength, or 10\% of one full fringe cycle.

\subsection{Intensity noise}
\label{sec:noisesource}
The measured signal has units of power and can be represented as the interferometric signal with additive noise:
\begin{equation}
\Im\pxn = \Ixn + \ni\pxn,
\label{eq:thermalnoise}
\end{equation}
with $\ni$ being the difference of the noise in the two outputs of the interferometer, $\ni = \nA-\nB$. When the beam combiner, optical train, and detectors are symmetric, the residual $\ni$ has zero mean. 
The total noise in $\Im\pxn$, expressed in Noise Equivalent Power, $\NEPtot$, is the sum of the three noise variances: 
\begin{equation}
\NEPtot^2 = 2\NEPph^2 +2\NEPdet^2 +  2\NEPsou^2,
\end{equation}
where $\NEPph$ and $\NEPsou$ are the thermal noise from the background (e.g. sky and warm optics in the case of a far-IR instrument)
and source photon noise, respectively, in one output, and $\NEPdet$ is the noise-equivalent power characterizing each detector's noise (including phonon, readout and Johnson noise). The factor of 2 multiplies each term since we are considering the difference of both outputs.
The relation between $\NEPtot$ and the variance $\varI$ of the noise $\ni$ during an interval $\Dt$ is \citep{Sromovsky:2003in}:
\begin{equation}
\label{eq:sigI}
\varI = \frac{\NEPtot^2}{2\Dt}.
\end{equation}
For space instruments, the noise will likely be dominated by the sky background (zodiacal light, galactic cirrus emission, or optics thermal emission) and detector for a very large fraction of astronomical targets, which tend to be faint; for balloon instruments, emission from
warm optics and the atmosphere sets the noise level in the far-IR.

%There are three dominant sources of thermal photons at balloon float altitude for BETTII: the atmosphere, the dewar window, and the external optics (or telescope). Additional sources, such as the radiation coming from the Cosmic Microwave Background and the zodiacal disk, are much less significant at our observing wavelengths: our estimates show a five orders of magnitude difference. Hence, we will focus on the three main components when working out the noise budget of our instrument. The results are shown in Table \ref{tab:noise}. The values in the table correspond to the background levels per pixel. The sky radiance at float in our bands is taken from \citet{Harries:1980cva}. The radiance from the window and telescope assumes that they are blackbodies at 240~K, with respective emissivities of 0.02 and 0.08. The total optical efficiency assumed for the telescope is 30\%. Combining all these noise sources leads to a background NEP of $1.4\times 10^{-15}$ W.Hz$^{-0.5}$ in band 1 and $6.4\times 10^{-16}$ W.Hz$^{-0.5}$ in band 2. The detector NEP is expected to be less than $3\times 10^{-16}$ W.Hz$^{-0.5}$ in band 1 and $3\times 10^{-16}$ W.Hz$^{-0.5}$ in band 2.

\subsection{OPD noise}
\label{subsec:phnoise}
Observing from the ground at optical wavelengths with a double-Fourier interferometer is limited by the phase coherence 
between the apertures, which is related to the atmospheric coherence time, as discussed by \cite{Mariotti:1988vea}. The short coherence time forces fast scan rates, which degrades the sensitivity of the instrument due to short integration times and phase shifts between sequential scans. 
This is not a problem for flying platforms, since even at balloon altitudes the atmospheric coherence is not a significant 
issue \citep{Rizzo:2012jp}. The major concerns for balloon and space missions are overall instrumental stability, knowledge of ZPD, and pointing errors, which can all contribute to OPD noise.

OPD noise arises in an interferogram when the OPD at the time of a measurement is uncertain, hence compromising the reconstruction of the true $x$-value. Since this uncertainty is a physical delay $\delta_x$, 
the error in phase is wavenumber dependent: $2\pi\delta_x\s$. $\delta_x$ is the difference between the estimated $x$
and the true $x$.
% LGM: Added the sentence above.
% LGM I deleted the "_k$ on the wavenumber because this is not a discrete spectrum statement.
For single-beam FTS instruments, internal laser metrology can provide optical path length 
measurements to high accuracy \citep[e.g.][]{Griffiths:2007uu}, and the separate paths the split beams need to travel can be kept small. 
For double-Fourier instruments, the entire optical paths upstream of the beam combiner affect the OPD, hence it is more challenging to accurately measure and estimate the OPD contributors. 
In addition, common-mode pointing errors of the collectors are directly converted to geometrical delay errors. 
Hence, it is critical to know the position and orientation of the baseline vector with respect to the 
astronomical target with high accuracy in order to properly reconstruct the interferogram.

For this analysis, we identify three timescales that can be used to examine the effects of OPD noise on the interferogram. These timescales are important to consider in the design of the OPD control system of any double-Fourier interferometer. Timescale~1 is the shortest and corresponds to the integration time for a single data point, typically a few milliseconds. In practice, this kind of OPD noise could be created by high-frequency mechanical jitters in the instrument (including the delay line bearing and motor, stiction behaviors and resonant modes, reaction wheels and other self-induced vibrations...). Timescale~2 is the time it takes to acquire one single interferogram over the full field of view and at the desired resolving power, typically on the order of seconds. The sources of noise that can affect this timescale include for example pointing errors and drifts, as well errors in the knowledge of the delay line position relative to
a reference ZPD. Finally, the longest timescale to be considered, timescale~3, is the time it takes to complete one full "track" by co-adding several consecutive interferograms to achieve the desired $\SNR$, typically a few minutes long. During this timescale, it is expected that the change in baseline orientation on the sky does not produce any significant change in the source spatial visibility function. The latter timescale is most importantly influenced by thermal variations and time-varying gradients that could change the optical alignment and mechanical configuration between the two arms. 


%[It is important to realize that while the phase noise can be large during the data acquisition, some can be largely reduced in processing of the data with good models of the noise sources. With this in mind, the following noise derivations apply more on the residual uncertainties on a post-processed data product, rather than in-flight noise control. (PUT THIS SOMEWHERE ELSE?)]

%\subsection{Other sources of noise}

%Talk here about all other sources of noise? What are they actually?

%Our instrument has three relevant timescales for phase noise: 1) the integration time for a single data point (2 to 10~ms), 2) the time that it takes to gather one interferogram ($1-10$~s), and 3) the time over which $M$ interferograms are gathered to complete an observation ($\sim$~10~min). These timescales correspond to high frequency vibrations in the structure, pendulation motion of the instrument package hanging under the balloon, and a combination of pendulation, vertical "bobbing" motion of the balloon, and thermal changes, respectively. BETTII is designed and instrumented to mitigate the amplitude of these effects, but there will be residuals. 

%The interferometric delay $\delaytot$ of an astronomical source consists in the sum of the delays created by the position of the baseline with respect to the source, $\delayext$, and the delays that are internal to the instrument, $\delayint$. External delays change with the pendulum motion of the payload (intermediate timescale), while internal delays will change due to vibrations (short timescale) and thermal variations (long timescale). 

%The magnitude and frequencies of the perturbations for each timescale are difficult to estimate in detail. However, we will make some reasonable assumptions in order to predict sensitivities in the following sections. For the short timescale, the truss and support structure are designed to have lowest vibration modes above 20~Hz where dissipation in the structure will strongly damp oscillations. We expect the phase perturbations to be small over the 1-10~ms timescale. 

%The pendulum modes have been measured before on multiple experiments \cite[e.g.][]{Fixsen:1996kha} and have periods from 2 to 30 seconds and amplitudes up to 10~arcminutes. The highest frequency modes damp faster, especially for an experiment like BETTII that will be staring at inertial targets and not scanning the sky. The gondola pointing control system will compensate for most of these perturbations; we expect to be able to keep the gondola pointed at a given target to within 10-15 arcseconds, once the most powerful modes have damped. The pointing control system is described in detail in \cite{Rizzo:spie2014a}. The residual pointing errors of the gondola will result in external delay errors, and we can expect these errors to be generated at the same frequencies as the perturbations. BETTII will utilize
%measurements from the star camera, the fiber optics gyroscopes, and the tip-tilt pointing corrections to determine these residual errors with an expected accuracy of 1 arcsecond during flight. 
%MAXIME --- IS 1 arcsecond correct??? Put in the correct number.

%A second source of phase error in the intermediate timescale is self-created perturbations in the control loops for the pointing system and the delay lines. These perturbations can be minimized with optimal setting of the gain and bandwidth of the control loops. Post-flight knowledge of the pointing is expected to be accurate to $\sim$0.1~arcsec over a few minutes, and can be held longer when fringes are seen. Knowledge of the error in the commanded position of the delay stages will be measured in real time on BETTII with an accuracy of 0.2~$\um$ with capacitive sensors.

%Finally, thermal distortions are the most difficult to predict. They will depend on the radiative environment during ascent and at float, which is very hard to reproduce in a test environment without using a large vacuum chamber. We expect that the payload will be thermally stable on the order of 10~minutes but it is expected to be systematically cooling throughout most of the flight. 

%The pendulum modes can be very well estimated since they are much slower than any of our sensors (periods of 2~s minimum). Internal vibrations are hard to predict at this point, but they will be predictable we will be able to mitigate them on the ground. Further, they should be minimized by the good symmetry of the payload and our constant effort for synchronous control and command. Finally, thermal distortions are extremely hard to predict. They will depend on the radiative environment at float, which are very hard to reproduce in a test environment without using a large vacuum chamber. We expect that the payload will be stable on the order of 10~min, before a re-calibration of the phase is necessary. 

%\subsection{Important timescales}

%On BETTII, we expect to see three major timescales. The first timescale describes perturbations that occur much faster than the duration of one scan, $\leq 3$~s. We expect to generate most of these perturbations within the payload, from the wheels, the rotation stages, and our other mechanisms, and so they should be predictable. The fastest pendulum modes can be of the order of a scan duration, but these modes damp on the order of a few minutes. The second timescale is between the duration of a scan and the duration of one track. This regime is dominated by the pendulum modes which are on the order of 10-30 seconds. We assume that during one track, there is no significant change in the geometry of the optical train. Finally, the third timescale is beyond the duration of a track, $\geq 10$~minutes. This regime is dominated by thermal changes within the structure, which will inevitably impact the optics and the alignment of the sensors. 



%As discussed in the introduction, errors in pointing of the truss structure result in phase errors. Because BETTII does not have high precision independent metrology of the internal optical paths, errors in knowledge of position of the delay line mirrors also result in phase errors. These two are the main contributors of phase noise on short timescales. On longer timescales, thermal shifts dominate and affect the phase through changes of the structure and the optics caused by asymmetric thermal loads. Hence, two levels of control need to be achieved: a fast control of the relative phase over short timescales for stability; and a slow loop that controls the absolute phase to avoid drifts in ZPD over long timescales. Timescales are discussed in more detail in Section \ref{sec:implications}


\section{Spectral signal-to-noise ratio}
\label{sec:spectralSNR}

\subsection{Effects of Gaussian intensity noise}

%The primary sources of intensity noise in the balloon environment are the photon noise from the incoming light and the detector noise as discussed in Section \ref{sec:noisesource}. Both are expected to have Gaussian distributions.
In the presence of Gaussian intensity noise (thermal background and detector noise), the measured interferogram is of the form of Eq. \ref{eq:thermalnoise}. We suppose that the noise has a variance $\varI$ and zero mean, and is independent of delay position.  In particular, this assumes that the source photon noise is negligible.
The noise in the spectral domain is the transform of the noise in the interferogram domain:
%To determine the SNR in the spectral domain, we start by a Fourier transform of the interferogram:
%\begin{equation}
%\DFT(\Im (\xn) ) = \sum_{n=0}^{N-1}(\Ixn + \ni(\xn)) e^{2i\pi n k/N}.
%\end{equation}
\begin{equation}
\Dx\DFT(\ni) =  \Dx\sum_{n=-N/2}^{N/2-1}\ni(\xn)e^{2i\pi n k/N},
\end{equation}
% WHY IS THERE A DELTA-X RUNNING AROUND HERE WHEN IT WAS NOT PRESENT IN SECTION 3.1
%Half of the noise is in the imaginary domain and half is in the real domain. 
where the $\Dx$ factor is to normalize the noise to a sampling bin \citep{Press:1992vya}, 
and $k$ indexes the $N$ discrete wavenumbers in the spectral domain.
%LGM: Added line above 
The interferogram interval is symmetric with about ZPD (n=0). The noise variance is equal in the imaginary and the real domain, and can be expressed as the variance of the noise transform:
\begin{equation}
\varspec = \Dx^2\VAR\left(\real(\DFT(\ni))\right),
\end{equation}
where $\VAR$ is the variance operation. By writing out the variance we obtain:
\begin{equation}
\varspec = \Dx^2\varI\sum_{n=-N/2}^{N/2-1}\cos^2(2\pi nk/N) =\frac{N}{2}\Dx^2\varI ,
\end{equation}
where we used $\sum_{n=-N/2}^{N/2-1}\cos^2(2\pi nk/N) = N/2$ for $k~\neq~0$. 

The signal at wavenumuber $\s_k$ in the discrete spectrum $\S_e(\s_k)$ is:
\begin{equation}
 \S_e(\s_k) = \frac{1}{\delta\s}\int^{\s_k+\delta\s/2}_{\s_k-\delta\s/2} \S_e(\s)d\s,
\label{eq:signal}
 \end{equation}
where $\delta\s = (N\Dx)^{-1}$. A line of power $P_e$ at $\s_{k_0}$  will thus have an apparent flux density $\S_e(\s_k) = N\D xP_e$ at $k=k_0$ and $0$ for all other $k$. The signal-to-noise ratio in the spectrum can be expressed in general as:
\begin{equation} 
\SNR_k  = \frac{\S_e(\s_k)}{\sigspec} =\sqrt{\frac{2}{N}}\frac{\S_e(\s_k)}{\Dx\sigI} .
\label{eq:spectralSNR}
\end{equation}
Using Eq.~\ref{eq:sigI} and the definition $x_{\textrm{max}}=N\Dx/2$, this becomes:
\begin{equation}
\SNR_k = \frac{\S_e(\s_k)}{x_{\textrm{max}}\NEPtot} \sqrt{N \Dt},
\end{equation}
where $\Dt$ corresponds to the integration time of one data point on the interferogram. As expected the $\SNR$ improves as the square-root of the total integration time, $\sqrt{N \Dt}$, and is adversely affected by increasing NEP and scan length. 
%The signal $\S_e$ in the recovered spectrum is only half the signal of the physical spectrum, because of the definition of $\Be$. 

%The signal in the delay domain is $\I(0) = \dsig\sum_{k}\S(\s_k)=\dsig N\oS$, where $\oS$ is the mean signal in the spectral domain. We also have $\dsig = (N\Dx)^{-1}$, so $\I(0) = \oS/\Dx$ which leads to:
%\begin{equation} 
%\SNR_\mI = \frac{I(max)}{\sigI} = \sqrt{\frac{N}{2}}\frac{\oS}{\sigspec},
%\end{equation}
%and finally:
%\begin{equation} 
%\SNR_\S (\s_k)= \sqrt{\frac{1}{\R}}\frac{\S(\s_k)}{\oS} \SNR_\mI,
%\end{equation}
%where $\R = N/2$ is the spectral resolution of the interferometer. This is the same result as the one derived by \cite{Davis:2001tr} and others. \textcolor{red}{[CHECK IF THERE IS A FACTOR OF 2, WITH SIMULATIONS]}

Defining the central wavenumber of the band as $\s_0$, the spectral resolving power of the transformed interferogram is $\R = \Dx N\s_0/2$. We introduce the sampling parameter $\samp = (\s_0\Dx)^{-1}$ which is the number of data samples per fringe for the central wavenumber in the band. The spectral resolving power at the band center can now be written $\R = \frac{N}{2\samp}$.  In practice one wants to pick a value of $\samp$ that ensures Nyquist sampling on the fringe for all wavenumbers in the band so $\samp \sim 3$ or greater is typically preferred. For a given integration time per data point (given $\SNR_\mI$), increasing the fringe sampling effectively increases the amount of time spent on the fringe, so the spectral $\SNR$ should increase with $\sqrt{\samp}$. Note that as long as we Nyquist-sample the fringe, there is no difference between multiplying the fringe sampling by some factor, and increasing the integration time per data point by the same factor, since in both cases the effective time on the fringe is equally increased.  %Hence, for a constant number of data points in the scan, increasing the sampling comes at a cost of decreasing the spectral resolving power. %Note that $\R$, $\samp$, $N$ and $x_{\textrm{max}$ are not independent parameters. 

It is useful to relate $\SNR_k$ to the $\SNR$ in the interferogram at the location of maximum intensity of the fringe, using physical quantities. The noise in each discrete measurement of the interferogram is $\sigI$. The signal at maximum intensity is $\mI_\textrm{max} = \D\s\oS$, where $\D\s$ is the width of the bandpass filter and $\oS$ is the average value of the signal in the band. Defining $\SNR_\mI = \mI_\textrm{max}/\sigI$, and noting that $\sqrt{N\Dx^2/2} = \frac{1}{\s_0}\sqrt{R/s}$, we obtain:
% PREVIOUS VERSION OF EQUATION
%\begin{equation}
%\SNR_k= \frac{\S(\s_k)}{\oS}\frac{\SNR_\mI}{2\Dx\sqrt{\samp\R}\Delta\s}.
%\label{eq:SNRratio}
%\end{equation}
\begin{equation}
\SNR_k = \frac{\S_e\sqrt{2}}{\sqrt{N}\Dx\sigI} = \frac{\S_e(\s_k)}{\oS}\sqrt{\frac{s}{\R}}\frac{ \s_0}{\D\s} \SNR_\mI .
\label{eq:SNRratio}
\end{equation}
Thus, the $\SNR$ in a channel of the final spectrum depends inversely on the square root of the resolving power $\R$ and the fractional bandwidth $\frac{\D\s}{\s_0}$; and it depends directly on the square root of the number of samples per fringe $\sqrt{\samp}$. 

%For BETTII's short spectral band, $\frac{\Delta\s}{\s_0}$ is in the range of 0.4 to 0.5 and we will be using $\samp=4$. The approximate SNR relationship for BETTII for a flat spectrum source is:  $\SNR_k \approx \frac{2}{\sqrt{\R}} \SNR_\mI$. 
%This result assumes a constant integration time at each point in the interferogram, so each data point has a set $\SNR$.  % and that the interferogram does not extend well beyond the region with coherence. In the latter case, $\SNR_k$ reverts to being proportional to $\SNR_\mI$ over the resolving power, i.e. there is a signal-to-noise penalty for including data where there is no coherence in the interferogram. 
%The effective length of the interferogram used for the FT can be reduced during data analysis in order to optimize the spectral $\SNR$.



\subsection{Effects of Gaussian OPD noise}

%Phase noise arises in an interferogram when the exact path offset from ZPD at the time a measurement is uncertain. The phase error is 2$\pi$ times position error divided by the central wavelength of the bandpass. For most FTS instruments, the phase noise is not a critical consideration since laser metrology can provide position measurements to high accuracy. For BETTII, the position relative to ZPD is a challenge because it depends on the entire pathlength through the instrument and the orientation of the baseline vector relative to the source direction. This problem will be similar for any space-based interferometry.

%An FTS instrument has three relevant timescales for phase noise: the integration time for a single data point, the time that it takes to gather one complete interferogram, and the time over which $M$ interferograms are gathered to complete an observation. These timescales can correspond to high frequency vibrations in the structure, pendulation motion of the instrument package hanging under the balloon, and a combination of pendulation and thermal changes, respectively. BETTII is designed and instrumented to mitigate the amplitude of these effects, but there will be residuals.

%The primary source of phase noise on BETTII is expected to be variations in the relative optical pathlengths in the two arms.
%If we assume that the pathlength noise follows a Gaussian distribution, then it is possible to derive an analytic expression for the effects. Systematic effects require a much more problem-specific data analysis scheme \citep{Fixsen:1994cs}; such analysis is strongly dependent on the detailed characteristics the instrument, which cannot be known before BETTII flies since the environment is very hard to reproduce for a system that large. However, in this discussion it is assumed that most systematic artifacts created in the phase domain can be fitted out and that the residuals are Gaussian.

%Consider the first case for noise within a single integration. The delay stage in a typical FTS is in constant motion so that each measurement point in the interferogram is an integral over the pathlengh $\Dx$ during the integration $\Dt$. As discussed in section \ref{sec:formalism}, we chose to include the signal lost due to this linear motion within $\Dx$, $\sinc(\pi\s\Dx)$, as part of the instrument passband response. It is also possible that the stage position relative to ZPD jitters at frequencies higher than $1/\Dt$. 
This section derives analytic expressions for the effects of Gaussian-distributed OPD noise. We look at the general case in order to derive sensitivities for double-Fourier instruments. Here, we suppose that the OPD from the delay line, the OPD within each arm of the instrument, and the OPD caused by an off-axis source are all measured or estimated with some residual error. Hence, the data points measured in the interferogram are associated with a delay value relative to ZPD, and if necessary, resampled to produce an evenly-spaced delay axis. This is necessary to use the FT and retrieve the spectrum. The noise on the delay estimate can be
characterized as a wavenumber-dependent phase error in the interference on the two beams. In the following, we quantify the impact of this noise on the spectral SNR, in order to understand how good our knowledge of the OPD needs to be to make sure the OPD noise effects are not dominant.
%LGM Added a little in here to make the connection between OPD errors and phase errors
%, and refer to the remaining phase errors after all systematics on all three noise timescales have been modeled and accounted for to produce the best estimate of the phase axis of the interferogram. This processing is necessary before the FT, because the transform expects samples that are evenly spaced on a known phase axis. 

%Although phase noise can be characterized as an uncertainty in OPD on three timescales (see Section~\ref{subsec:phnoise}), its effects on the spectral $\SNR$ can be characterized by the quadrature sum of the variances of the noise residuals on each timescale, due to the linearity of the FT process. We suppose that we have done our best to model the noise on all timescales, and that we are working with one final interferogram with all phase noise added to it.

%As described in Section~\ref{subsec:phnoise}, phase noise can be
%characterized as an uncertainty in OPD on three timescales:
%1) the time within a single integration, 2) the time to complete a single interferogram scan, and 3) the time to accumulate a
%full measurement of multiple scans.

%First, the effects of phase noise across the various timescales are identical. Indeed, the Fourier transform is linear and in the end, the FT of a co-addition of interferograms is the same as a co-addition of FTs of individual interferograms.
%LGM Added sentence below and deleted \sigma_x later in the paragraph.
Let's consider a single frequency signal first, so that the phase is proportional to the OPD. 
If we suppose that these residual phase errors $\Phir(x)$ are represented by a Gaussian distribution with zero mean and variance $\varPhir$, then the primary effect of the noise is to change the instantaneous power in $\I(x)$ by the factor $e^{i\Phir(x)}$. Now we consider a large ensemble of realizations of this noise distribution in order to predict its effect on the $\SNR$. Using the expression from \cite{Richards:2003bp}, for sufficiently small phase errors ($<\pi$ radians), the intensity of the coherent signal is reduced, on average, by a factor $e^{-\varPhir/2}$. For Gaussian-distributed OPD uncertainties with standard deviation $\lambda/20$, where $\lambda$ is the wavelength, the signal intensity is reduced by 5\%; for $\lambda/10$ the amplitude is reduced by 18\%. To give a practical example of the impact of this effect, we can consider the case of BETTII: if we assume that the uncertainty in the attitude of the payload is the only source of OPD noise, then knowing the attitude to within 0.1" rms will reduce the signal, on average, by 18\% at 40~$\um$.

For the polychromatic case, the delay position uncertainty, $\delta_x$, creates larger phase errors the shorter the wavelength, 
$\Phir(k) = 2\pi\delta_x \s_k$. A given error distribution of variance $\varopd$ in position yields a degradation across the band, $e^{-\varPhir(k)/2}$, with $\varPhir(k) = (2\pi)^2\varopd\s_k^2$.
% Figure \ref{fig:PhaseNoiseSim} shows how the signal decreases for different phase noise levels at a reference frequency.

Of course, the power lost from the coherent fringe pattern is still present in the scan; 
it becomes part of the incoherent signal seen by each output. 
In the limit where there is no spectral noise from the background or detectors, defining $\S_k\equiv\S_e(\s_k) $ we have:
\begin{equation}
\SNR_k= \frac{\S_k e^{-\varPhir(k)/2}}{\sqrt{\frac{1}{2\samp\R}\sum_{k'} \left[\S_{k'}^2(1-e^{-\varPhir(k')})\right]}},
\label{eq:noiseph}
\end{equation}
where $k'$ designates an index on all positive wavenumber bins. Note that $N=2\samp\R$. This relationship is identical to the one derived by \citet{Meynart:1992fv}, and we suggest an alternate and more detailed justification for it (see Appendix \ref{ap:phasenoise}). Studying this relationship, all the wavenumbers contribute to the white noise at a given wavenumber $\s_k$. The strongest lines (strongest $\S^2_{k'}$) and the shortest wavelengths (strongest $1-e^{-\varPhir(k')}$) contribute the most to the overall noise.
To summarize, considering an ensemble average of interferograms, OPD noise degrades the spectral $\SNR$ in two ways: first, it reduces the overall signal in the interferogram; second, it converts this lost power into white noise.

%\subsection{Combination of intensity and phase noise for co-added interferograms}

More realistically, observations will have
both intensity and OPD-generated spectral noise. In this case, the intensity noise and the scattered power
add in quadrature to give:
\begin{equation}
\SNR_k = \frac{\S_k e^{-\varPhir(k)/2}}{\sqrt{\frac{1}{2\samp\R}\sum_{k'} \left[\S_{k'}^2(1-e^{-\varPhir(k')})\right] + \samp\R\Dx^2\varI}}.
\label{eq:noisephth}
\end{equation}

The numerator of Eq.~\ref{eq:noisephth} shows that any amount of OPD noise will reduce the spectral $\SNR$. However, the impact of OPD noise is even greater when the power lost from the fringe is comparable to the intensity noise, as the first term of the denominator starts to matter. In fact, for arbitrarily large source fluxes, this equation reaches an asymptotical value which depends only on the OPD noise, and sets the maximum $\SNR$ achievable on average in a single scan. This is relevant for astronomical calibrators which can be so bright that the intensity noise term is negligible. In that case, assuming constant OPD noise, more $\SNR$ is only achievable by co-adding consecutive scans, as we discuss in the next section and in Appendix C. For most astronomical applications, where targets are usually faint compared to the intensity noise, it is expected that the first term of the denominator will be negligible.



\subsection{Co-adding interferograms}

Eq.~\ref{eq:noisephth} is the general case of a single interferogram with OPD and intensity noise. In practice, we would co-add $M$ interferograms in one ``track" to build up $\SNR$, but this puts stringent requirements on the performance of the control system and OPD estimator, because consecutive interferograms need to stay aligned with each other to within a small fraction of the carrier wavelength, to avoid causing OPD noise. The design and performance of the OPD estimator is highly implementation-specific, but most balloon and space designs will likely include an estimator that either directly measures the OPD, or indirectly infers it from the measurement of another quantity. 

A direct OPD measurement can be achieved for example with a fringe-tracking instrument, while an indirect OPD estimate can be an attitude measurement, which can be related to the OPD by simple geometry by using some assumptions. The latter scheme only works if the OPD errors are only influenced by pointing uncertainties over the timescale of a track, and that all other OPD contributors are modeled and corrected with comparatively high fidelity. The spectral $\SNR$ over $M$ scans can be determined from Eq.~\ref{eq:noisephth} by multiplying the whole equation by a factor of $\sqrt{M}$. The OPD noise term causing the phase noise variance $\varPhir$ then corresponds to the variance of the OPD uncertainties for each point of a scan, plus the variance of the OPD estimation error in determining the position of the center of each scan, which is necessary to properly co-align them (Appendix C).



%However, we are discussing two concepts that could be used to design double-Fourier instruments, namely a "proportional" phase estimator, and an "integral" phase estimator.

%The "proportional" estimators have no or negligible drift over a track. This would be the case of an estimator that determines the phase with a ZPD measurement at another wavelength (e.g. using a fringe tracker), or with an indirect attitude measurement at a high rate compared to the time it takes to acquire a single scan (timescale 2). On this timescale, it is reasonable to assume that most of the phase noise will originate from pointing uncertainties, so the phase is proportionally related to the attitude. In the case of an indirect method, though, one needs to be careful that the effects of all drifts that occur on longer timescales are properly corrected. For this type of estimator, 
%"Integral" estimators can have unpredictable drifts over a single track. This would be the case of an estimator that would use phase velocity measurements instead of phase position measurements, for example by integrating gyroscope information to obtain an attitude (and hence phase) estimate. In this case, there is a noise penalty for co-adding more interferograms, since the phase is an integral of a noisy term and will drift with time. Characterizing the variation of the phase noise as a function of $M$ is necessary to determine the spectral $\SNR$.




%Hence, the noise on the phase or attitude measurement is directly related to the actual phase noise in the interferogram. The advantage with this estimator is that it does not drift over the integration of multiple scans in a track, assuming the rest of the system is well-behaved. For this type of estimator, Eq.~\ref{eq:noisephth} can be multiplied by $M$ to measure the spectral $\SNR$ in a stack of $M$ interferograms, with $\varPhir$ corresponding to the variance of the error within each single interferogram, plus the error made by the estimator in determining the phase for each interferogram.

%A second type of estimator can consist of an attitude velocity measurement. This is common with payloads integrating gyroscope signals to determine their true attitude. In this case, there is a noise penalty for co-adding more interferograms, since the phase is an integral of a noisy term and will drift with time. This situation also occurs in the first type of estimator if there is an uncontrolled drift that an indirect attitude measurement could not measure (e.g. from a thermal variation that would change the alignment of the optics). This problem goes away if one uses a fringe tracker or any instrument that directly measures the absolute phase.


%The longer the integration time for $M$ interferograms, the larger the phase noise from timescale 3 is injected into the final product.

%It is possible to mitigate this problem by self-calibrating the phase between subsets of the $M$ interferograms, to relieve some of the estimator requirements. Indeed, finding the center of a fringe envelope can be done with a smaller $\SNR$ than what would be required to achieve good spectral $\SNR$. With noisy data, we estimate that a reasonable parabolic fit on the envelope would lead to finding the fringe center with an error variance of $\varPhir(\s) = (2\pi)^2 {\s^2/\s_0^2}/ (N_f\times\samp\times\SNR_\mI^2)$, where $N_f$ is the number of fringes with good $\SNR$ so that $N_f\times\samp$ corresponds to the number of points effectively used for the fit. This simple expression merely states that our ability to find the fringe center improves with the square root of the number of samples with good $\SNR$.

%The total phase error variance to be used in Eq.~\ref{eq:noisephth} is the sum of the error variance in estimating the fringe center of the subset, derived in the previous paragraph, plus the error variance intrinsic to each subset, which is caused by our drifting estimator. 
%The latter is mostly due to noise occurring on timescale 3 (between different scans), and we will assume that the residual phase uncertainties within each scan is negligible.

%Let's assume that the instrument's control system and phase estimator can maintain an OPD knowledge error of $\phi_0$ over $M_0$ consecutive scans, where $\phi_0$ is expressed in percentage of the central wavelength $1\over\s_0$, for convenience. In the case of a estimator's error characterized by a Gaussian distribution, the phase error variance after "blindly" co-adding $M$ interferograms is $\varPhir(\s) = (2\pi)^2\phi_0^2(\s^2/\s_0^2)(M/M_0)$. Let's also assume that $\F_0$ is the source flux for which $\SNR_\mI =1$ for a single interferogram. Then we have $\SNR_\mI^2 = M (\F/\F_0)^2$ as the $\SNR$ of the sum of $M$ scans for a source flux $\F$, and the total phase error variance is:
%\begin{equation}
%\varPhir(\s) = (2\pi)^2{\s^2\over\s_0^2}\left[\phi_0^2 {M\over M_0} + \frac{\F_0^2}{MN_f\samp \F^2}\right].
%\end{equation}
%To minimize the phase noise, an optimal number of scans per subset $M_s$ should be chosen. With the expression above, we find that $M_s = [M_0\F_0^2/ (\phi_0^2 N_fs\F^2)]^{1/2}$, and the minimum phase noise variance is:
%\begin{equation}
%\varPhir(\s) = (2\pi)^2{\s^2\over\s_0^2}\frac{2\phi_0\F_0}{\sqrt{N_fsM_0}\F}.
%\end{equation}

%This can be used in Eq.~\ref{eq:noisephth} to calculate the spectral $\SNR$ of a subset of $M_0$ scans, with $\varI$ corresponding to the variance of the intensity noise in $M_0$ scans. Stacking $M/M_0$ subsets in a full track leads to a spectral $\SNR$ increase of a factor $(M/M_0)^{1/2}$.

%This is usually a significant hit to the spectral $\SNR$, compared to an estimator with no drift, which is then favored in any double-Fourier instrument implementation. However, a robust, no-drift estimator with high precision can be very challenging and costly to implement on a flying or orbiting platform, and this analysis can be useful to provide a degraded performance estimate in the event that the main estimator implementation fails (e.g., if there is no guide star with enough $\SNR$ to use for fringe tracking, or if there are unknown drifts in the case of an attitude-only estimator). 
%In addition, for very bright sources like calibrators, each interferogram can be so bright that it is possible to measure the center of the fringe packet to better accuracy than the estimator itself. In this case, the phase noise variance will start to decrease linearly with increasing flux. This is a useful property to test the performance of an estimator before slewing to a fainter science target.
%However, during the time it takes to co-add these interferograms, there are residual uncertainties coming from phase noise over timescale 3, which prevents from perfect alignment of consecutive scans. 

%In this case, this expression applies to each interferogram in the track, and the total $\SNR_k$ needs to be multiplied by a factor $\sqrt{M}$. The phase noise term will then characterize to the variance of the noise residuals from timescale 1 and 2, plus the residual error variance made when aligning the consecutive interferograms on each other. This latter variance should increase linearly with $M$ if the same error is made each time we add one interferogram, but some strategies can be used to keep this error to a minimum (see Section~\ref{subsec: noisemitigation})

%If the interferograms can be added
%If, on the other hand, $M$ consecutive interferograms can be added in phase (perfect alignment of ZPD) then $\SNR_k$ improves as $\sqrt{M}$. Then the following expression would apply to each interferogram in a track, 
%In practice, an additional phase offset error is made when co-adding the interferograms, and the variance of this error needs to be added to the phase noise variance within the interferogram $\varPhir$. Hence, the total phase noise becomes a function of $M$ and can increase with time due to various drifts in the system. This is particularly important for observations where fringes cannot be seen in one single interferogram due to low SNR.

% In the case of BETTII, the anticipated $\NEPtot$ is large enough that most science sources will not contribute significantly to the incoherent signal. %(compare the total power from sky window and telescope to total power from a 1 Jy source in Appendix \ref{ap:system}).
%The source noise contribution is likely to be significant for space missions where the background is much lower and the detectors can be tuned to have lower NEP.



%In the case of $M$ averaged interferograms, the expression becomes:
%\begin{equation}
%\SNR_k = \sqrt{\textrm{M}}\frac{\S_k e^{-\varPhir(k)/2}}{\sqrt{\frac{1}{2\samp\R}\sum_{k'} \left[\S_{k'}^2(1-e^{-\varPhir(k')})\right] + \samp\R\Dx^2\varI}},
%\end{equation}
%where the phase noise $\varPhir(k)$ now represents the total phase noise over the $M$ interferograms. Note that the phase noise residuals are likely to increase with $M$, as drifts are introduced into the system by longer-timescale perturbations.


%[I SUGGEST TO DELETE THE FOLLOWING PARAGRAPH]
%This relation can be studied in a little more detailed in order to understand the impact of phase noise. For example, it is straightforward to study the case of a flat spectrum in band 1 of the instrument (see Table 1). We have $\S_e(\s_k) = \S_e$ in the band, and $\S_e(\s_k) = 0$ zero everywhere else. In that case, let's name $\SNRnp$ for the $\SNR$ that one would obtain with no phase noise. When we add phase noise, the minimum $\SNR$ is obtained for the smallest wavelength in the band, $\s_k = \s_\textrm{max}$. The equation simplifies to [MAYBE GET RID OF THIS EQUATION???]:
%\begin{equation}
%\SNR_\textrm{min} = \frac{e^{-\Delta^2_{\Phi, \textrm{max}}/2}}{\sqrt{\frac{1}{2\R}\sum_{k'} \left[(1-e^{-\varPhir(k')})\right] + 1/\SNRnp^2}}.
%\end{equation}

%%Note that the quantity $\frac{1}{2\R}\sum_{k'} \left[(1-e^{-\varPhir(k')})\right] $  does not actually depend on $\R$, since it is just the expression of the integral of $(1-e^{-\varPhir(k)})$. 
%
%\begin{figure}[ht!]
%\begin{center}
%\includegraphics[width=0.49\textwidth]{phasenoise_1.png}
%\caption[Impact of phase noise]{Impact of phase noise on the spectral SNR, in the case when no phase noise corresponds to a SNR of 5. The percentage is taken as a fraction of the wavelength corresponding to the central wavenumber of the band. }
%\label{fig:PhaseNoiseSim}
%\end{center}
%\end{figure}
%
%\subsection{Observing strategies}




%the phase noise manifests itself in two ways. First, there could be phase noise while scanning each individual packets within a scan. Second, there could be a phase uncertainty when lining up each consecutive scan, which corresponds to an error in identifying the precise location of the center of the fringe packet.

%In the first case, we expect the phase noise within each fringe packet to be very small. Each sample is taken every 2.5~ms and only a few fringes have decent signal to noise ratio, so the time spent on any given source is only a few tens of milliseconds. We do not expect to observe any noise at these high frequencies. Most phase noise sources are two to three orders of magnitude slower, hence will be very well behaved over this time period. For the purpose of this analysis, we will not consider this source of noise any further.

%In the second case, however, the timescale between scans is comparable to the largest phase noise source: the pendulum modes. While we will be actively controlling the payload, and do our best to control the delays in order to freeze the fringes, phase noise while co-adding two consecutive interferograms is inevitable. When looking at dim astronomical sources, where no fringes can be seen in one single scan, then we have no phase reference until we go back to a phase calibrator, and need to co-add the scans "blindly". Equation \ref{eq:noisephth} then is multiplied by a factor $\sqrt(M)$ and the phase noise corresponds to our ability to co-align M consecutive scans. 

%However, this analysis is focusing on the  residual errors \textit{after} all the corrections have been applied, and allows us to understand how well we need to correct for phase errors. 

%Let's suppose that both types of noise can be characterized by random processes, just as in the case of errors within individual scans discussed previously. In this case, the two types of errors are indistinguishable from each other as their variances simply add up and they contribute to an increase in the white spectral noise and a decrease in the observed signal. However, the white spectral noise is incoherent and averaged over $M$ interferograms, so the spectral $\SNR$ is increasing as the square root of the number of scans:
%\begin{equation}
%\SNR_k = \sqrt{\textrm{M}}\frac{\S_k e^{-\varPhir(k)/2}}{\sqrt{\frac{1}{2\samp\R}\sum_{k'} \left[\S_{k'}^2(1-e^{-\varPhir(k')})\right] + \samp\R\Dx^2\varI}}.
%\end{equation}
%Three regimes are noticed when plotting this equation (see Fig. \ref{fig:SpectralSNR}). In most scientific observations on BETTII, the source's spectral density $\S_k$ is small compared to the noise created by the intensity noise, so the effects of phase noise are small and the spectral $\SNR$ is a linear function of the spectral density. For increasingly larger source fluxes, the phase noise pushes the $\SNR$ towards an asymptotical behavior. But when the source is bright enough to have a $\SNR = 2.5$ in the interferogram, we can then control the phase noise and the increase of the $\SNR$ with spectral density is again linear. For normal BETTII operations and 3 second scans that cover the whole field of view, this value of $\SNR_\mI=2.5$ happens for around 50~Jy, which is considerable.

\subsection{Implications for spectroscopy}
A primary application for BETTII and proposed missions like SPIRIT will be the measurement of the spectral energy distribution
from warm dust associated with star formation in different environments. These types of measurements require broad wavelength
coverage but not especially high spectral resolution since the emission can be characterized as a sum of Planck functions over
a range of temperatures. For an instrument like BETTII, covering from 30-50~$\um$ and 60-90~$\um$ simultaneously,
$\R\sim 10$ in each band is sufficient to accomplish much of the science.

Spectral measurement with $\R\sim 10$ requires covering a delay range of $\pm 10~\lambda_0$ for a single source. On the other
hand, a delay range of 35-70~$\lambda$ (see Eq.~\ref{eq:delay}) is needed to move ZPD across 1 arc-minute of sky. Hence, typically,
the delay requirements for spatial coverage creates interferograms with higher resolution than needed to measure the continuum, and the full scan needs to be cut into smaller arrays around each target in the field. The size of these smaller arrays depends on the desired spectral resolving power $\R$, and the required sensitivity, as shown in Eq.~\ref{eq:SNRratio}. However, the additional data can be used for higher-resolution spectroscopy, for example to measure specific atomic lines in the far-IR. The $\SNR$ for lines is actually increasing with the square root of the number of data points in the interferogram, as the broadband noise gets more diluted in increasingly narrower spectral bins (see Eq. \ref{eq:signal}, \ref{eq:spectralSNR}). 
%(see Eq.~\ref{eq:noisephth} for a line of power $P$).

%As indicated by Eq.~\ref{eq:noisephth}, in the intensity noise-dominated regime, the $\SNR$ in the spectrum is proportional to ${ 1 \over \sqrt{R}}$.

As discussed for FTS instruments \citep[e.g.][]{Davis:2001tr},
apodization, the weighting of the points of the measured interferogram before applying the DFT, is one method for optimizing the $\SNR$
in the spectrum.
 The weight scheme is optimized to measure a specific type of spectrum: narrow line, broad features, continuum. 
The method relies on the fact that the data points close to the center or edges of a fringe packet contain information about low or high spectral frequencies, respectively. For example, if the purpose of an observation is to study continuum, it is appropriate to apply smaller weights to data points far away from the central fringe, since they add noise and very little $\SNR$. 

A common low-resolution spectroscopy case can be derived analytically
if a source has a spectrum following a power law distribution over the covered band. We can 
write $\S(\s) \propto \s^\alpha$ where the exponent $\alpha$ is the quantity of interest. 
Several methods have been developed to properly fit these power laws using maximum entropy and other 
techniques \citep[e.g.][]{Clauset:2007iy}. Here we use a simple estimator and provide 
a ready-to-use formula to help quantify the sensitivity of double-Fourier instruments.

By taking the logarithm of the spectrum, the problem is turned into a weighted linear fit in log-log space, where we want to determine the slope of a line. The noise in the new domain is $\sigL = \left|\frac{d(\ln(\S))}{d\S}\right|\sigspec = \sigspec/\S = 1/\SNR_\S$. The weights $\wk = 1/\sig_k^2$ of the linear fit are then simply the values of the spectral $\SNR$ squared at each data point, $\SNR^2_k$. The error on the weighted least square estimate of the slope is \citep{Bevington:2003tc}:
\begin{equation}
\sigalpha^2 = \frac{\sum \wk}{\sum \wk\sum \wk X_k^2 - \left(\sum\wk X_k\right)^2},
\end{equation}
where $X_ k \equiv \ln(\s_k)$ is the natural logarithm of the wavenumber for data point $k$. In the case of uniform spectral signal-to-noise ratio $\SNR_\S$ over $m$ points of the spectrum, this expression simplifies to:
\begin{equation}
\sigalpha^2 = \frac{1}{m\times\SNR^2_\S\times\VAR(X_k)}.
\end{equation}
This equation indicates that the variance of the spectral index estimate decreases with the number of points used to calculate the estimate, the spectral $\SNR$ squared, and the variance of the points distribution on the logarithmic wavenumber axis. For example, for 10 data points spread evenly from 30 to 55~$\um$, each with a spectral $\SNR$ of 5, we obtain an error on the slope determination $\sigalpha\sim 0.3$.

\section{Spectral sensitivity analysis for BETTII}
\label{sec:implications}
This section applies elements of the above discussion to BETTII. A general discussion on the details of BETTII can be found in \cite{2014PASP..126..660R}.
On BETTII, two mirrors collect light with an altitude-azimuth pointing system. The truss that holds the two mirrors moves in azimuth and determines the baseline vector, while the mirrors themselves move only in elevation. While BETTII does not physically rotate about the line of sight to cover different baseline angles, the payload always stays horizontal and the projection of its baseline vector changes as a source moves across the sky, hence covering different angles in the ($u, v$)-plane. The absolute OPD and ZPD of the instrument cannot be
measured, maintained, or known with perfect accuracy, especially during the flight itself, due to attitude estimation errors leading to our inability to perfectly estimate the orientation of the baseline vector in real time. In fact, a significant component of the mission's
design and implementation involves the selection and coordination of the suite of instruments which provide attitude measurements to construct the OPD estimator.

A second relevant aspect of BETTII is that the detectors are cryogenic bolometers \cite[see][for similar architectures]{2014ApJ...790...77S} with 1/f noise which
sets an optimal read-out time for the detectors of around 2.5 milliseconds (timescale 1). With BETTII's designed field coverage
of 2 arcminutes, full field scans consist of 1024 points and take 3 seconds to complete (timescale 2). Due to thermal emission from the atmosphere,
warm mirrors, and cryostat windows, BETTII will be in the background noise limited case for all science targets.
It is anticipated that 200 scans will typically be co-added to create one single visibility measurement over 10 minutes (timescale 3). 
For most source locations, the variation of the baseline orientation due to change in parallactic angle is not significant over this period.

\subsection{Noise sources and control system}
%The balloon environment makes observations in this wavelength range possible, but a large amount of background noise is still created by the residual atmosphere and the optics that are at ambient temperature.
Table~\ref{tab:noise} shows our estimates of the
background power levels associated with the atmosphere, warm optics, and windows in the two BETTII bands. 
The detectors themselves have been designed to have a noise level comparable to the background to optimize the use of the
dynamic range of the devices. The total NEPs of the short and long bands are expected to be $\sim 2\times 10^{-15}$ W.Hz$^{-0.5}$ and $\sim 1\times 10^{-15}$ W.Hz$^{-0.5}$, respectively. The source photon noise is negligible compared to the total NEP. 

Balloon instruments are subject to low frequency ($<~0.5$~Hz) pendulum modes and other oscillations introduced by the system's geometry and mass distribution, which make pointing a challenge. However, it is expected that the balloon environment is free of perturbations at any higher frequency (other than the instrument specific perturbations). Hence, sensors with high electrical bandwidth can robustly estimate the pendulum modes to gain accurate knowledge of the attitude, which can be used as our indirect OPD estimator since it is geometrically related to the phase on sufficiently short timescales.

The BETTII control system is organized with three different levels of control loops \citep{2014SPIE.9143E..3HR}: the coarse pointing loop, the fine pointing loop, and the OPD loop. The coarse pointing loop uses gyroscopes and star cameras to keep the baseline oriented within 10-15" of an appropriate near-IR guide star. A dichroic splits the near-IR (1-2$~\um$) from the far-IR (30-90$~\um$) inside the cryostat before the scanning delay line. The guide star is imaged through each of the two arms on two separate readout windows of a near-IR detector array that shares most of the optical path with the science channels. The fine control loop uses fast-steering tip-tilt mirrors, located at the pupils of each arm, to control the guide star image on each window and maintain good overlap of the beams at the science detectors. This loop reads the near-IR detector and generates a tip/tilt correction at 100~Hz. We expect to achieve beam overlap to within better than 1.5" at all times when a guide star is available. The spatial resolution of an individual BETTII beam
is 17" in the short wavelength band so this is a little better than 1/10th of a resolution element. The interferometric visibility loss
for this overlap error is anticipated to be less than 0.5\%.

We do not expect to be able to maintain the three dimensional orientation of the truss, and hence the baseline
vector, to much better than 10" rms, due to the various pendulum modes mentioned above and large inertia of the payload.
However, the errors in OPD introduced by pointing errors can be corrected directly using a delay line. BETTII uses a delay line external to the cryostat to correct the OPD at the entrance of the cryogenic volume. This delay line is completely separate from the science delay line which scans the OPD to produce the interferogram. Two delay lines are not a requirement for a double-Fourier instrument in general as the job can be done in theory by a single mechanism, with sufficient range and mechanical bandwidth. The external delay line on BETTII allows for the possible future upgrade
of correcting and monitoring the OPD outside of the cryostat using the near-IR channel by implementing a fringe tracker \citep{Rizzo:2012jp}.
%LGM: added the last sentence and made a couple other small changes.

For the OPD loop on BETTII, the angles of the tip/tilt mirrors which are used to maintain overlap of the beams act as an estimator of the baseline orientation, and hence as an indirect estimator of the OPD. The attitude estimates computed from these angles are fed to the external delay line so that the OPD at the entrance of the cryostat stays as constant as possible. Because the pendulation modes have periods of a few to tens
of seconds and should be well-behaved, we expect to be able to trust the control signals and estimate the attitude of the baseline vector to $\sim 0.12$" rms, which corresponds to a fifth of a detector pixel in the near-IR tracking array. A 0.12" attitude error indirectly corresponds to a delay uncertainty of 5~$\um$, or 12\% of a wavelength at 40 microns. This is a critical consideration when co-adding consecutive interferograms. With this amount of OPD noise, we expect, on average, a $\sim 25\%$ degradation in $\SNR$ for all sources in the short band, simply from the effects of phase noise in reducing the coherent signal (see Eq. \ref{eq:noisephth}).

Even with a stable OPD estimator, the absolute ZPD of the instrument must be measured during flight and tracked over long timescales as the instrument and the truss cool down to ambient temperatures ($\sim$240~K). This can be accomplished by observing a bright point source with known position periodically during a flight and identifying the center of the interferogram response (see Appendix C).

\subsection{Derived sensitivity and faintest detectable targets}

Incorporating these sources of noise with the formulas derived in the previous sections leads to the sensitivity values shown in Table~\ref{tab:sensitivity}. In this table we show the sensitivity in the two bands. The minimum detectable flux density (MDFD), which is the flux that provides $\SNR_\mI = 1$ in a single interferogram, is 15~Jy and 26~Jy in band 1 and 2 respectively. For 200 scans averaged with a OPD noise between scans of $5~\um$, the MDFD is 1~Jy and 2~Jy, using a matched filter efficiency of 0.5 and 0.4, respectively \citep{Mighell:2005fwa}. The faintest detectable spectroscopic point source that leads to a spectral $\SNR=5$ is 25~Jy and 13~Jy, respectively. These are determined for ``normal observing", which consists of co-adding 200 scans in 10 minutes that span the whole 2'x2' field of view, using a spectral resolution of $\R=10$ and a nominal OPD noise of $5~\um$~rms. 

At the bottom of the table, we also show the results in case we were using the instrument in an ``enhanced sensitivity" mode. This mode is mentioned here to illustrate the flexibility of the interferometer and its observing modes. It consists of increasing the individual integration time for each point in the interferogram by a factor of 3, while reducing the interferometric field of view by the same factor of 3: while the intrinsic field of view is unchanged at the detector, for the same scan time we only cover enough OPD range to cross ZPD for a subset of the pixels of the detector (and obtain a scan of the same length). This mode could be used for example for isolated targets which are located in less crowded star fields, by optimizing the time spent close to ZPD, where there is more signal (as we are interested in low-resolution spectroscopy). BETTII's observing parameters can be changed during flight so that the instrument stays flexible to optimize the chance of seeing fringes.

Finally, we show the overall sensitivity as a function of point source flux density (Eq.~\ref{eq:noisephth}) for both observing modes and both bands in Figure~\ref{fig:SpectralSNR}. In normal background-limited regime, the sensitivity curves should be straight lines. Here, OPD noise creates a decrease in overall sensitivity as a reduction in coherent power, but also, for brighter targets, from the power lost from the fringe that is converted to white noise (which causes a deviation from straight lines). For very bright targets of 50~Jy or more, it is possible to measure the OPD accurately within each interferogram by tracking the fringes in the science channels themselves (see Appendix C). For sufficiently large $\SNR$, this process has less error than the assumed $5~\um$ OPD noise coming from the indirect OPD estimation, so the OPD noise decreases for these very large fluxes to become negligible. This is particularly attractive for in-flight testing and calibration.

It is important to note that for sufficiently faint targets, it is impossible to accurately measure the OPD using single scans or co-adds of scans: we rely on the OPD estimator to have sufficient stability to properly co-add scans until the next calibration measurement. This needs to be considered carefully when planning the observation strategy, as long stretches without calibration could lead to a total loss of the OPD information (hence a total loss in scientific data), due to other OPD noise contributors such as thermal drifts that impact the payload on long timescales.
%\subsection{Phase noise mitigation strategies}
%\label{subsec: noisemitigation}

%There are two natural regimes for BETTII's interferograms. First, the high-$\SNR_\mI$ regime exhibits enough SNR that the fringes are visible in one single delay line scan. Second, the low-$\SNR_\mI$ regime does not have enough SNR and requires multiple consecutive interferograms to obtain a usable dataset. 

%In the high $\SNR_\mI$ case, when the source is bright enough to display a detectable fringe pattern in one single scan, then we can self-calibrate each scan with respect to each other by aligning each scan on the center of the fringe packet. We estimate that $\SNR_\mI=2.5$ would allow this. Also, we estimate that our ability to find the center of the fringe with a simple parabolic fit on the envelope has a variance related to $\SNR_\mI$ with the relation $\varPhir = (2\pi)^2 / (N_f\times\samp\times\SNR_\mI^2)$, where $N_f$ is the number of fringes with good $\SNR$ so that $N_f\times\samp$ corresponds to the number of points used to find the fringe center. We do our calculations supposing $N_f=3$ as a minimum. This simple expression merely states that our ability to find the fringe center improves with the square root of the number of samples with good $\SNR$.

%For the low $\SNR$ case, a number of scans are co-added in post-processing, using the post-flight best estimates for the attitude and the ZPD. The phase noise effectively stems from the residual uncertainty in the pointing solution, originating mostly in noise created over timescale~3. One can co-add the smallest amount $M$ of consecutive scans that would average to an overall $\SNR_\mI = 2.5$, effectively binning the 200 interferograms into subsets of $M$ within each track. The overall phase noise residuals (after post-processing) within each subset of scans is what influences the spectral $\SNR$, since a fringe is now visible within each subset, allowing for coherent co-adding of the subsets within a track. This method has the advantage that it relaxes the requirements on the phase estimator, since it can reduce the duration a certain phase noise requirement needs to hold.

%For 200 consecutive interferograms which are binned into subsets of $M$, the phase noise variance is the sum of the phase noise variance within each subset, plus the variance of the error in locating the center of the fringe packet within each subset. The error in locating the fringe center for $\SNR_\mI \sim 2.5$ corresponds to approximately 12\% phase noise for the short band (that is, 12\% rms of the central wavelength $1\over\s_0$). 

%For the short band, one single interferogram has a $\SNR_\mI=1$ for a source flux of 40~Jy. For the long band, it corresponds to about 80~Jy. For example a 20~Jy source in the short band requires $\sim~30$ consecutive scans to reach $\SNR_\mI \sim 2.5$. In order for the estimator not to be the limiting noise factor, it is reasonable to require that its residual noise to be about 12\% phase noise as well. At 3 seconds per scan, this means that the phase estimator needs to be good down to $\sim 5~\um$~rms over 1.5~min, for the short band. This also corresponds to relative attitude knowledge of 0.12~arcsec~rms over the same period. 

%In order to derive an expected spectral sensitivity curve, we suppose that in the short band the phase estimator error within a subset of $M$ scans has a variance that increases linearly with $M$ as $\sigma_M^2 = (2\pi)^2\times (0.12)^2\times (M/30)$, where we supposed a 12\% noise for $M=30$ scans. For sources less than 20~Jy, we suppose a fixed phase noise of 15\%, as fainter sources will benefit from the phase alignment from the brightest sources in the field of view. 

%Once the signal is so bright that a $\SNR_\mI=2.5$ is reached within one scan, the $\SNR_\S$ increases linearly with flux density, since the phase noise becomes quickly negligible.


%\subsection{Calibration}

%Calibration of the transmission profile and instrumental effects will also play an important role and could jeopardize the recovery of the source spectrum. However, it is difficult to measure the overall system's properties from the ground, due to the size of the instrument, the perturbations of the environment and the extremely large far-IR thermal background of ambient-temperature laboratories: interferometry from the ground with far-IR double-Fourier instruments of the size of BETTII is extremely challenging. In-flight calibration observations seem to be the best method of properly understanding the instrument at the system level. 

%Most instrumental effects can be determined by observing bright point sources. This is routinely done in ground-based radio interferometry. However, with BETTII, the sensitivity is such that these sources are rare. In addition it is often difficult to know if a bright far-IR astronomical source is actually a point source when observed at such resolution: this is precisely the problem that BETTII is trying to study. In order to characterize the instrument, we suggest to use bright solar system objects for our bandpass and flux calibrators. Although all the planets are too resolved for BETTII's 8~meter baseline, bright asteroids are not and can have typical fluxes of hundreds of Janskies in the BETTII bands, which provides excellent SNR even in one single delay line scan. 

%\section{Implications for BETTII Observations}
%
%
%
%\subsection{Observing strategies}
%Although it is not possible to achieve better sensitivity than the one imposed by the thermal noise background, there are strategies to keep the phase noise to a minimum over various timescales. With time-domain Fourier Transform spectrometers \citep{Griffiths:2007uu} it is often the case that delay space is scanned rapidly to optimize the integrity of the interferogram and multiple scans are collected to achieve the desired $\SNR$. For the bolometer-type detectors commonly used at infrared wavelengths, and planned for BETTII, short integrations are also favored due to the intrinsic $1/f$ noise \citep{deKorte:2003km}. In our case, however, we expect minimal residual phase noise within each individual scan. This naturally sets two regimes of observations: low $\SNR_\mI$ where the astronomical source is weakly detected or not detected in a single interferogram, and high $\SNR_\mI$ where individual interferograms can be used to estimate ZDP. 
%
%The observations will be divided in tracks of approximately 10 minutes (200 scans of 3 seconds). The scans are composed of 1024 points of 2.5~ms integration time each, in normal observing mode. In the short band, this corresponds to a fringe sampling of $\samp=4$, and in the long band $\samp \sim 7$. We assume that over one track the spatial visibility of the source is unchanged. Observing tracks at multiple baseline angles throughout the night gives us information about the source's spatial distribution. Hence the relevant sensitivity number corresponds to that of one single track of 10 minutes. Of course, this sensitivity can be improved by observing a point source at multiple occurrences during the night.
%
%\subsection{Thermal noise mitigation}
%
%In order to improve significantly the overall sensitivity, it is possible to take advantage of the fact that the delays creating phase noise are well behaved on short timescales. For example, we can decide to reduce the field of view and not cover the full stroke of the delay line, while keeping the scan speed constant. If we shorten the field of view by a factor of 3, then this gives us 3 times the number of interferograms for the same time period, which allows us to pick up interferograms (1-$\sigma$)  in one scan at $\sim~25$~Jy, instead of $\sim~40$~Jy for normal observing. For the first flight, where we will learn more about the perturbations and their exact timescales, it will be possible to tune the scan speed and stroke parameters as the payload is floating.
%
%This mitigation method provides an "enhanced sensitivity" mode and is particularly relevant for fields where we know the emission is localized within a small given region, and not extended over the full $2\times 2$ arcminutes. Note that the region of interest does not need to be centered around the middle of the delay line range in particular.
%
%The prospect of slowing down the scan, rather than reducing the field of view, is more complicated. Indeed, cold delay line is intimately connected to a number of different gains within the detector readout electronics. A good ground calibration is necessary, and it is much harder to change things while in flight, as changing the scan speed without re-tuning the gains could create undesirable effects \citep{Fixsen:1994cs}.
%
%
%
%\subsection{Phase noise mitigations}
%In the high $\SNR_\mI$ case, when the source is bright enough to display a detectable fringe pattern in one single scan, then we can self-calibrate each scan with respect to each other by aligning each scan on the center of the fringe packet. We estimate that $\SNR_\mI=2.5$ would allow this. Also, we estimate that our ability to find the center of the fringe with a simple parabolic fit on the envelope has a variance related to $\SNR_\mI$ with the relation $\varPhir = (2\pi)^2 / (N_f\times\samp\times\SNR_\mI^2)$, where $N_f$ is the number of fringes with good $\SNR$ so that $N_f\times\samp$ corresponds to the number of points used to find the fringe center. We do our calculations supposing $N_f=3$ as a minimum. This simple expression merely states that our ability to find the fringe center improves with the square root of the number of samples with good signal-to-noise ratio..
%
%For the low $\SNR$ case, $M$ scans are co-added blindly in post-processing, using the post-flight best estimates for the attitude and the ZPD. The phase noise effectively corresponds to the error in the pointing solution. One can co-add the smallest amount of consecutive scans that would allow for an overall $\SNR_\mI = 2.5$, effectively binning the 200 interferograms into subsets within each track. The overall phase noise residuals (after post-processing) within each subset of scans is what influences the spectral $\SNR$, since a fringe is now visible within each subset, allowing for coherent co-adding of the subsets within a track. This method has the advantage that it relaxes the requirements on the phase noise estimator, since it can reduce the duration a certain phase noise requirement needs to hold.
%
%For 200 consecutive interferograms which are binned into subsets of $M$, the phase noise variance is the sum of the phase noise variance within each subset, plus the variance of the error in locating the center of the fringe packet within each subset. The error in locating the fringe center for $\SNR_\mI \sim 2.5$ would be approximately equal to the phase noise within each subset for about 12\% phase noise, for the short band. 
%
%For the short band, with normal observing parameters, one single interferogram has a $\SNR_\mI=1$ for a source flux of 40~Jy. For the long band, it corresponds to about 80~Jy. Hence, for example a 20~Jy source in the short band requires $\sim~30$ consecutive scans to reach $\SNR_\mI \sim 2.5$. At 3 second per scan, this means that the phase estimator needs to be good down to $\sim 5~\um$~rms (12\% of the short band central wavelength) over 1.5~min. This also corresponds to an attitude knowledge of 0.12~arcsec~rms. 
%
%In order to derive an expected spectral sensitivity curve, we suppose that in the short band the phase estimator error within a subset of $M$ scans has a variance that increases with $M$ as $\sigma_M^2 = (2\pi)^2\times (0.1)^2\times (M/30)$, where we supposed a 10\% noise for $M=30$ scans. For sources less than 20~Jy, we suppose a fixed phase noise of 15\%, as fainter sources will benefit from the phase alignment from the brightest sources in the field of view. 
%
%Once the signal is so bright that a $\SNR_\mI=2.5$ is reached within one scan, the $\SNR_\S$ increases linearly with flux density, since the phase noise becomes quickly negligible.
%
%
%
%
%\subsection{Faintest detectable targets}
%
%The BETTII 10-minutes sensitivity is estimated in Table \ref{tab:sensitivity}, for $\R=10$ in both bands, assuming the strategies mentioned in the previous paragraphs. The enhanced sensitivity mode shown here assumes a factor of 3 longer integration time per data point. %The complete list of parameters, including efficiencies, is summarized in the appendix in \ref{tab:params}.
%
%
%The 200 scans are co-added assuming that the object's visibility at that baseline orientation is not changing during the integration ($\Vb = $ constant). Those 200 scans correspond to roughly 10 minutes of on-source time. It is important to point out that BETTII would go back to that target in order to observe it at another angle. If that is the case, and the object is a point source for the interferometer, the new 200 scans can be co-added to the previous 200 scans, hence improving significantly the spectral $\SNR$. 
%\section{Calibration}
%\subsection{Overall strategy}
%
%BETTII will need calibrator targets to help understand the properties of the instrument. For BETTII, ground testing at far-IR wavelengths is extremely challenging, so we will have to spend a significant portion of the first flight looking at a bright calibrator targets in order to characterize systematics like the visibility losses, the bandpass filter shape, etc. For this step, we seek a bright calibrator with known spatial distribution and spectrum.
%
%One calibration step which is of paramount importance occurs before we make any science observations. It consists in looking at a bright point source that allows us to locate ZPD for the first time. Although the instrument will be aligned on the ground so that ZPD falls precisely at the middle of the range of the delay line, thermal changes during ascent will certainly change the location of ZPD. By measuring the location of the fringe center on that first delay line scan, it is possible to adjust the external delay line in order to put ZPD in the center of the stroke of the internal delay line. When ZPD is at the center of the scan range, one delay line stroke will go through ZPD for every pixel on the detector.
%
%Between tracks, BETTII will need to look at calibrator targets in order to monitor the variation of instrumental properties with time. To monitor the timescales of changes in phase and tune our pointing and phase estimators, it would be ideal to measure this pointing error as frequently as possible, which is once every scan. These calibrators do not need to be extremely bright - although a minimum number of scans should be required to reach rapid ZPD finding. In addition, we want to limit the slew range from our scientific targets and that calibrator, in order to optimize the time on source and minimize the disturbances to the instrument between science and calibrator targets.
%%\subsection{Errors in the bandpass profile calibration}
%%
%%\textcolor{red}{If we can measure the bandpass profile only to within n\%, how does that impact our estimate of the spectral index? In practice, how do we do this? Do we do some sort of deconvolution, or we simply divide by the bandpass profile?}
%%
%%\textcolor{red}{Ask Lee how this is done most likely, and run simulations.}
%
%%\subsection{Directly measuring the phase}
%
%%It would be convenient to directly measure the group delay of the incoming light, which would provide an absolute measurement of the location of ZPD. We have investigated the feasibility of a near-IR fringe tracking system with simple envelope tracking, as well as dispersed fringe tracking \citep{Rizzo:2012jp}. Although this system would be robust enough to work at low signal-to-noise ratios, it puts considerable requirements on the differential wavefront errors between the two arms, which are hard to control because of the thermal environment at float. Although BETTII will fly this near-IR instrument, its success does not determine the success of the overall mission. [Stephen's comment: Although the technical success of BETTII is not dependent upon the use of such a near-IR instrument, it will include it in order to….
%%(why are we including it?  hope?  future work?  interesting characterization data?)]
%
%
%
%%\subsection{Bandpass profile measurement}
%%% THIS SEEMS ODD HERE... THINKING ABOUT WHERE TO MOVE IT
%%In recovering the scientific signal $\Be\Vb$, one needs to know the bandpass profile, $\eta\Tbp$ (see Fig \ref{fig:Transmission}). While this will be determined to first order from laboratory calibration of the filters, cryostat windows, and detectors, it could be hard to estimate or measure with enough accuracy once the system is fully assembled. 
%%
%%\begin{figure}[ht!]
%%\begin{center}
%%\includegraphics[width=0.49\textwidth]{phasenoise_3.png}
%%\caption[transmission]{Typical transmission profile $\eta\Tbp$ \citep[inspired by the IRAS transmission spectrum,][]{Whelan:2011}. [NEED TO CHANGE THIS WITH CARDIFF DATA]}
%%\label{fig:Transmission}
%%\end{center}
%%\end{figure}
%%
%%To ensure that we obtain a proper end-to-end characterization of our bandpass profile, we plan on including regular calibrator observations during the flight. This calibrator can be used as a bandpass, flux, and phase calibrator at once. Depending on the online stability of the system, these calibrator observations will be more or less frequent. Calibrator choices are discussed in section~\ref{sec:calibrators}.
%
%\subsection{Calibrator selection}
%\label{sec:calibrators}
%
%BETTII requires bright targets of known spectrum and spatial extension in order to achieve good calibration. While some astronomical objects such as YSOs can sometimes provide the required fluxes (more than 60 Jy), their spatial extension is often uncertain - it is precisely the goal of BETTII to determine their spatial extension, or whether these bright sources are composed of multiple, dimmer sources. Previous missions might not have the angular resolution or the detector properties to draw conclusions on the brightest sources (most detectors are optimized for faint targets and hence saturate on the brightest sources, like Spitzer and WISE). 
%
%Among the objects that are very bright within our bands are the planets of the solar system. We have computed the exact distances of the planets during the month of October, 2015, which is approximately the center of our launch window. Only Uranus and Neptune are up in the sky by night, and their respective extensions are 3.7" and 2.3". This is a significant hit on the visibility, and although Uranus is $~1000$ Jy, it is not usable as a reliable calibrator (see Fig. \ref{fig:Visibility}). If they were available in the night sky in 2015, the moons of Jupiter or Saturn would be excellent candidates, provided that the leaking light from the planet does not disturb the observations too much. The moons are point sources for BETTII and there will often be more than one moon within the instrument's field of view, which would help calibrate the drifts within each individual scan, and show how accurately BETTII can measure the relative distance between two sources.
%
%
%
%Since the planets are not useful for next summer, we need to look towards other types of calibrators. Our favorite option is to look at the solar system's asteroids. Asteroids have been used for calibration before, and their thermal emission properties is well understood \citep{1998A&A...338..340M}. Although they are typically one tenth of the flux of Uranus, they are all unresolved so their visibility is unity. In particular, Ceres is a point source about 800 Jy in the short band, which should provide a $\SNR\sim 20$ in one single interferogram scan. In addition, Ceres is an available target during our launch window. Other asteroids, like Vesta and Pallas, are also excellent candidates for calibrating the instrument. Their spectrum has been extensively studied and successfully used as calibrators by other far-IR facilities such as Herschel \citep{Muller:2002je}.



%A mix of other potential targets are discussed in Table~\ref{tab:calibrators}. These targets are all available from our launch site in the night sky throughout the launch window in 2015. All these calibrators are bright enough in the far-IR bands, as well as in the near-IR bands that allow us to track each individual arm and overlap the beams.

%\begin{table*}[ht!]
%
%\begin{center}
%\begin{tabular}{|c|c|c|c|c|c|c|}
%\hline
%\textbf{Name of source} & \textbf{Type} & \multicolumn{3}{|c|}{\textbf{Published Fluxes (Jy)}}& \textbf{Spatial} &\textbf{ Comment} \\
%&  & \textbf{$25\um$} & \textbf{$60 \um$} &\textbf{$100\um$}  &  \textbf{extension?} &\\
%\hline
%
%Betelgeuse & Red giant & 1700 & 299 & 96 & 40 mas &  \\
%\hline
%V* IK Tau & Variable star & 2380 & 332 &103 & ?? &  \\
%\hline
%IC342  & Starburst galaxy & ?? & ?? & ??  & ?? & A. Schulz et al.: The nucleus of the nearby galaxy IC342  \\
%\hline
%EM* LkHA 101  & Star (?) & 364 & 3236 & 4575  & ?? & Looks like a good candidate - A dusty torus around the luminous young star LkHalpha 101 (Tuthill et al.)  \\
%\hline
%HD 36982  & Variable star & 367 & 4800 & 24  & ?? & Promising - see the paper from bob \\
%\hline
%UGCA 39  & Galaxy & 9 & 93 & 227  & ?? &  \\
%\hline
%
%\end{tabular}
%\caption{Calibrators}
%\label{tab:calibrators}
%\end{center}
%\end{table*} 
\section{Conclusion}

%This paper proposes analytical tools to predict the spectral SNR of single-baseline spatio-spectral interferometers, as a function of intensity and phase noise. The phase noise is divided in three major timescales that are relevant to all spatio-spectral instruments. The derived expressions are applied to the case of BETTII, a 8~meter baseline interferometer that pioneers this technology from a balloon platform to study bright star forming regions. We also present methods to mitigate some of the phase noise effects expected in BETTII's data products. 

Spatio-spectral interferometry can enable high resolution spectral imaging of wide fields at
far-IR wavelengths. Implementation of the technique provides some new instrumental
challenges compared to traditional Fourier Transform Spectroscopy, such as the fact that the measured spectrum is a mix of the source's spectral and spatial information.
%complex nature of the source visibility function. The intrinsic interferogram
%has odd symmetry, rather than even symmetry for an FTS, due to the symmetry of the two light
%paths. The source visibility, which can be a complex function, mixes with the spectral
%response to create an interferogram with odd and even symmetry components. The resulting source
%spectrum, which is a mix of spectral and spatial information, can be complex valued function of wavenumber.

%In a double-Fourier system, the zero path difference occurs at a different delay setting for
%each pixel column perpendicular to the baseline vector. As the delay line sweeps the optical path difference
%between the two light paths to create the interferogram for each pixel, it also covers ZPD for
%each pixel. For scientifically interesting field coverages, the delay stroke needed to cover the
%field is equivalent to a spectral resolving power of 100's to 1000's.

In a double-Fourier system, the zero path difference for each detector pixel occurs at a different delay setting of the delay line. The delay stroke needed to cover a scientifically interesting field of view is equivalent to a spectral resolving power of 100's to 1000's for the central pixels.

We present an analysis of the impact of Gaussian intensity and OPD noise on the spectral sensitivity.
Intensity noise, essentially thermal noise from the optics, sky, astrophysical background, and detector, is similar to noise in 
FTS systems with the exception that the longer scan lengths required to cover the spatial
field add noise; this can be mitigated by cutting the interferogram for each pixel  into smaller arrays centered on each source's ZPD to match the desired spectral resolving power, and by apodizing the interferogram to increase sensitivity to the spectral properties of interest. OPD noise is not usually relevant for FTS systems, but is intrinsic to double-Fourier instruments, since the two incoming beams go through long separate paths before combination. For instruments on balloons or in space, the OPD noise is expected to be dominated by disturbances from the instrument and from pointing errors. On average, OPD noise reduces the coherent power in the interferogram, and converts the power lost from the fringe into additional white noise in the spectrum. We argue that there are three relevant noise timescales: the time to take a single data point, the time to collect a complete interferogram, and the time to co-add $M$ interferograms together in a track. The latter corresponds to the timescale that the source spatial visibility function changes significantly, due to the rotation of the baseline angle on the sky.

We derive the spectral sensitivity of double-Fourier instruments to intensity and OPD noise. The expressions in this paper are derived in the general case and can be used to design any instrument that implements this method.

Applied to the case of BETTII, these equations lead to spectral sensitivity estimates of 25 and 13~Jy in its 30-50~$\um$ and 60-90~$\um$ bands, respectively, to achieve a spectral $\SNR=5$ in 10 minutes with $\R=10$ and an assumed OPD noise of $5~\um$ rms.




%BETTII is designed as symmetric double Fourier interferometer to obtain spatial-spectral data over an extended field. The zero path difference in the two arms is swept across the detector array to create a delay spectrum for each detector. 
%
%This paper presents an analytic form for the interferometric signals and their Fourier Transforms, which correspond to the usable scientific data product that BETTII will obtain. The signal-to-noise ratios is greatly influenced by both thermal noise and phase noise components, which both occur in the interferogram domain but have a severe impact in the spectrum. 
%
%We present ways to mitigate both types of noise by strategically controlling the observation and the data reduction process. Using the current estimated parameters of the system, we derive an expected spectral sensivitity of the instrument over 10 minutes, which amounts to 34 Jy in the short band, and 32 Jy in the long band, with a spectral resolution of $\R=10$. Systematic effects could degrade this sensitivity further, and data processing methods like apodization of the interferograms could improve it.
%
%Finally, we identify Ceres and the other bright asteroids as our best candidates for calibrator targets for BETTII's first flight in 2015.
%

%The ability to determine the location of a source on the sky is discussed 



%The optical system is being designed with a high degree of symmetry and a careful selection of structural materials to minimize changes in differential path length along the two arms as the instrument changes temperature from ground-level to flight altitude ($\Delta T$ of 50~K or more). Path length differences can be caused by temperature gradients across the structure; the impact is being minimized by use of carbon-fiber trusses, thermal shielding, and thinned, low heat capacity structures to facilitate rapid cooling.

%The noise power for BETTII at flight altitude a combination of contributions from the sky, the cumulative emissivity of the mirrors outside the dewar, and the dewar window. It is anticipated that the mirror component will dominate in the 30-50 micron band, and the mirrors and sky will contribute significantly in the 60-90 micron band.

%Phase noise in BETTII arises primarily from error in the instantaneous position of the delay line mirrors and uncertainty in the orientation of the baseline vector between the siderostats.  Knowledge of the baseline will be obtained precision gyroscopes and pointing corrections from the tip-tilt mirrors tracking a guide star during observations. The critical measure is one arc-second in truss angle corresponds to 40 microns in optical path difference.



% \acknowledgments

% The material presented in this paper is based upon work supported by NASA Science Mission Directorate through the ROSES/APRA program, with additional support from NASA Goddard Space Flight Center, and NASA GSFC grant NNX11AG92A to the University of Maryland. Work by T. Veach was supported by an appointment to the NASA Postdoctoral Program at GSFC, administered by the Oak Ridge Associated Universities under contract with NASA. We would like to thank the anonymous referee for suggested improvements to the paper.


\begin{table}[ht!]
\begin{center}
\begin{tabular}{|c|c|c|c|}
\hline 
\textbf{Parameter} & \textbf{Band 1} & \textbf{Band 2} & \textbf{Comment} \\ 
\hline 
Window emissivity & 0.02 & 0.02 & Measured in the lab \\ 
\hline 
Telescope emissivity & 0.077 & 0.077 & 10 mirrors at 0.992 reflection\\ 
\hline 
Sky radiance & 0.16 W.m$^{-2}$.sr$^{-1}$ & 0.07 W.m$^{-2}$.sr$^{-1}$ & \cite{Harries:1980cva} \\ 
\hline 
Window radiance & 0.17 W.m$^{-2}$.sr$^{-1}$ & 0.04 W.m$^{-2}$.sr$^{-1}$ & Blackbody at 240~K \\ 
\hline 
Telescope radiance & 0.17 W.m$^{-2}$.sr$^{-1}$ & 0.04 W.m$^{-2}$.sr$^{-1}$ & Blackbody at 240~K \\ 
\hline 
Total optical efficiency & 0.3 & 0.3 & Per arm, includes detectors \\ 
\hline 
Photon power from the sky & 35 pW & 36 pW & • \\ 
\hline 
Photon power from the window & 40 pW &  18 pW & • \\ 
\hline 
Photon power from the telescope & 108 pW & 38 pW & • \\ 
\hline 
Total absorbed photon NEP & 1.4$\times 10^{-15}$ W.Hz$^{-0.5}$ & 7$\times 10^{-16}$ W.Hz$^{-0.5}$ & From each arm \\ 
\hline 
Detector NEP & 3$\times 10^{-16}$ W.Hz$^{-0.5}$ & 3$\times 10^{-16}$ W.Hz$^{-0.5}$ & For each detector \\ 
\hline 
Total NEP & 2$\times 10^{-15}$ W.Hz$^{-0.5}$ & 1$\times 10^{-15}$ W.Hz$^{-0.5}$ & For the sum of both arms \\ 
\hline 
\end{tabular} 
\caption{BETTII noise parameters}
\label{tab:noise}
\end{center}
\end{table} 
\begin{table}[ht!]
\begin{center}
\begin{tabular}{|c|c|c|c|}
\hline \multicolumn{4}{|c|}{\textbf{Single scan}} \\
\hline 
     & Band 1 &  Band 2 & SNR Target \\
\hline
MDFD & 15 Jy  & 26 Jy & $\SNR_\mI = 1$\\ 
\hline
\hline \multicolumn{4}{|c|}{\textbf{Normal observing (200 scans, 10 min)}} \\
\hline
MDFD & 1 Jy  & 2 Jy & $\SNR_\mI = 1$\\ 
\hline 
Faintest pt. source & 25 Jy  & 13 Jy & $\SNR_k = 5$\\ 
\hline 
\hline
\multicolumn{4}{|c|}{\textbf{Enhanced sensitivity (200 scans, 10 min)}} \\
\hline
Faintest pt. source & 14 Jy  & 7 Jy & $\SNR_k = 5$\\ 
\hline 
\end{tabular} 
\caption{BETTII sensitivity estimates}
\label{tab:sensitivity}
\end{center}
\end{table} 


\begin{figure}[ht!]
\begin{center}
\includegraphics{Figures/f1.eps}
\caption[WideField]{Concept of wide-field double-Fourier interferometry. Light from the instrument is focused after combination to an image of the sky on the detector array (represented as the grid). Each column of the detector has a distinct ZPD so the interferometric responses (right side) of two sources on different columns are centered around different delay positions. The gray stripe represents the central column on the detector array and its corresponding ZPD on the interferograms.}
%The instrument is represented by the double-headed arrow to the left. This encompasses the light collectors, the telescopes, the tip/tilt correction stages, the pupil re-imagers, and the beam combiner. After the beam combiner, the beam is focussed onto the detector array shown in this picture. Changing the OPD modulates the intensity of the Airy disks of each source in the field, and we get interferograms, as shown to the right. In this sketch we suppose that the baseline is aligned with the rows on the detector. Hence, each column of pixels always has the same instrumental OPD. In the general case, pixels along a line perpendicular to the baseline have the same OPD. Further, each source at an angle $\theta$ from the central column has an external OPD equal to $|\baseline |\sin\theta$. When the instrumental OPD approaches the opposite of this value, fringes can be seen for the source.}
\label{fig:widefield}
\end{center}
\end{figure}

\begin{figure}[ht!]
\begin{center}
\includegraphics{Figures/f2.eps}
\caption[optics]{Optical train diagram of a typical far-IR, double-Fourier instrument. The K-mirror rotates the beam to align the fields of view of the two sides. Inside the cryostat, a set of optics re-image the pupil, implement a controlled instrumental delay between them with the Cold Delay Line, and relay them towards the central beam combiner. After the combiner, the beams are imaged onto the detectors. To see the BETTII-specific implementation of this design, see \cite{2014PASP..126..660R}.}
\label{fig:optics}
\end{center}
\end{figure}

%\begin{comment}
\begin{figure}[ht!]
\begin{center}
\includegraphics{Figures/f3.eps}
\caption[interfero]{Effects of phase and intensity noise on the recovered spectrum (single realization of the noise). Left column: normalized interferograms, intensity as function of OPD. Right column: normalized DFT of interferograms. Solid: input spectrum multiplied by anti-symmetric transmission function; Solid circles: Imaginary part of DFT from interferogram; Dotted: Real part of DFT. First row: ideal measured signal, no noise; used for normalization of all other plots. Second row: results with a realization of phase noise of 10\% at each point of the interferogram. Third row: results with a realization of intensity noise and $\SNR_\mI=10$.}
\label{fig:interfero}
\end{center}
\end{figure}

%\begin{figure}[ht!]
%\begin{center}
%\plotone{f4.eps}
%\caption[Impact of phase noise]{The phase noise affects the signal more for larger wavenumbers. This shows the simulation for band 1 of BETTII. The amount of phase is quantified by its amount in terms of percentage RMS of the wavelength corresponding to the central wavenumber, about 39 microns (256 cm${-1}$) in our case.}
%\label{fig:PhaseNoiseSim}
%\end{center}
%\end{figure}

%\begin{figure}[ht!]
%\begin{center}
%\plotone{f5.eps}
%\caption[exinter]{Example of simulated recovered spectrum of a 40 Jy source over one 10-minute track with $\R=10$, in the presence of  expected thermal noise levels and phase noise of 10\%. The chosen source spectrum follows a simple power law $\S(\s)\propto\s^2$.}
%\label{fig:exinter}
%\end{center}
%\end{figure}

%\begin{figure}[ht!]
%\begin{center}
%\plotone{f6.eps}
%\caption[Phase Noise]{Residual phase noise profile. The phase noise is expressed in percentage of the central wavelength of the short band (\% of 40~$\um$). }
%\label{fig:SpectralSNR}
%\end{center}
%\end{figure}

%\begin{figure}[ht!]
%\begin{center}
%\plotone{f7.eps}
%\caption[Spectral SNR]{Effects of phase noise and background noise on the spectral $\SNR$. This simulation is computed for BETTII's band 1 parameters and noise estimates, for R=10. We show the results in both the normal science observing mode and the enhanced sensitivity mode. Both simulations are for 200 averaged scans, or 10 minutes on a source. It is important to note that one source can be observed at multiple times during the course of one flight, hence increasing the spectral $\SNR$.}
%\label{fig:SpectralSNR}
%\end{center}
%\end{figure}
%\end{comment}
%\begin{figure}[ht!]
%\begin{center}
%\plotone{f8.eps}
%\caption[Visibility]{Effects of source extension with an 8~m baseline interferometer, for band 1. The visibility at the central wavelength is larger than the integrated visibility over the band, especially given that our band is extremely wide.}
%\label{fig:Visibility}
%\end{center}
%\end{figure}
\begin{figure}[ht!]
\begin{center}
\includegraphics{Figures/f4.eps}
\caption[BETTII Spectral SNR]{%Effects of phase noise and background noise on the spectral $\SNR$. This simulation is computed for BETTII's band 1 parameters and noise estimates, for R=10. We show the results in both the normal science observing mode and the enhanced sensitivity mode. Both simulations are for 200 averaged scans, or 10 minutes on a source. It is important to note that one source can be observed at multiple times during the course of one flight, hence increasing the spectral $\SNR$.
BETTII's spectral sensitivity. Solid: Normal observing mode, band 1; Dashed: Enhanced sensitivity mode, band 1; Dotted: normal observing mode, band 2; Dot-dashed: Enhanced sensitivity mode, band 2. This plot includes the technique of fringe tracking in the science channel for sufficiently bright sources (see Appendix C). As the source flux rises, the effects of the phase noise become larger and the SNR should reach an asymptotical value. However, with fringe tracking, the phase noise itself becomes smaller since one can see fringes in one single or a few consecutive scans, so the co-adding becomes easier. Thanks to the fringe-tracking, there is no regime where the phase noise is expected to be dominant on BETTII, provided that the control system performs according to expectations.}
\label{fig:SpectralSNR}
\end{center}
\end{figure}
 
% Chapter 3

\chapter[Attitude estimation and control for BETTII]{\setstretch{1}Attitude estimation and control for BETTII} % Main chapter title

\label{chap:controls} % For referencing the chapter elsewhere, use \ref{Chapter1} 

\epigraph{\setstretch{1}\small\itshape Don't bother just to be better than your contemporaries or predecessors. Try to be better than yourself.}{W. Faulkner}



\usetikzlibrary{decorations.markings}





\newcommand{\savedx}{0}
\newcommand{\savedy}{0}
\newcommand{\savedz}{0}
\newcommand{\bettii}[1]%
{   
\pgfmathsetmacro{\boxsize}{#1}
\coordinate (O) at (0,0,0);
\coordinate (Ox) at (#1,0,0);
\coordinate (Oy) at (0,#1,0);
\coordinate (Oz) at (0,0,#1);

\coordinate (a) at (0.5*\boxsize,-4.5*\boxsize,0);
\coordinate (b) at (0.5*\boxsize,4.5*\boxsize,0);
\coordinate (c) at (-0.5*\boxsize,-4.5*\boxsize,0);
\coordinate (d) at (-0.5*\boxsize,4.5*\boxsize,0);
\coordinate (e) at (0.5*\boxsize,-3.5*\boxsize,-\boxsize);
\coordinate (f) at (0.5*\boxsize,3.5*\boxsize,-\boxsize);
\coordinate (g) at (-0.5*\boxsize,-3.5*\boxsize,-\boxsize);
\coordinate (h) at (-0.5*\boxsize,3.5*\boxsize,-\boxsize);
\coordinate (zc1) at (0.5*\boxsize,0.5*\boxsize,\boxsize);
\coordinate (zc2) at (0.5*\boxsize,-0.5*\boxsize,\boxsize);
\coordinate (zc3) at (-0.5*\boxsize,-.5*\boxsize,\boxsize);
\coordinate (zc4) at (-0.5*\boxsize,.5*\boxsize,\boxsize);
\coordinate (Oc1) at (0.5*\boxsize,0.5*\boxsize,0);
\coordinate (Oc2) at (0.5*\boxsize,-0.5*\boxsize,0);
\coordinate (Oc3) at (-0.5*\boxsize,-.5*\boxsize,0);
\coordinate (Oc4) at (-0.5*\boxsize,.5*\boxsize,0);
\coordinate (Bc1) at (0.5*\boxsize,0.5*\boxsize,-\boxsize);
\coordinate (Bc2) at (0.5*\boxsize,-0.5*\boxsize,-\boxsize);
\coordinate (Bc3) at (-0.5*\boxsize,-.5*\boxsize,-\boxsize);
\coordinate (Bc4) at (-0.5*\boxsize,.5*\boxsize,-\boxsize);
\coordinate (Dc1) at (0.5*\boxsize,1.5*\boxsize,0);
\coordinate (Dc2) at (0.5*\boxsize,-1.5*\boxsize,0);
\coordinate (Dc3) at (-0.5*\boxsize,-1.5*\boxsize,0);
\coordinate (Dc4) at (-0.5*\boxsize,1.5*\boxsize,0);




   \draw[thin,densely dashed] (a)--(b) (c)--(d) (a)--(c) (b)--(d) (e)--(f) (g)--(h) (e)--(g) (f)--(h) (a)--(e) (c)--(g) (b)--(f) (d)--(h) (zc1) -- (zc2) (zc2) -- (zc3) (zc3) -- (zc4) (zc4) -- (zc1) (zc1) -- (Dc1) (zc2) -- (Dc2) (zc3) -- (Dc3) (zc4) -- (Dc4) (Dc1) -- (Dc4) (Dc2) -- (Dc3) (zc1) -- (Bc1) (zc2) -- (Bc2) (zc3) -- (Bc3) (zc4) -- (Bc4) (Bc1) -- (Bc4) (Bc2) -- (Bc3) ;
   \draw[thick](zc1) -- (Oc2) (zc4) -- (Oc3) (Oc3) -- (Bc4) (Oc2) -- (Bc1);
%   \draw[->,thick] (O) -- (Ox) node[right] {x};
%   \draw[->,thick] (O) -- (Oy) node[right] {y};
%   \draw[->,thick] (O) -- (Oz) node[right] {z};
%    \fill (a) circle (0.1cm);
%    \fill (d) ++(0.1cm,0.1cm) rectangle ++(-0.2cm,-0.2cm);
}


\tikzset{
%Define standard arrow tip
>=stealth',
%Define style for different line styles
help lines/.style={dashed, thick},
axis/.style={<->},
important line/.style={thick},
connection/.style={thick, dotted},
}

\renewcommand*{\arraystretch}{0.75}
%----------------------------------------------------------------------------------------

%\section{Derivation of requirements from the science}
\subsection{	Relevant timescales (flows well because it is relevant to chapter 2)}
\subsection{	Expected perturbations}
\subsubsection{	High-altitude winds and pendulum motions}
\subsubsection{	High-frequency embedded perturbations}
\subsection{	Perturbation rejection requirements }
\subsection{	Required control performance}
\subsection{	Required knowledge performance}


%----------------------------------------------------------------------------------------

\section{Control system architecture}
\subsection{Overall strategy}
\subsection{Modes of operation}
\subsection{PID control loops}
\subsection{Actuator description and characterization}
\subsection{Sensor description and characterization}
\subsection{Control electronics}
\subsection{Software architecture}


%----------------------------------------------------------------------------------------

%\input{Chapters/Chap3-AttitudeRepresentation}

%----------------------------------------------------------------------------------------

\section{3D attitude estimation and sensor fusion}
\label{sec:KalmanFilter}
\renewcommand*{\arraystretch}{0.75}


The attitude estimation consists of combining high-frequency angular velocity measurements of the payload with low-frequency attitude measurements. The high-frequency measurements, usually from gyroscopes, are relative measurements, and exhibit biases. The attitude measurements are absolute. The Kalman filter \citep{Kalman:1960ii} combines these two types of measurements in a mathematical formalism that uses a model of the physical relationship between them. For the simplest version of this filter, the goal is to estimate the bias of the high-frequency measurements, hence providing bias-corrected, drift-less, trustworthy dynamical information that can be used to estimate the attitude at all times, even when there is no absolute measurement. In the general sense, the filter finds the state parameters that minimize the covariance of the error between a predicted quantity (in our case, the predicted attitude from integrated gyroscope velocities), and a measured quantity (in our case, an absolute attitude measurement from the star cameras). 

%In our situation, gyroscopes will provide high-frequency velocity data (typically at 100~Hz), while star cameras will provide absolute measurements every few seconds. The gyroscopes are much more trustworthy over short periods, so they won't be influenced by just a few star camera measurements. However, on long timescales, the attitude as propagated using the gyroscopes will drift with respect of the star cameras because of the gyroscopes' inherent biases. The Kalman filter will estimate those biases and provide a bias-corrected angular velocity to make sure that the gyroscopes do not drift away from the star camera results.

This filter is very common for spacecraft attitude and control, although a large number of variations exist. It was first popularized in the 60's in the United States during the Apollo missions, when it was used to determine the attitude of the Apollo capsules in inertial space. It also continues to be an active field of research today \citep[e.g.,][]{Crassidis:2011ud,Markley:2014dn}. Under certain circumstances and assumptions, the Kalman filter is the optimal filter, which means that it is the filter that has the fastest possible convergence towards the minimum steady-state error covariance.


One of the complexities of the Kalman filter is that it involves inverting matrices to find the optimal solution when new absolute measurements are received. This has implications in terms of numerical complexity which often will limit the bandwidth of the filter, especially in the context of resource-limited FPGA computers on spacecraft. In practice the trade-off is the following: either limit the bandwidth of the filter, or limit the number of state parameters (\textit{i.e.} limit the rank of the matrix to invert). On the ground, this limitation is usually not an issue. But even our powerful embedded computer will be limited in the speed at which it can find solutions.

In order to set up the Kalman filter, we choose quaternions to describe our attitude, which are discussed extensively and compared to other attitude representations in Appendix~\ref{sec:attituderepresentation}. In addition to the various advantages explained in this appendix, quaternions have a nice behavior when it comes to small angles, so we can use them in their linear, small angle approximation to create a \textit{multiplicative}, \textit{extended} Kalman filter (MEKF) \citep{Lefferts:1982dx}. It is \textit{extended} because it operates in the small angle approximation, hence it is a local approximation of a non-linear relationship. And it is \textit{multiplicative}, instead of being additive, because we use the quaternion multiplication operation to describe the "difference" or error between measured and predicted attitudes. One popular instance where this filter was successfully implemented on board the WMAP spacecraft \citep{Harman:2005ux}.

First, we need to choose a representation for our sensor suite: the gyroscopes and the star cameras. Second, we describe the equations that govern the physics of our system and connect the sensors together: this is critical for the Kalman filter to produce robust estimates, and the more accurate our representation is, the more accurate our predictions can be. Third, we discuss the Kalman setup, and two phases of the algorithm: prediction and update. And finally, we discuss potential improvements of the filter that can be used for ground-based analysis of the data.

\subsection{Sensor models}
\label{sec:SensorModels}

This section describes the chosen mathematical model that will be used to represent our sensors. These are necessarily approximations, as they do not encompass all of the possible physical effects that will be observed. The residual, non-modeled physical effects need to be small with respect to this representation in order for this filter to work optimally. In the ideal case, when all physical contributors are accounted for, and when the residual noises have a normal distribution, the Kalman filter is the optimal filter. 

\subsubsection{Gyroscope model}

For our baseline design, the gyroscope model that we use is: $\gyroVecMeas  =  \gyroVec + \bias + \nGyros$, where $\gyroVecMeas$ is the measured angular velocity vector, $\gyroVec$ is the true angular velocity vector, $\bias$ is the bias vector, and $\nGyros$ is the angular velocity noise vector (also called the "rate noise"). This implies that we have ideal alignment between each gyroscope and what we define to be the gyroscope reference frame. We consider that $\nGyros$ is a white noise process with a diagonal covariance matrix $\N_\gyro = \sigma_{c,\gyro}^2\bI_{3\times 3}$. 

We consider that the derivative of the bias $\bias$ is also a white noise process: $\dot{\bias} = \nBias$, where $\nBias$ has a diagonal covariance matrix $\N_\bias = \sigma_{c,b}^2\bI_{3\times 3}$.

Assuming that the covariance matrices are diagonal help to set up the filter, but is not a necessary assumption in the general case. The following implementation is not relying on this assumption.

%The angular random walk of the gyroscope assemblies is 0.2~deg.sec$^{-1}$ with an effective bandwidth of 50~Hz (100~Hz sampling). 
The angular random walk (ARW) that we measure is ARW~$\approx\SI{5e-4}{\deg\per\hour}$. This means that if we integrate the gyroscope's rate for 1~hour, the $1\sigma$ uncertainty on our position would be 
$\SI{5e-4}{\deg}\sim\ang{;;1.8}$. For an integration time of 1 second, it would be \ang{;;0.03}. For a single integration time step $\Deltat = \SI{0.01}{\second}$, it would be \ang{;;0.003}. 

The units required for $\sigma_\gyro$ are [\si{\radian\raiseto{-0.5}\second}], so we convert:
\begin{equations}
\sigma_\gyro\units{\si{\radian\raiseto{-0.5}\second}} = \frac{\pi}{60\times180}\times \textrm{ARW}\units{\si{\deg\raiseto{-0.5}\hour}} \sim 1.5\times 10^{-7}~\si{\radian\raiseto{-0.5}\second}.
\end{equations}

Note that we can relate the ARW to the measured discrete rate noise uncertainty $\sigma(\nGyros^\textrm{meas})$ with:
\begin{equations}
\sigma(\nGyros^\textrm{meas})\units{\si{\deg\per\second}} = \textrm{ARW}\units{\si{\deg\raiseto{-0.5}\hour}}\times 60\sqrt{\textrm{BW}\units{\si{\hertz}}},
\end{equations}
where $\textrm{BW}\units{\si{\hertz}}$ is the gyroscope's bandwidth, equal to \SI{50}{\hertz} for our system. We obtain a quantity close to the measured quantity, $\sigma(\nGyros^\textrm{meas})\sim\SI{0.2}{\arcsec\per\second}$.

The bias instability units are [\si{\radian\raiseto{-3/2}\second}]. We adopt the manufacturer's specification for a worst-case scenario bias instability over a wide range of temperatures equal to \SI{0.005}{\deg\per\hour}. This is for a bandwidth of \SI{50}{\hertz}, so we obtain the bias instability term, which also corresponds to the process noise of our Kalman filter:
\begin{equations}
\sigma_\bias\units{\si{\radian\raiseto{-3/2}\second}} = \SI{0.005}{\deg\per\hour}\times \sqrt{\textrm{BW}\units{\si{\hertz}}}  \sim \SI{1.8e-7}{\radian\raiseto{-3/2}\second}.
\end{equations}
This represents how much what we are trying to estimate is expected to vary. While this drift appears to be very slow, it increases linearly with time (as opposed to the ARW which increases as the square root of the time). Hence, the bias drift quickly increases the position uncertainty when integrating the gyroscopes, which justifies the efforts in trying to properly estimate its properties and correct for it as often as possible.

\subsubsection{Star camera model}

The star camera takes a picture of the sky to make noisy measurements of the right ascension (RA) and declination (DEC) of the boresight, as well as the roll angle (ROLL) in which the frame is taken. The RA and DEC typically are much more accurate than the roll angle. Each angle can be used as en Euler angle to define the attitude of the payload in the inertial frame (or equivalently, the rotation from the inertial frame to the current attitude). Each angle corresponds to a quaternion rotation about a single axis: 
\begin{eqnarrays}
\quat{q}_\textrm{RA} &=& [0, 0, \sin(\textrm{RA}/2),\cos(\textrm{RA}/2)]^T,\\
\quat{q}_\textrm{DEC} &=& [ 0, \sin(\textrm{DEC}/2), 0,\cos(\textrm{DEC}/2)]^T,\\
\quat{q}_\textrm{ROLL}& = &[  \sin(\textrm{ROLL}/2),0,0,\cos(\textrm{ROLL}/2)]^T,\\
\Attitude^{\textrm{meas}}_{\starcam}& = &\quat{q}_\textrm{ROLL}\quat{q}_\textrm{DEC}\quat{q}_\textrm{RA}.
\end{eqnarrays}

The errors associated with the three Euler angles are assumed to be a random vector $\nStarcam$, also with a diagonal covariance matrix $\measCovMat$. Typical star camera noises are 1-\ang{;;2} in RA and DEC and \ang{;;100} r.m.s. in ROLL (see Chapter~\ref{chap:implementation}). 

The star camera is oriented at a fixed position on the payload, which is not necessarily aligned with the gyroscope reference frame. In that case, the attitude quaternion needs to be rotated by the quaternion representing the transformation between both reference frames. In addition, the covariance matrix needs to be rotated by the direction cosine matrix corresponding to the same transformation. This would not have an effect if the covariance matrix was a multiple of the identity matrix, but it usually is not the case since the Roll measurement is often much less sensitive. This can have implications while designing the balloon payload and deciding on the placement and orientation of the star camera: the attitude estimation will be less precise about the Roll axis of the star camera.

%The fundamental problem is trying to determine the attitude of the payload in the inertial reference frame, using biased gyroscope measurements at high frequency and absolute star camera measurements at low frequency. 

% \subsection{[PUT THIS IN APPENDIX?] Tools required for the Kalman Filter}
% \subsubsection{Quaternion propagation}
% \subsubsection{Quaternion operations}



\subsection{Continuous state equation and error}

We want to use the Kalman filter to obtain an estimate of the attitude quaternion $\Attitude_k \equiv \fromto{I}{G}\Attitude(t)$, but also use it to estimate the gyroscope biases $\bias(t)$ to improve our attitude predictions and lower the errors between predicted and measured. The "state" of our system is described by the vector:
\begin{equations}
\stateVec(t) = \begin{bmatrix} \fromto{I}{G}\Attitude(t) \\ \bias(t) \end{bmatrix}.
\end{equations}

The evolution of the state is governed by the two differential equations that follow:
\begin{eqnarrays}
\fromto{I}{G}\dotAttitude(t) & = &\frac{1}{2}\matOmega(\gyroVec(t))\fromto{I}{G}\Attitude(t),\\
\dot{\bias}(t) & = & \nBias(t),
\end{eqnarrays}
with $\gyroVec = \gyroVecMeas - \bias - \nGyros$. These equations represent the exact relationship between our quantities of interest, assuming that the noise values are known. In practice, we will create an \textit{estimator} that is used to evaluate the expected value of these quantities. This estimator, $\EstStateVec = \left[\EstAttitude(t) , \EstBias(t)\right]^T$, is governed by the following equations:
\begin{eqnarrays}
\fromto{I}{G}\dotEstAttitude(t) & = &\frac{1}{2}\matOmega(\EstGyroVec(t))\fromto{I}{G}\EstAttitude(t),\\
\dot{\EstBias}(t) & = & \boldsymbol{0},
\end{eqnarrays}



%\subsection{Error state representation}
%The state $\stateVec$ has 7 components, but it is numerically more stable to reduce it to 6 by using the unity constraint within the attitude quaternion. It is also possible to linearize the system using the error representation of the state. In this representation, our model is the difference between the true and estimated state. Since we use a quaternion representation for the attitude, it is convenient to express this difference in a multiplicative form:

The Kalman filter's goal is to minimize the variance of the estimator's error - that is, the covariance of the error vector $\stateVec-\EstStateVec$. However, in our case, we have constraints in the system since we force the quaternion to be of unit length: this introduces a singularity in the covariance matrix  of the error vector, and is prone to numerical complications. It is possible to circumvent this problem by using the multiplicative properties of the quaternion used in the small angle approximation. This is called a "multiplicative" Kalman filter, as opposed to a more traditional "additive" filter.

To do this, instead of following the evolution of the state $\EstStateVec$ itself, we will follow the evolution of the error vector $\ErrorState = [\deltaTheta,\deltaBias]^T$, where $\deltaTheta$ corresponds to the 3-dimensional angular error between true and estimated attitude quaternion taken from the difference quaternion $\fromto{\hat{G}}{G}\delta\Attitude = \fromto{I}{G}\Attitude\otimes\fromto{I}{\hat{G}}\EstAttitude^{-1} \approx [1,\frac{1}{2}\deltaTheta]^T$, and $\deltaBias = \bias - \EstBias$. 

The evolution of $\ErrorState$ as a function of time can be obtained by taking the quaternion derivative of the true attitude quaternion $\dotAttitude = \dot{\delta\Attitude} \otimes \EstAttitude + \delta\Attitude \otimes \dotEstAttitude$. With our gyroscope model, we can write:
\begin{eqnarrays}
\gyroVecMeas & = & \gyroVec + \bias + \nGyros,\\
 \EstGyroVec & = & \gyroVecMeas - \EstBias, \\
\textrm{so} \quad \gyroVec^\textrm{true} & = & \EstGyroVec - \nGyros - \deltaBias.
\end{eqnarrays}


% Traditionally, the Kalman filter uses a state, $\stateVec$, and tries to minim
% In order to define the Kalman filter, we need to understand the errorThe state $\stateVec$ has 7 components
% \begin{equations}
% \fromto{I}{G}\Attitude = \fromto{\hat{G}}{G}\delta\Attitude \otimes\fromto{I}{\hat{G}}\EstAttitude.
% \end{equations}
% For small angle approximations, we can write $\fromto{\hat{G}}{G}\delta\Attitude \approx [1,\frac{1}{2}\deltaTheta]^T$. $\deltaTheta_k$ represents the vector of angles that represent the rotation between the estimated and true reference frame in the angle-axis representation, and is expressed in the gyroscope reference frame.
% Similarly, we define $\deltaBias = \bias - \EstBias$, and form the 6-dimension error state representation $\ErrorState = [\deltaTheta,\deltaBias]^T$. It is now important to establish which equations govern this error state in order to understand the expected evolution of the system.

After a lengthy derivation to express $\dot{\deltaTheta}$ from $\dot{\delta\Attitude} = [0, \frac{1}{2}\dot{\deltaTheta}]^T$ \citep{Trawny:2005va}, we obtain:
\begin{equations}
\dot{\deltaTheta} = -\EstGyroVec\times\deltaTheta - \deltaBias - \nGyros.
\end{equations}
Note that the cross-product $\EstGyroVec\times\deltaTheta$ is equal to the matrix multiplication $\omegaCross\deltaTheta$, where $\omegaCross$ is the skew-symmetric matrix made out of the elements of $\EstGyroVec$.

The bias equation is: 
\begin{equations}
\dot{\deltaBias} = \dot{\bias} - \dot{\EstBias} = \nBias.
\end{equations}
We can now write the linearized equations representing the evolution of the error state $\ErrorState$:
\begin{equations}
\dot{\ErrorState} = \begin{bmatrix} \dot{\deltaTheta} \\ \dot{\deltaBias} \end{bmatrix} = \Fc \begin{bmatrix} \deltaTheta \\ \deltaBias\end{bmatrix} + \Gc \begin{bmatrix} \nGyros \\ \nBias \end{bmatrix},
\end{equations}
with
\begin{equations}
\Fc = \begin{bmatrix} \omegaCross & -\bI_{3\times 3} \\ \bzero_{3\times 3} & \bzero_{3\times 3} \end{bmatrix},
\end{equations}
and:
\begin{equations}
\Gc = \begin{bmatrix} -\bI_{3\times 3} & \bzero_{3\times 3} \\ \bzero_{3\times 3} & \bI_{3\times 3} \end{bmatrix}.
\end{equations}

It is important here to introduce the expression of the propagation error covariance matrix of this continuous representation. Writing the noise vector $\vectors{n} = \begin{bmatrix} \nGyros \\ \nBias \end{bmatrix}$, the covariance matrix is the expected value of the product of two noise vectors taken at different times \citep{Trawny:2005va}, but since we suppose that the noise samples are independent, the covariance is not a function of this time difference $\tau$:
\begin{equations}
\noiseCovMat_c = E[\vectors{n}(t+\tau)\vectors{n}^T(t)] = 
\begin{bmatrix} \sigma_{c,\gyro}^2\bI_{3\times 3} & \bzero_{3\times 3}\\ \bzero_{3\times 3} & \sigma_{c,b}^2\bI_{3\times 3}\end{bmatrix}.
\end{equations}

\subsection{Integration of continuous equations}

Since our system has a fast sampling rate compared to the characteristic times of the system, we can consider that $\Fc$ is constant over a time step in order to express the state evolution in a discrete sense, which is appropriate for a computer implementation. We can integrate the state equation between $t_{k-1}$ and $t_k=t_{k-1}+\Deltat$, which leads to a discrete state transition matrix $\StateTransitionMat_k$:
\begin{equations}
\StateTransitionMat_k = \StateTransitionMat(t_k, t_{k-1}) = \exp\left(\Fc\Delta t\right) \equiv \begin{bmatrix} \boldsymbol{\Theta_k} & \boldsymbol{\Psi_k} \\ \bzero_{3\times 3} & \bI_{3\times 3}\end{bmatrix},
\end{equations}
with $\boldsymbol{\Theta_k} \sim \bI_{3\times 3} - \Deltat\omegaCross + \frac{\Deltat^2}{2}\omegaCross^2$ and $\boldsymbol{\Psi_k} \sim \bI_{3\times 3}\Deltat +  \frac{\Deltat^2}{2}\omegaCross - \frac{\Deltat^3}{6}\omegaCross^2$. The exponential function in this equation refers to the matrix exponential.

These expressions are now what we need to establish a discrete version of the state equations, which are based on this transition matrix $\StateTransitionMat_k$. 

\subsection{Discrete covariance matrices}
Since we have a discrete system, it is also necessary to also represent the propagation error covariance matrix discretely. The discrete propagation covariance matrix $\noiseCovMat$ sampled between time $t_k$ and $t_{k+1} =t_{k}+\Deltat$ is related to the continuous matrix $\noiseCovMat_c$ through the relationship \citep{Maybeck:1982vh}:
$$
\displaystyle\noiseCovMat = \int^{t_{k+1}}_{t_k} \StateTransitionMat(t_{k+1},\tau)\Gc(\tau)\noiseCovMat_c\Gc^T(\tau)\StateTransitionMat^T(t_{k+1},\tau)d\tau.
$$
The full result of this integration is given in \citep{Trawny:2005va}. To the second order in $\Deltat$, the equations simplify when $\gyroVec\to \boldsymbol{0}$ into:
\begin{eqnarrays}
\noiseCovMat_{11} &=& \sigma_\gyro^2\Deltat\cdot\bI_{3\times 3},\\
\noiseCovMat_{12} &=& -\sigma_\bias^2\frac{\Deltat^2}{2}\cdot\bI_{3\times 3},\\
\noiseCovMat_{22} &=& \sigma_\bias^2\Deltat\cdot\bI_{3\times 3},
\end{eqnarrays}
with 
\begin{equations}
\noiseCovMat = \begin{bmatrix} \noiseCovMat_{11} & \noiseCovMat_{12} \\ \noiseCovMat_{12}^T & \noiseCovMat_{22}\end{bmatrix}.
\end{equations}
\subsection{Discrete Kalman filter setup}
%A truth model is a description of how the true state evolves physically. While the state representation can be given in continuous terms, here we immediately use a discrete approach. We have:
Now that we obtained all discrete representation of our system, we can write the algorithm's steps. The Kalman filter will estimate the current attitude quaternion and gyroscope bias value, while minimizing the covariance of the error $\ErrorState$. Below, we summarize the relevant physical equations that are used to set up this filter. This is useful if one wants to build a physical model of the dynamic system.
\begin{enumerate}
\item \textbf{Velocity estimate}: $\EstGyroVec_{k} = \gyroVec_k^{\textrm{meas}} - \EstBias_{k}$,
\item \textbf{Attitude propagation}: $\EstAttitude_{k} = \exp\left(\frac{1}{2}\matOmega(\EstGyroVec_{k})\Deltat\right)\EstAttitude_{k-1},$
\item \textbf{Error state evolution}: $\ErrorState_{k}  = \StateTransitionMat_{k}\ErrorState_{k-1} + \Gc_{k}\vectors{n}_{k}$,
\item \textbf{Error covariance to be minimized}: $\stateCovMat_{k} = \cov{\ErrorState_{k}}$,
\item \textbf{Error covariance evolution}: $\stateCovMat_{k}  =   \StateTransitionMat_k \stateCovMat_{k-1}\StateTransitionMat^T_k + \noiseCovMat_k$,
\item \textbf{New attitude measurement}: $\Attitude^{\textrm{meas}}_k$, 
\item \textbf{State error measurement}: $\zMeasurement_k = \measErrMat_k\ErrorState_k + \nMeas_k$. 
\end{enumerate}

Note that in that last step, the error measurement $\zMeasurement_k$ is determined by extracting $\deltaTheta_k^{\textrm{meas}}$ from the difference quaternion $\delta\Attitude_k = \Attitude^{\textrm{meas}}_k \otimes \EstAttitude^{-1}_{k}$ using the small angle approximation. Furthermore, we have $\vectors{n}_k = \begin{bmatrix} \nGyros & \nBias \end{bmatrix}^T$, $\nMeas_k$ is the measurement noise, and in our case $\measErrMat_k = \begin{bmatrix} \bI_{3\times 3} & \bzero_{3\times 3} \end{bmatrix}$.

At each step, we will attempt to produce our best estimate of the state $\EstStateVec$, and keep track of the evolution of the state error $\ErrorState$ and its covariance matrix $\stateCovMat$. There are two distinct phases in the Kalman filter: the prediction, and the update.

In the prediction phase, we use our best estimates from the previous step, along with the velocity measurements and the expected propagation relationships to predict what the estimates should be at the current step. If we don't get a new attitude measurement at that step, then these new estimates are the best we can do.

When we do get a new attitude measurement, then in addition to the prediction phase, we also do an update phase. We compare the best estimate from the prediction phase to our new measurement, and use the difference to compute a correction to our state. This uses the weights of the various noise contributors in the system, as well as additional weights that can be defined by the user. This phase most importantly estimates the bias of the gyroscopes, to allow robust propagation of the state from one step to the next.

In this section, however, we assume that the attitude measured by the star camera $\Attitude^{\textrm{meas}}_k$ corresponds to the attitude at the current step. In reality, when we receive the star camera, it represents an attitude that was taken some number of steps ago. This is due to the slow processing of the star camera images and the catalog search. Our software cannot solve the star camera position in one single loop iteration. We tackle this issue in Section~\ref{subsec:delayed}. 

\subsection{Kalman filter: prediction}

% The Kalman filter propagation equations for the state error can be written:
% \begin{eqnarrays}
% \ErrorState_{k|k-N} & = & \StateTransitionMat_{k}\ErrorState_{k-1|k-N} \\
% \stateCovMat_{k|k-N} & =  & \StateTransitionMat_k \stateCovMat_{k-1|k-N}\StateTransitionMat^T_k + \noiseCovMat_k
% \end{eqnarrays}

The notation $\ErrorState_{k|k-N}$ corresponds to the estimate made at step $k$ knowing the value at step $k-N$, where $k-N$ corresponds to the step at which we received the last absolute attitude measurement. 
% In our implementation, we receive a new gyroscope measurement $\gyroVec_k^{\textrm{meas}}$, and we suppose that we already have an estimate of the error state $\EstErrorState_{k-1|k-1}$ (through the estimate of the attitude $\EstAttitude_{k-1|k-1}$ and the bias $\EstBias_{k-1|k-1}$), and the state covariance matrix $\stateCovMat_{k-1|k-1}$. The propagation steps of the Kalman filter are aimed to form our best estimate of the state at step $k$, knowing the state at step $k-1$:

The algorithmic steps for this phase are:
\begin{enumerate}
\item \textbf{Predict the bias}: $\EstBias_{k|k-N} = \EstBias_{k-1|k-N}$ since there is no new information to allow us to update the bias.
\item \textbf{Estimate the angular velocity}: $\EstGyroVec_{k|k-N} = \gyroVec_k^{\textrm{meas}} - \EstBias_{k|k-N}$.
\item \textbf{Predict the attitude}: $\EstAttitude_{k|k-N} = \exp\left(\frac{1}{2}\matOmega(\EstGyroVec_{k|k-N})\Deltat\right)\EstAttitude_{k-1|k-N}.$
\item \textbf{Compute the state transition matrix}: $\StateTransitionMat_k = \begin{bmatrix} \boldsymbol{\Theta}_k & \boldsymbol{\Psi}_k \\ \bzero_{3\times 3} & \bI_{3\times 3}\end{bmatrix}$ using $\EstGyroVec_{k|k-N}$ in the expressions of $\boldsymbol{\Theta}_k$ and $\boldsymbol{\Psi}_k$.
\item \textbf{Compute the added noise covariance matrix}: $\noiseCovMat_k$. This corresponds to the noise that is added by the new gyro measurement.
\item \textbf{Update the state covariance matrix}: $\stateCovMat_{k|k-N}  =   \StateTransitionMat_k \stateCovMat_{k-1|k-N}\StateTransitionMat^T_k + \noiseCovMat_k$
\end{enumerate}
We have now propagated our system from step $k-1$ to step $k$, and we have three new quantities: the bias $\EstBias_{k|k-N}$, the attitude estimate $\EstAttitude_{k|k-N}$, and the state covariance matrix $\stateCovMat_{k|k-N} $. If we do not get any star camera measurement, then at the next step we will just continue propagating with this procedure.

\subsection{Kalman filter: update}
\label{subsec:EKFUpdate}

The star camera information provides us with a measurement of the attitude $\Attitude^{\textrm{meas}}_{k}$, which is compared to our predicted attitude. We use the difference between our prediction and the measurement to update the bias and the state covariance matrix. Under certain circumstances, the Kalman filter is the optimal estimator: it converges towards the correct solution with the minimum amount of iterations.

For the Kalman filter update procedure, we form a measurement vector $\zMeasurement_{k}$ that corresponds to the difference of an attitude measurement at step $k$ and the predicted attitude at step $k$.

\begin{enumerate}
\setcounter{enumi}{6}
\item \textbf{Compute the innovation}: $\zMeasurement_{k} =  \deltaTheta^\textrm{meas}_{k}$ with $\deltaTheta^\textrm{meas}_{k}$ extracted from the difference quaternion $\delta\Attitude_{k} = \Attitude^{\textrm{meas}}_{k} \otimes \EstAttitude^{-1}_{k|k-N}$. 
\item \textbf{Compute the innovation covariance}: $\measErrCovMat_{k} = \measErrMat_{k}\stateCovMat_{k|k-N}\measErrMat^T_{k} + \measCovMat_{k}$.
\item \textbf{Compute the Kalman gain}: $\KalmanGain_{k} = \stateCovMat_{k|k-N}\measErrMat^T_{k}\measErrCovMat^{-1}_{k}$.
\item \textbf{Update error state}: $\ErrorState_{k|k} = \KalmanGain_{k} \zMeasurement_{k} = \begin{bmatrix} \deltaTheta \\ \deltaBias\end{bmatrix} = \begin{bmatrix} 2\DeltaQuatVec \\ \deltaBias\end{bmatrix}$
\item \textbf{Update attitude estimate}: $\EstAttitude_{k|k} =\delta\Attitude\otimes \EstAttitude_{k|k-N}$ with $\delta\Attitude = \begin{bmatrix} \sqrt{1-\DeltaQuatVec^T\DeltaQuatVec} \\ \DeltaQuatVec\end{bmatrix}$ if $\DeltaQuatVec^T\DeltaQuatVec \leqslant 1$, or $\delta\Attitude = \frac{1}{\sqrt{1+\DeltaQuatVec^T\DeltaQuatVec}}\begin{bmatrix} 1 \\ \DeltaQuatVec \end{bmatrix}$ otherwise.
\item \textbf{Update the bias}: $\EstBias_{k|k} = \EstBias_{k|k-N} + \deltaBias$.
\item \textbf{Update the angular velocity estimate}: $\EstGyroVec_{k|k} = \gyroVec_k^{\textrm{meas}} - \EstBias_{k|k}$
\item \textbf{Update state covariance matrix with Joseph's form}: $\stateCovMat_{k|k} = (\bI_{6\times 6} - \KalmanGain_{k}\measErrMat_{k})\stateCovMat_{k|k-N}(\bI_{6\times 6} - \KalmanGain_{k}\measErrMat_{k})^T + \KalmanGain_{k}\measCovMat_{k}\KalmanGain_{k}^T.$
\end{enumerate}

\subsection{Delayed star camera solution}
\label{subsec:delayed}

In general, the star camera takes much longer than one single loop cycle to produce an attitude estimate. Between the time we trigger the star camera frame and the time we receive the attitude measurement, we need to keep track of the propagation matrices that will allow to express both the attitude and its covariance matrix in the current reference frame, where the measurement can be combined with the a priori estimate from the Kalman filter.

While no new star camera measurement is available, the attitude transition is expressed by $\EstAttitude_{k} = \exp\left(\frac{1}{2}\matOmega(\EstGyroVec_{k})\Deltat\right)\EstAttitude_{k-1},$ and the new covariance is $\stateCovMat'_{k}  =  \StateTransitionMat_k \stateCovMat_{k-1}\StateTransitionMat^T_k + \noiseCovMat_k$, where we assume that $\noiseCovMat_k$ is a constant. We can consider that the gyroscope bias does not change significantly during the time between two star camera measurement (typically on the order of a few seconds). With this we can create a recursive relationship and $\Attitude_{k} = \left[\boldsymbol\Pi_{i=k-N}^k\exp\left(\frac{1}{2}\matOmega(\EstGyroVec_{i})\Deltat\right)\right]\Attitude^{\textrm{meas}}_{k-N}$ where $k-N$ again represents the index at which the star camera image was taken. Similarly, we have: $\stateCovMat_{k} = \A_k\stateCovMat_{k-N}\A_k^T + \B_k$ where $\A_k$ and $\B_k$ are defined recursively as $\A_k = \StateTransitionMat_{k}\A_{k-1}$ with $\A_0 = \bI_{6\times 6}$, and $\B_k = \noiseCovMat_k + \StateTransitionMat_{k}\B_{k-1}\StateTransitionMat_{k}^T$ with $\B_0 = \boldsymbol{0}_{6\times 6}$. $\A_k$ can also be written  $\A_k = \StateTransitionMat_{k}\StateTransitionMat_{k-1}\cdots\StateTransitionMat_{k-N} = \left[\boldsymbol\Pi_{i=k-N}^k\StateTransitionMat_{i}\right]$.

Hence, once we trigger the star camera, we need to start keeping track of the matrices $\A_k$, $\B_k$, and $\C_k = \boldsymbol\Pi_{i=k-N}^k\exp\left(\frac{1}{2}\matOmega(\EstGyroVec_{i})\Deltat\right)$, appropriately reset them when a new star camera trigger has occurred, and propagate them until the estimator receives the star camera value.

\subsection{Enhancing the Kalman filter models}
\label{subsec:enhancedKalman}

The simple gyroscope model that we adopt is incomplete, and can cause some issues that need explanation. In our simplified representation, gyroscope models using only a bias to account for the measurement errors. The bias, which combines linearly with the measured velocity, is adjusted by the Kalman filter to correct the errors and minimize the covariance of the error.

However, this supposes that the gyroscopes are perfectly orthogonal, with unity scale factor, and the transformation between the absolute measurement sensor (the star camera) reference frame and the gyro reference frame is known perfectly. An alignment error in either of these two components will translate to multiplicative errors on the velocities, which will have a large effect when the velocity dramatically changes (for example, after a slew) and will not be accounted for by a simple bias model. Eventually, the bias would adjust to be in agreement with the star camera measurements - but it can take a while, and during this time, the velocity that we think we are moving at is incorrect. To put this in perspective, a 1\% error on the gyroscope velocity in one axis for a \SI{10}{\degree} slew at \ang{;;400}\si{\per\second} corresponds to a position error of 6 arcminutes, a considerable amount given our pointing requirements.

For spacecraft projects, alignment issues and calibrations are allocated a large amount of resources to minimize these issues and come close to the ideal configuration. Our project has not dedicated enough resources to ensure exquisite alignment and calibration between the gyroscopes and the star camera, due to lack of time and resources. We nevertheles propose elements of solution in the next section.

If this error persists during flight, the poor man's solution is as follows. Instead of tracking the Kalman filter during the entire duration of the slew, we discard the star camera measurements during the slew and reset the estimator after the slew is complete. This resets our starting position with the first solution from the star camera. Since we will be off our target, we will slew again to the desired target, which will be much closer. Each time this needs to be repeated, we minimize the effects of the alignment errors.

For our scientific purpose, even a 1\% error in the gyroscope scale factor or angular velocity alignment is not a deal breaker, since their main purpose is to maintain sufficient stability to lock onto a guide star with the fine guiding sensor. The fine guiding sensor is by definition in the correct reference frame, since it observes through the optical train. 

\subsubsection{Estimating angular error between reference frames}

Here, we propose an appropriate approach to estimate the gyroscope misalignment using a different Kalman filter. In this filter, the global misalignment error of the entire reference frame is set as part of the state, and is being estimated at each step. A global misalignment error can be represented by a rotation matrix which, in the small angle approximation, can be written $\boldsymbol{C'} \approx \bI_{3\times 3} + \boldsymbol{C} $, with:
\begin{equations}
\boldsymbol{C} = \begin{bmatrix} 0 & c_{xy} & c_{xz} \\   -c_{xy} & 0 &c_{yz} \\  -c_{xz} & -c_{yz} & 0 \end{bmatrix}.
\end{equations}
We now have $\gyroVec^\textrm{true} = (\bI_{3\times 3} - \textbf{C})\gyroVec^\textrm{meas}$, so the gyro error introduced by the misalignment is $\Delta\gyroVec=-\textbf{C}^T\gyroVec^\textrm{meas}$. The new state components are $\textbf{c} = \begin{bmatrix} c_{xy}& c_{xz} & c_{yz}\end{bmatrix} ^T$, and we can rearrange the matrix terms to express $\boldsymbol{\Omega}_c$ as a function of the components of $\gyroVec^\textrm{meas}$ and write: $\Delta\gyroVec =\boldsymbol{\Omega}_c \textbf{c}$. Similarly to the error-representation equations in the Kalman filter model expressed in the previous sections, we then obtain the new upper right block of the transition matrix:
\begin{equation}
\boldsymbol{\Psi_k} = -\boldsymbol{\Omega}_c\Delta t.
\end{equation}

This is useful because the three additional state elements can replace the gyroscope bias for initial calibration and determination. When on the ground, it is then possible, with minimum software changes, to estimate the three components of a rotation matrix instead of the three components of a bias vector. 

\subsubsection{Estimating the orthogonalization error and scale factor error}

The full orthogonalization matrix for the three gyroscopes is a non-orthogonal matrix $\textbf{M}$:
\begin{equations}
\boldsymbol{M} = \begin{bmatrix} k_x & m_{xy} & m_{xz} \\   m_{yx} & k_y &m_{yz} \\  m_{zx} & m_{zy} & k_z \end{bmatrix},
\end{equations}
where $\textbf{k} = \begin{bmatrix}k_{x} &k_{y} &k_{z}\end{bmatrix}^T$ is the scale factor of the gyroscopes, and the cross terms correspond to the misalignments between the different axes. This can also be rearranged and rewritten in terms of the three scale factor unknowns and the 6 cross terms unknowns $\textbf{m} = \begin{bmatrix}m_{xy} &m_{xz} &m_{yx} &m_{yz} &m_{zx} &m_{zy}\end{bmatrix}^T$:
\begin{equations}
\Delta\gyroVec_k = \boldsymbol{\Omega}_k\textbf{k},
\Delta\gyroVec_m = \boldsymbol{\Omega}_m\textbf{m},
\end{equations}
for a total error in velocity $\Delta\gyroVec = \Delta\gyroVec_k + \Delta\gyroVec_m + \Delta\gyroVec_\bias$ if we also include the bias that we discussed in our standard estimator. We have here:
\begin{equations}
\boldsymbol{\Omega}_k = \begin{bmatrix} \omega_x & 0 & 0 \\   0 &  \omega_y  &0 \\  0 & 0 & \omega_z \end{bmatrix}
\end{equations}
and
\begin{equations}
\boldsymbol{\Omega}_m = \begin{bmatrix} \omega_y & \omega_z & 0 & 0  & 0 & 0 \\    0 & 0 & \omega_x &\omega_z  & 0 & 0  \\  \ 0 & 0 & 0  & 0 & \omega_x &\omega_y  \end{bmatrix}.
\end{equations}
This is now a 15-state Kalman filter, with the error state:
$\ErrorState = \begin{bmatrix}\deltaTheta &\Delta\textbf{k}  &\Delta\textbf{m} &\Delta\bias\end{bmatrix}^T$. The top right block of the transition matrix can be written:
\begin{equation}
\boldsymbol{\Psi_k} = -\Delta t\begin{bmatrix}\boldsymbol{\Omega}_k&\boldsymbol{\Omega}_m & \bI_{3\times 3}\end{bmatrix}.
\end{equation}

This is handy for data analysis on the ground, but not appropriate for flight since increasing the state vector size quickly increases the computational cost of the filter. While running this Kalman filter implementation in the Real Time OS on \boop, we measured average run times of $\sim\SI{0.4}{\second}$, largely caused by the $15\times 15$ matrix inversion process that happens during the update phase of the filter.

%It is nonetheless possible to use a more refined model of the gyroscope velocity measurement. There are three effects that can be included in a linear model: the scale factor error, the orthogonalization error, and the alignment error. The orthogonalization error comes from the fact that the individual gyroscopes do not form an orthogonal basis. The alignment error is an error of the orthogonal gyro reference frame from its expected position. We can write:
%
%\begin{equations}
%\gyroVec^\textrm{meas} = \boldsymbol{C'}\boldsymbol{M'}\gyroVec^\textrm{true},
%\end{equations}
%where $\boldsymbol{M'}$ is is the orthogonalization matrix and $\boldsymbol{C'}$ is the rotation matrix in the small approximation:
%\begin{equations}
%\boldsymbol{M} = \begin{bmatrix} M_x & m_{xy} & m_{xz} \\   m_{yx} & M_y &m_{yz} \\  m_{zx} & m_{zy} & M_z \end{bmatrix}, \boldsymbol{C'} = \begin{bmatrix} 1 & c_{xy} & c_{xz} \\   -c_{xy} & 1 &c_{yz} \\  -c_{xz} & -c_{yz} & 1 \end{bmatrix}.
%\end{equations}
%
%In order to convert this to a linear model, we use the small angle approximation $\boldsymbol{C'} = bI_{3\times 3} + \boldsymbol{C} $ and develop the expression ignoring second order terms:

\subsection{Conclusions on sensor fusion}

We have defined, designed, implemented and tested a complete sensor fusion algorithm based on an multiplicative, extended Kalman filter, which has several steps summarized in Fig.~\ref{fig:kalmanFilterSteps}.

\begin{figure}[!h]
	\centering
	\includestandalone{Figures/KalmanFilter}
\caption{Kalman filter steps}
\label{fig:kalmanFilterSteps}
\end{figure}

This software, implemented entirely in Labview Real Time OS, has been the workhorse of our testing of the control system. Similarly, we have implemented several variations of the software for use on the ground, in order to estimate the residual misalignments between the individual gyroscopes, as well as between the gyroscope frame and the star cameras.

The software merges the information gathered from the gyroscopes and the star cameras, while appropriately correcting for the lag in the star camera measurements, and accounting for user-defined weights. 

While this software is deeply integrated with our hardware and flight software architecture, its critical components are quite independent. We plan on sharing this software with an open-source license after the pointing test results are published. It is quite versatile and allows for many user improvement and modifications.

%----------------------------------------------------------------------------------------

\section{Practical implementation and test results}


\subsection{Testing setups and limitations}
\subsubsection{Talk about the way we test in the high bay, etc}
\subsubsection{List of test setups: gyro only, gyro+star camera, gyro+star camera+tip/tilts with CCD cameras, gyro+star camera with H1RG;}
\subsubsection{Explain the communication/data recording approach}
\subsection{Implementation}

\documentclass{standalone}
\usepackage{tikz}
\usetikzlibrary{shapes,arrows}

\begin{document}
\tikzstyle{block} = [draw, fill=black!20, rectangle, 
    minimum height=3em, minimum width=6em,align=center]
\tikzstyle{input} = [node distance=1cm]
\tikzstyle{output} = [node distance=1cm]
\newpage
\begin{tikzpicture}[auto, >=latex',scale=0.8, every node/.style={transform shape}]
\linespread{1}


% Inputs
\node [input,name=measuredVelocity,shift={(2cm,0cm)}] {$\gyroVec^{\textrm{meas}}_k$};
\node [input,above of=measuredVelocity,name=bias] {$\EstBias_{k|k-N}$};
\node [input,above of=bias,name=attitude,node distance=1cm] {$\EstAttitude_{k-1|k-N}$};
\node [input,above of=measuredVelocity,name=covariance,node distance=3cm] {$\noiseCovMat_k$};
\node [input,below of=measuredVelocity,name=propagation matrices,node distance=4cm] {$\A_{k-1},\B_{k-1},\C_{k-1}$};

% line break
\node [input,name=line break left,above of=covariance,node distance=2cm] {};
\node [input,name=line break left2,below of=line break left,node distance=0.5cm,right] {\large \textbf{Kalman Filter: Prediction}};
\node [input,name=line break right,right of=line break left,node distance=15cm] {\large { }};
\draw[dashed] ([xshift=-1cm]line break left.north west) -- (line break right.north east);


% blocks
\node [block, right of=measuredVelocity,node distance=3cm] (estimate velocity) {Estimate \\ Velocity};
\node [block, right of=estimate velocity,node distance=4cm] (predict attitude) {Predict \\ Attitude};
\node [block, below of=predict attitude,node distance=2cm] (state transition) {State \\ transition};
\node [block, right of=state transition,node distance=4cm] (estimate covariance) {Estimate \\ covariance};
\node [block, below of=estimate covariance,node distance=2cm] (update propagation matrices) {Propagate \\ matrices};

% outputs
\node [output,right of=predict attitude,node distance=3cm,name=estattitude]{};
\node [output,right of=estimate covariance,node distance=3cm,name=estcovariance]{};
\node [output,right of=update propagation matrices,node distance=3cm,name=propmat]{};

% arrows
\draw[->] (bias.east) -| (estimate velocity.north);
\draw[->] (measuredVelocity.east) -- (estimate velocity.west);

\node[name=mid,right of=estimate velocity,node distance=2cm,draw,fill,minimum size=3pt,circle]{};
\node[above] at (mid){$\EstGyroVec_{k|k-N}$};
\draw[->] (estimate velocity.east) -- (mid) -- (predict attitude.west);
\draw[->] (attitude.east) -| (predict attitude.north);
\draw[->] (mid) |- (state transition.west) ;
\node[name=mid,right of=state transition,node distance=2cm,draw,fill,minimum size=3pt,circle]{};
\node[above] at (mid){$\StateTransitionMat_k$};
\draw[->] (state transition.east) -- (mid) -- (estimate covariance.west) ;
\node[name=mid2,below of=mid,node distance=1cm,outer sep=0cm,inner sep=0cm]{};
\draw[-] (mid) --  (mid2.north) ;
\draw[->] (mid2.north) -|  (update propagation matrices.north) ;

\node[name=mid,right of=estimate covariance,node distance=3cm]{};
\node[above] at (mid){$\stateCovMat_{k|k-N}$};
\draw[->] (estimate covariance.east) -- (estcovariance) ;
\draw[->] (covariance.east) -| (estimate covariance.north) ;

\node[name=mid,right of=predict attitude,node distance=3cm]{};
\node[above] at (mid){$\EstAttitude_{k|k-N}$};
\draw[->] (predict attitude.east) -- (mid);

\draw[->] (propagation matrices.east) -- (update propagation matrices.west);
\draw[->] (update propagation matrices.east) -- (propmat);
\node[name=mid,right of=update propagation matrices,node distance=3cm]{};
\node[above] at (mid){$\A_{k},\B_{k},\C_{k}$};


% Kalman Update
% line break
\node [input,name=line break left,below of=propagation matrices,node distance=1.5cm] {};
\node [input,name=line break left2,below of=line break left,node distance=0.5cm,right] {\large \textbf{Kalman Filter: Update}};
\node [input,name=line break right,right of=line break left,node distance=15cm] {\large { }};
\draw[dashed] ([xshift=-1cm]line break left.north west) -- (line break right.north east);

% Inputs
\node [input,name=measured attitude,below of=line break left,node distance=3cm] {$\Attitude^\textrm{meas}_{\starcam}$};
\node [input,name=final matrices,above of=measured attitude,node distance=1cm] {$\A_{k},\B_{k},\C_{k}$};
\node [input,name=covmat,below of=measured attitude,node distance=6cm] {$\stateCovMat_{k|k-N}$};
\node [input,name=meascovmat,below of=measured attitude,node distance=2cm] {$\measCovMat_{k}$};

% blocks
\node [block, right of=measured attitude,node distance=3cm] (propagate attitude) {Rotate \& \\ Propagate};
\node [block, below of=propagate attitude,node distance=2cm] (innovation) {Innovation};
\node [block, below of=innovation,node distance=2cm] (innovation covariance) {Innovation \\ covariance};
\node [block, below of=innovation covariance,node distance=2cm] (kalman gain) {Kalman \\ gain};
\node [block, right of=kalman gain,node distance=4cm] (calculate error) {Calculate \\ error};
\node [block, right of=calculate error,node distance=4cm] (update bias) {Update \\ bias};
\node [block, below of=update bias,node distance=2cm] (update attitude) {Update \\ attitude};
\node [block, above of=update bias,shift={(3cm,1cm)}] (update velocity) {Update \\ velocity};
\node [block, below of=update attitude,node distance=2cm] (update covariance) {Update \\ covariance};

% arrows
\draw[->] (measured attitude) -- (propagate attitude);
\draw[->] (final matrices) -| (propagate attitude);
\draw[->] (propagate attitude.south) -- (innovation.north);
\draw[->] (innovation covariance.south) -| (kalman gain.north);
\draw[->] (covmat.east) -- (kalman gain.west);
\draw[->] (covmat.north) |- (innovation covariance.west);
\draw[->] (meascovmat.east) -- (innovation.west);
\draw[->] (covmat.south) |- (update covariance.190);

% intermediary nodes
\node[name=mid,right of=kalman gain,node distance=2cm,draw,fill,minimum size=3pt,circle]{};
\node[above] at (mid){$\KalmanGain_{k}$};
\draw[->] (kalman gain.east) -- (mid) -- (calculate error.west);
\draw[->] (mid) |- (update covariance.165);

\node[name=mid,below of=propagate attitude,node distance=1cm]{};
\node[left] at (mid){$\Attitude^{\textrm{meas}}_{k}$};


\node[name=mid,below of=innovation,node distance=1cm,draw,fill,minimum size=3pt,circle]{};
\node[left] at (mid){$\zMeasurement_{k}$};
\draw[->] (mid) -| (calculate error.north);
\draw[->] (innovation.south) -- (mid) -- (innovation covariance.north);

\node[name=mid,below of=innovation covariance,node distance=1cm]{};
\node[left] at (mid){$\measErrCovMat_{k}$};


\node[name=mid,right of=calculate error,node distance=2cm,draw,fill,minimum size=3pt,circle]{};
\node[above] at (mid){$\ErrorState_{k|k}$};
\draw[->] (calculate error.east) -- (mid) -- (update bias.west);

\node [input,name=bias estimate,above of=mid,shift={(1cm,0cm)}] {$\EstBias_{k|k-N}$};
\draw[->] (bias estimate) -| (update bias.north);

\node [input,name=attitude estimate,below of=bias estimate,node distance=2cm] {$\EstAttitude_{k|k-N}$};
\draw[->] (attitude estimate) -| (update attitude.north);


\node [input,name=velocity estimate,above of=bias estimate,node distance=2cm] {$\EstGyroVec_{k|k-N}$};
\draw[->] (velocity estimate) -| (update velocity.north);
\draw[->] (mid) |- (update attitude.west);

\node [output,name=velocity,right of=update velocity,node distance=2cm]{}; \node [output,right of=update velocity,node distance=2cm,above] {$\EstGyroVec_{k|k}$};
\draw[->] (update velocity.east) -- (velocity);

% outputs
\node [output,name=mid,right of=update bias,node distance=3cm,draw,fill,minimum size=3pt,circle] {};
\draw[->] (update bias.east) -- (mid);
\draw[->] (mid) -- (update velocity.south);
\node [output,name=bias,right of=mid,node distance=2cm] {};
\node [output,right of=mid,node distance=2cm,above] {$\EstBias_{k|k}$};
\draw[->] (mid) -- (bias.west);
\node [output,name=attitude,right of=update attitude,node distance=2cm] {};
\node [output,right of=update attitude,node distance=2cm,above] {$\EstAttitude_{k|k}$};
\draw[->] (update attitude) -- (attitude);

\node [output,name=covariances,right of=update covariance,node distance=2cm] {};
\node [output,right of=update covariance,node distance=2cm,above] {$\stateCovMat_{k|k}$};
\draw[->] (update covariance) -- (covariances);

\end{tikzpicture}
\end{document}

\subsubsection{Star camera software}
This software was not written by us, so maybe not include this part in detail?
\subsubsection{Autofocus algorithm and performance}
\subsubsection{	Gondola attitude estimator}
\subsubsection{Telescope attitude estimator}
\subsubsection{Phase estimator [is that Arnab’s realm?]}
\subsubsection{Track mode}
\subsubsection{	Slew mode}
\subsubsection{	Acquire mode [this one is contingent on the telescopes working, and is not perfectly representative when only using one single side of the payload]}
\subsection{	Pointing tests and performance results}
\subsubsection{Gyro only}
Spectral analysis when attached on the gondola
Noise caracteristics
\subsubsection{	Gyros+star camera}
\subsubsection{	gyro+star camera+tip/tilts with CCD cameras}
\subsubsection{	gyro+star camera with H1RG;}
\subsection{	Using the test results to estimate the flight performance (have to think more about that section)}
\subsubsection{	Perturbation rejection estimates}
\subsubsection{	Pointing knowledge predictions}
\subsubsection{	Pointing control predictions}


% Chapter 4

\chapter[Implementation and on-sky testing]{\setstretch{1}Implementation and on-sky testing} % Main chapter title


%\epigraph{\setstretch{1}\small\itshape Ever tried. Ever failed. No matter. \\ Try again. Fail again. Fail better.}{S. Beckett, \textit{Worstward Ho!}}

\section{Key pre-flight procedures}
\subsection{Inertia measurement}
While CAD models allowed to us to estimate the moment of inertia of the payload, this is only an approximation. For testing and for launch, the payload will be different than the model we have: we will either miss some components because they are not yet installed, or have additional components such as the ballasts, the crush pads, or the weights that are used to balance the payload.

We use a simple procedure to estimate the moment of inertia about $\gyroVec{z}_\gyro$ of the payload while hanging from a crane. For this purpose, we command the CCMG to input a torque to the payload by moving the gimbal at a constant velocity. According to Eq.~\ref{eq:CCMGTorque}, $\ccmgtorque =  20.8\times \dot{\theta}\cos\theta$. According to conservation of angular momentum, the rate of change of the total angular momentum about $\vectors{z}_\gyro$ is $\inertia_\vectors{z}\dot{\gyroVec_\vectors{z}} = \ccmgtorque = 20.8\times \dot{\theta}\cos\theta$.

We measure the inertia $\inertia_\vectors{z}$ by averaging measurements of the angular acceleration $\dot{\gyroVec_\vectors{z}}$, divided by the instantaneous input torque, which is numerically more stable than averaging its inverse since the accelerations, expressed in \si{\radian\per\second} are typically very small. A measure of the inertia is then the inverse of this average. By repeating the measurement over multiple accelerations and deceleration cycles, we can also obtain an uncertainty to this estimate.

[PUT HERE A TABLE WITH MEASURED INERTIA AND UNCERTAINTIES]

\subsection{Sensor alignment and calibration}
While the intrinsic noise of our sensors has been characterized in Section~\ref{subsec:gyros}, it is important to test them while mounted to the payload, align their axes to the other reference frames, and study their spectral energy distribution. Mounting the gyroscopes in a 3-dimensional mount on the truss will inevitably lead to alignment errors and the contribution of new vibration frequencies present in the structure and excited by the moving parts on the payload.


\subsubsection{Gyroscope spectral analysis in flight configuration}

While the payload is on the ground, without wheels;

While the payload is hanging, without wheels

While the payload is hanging, with wheels on;

\subsubsection{Orthogonalization of gyroscope mount}
\label{ap:gyroOrth}
\subsubsection{Alignment of gyroscope mount to star camera mounts}

BETTII will have two flight star cameras for redundancy. It is important to understand the transformation between the gyroscope reference frame and the two star camera reference frames to minimize the propagation errors within the Kalman filter. 

A first transformation matrix can be estimated roughly by assuming that the star camera is in the ($\vectors{x}_\gyro$,$\vectors{z}_\gyro$) plane. The elevation of the star camera, nominally around \SI{45}{\degree}, can be estimated if $\vectors{z}$ is assumed to be aligned with the gravity vector while the payload is sitting on the ground, which is a good approximation. By taking some star camera measurements, solving the fields, and converting these fields to a local azimuth and elevation (using the time at which the frames were taken and the geographical location), the elevation can be estimated, which corresponds to the angle of the line of sight vector of the star camera with respect to the horizontal plane. This gives us a first estimate of our star camera-to-gyroscope reference frame rotation, but is not very precise given our assumptions. This rotation is critical during the update phase of the Kalman filter when combining the star camera measurement with the propagated estimate.

But if this matrix is slightly off, the Kalman filter will attribute the difference between estimated position and measured position as an additive bias error in the gyroscope velocity measurement, which should converge to a steady-state value after a few tens of star camera measurements, depending on the Kalman filter gains. The filter then uses this additive bias to reconstruct an estimated velocity at each time step. Admittedly, this value of the bias also contains the value of the real gyroscope bias - but we anticipate  this bias value to be very small compared to the potential effects of misalignments.

Our procedure involves calculating the quaternion representing the rotation from the steady-state estimated angular velocity vector, and the measured, orthogonalized angular velocity vector. While sitting on the ground, this corresponds to measuring the Earth's angular velocity. This uses the mathematical technique that we derived in Appendix~\ref{ap:rotBetweenTwoVec}. Each star camera measurement shall be further rotated by this quaternion to ensure that it is properly expressed in the gyroscope reference frame. 

Below, we summarize the steps to properly align the gyroscopes to the star camera:
\begin{enumerate}
 \item Orthogonalize the gyroscope mount according to Appendix~\ref{ap:gyroOrth}, and find $M_\textrm{orth}$. This now gives a velocity vector in an orthogonal gyro reference frame, $\gyroVec_\gyro$ = $M_\textrm{orth}\gyroVec_\gyro^\textrm{meas}$. 
 \item While the truss is sitting horizontal on the ground, calculate the elevation of the star camera by converting solutions to the local East-North-Up reference frame, using the sidereal time and the geographical latitude. This gives a matrix $M_\textrm{coarse}$  or a quaternion $\quat{q}_\textrm{coarse}$ representing the transformation from the star camera reference frame to the gyroscope reference frame.
\item Still while sitting on the ground, use $M_\textrm{coarse}$ or $\quat{q}_\textrm{coarse}$ in the flight Kalman filter to estimate the gyro bias vector. Record the steady-state value of the estimated velocity vector.
 \item Calculate the transformation between the estimated and measured angular velocity vector, this is a matrix $M_\textrm{fine}$ or a quaternion $\quat{q}_\textrm{fine}$.
 \item For each star camera reference frame solution, rotate the solution $\quat{q}_\textrm{fine}\quat{q}_\textrm{coarse}$.
 \item Use this new matrix in the flight Kalman filter to make sure the estimated biases are close to zero. 
 \end{enumerate} 
\subsection{Star camera}
\subsubsection{Software tuning}

\begin{landscape}
\begin{figure}[!ht]
	\centering
	\includegraphics[width=1.5\textwidth]{Figures/starcam_images.pdf}
	\caption[Star camera example WISE]{\textit{Left}: Example of a background-subtracted star camera image with identified $>5\sigma$ sources circled in red. The orientation of the image on the celestial sphere is the one provided by BETTII's embedded star camera solver. This image corresponds to a field in the Scorpius constellation. \textit{Right}: WISE \SI{3.4}{\um} mosaic from the online archive, centered on the same location. This image is composed of 9 individual WISE images that we patched into a mosaic using the \textit{Montage}[CITE] software package.}
	\label{fig:starcamexample}
    \end{figure}
\end{landscape}
\begin{landscape}
\begin{figure}[!ht]
	\centering
	\includegraphics[width=1.5\textwidth]{Figures/starcam_SDSSr_zoom.pdf}
	\caption[Star camera individual star]{\textit{Left}: Snapshot of a bright star seen within the background-subtracted star camera frame. \textit{Right}: Snapshot taken at the same location from the WISE \SI{3.4}{\um} archive.}
	\label{fig:starcamzoom}
    \end{figure}
\end{landscape}

Discuss about tuning, catalog, filters, etc

Table with the star camera parameters

Show mean deviation of star camera image with optimized parameters;
show time, statistics, etc.

\subsubsection{Calibration}

\section{Test setups and limitations}
\subsection{Talk about the way we test in the high bay, etc}
\subsection{List of test setups: gyro only, gyro+star camera, gyro+star camera+tip/tilts with CCD cameras, gyro+star camera with H1RG;}
\subsection{Explain the communication/data recording approach}

\subsection{Autofocus algorithm and performance}

\section{Estimator implementation}
\subsection{Gyro attitude estimator}
\documentclass{standalone}
\usepackage{tikz}
\usetikzlibrary{shapes,arrows}

\begin{document}
\tikzstyle{block} = [draw, fill=black!20, rectangle, 
    minimum height=3em, minimum width=6em,align=center]
\tikzstyle{input} = [node distance=1cm]
\tikzstyle{output} = [node distance=1cm]
\newpage
\begin{tikzpicture}[auto, >=latex',scale=0.8, every node/.style={transform shape}]
\linespread{1}


% Inputs
\node [input,name=measuredVelocity,shift={(2cm,0cm)}] {$\gyroVec^{\textrm{meas}}_k$};
\node [input,above of=measuredVelocity,name=bias] {$\EstBias_{k|k-N}$};
\node [input,above of=bias,name=attitude,node distance=1cm] {$\EstAttitude_{k-1|k-N}$};
\node [input,above of=measuredVelocity,name=covariance,node distance=3cm] {$\noiseCovMat_k$};
\node [input,below of=measuredVelocity,name=propagation matrices,node distance=4cm] {$\A_{k-1},\B_{k-1},\C_{k-1}$};

% line break
\node [input,name=line break left,above of=covariance,node distance=2cm] {};
\node [input,name=line break left2,below of=line break left,node distance=0.5cm,right] {\large \textbf{Kalman Filter: Prediction}};
\node [input,name=line break right,right of=line break left,node distance=15cm] {\large { }};
\draw[dashed] ([xshift=-1cm]line break left.north west) -- (line break right.north east);


% blocks
\node [block, right of=measuredVelocity,node distance=3cm] (estimate velocity) {Estimate \\ Velocity};
\node [block, right of=estimate velocity,node distance=4cm] (predict attitude) {Predict \\ Attitude};
\node [block, below of=predict attitude,node distance=2cm] (state transition) {State \\ transition};
\node [block, right of=state transition,node distance=4cm] (estimate covariance) {Estimate \\ covariance};
\node [block, below of=estimate covariance,node distance=2cm] (update propagation matrices) {Propagate \\ matrices};

% outputs
\node [output,right of=predict attitude,node distance=3cm,name=estattitude]{};
\node [output,right of=estimate covariance,node distance=3cm,name=estcovariance]{};
\node [output,right of=update propagation matrices,node distance=3cm,name=propmat]{};

% arrows
\draw[->] (bias.east) -| (estimate velocity.north);
\draw[->] (measuredVelocity.east) -- (estimate velocity.west);

\node[name=mid,right of=estimate velocity,node distance=2cm,draw,fill,minimum size=3pt,circle]{};
\node[above] at (mid){$\EstGyroVec_{k|k-N}$};
\draw[->] (estimate velocity.east) -- (mid) -- (predict attitude.west);
\draw[->] (attitude.east) -| (predict attitude.north);
\draw[->] (mid) |- (state transition.west) ;
\node[name=mid,right of=state transition,node distance=2cm,draw,fill,minimum size=3pt,circle]{};
\node[above] at (mid){$\StateTransitionMat_k$};
\draw[->] (state transition.east) -- (mid) -- (estimate covariance.west) ;
\node[name=mid2,below of=mid,node distance=1cm,outer sep=0cm,inner sep=0cm]{};
\draw[-] (mid) --  (mid2.north) ;
\draw[->] (mid2.north) -|  (update propagation matrices.north) ;

\node[name=mid,right of=estimate covariance,node distance=3cm]{};
\node[above] at (mid){$\stateCovMat_{k|k-N}$};
\draw[->] (estimate covariance.east) -- (estcovariance) ;
\draw[->] (covariance.east) -| (estimate covariance.north) ;

\node[name=mid,right of=predict attitude,node distance=3cm]{};
\node[above] at (mid){$\EstAttitude_{k|k-N}$};
\draw[->] (predict attitude.east) -- (mid);

\draw[->] (propagation matrices.east) -- (update propagation matrices.west);
\draw[->] (update propagation matrices.east) -- (propmat);
\node[name=mid,right of=update propagation matrices,node distance=3cm]{};
\node[above] at (mid){$\A_{k},\B_{k},\C_{k}$};


% Kalman Update
% line break
\node [input,name=line break left,below of=propagation matrices,node distance=1.5cm] {};
\node [input,name=line break left2,below of=line break left,node distance=0.5cm,right] {\large \textbf{Kalman Filter: Update}};
\node [input,name=line break right,right of=line break left,node distance=15cm] {\large { }};
\draw[dashed] ([xshift=-1cm]line break left.north west) -- (line break right.north east);

% Inputs
\node [input,name=measured attitude,below of=line break left,node distance=3cm] {$\Attitude^\textrm{meas}_{\starcam}$};
\node [input,name=final matrices,above of=measured attitude,node distance=1cm] {$\A_{k},\B_{k},\C_{k}$};
\node [input,name=covmat,below of=measured attitude,node distance=6cm] {$\stateCovMat_{k|k-N}$};
\node [input,name=meascovmat,below of=measured attitude,node distance=2cm] {$\measCovMat_{k}$};

% blocks
\node [block, right of=measured attitude,node distance=3cm] (propagate attitude) {Rotate \& \\ Propagate};
\node [block, below of=propagate attitude,node distance=2cm] (innovation) {Innovation};
\node [block, below of=innovation,node distance=2cm] (innovation covariance) {Innovation \\ covariance};
\node [block, below of=innovation covariance,node distance=2cm] (kalman gain) {Kalman \\ gain};
\node [block, right of=kalman gain,node distance=4cm] (calculate error) {Calculate \\ error};
\node [block, right of=calculate error,node distance=4cm] (update bias) {Update \\ bias};
\node [block, below of=update bias,node distance=2cm] (update attitude) {Update \\ attitude};
\node [block, above of=update bias,shift={(3cm,1cm)}] (update velocity) {Update \\ velocity};
\node [block, below of=update attitude,node distance=2cm] (update covariance) {Update \\ covariance};

% arrows
\draw[->] (measured attitude) -- (propagate attitude);
\draw[->] (final matrices) -| (propagate attitude);
\draw[->] (propagate attitude.south) -- (innovation.north);
\draw[->] (innovation covariance.south) -| (kalman gain.north);
\draw[->] (covmat.east) -- (kalman gain.west);
\draw[->] (covmat.north) |- (innovation covariance.west);
\draw[->] (meascovmat.east) -- (innovation.west);
\draw[->] (covmat.south) |- (update covariance.190);

% intermediary nodes
\node[name=mid,right of=kalman gain,node distance=2cm,draw,fill,minimum size=3pt,circle]{};
\node[above] at (mid){$\KalmanGain_{k}$};
\draw[->] (kalman gain.east) -- (mid) -- (calculate error.west);
\draw[->] (mid) |- (update covariance.165);

\node[name=mid,below of=propagate attitude,node distance=1cm]{};
\node[left] at (mid){$\Attitude^{\textrm{meas}}_{k}$};


\node[name=mid,below of=innovation,node distance=1cm,draw,fill,minimum size=3pt,circle]{};
\node[left] at (mid){$\zMeasurement_{k}$};
\draw[->] (mid) -| (calculate error.north);
\draw[->] (innovation.south) -- (mid) -- (innovation covariance.north);

\node[name=mid,below of=innovation covariance,node distance=1cm]{};
\node[left] at (mid){$\measErrCovMat_{k}$};


\node[name=mid,right of=calculate error,node distance=2cm,draw,fill,minimum size=3pt,circle]{};
\node[above] at (mid){$\ErrorState_{k|k}$};
\draw[->] (calculate error.east) -- (mid) -- (update bias.west);

\node [input,name=bias estimate,above of=mid,shift={(1cm,0cm)}] {$\EstBias_{k|k-N}$};
\draw[->] (bias estimate) -| (update bias.north);

\node [input,name=attitude estimate,below of=bias estimate,node distance=2cm] {$\EstAttitude_{k|k-N}$};
\draw[->] (attitude estimate) -| (update attitude.north);


\node [input,name=velocity estimate,above of=bias estimate,node distance=2cm] {$\EstGyroVec_{k|k-N}$};
\draw[->] (velocity estimate) -| (update velocity.north);
\draw[->] (mid) |- (update attitude.west);

\node [output,name=velocity,right of=update velocity,node distance=2cm]{}; \node [output,right of=update velocity,node distance=2cm,above] {$\EstGyroVec_{k|k}$};
\draw[->] (update velocity.east) -- (velocity);

% outputs
\node [output,name=mid,right of=update bias,node distance=3cm,draw,fill,minimum size=3pt,circle] {};
\draw[->] (update bias.east) -- (mid);
\draw[->] (mid) -- (update velocity.south);
\node [output,name=bias,right of=mid,node distance=2cm] {};
\node [output,right of=mid,node distance=2cm,above] {$\EstBias_{k|k}$};
\draw[->] (mid) -- (bias.west);
\node [output,name=attitude,right of=update attitude,node distance=2cm] {};
\node [output,right of=update attitude,node distance=2cm,above] {$\EstAttitude_{k|k}$};
\draw[->] (update attitude) -- (attitude);

\node [output,name=covariances,right of=update covariance,node distance=2cm] {};
\node [output,right of=update covariance,node distance=2cm,above] {$\stateCovMat_{k|k}$};
\draw[->] (update covariance) -- (covariances);

\end{tikzpicture}
\end{document}

\subsubsection{Testing the Kalman filter software with simulated data}
\subsubsection{Test results when sitting on the ground}
\subsection{Telescope attitude estimator}
Link to cross-elevation
\subsection{Phase estimator [is that Arnab’s realm?]}

\section{Operating modes}
\subsection{Track mode}
Gain tuning
Show wheel angle for long time
\subsection{Slew mode}
\subsection{Acquire mode [this one is contingent on the telescopes working, and is not perfectly representative when only using one single side of the payload]}

\section{Pointing tests and performance results}
\subsection{Gondola pointing stability}
\subsubsection{In the high bay}
 show azimuth stability data
show telescope rolling rms
\subsubsection{With the door open}
\subsection{Kalman filter performance}
Show data with the door open with the star camera acquiring frames
\subsection{gyro+star camera+tip/tilts with CCD cameras}
\subsection{gyro+star camera with H1RG;}

\section{	Using the test results to estimate the flight performance (have to think more about that section)}
\subsection{Perturbation rejection estimates}
\subsection{Pointing knowledge predictions}
\subsection{Pointing control predictions}
 
% Chapter 5

\chapter[Conclusion]{\setstretch{1}Conclusion} % Main chapter title
\label{chap:conclusion}


\epigraph{\setstretch{1}\small\itshape There is nothing like a dream to create the future.}{V. Hugo}


Over the course of 5 years, the BETTII project went from paper drawings to its first flight campaign. I have had the opportunity to be involved in all aspects of the project, which provided me with a unique view of how to build instruments to address a specific scientific question. In addition to the day-to-day engineering challenges, a global vision of the process was acquired, which made this experience irreplaceable.

Regardless of BETTII's first flight campaign, the mechanical, electrical, and software infrastructure developed from scratch for BETTII form a powerful pointed observatory platform that can host various instruments in the future. If BETTII succeeds and is able to obtain more funding over the years, the versatility of its subsystems make them relatively straightforward to repair, enhance, or adapt to future goals.

The work and thoughts spent on BETTII during the past years, combined with the approaching Decadal Survey discussions, have converged towards a new concept for a potential Probe-class space telescope: the Space High Angular Resolution Probe for the InfraRed (SHARP-IR, pronounced "sharper"). This new concept, which are currently going through the Architecture Design Lab and soon through the Instrument Design Lab at NASA GSFC, could see the full potential of double-Fourier interferometry come to fruition, and provide transformational science in the far-infrared. The concept was unveiled at the SPIE conference in Edinburgh (Rinehart, Rizzo et al. 2016). 






%----------------------------------------------------------------------------------------
%	THESIS CONTENT - APPENDICES
%----------------------------------------------------------------------------------------

\appendix % Cue to tell LaTeX that the following "chapters" are Appendices

% Include the appendices of the thesis as separate files from the Appendices folder
% Uncomment the lines as you write the Appendices

% Appendix A

\chapter{Far-IR double-Fourier interferometers and their spectral sensitivity} % Main appendix title

\label{AppendixA} % For referencing this appendix elsewhere, use \ref{AppendixA}

\section{Deriving the Interferogram Equation in a Double Fourier System}
\label{ap:interfero}
The interferogram from a double-Fourier system is different from the interferogram
for an FTS in several ways that derive from the fact
that the double-Fourier system starts with two independent input beams viewing the same astronomical target. For this derivation,
we will follow the convention in the FTS literature and consider the propagation of a single plane wave (radiation from a point source at infinity) at wavenumber $\s\equiv{1/\lambda}$ through the system.

Figure 2 in the main text shows the setup for a typical double-Fourier system with the K-mirror on one
arm to keep the sky images at the same rotation on the two paths, and the delay line
in the other arm to allow adjustment of the relative path lengths between path 1 and 2.
The plane wave travels a distance $x_1$ on path 1 from an entrance aperture an arbitrary distance
above the siderostat to the beam combiner: $a_1(\s) e^{-2\pi i \s x_1+\phi}$,
where $a_1$ is the amplitude of the electric field and $\phi$ corresponds to an arbitrary phase offset. For convenience of notation, in the following derivation we drop the amplitudes' dependence on wavenumber by writing $a_1$ instead of $a_1(\s)$.

The wave also undergoes phase shifts caused by reflections and partial reflections along
the path. A full reflection for light traveling in air or a vacuum causes a 180~$\deg$ phase shift;
a 50\% reflection at the beam splitter/combiner causes a 90~$\deg$ phase shift between reflected and transmitted beam \citep{Lawson:2000vf}. Since the instrument
measures the combined light at the detectors, what matters is the difference in the
numbers of reflections along path 1 and 2. In the case of the particular BETTII implementation, path 1 contains one more reflection than path 2.

The electrical fields arriving at the ``+" and ``-" detectors are then:
\begin{eqnarray}
A_- &=& a_1 e^{-2\pi i \s x_1+ i \pi + i\pi /2+\phi} + a_2 e^{-2\pi i \s x_2 +\phi },\\
A_+ &=& a_1 e^{-2\pi i \s x_1 + i \pi +\phi} + a_2 e^{-2\pi i \s x_2 + i\pi /2 +\phi},
\end{eqnarray}
where the $\pi$ phase shift on path 1 occurs because there is one extra reflection compared to path 2 (see Fig.~\ref{fig:optics}), and $\phi$ corresponds to an arbitrary phase offset. The detectors are power detectors so defining the intensity $ I = A^* A$:
\begin{eqnarray}
I_- &=& a_1^2 + a_2^2 + a_1 a_2 \left(e^{-2\pi i \s (x_1 - x_2) + 3i\pi /2} + e^{2\pi i \s (x_1 - x_2) - 3i \pi/ 2}\right),\\
I_+ &=& a_1^2 + a_2^2 + a_1 a_2 \left(e^{-2\pi i \s (x_1 - x_2)  + i\pi /2} + e^{2\pi i \s (x_1 - x_2) -  i \pi/ 2}\right).
\end{eqnarray}
Defining $x \equiv x_1 - x_2$ and expanding the complex exponentials, the equations can be simplified to:
\begin{eqnarray}
I_- &=& (a_1^2 + a_2^2 ) \left( 1 - {2 a_1 a_2 \over a_1^2 + a_2^2} \sin( 2 \pi \s x) \right),\\
I_+ &=& (a_1^2 + a_2^2 )\left ( 1 + {2 a_1 a_2 \over a_1^2 + a_2^2} \sin( 2 \pi \s x) \right),
\end{eqnarray}
where $x$ is now the difference in the physical length between the two light paths.
For the case of equal wave amplitudes on path 1 and 2 ($a_1=a_2=a$):
\begin{equation}
I_\pm = 2 a^2 ( 1 \pm \sin( 2 \pi \s x) ).
\end{equation}
The generalization of this equation to a source distribution on the sky requires the recognition that $a_1$ and $a_2$
are complex values such that $|a_1|^2(\s)$ and $|a_2|^2(\s)$ are
power from the source at wavenumber $\s$, while $a_1 a_2^*$ is the correlated power seen through the two apertures which is
the source spatial visibility, $\gamma(\baseline,\s)$, and is in general a complex valued function. $\gamma(\baseline,\s)$, which is a function of
the baseline vector $\baseline$ connecting the two light collectors, and $\s$, is the Fourier transform of the
source emission distribution on the sky.
For the general case, the previous equations become:
\begin{eqnarray}
I_- & = & |a_1|^2 + |a_2|^2 + \gamma(\baseline,\s) e^{-2\pi i \s (x_1 - x_2) + 3i\pi /2} + \gamma^*(\baseline,\s) e^{2\pi i \s (x_1 - x_2) - i 3\pi/ 2},\\
I_+ & = & |a_1|^2 + |a_2|^2 + \gamma(\baseline,\s) e^{-2\pi i \s (x_1 - x_2) + i\pi /2} +  \gamma^*(\baseline,\s) e^{2\pi i \s (x_1 - x_2) -i  \pi/ 2}.
\end{eqnarray}
The same simplification as before can be done except that $\gamma(\baseline,\s)$ is a complex-valued function. If we define the normalized spatial
visibility as
\begin{equation}
\Vb(\s) = {2\gamma(\baseline,\s)\over a_1^2 + a_2^2},
\end{equation}
then the equation for $I_\pm$ becomes:
\begin{eqnarray}
I_\pm & = & ( |a_1|^2 + |a_2|^2 ) \left[ 1 \pm (\real\left(\Vb(\s)\right) \sin(2 \pi \s x) - \imag\left(\Vb(\s)\right) \cos(2 \pi \s x))\right],\\
I_\pm & = & ( |a_1|^2 + |a_2|^2 ) \left[ 1 \pm \real\left( i \Vb(\s) e^{-2 \pi i \s x}\right)\right],
\end{eqnarray}
where $\real(f)$ is the real component of $f$ and $\imag(f)$ is the imaginary component.

The same style of derivation can be done with for a realistic instrument with a complex transfer function.  If
we characterize the spectral transmission function as $t_1(\s) = |t_1(\s)| e^{i\Phi_1(\s)} $ along path 1, and
$t_2(\s) = |t_2(\s)| e^{i\Phi_2(\s)} $ on path 2, then the amplitude mismatch of the spectral transmission function in each path reduces the power in the interferogram and
the phase differences introduce a phase factor $\Phi_{i} = \Phi_1 - \Phi_2$ into the exponential term.
As a result, the source visibility in the previous equations is multiplied by a normalized, instrumental visibility loss term, $\Vi = |\Vi(\s)|e^{i\Phi_i(\s)}$:
\begin{equation}
I_\pm = ( |t_1|^2 |a_1|^2 + |t_2|^2 |a_2|^2 ) \left[ 1 \pm \real( i \Vb(\s)\V_i(\s) e^{-2 \pi i \s x})\right].
\end{equation}
%where $\Vb(\s)$ now also includes the normalizations does to the instrument transfer function.


\section{Spectral noise in presence of gaussian phase noise}
\label{ap:phasenoise}

Suppose that the signal is a line of power density $2\S$ centered on bin number $k$ corresponding to wavenumber $\s_k$. In the complex interferogram, the line has a power density $\S$ in bin $k$ and $-\S$ at $-k$, and zero everywhere else. To simplify the analysis, let's focus on the positive frequencies, which only contain half the noise. The interferogram at delay $\xn = n\Dx$ is $\Ikxn =\S\dsig e^{-2i\pi \s_k\xn}$. Through a simple DFT, the value of the line in the spectrum in ideal conditions is:
\begin{equation}
\Dx\DFT(\Ikxn)[k'] = \Dx\sum_{n=-N/2}^{N/2-1}\S\dsig e^{-2i\pi \s_k\xn }e^{2i\pi n k'/N} = \Dx\sum_{n=-N/2}^{N/2-1}\S\dsig e^{-2i\pi (k-k')n/N },
\end{equation}
which is equal to $\Dx N\S\dsig = \S$ for $k=k'$ and zero everywhere else.
Note that we have $\s_k\xn = k\dsig n\Dx = kn/N$. and $\dsig = (N\Dx)^{-1}$. The phase noise degrades the effective power of the line, so it is now $\S e^{-\varPhir/2}$ \citep{Richards:2003bp}. The noisy interferogram is $\Ikxn =\S\dsig e^{-2i\pi kn/N}e^{i\Phir\pxn}$.

Designating the operator $\langle \rangle $ as the ensemble average, the noise $\varspec$ in the interferogram is the variance of the DFT:
\begin{eqnarray}
\varspec [k'] & = & \VAR(\Dx\DFT(\Ikxn)[k']) \\
& = & \Dx^2\left(\left\langle\left\vert \sum_n \Ikxn e^{2i\pi n k'/N} \right\vert^2\right\rangle - \left\vert\left\langle \sum_n \Ikxn e^{2i\pi n k'/N} \right\rangle\right\vert^2\right) ,\\
& = & \Dx^2\left(\sum_n\sum_{n'}\left\langle \Ikxn\Iksxn\right\rangle e^{2i\pi (n-n') k'/N} - \sum_n\sum_{n'}\left\langle \Ikxn\right\rangle\left\langle\Iksxn\right\rangle e^{2i\pi (n-n') k'/N}\right), \\
& = & \Dx^2\sum_n\sum_{n'} \left[ \left\langle \Ikxn\Iksxn\right\rangle - \left\langle \Ikxn\right\rangle\left\langle\Iksxn\right\rangle\right] e^{2i\pi (n-n') k'/N}.
\end{eqnarray}
We can write $\langle \Ikxn\Iksxn\rangle = \langle \S^2\dsig^2 e^{-2i\pi (n-n') k/N} e^{i(\Phir\pxn - \Phir\pxnp)}\rangle$. This quantity is equal to  $\S^2\dsig^2 e^{-2i\pi (n-n') k/N} e^{-\varPhir}$ when $n\neq n'$ and equal to $\S^2\dsig^2$ when $n=n'$. The quantity $\langle \Ikxn\rangle\langle\Iksxn\rangle$ is equal to $\S^2\dsig^2 e^{-2i\pi (n-n') k/N} e^{-\varPhir}$ for all $n$ and $n'$. Hence, the term in the sum is nonzero only for $n=n'$, for which it is $\S^2\dsig^2(1 - e^{-\varPhir})$. The value of the sum is then:
\begin{eqnarray}
\varspec [k'] & = & \Dx^2\sum_n\S^2\dsig^2(1-e^{-\varPhir}), \\
& = & \Dx^2N\S^2\dsig^2(1-e^{-\varPhir}),\\
& = & \frac{1}{N}\S^2(1-e^{-\varPhir}).
\end{eqnarray}
This quantity is independent of $k'$, so the noise is white. The negative frequencies contribute the same amount, doubling the noise variance. However, we are only considering the imaginary part of the spectrum, so only half the noise variance is important in our calculation of our $\SNR$. The last expression thus represents the variance of the noise that is useful for our $\SNR$ calculations.

\section{Fringe tracking in the science channels}
\label{apsec:fringeTracking}

For sufficiently bright sources, it is possible to self-calibrate the OPD between subsets of the $M$ interferograms in a track, to prevent the drift of an indirect OPD estimator. The idea is to bin consecutive interferograms in subsets in order to build up enough $\SNR$ to clearly see a fringe and be able to estimate its position with sufficient accuracy. Then, the different subsets within a track can be offset and co-added with better accuracy (smaller OPD noise) than if we were co-adding the $M$ interferogram individually with only the instrument OPD estimator noise. The best scenario would be when the fringe has a high $\SNR$ in each single interferogram - which will be the case of calibrators for BETTII.

There are many ways to fit the location of the fringe center, and the error associated with each method is highly implementation-specific. Here, we consider the simple example of a fringe tracking algorithm in two steps \citep{Rizzo:2012jp}: a Hilbert transform of the interferogram to obtain its envelope; and a centroid of the points of the envelope above a certain $\SNR_\mI$ threshold. The Hilbert transform doubles the error variance in the interferogram, and in the worst case, the centroid has an error variance of approximately $ (n\times\SNR_\mI^2)^{-1}$, where $n$ is the number of data points above the threshold $\SNR_\mI$. The conversion to a phase leads to a phase error variance equal to $[\varPhir(\s)]_\textrm{direct} \sim 2\times(2\pi)^2 {\s^2/\s_0^2}/ (n\times\SNR_\mI^2)$. This indicates that when the $\SNR$ is high enough, this direct estimate of the phase can become better than the estimate coming from an indirect OPD estimator with corresponding phase error variance $[\varPhir(\s)]_\textrm{indirect}$, like the attitude estimator used on BETTII. 

%With noisy data, we estimate that a reasonable parabolic fit on the envelope would lead to finding the fringe center with an error variance of $\varPhir(\s) = (2\pi)^2 {\s^2/\s_0^2}/ (N_f\times\samp\times\SNR_\mI^2)$, where $N_f$ is the number of fringes with good $\SNR$ so that $N_f\times\samp$ corresponds to the number of points effectively used for the fit. This simple expression merely states that our ability to find the fringe center improves with the square root of the number of samples with good $\SNR$.

In Chapter~\ref{chap:phasenoisepaper}, Fig.~\ref{fig:SpectralSNR}, we use Eq.~\ref{eq:noisephth} and a total phase error variance which is a combination of the phase noise from the direct and indirect methods, to ensure continuity:
\begin{equation}
\varPhir(\s) = \left( \frac{1}{[\varPhir(\s)]_\textrm{direct}} + \frac{1}{[\varPhir(\s)]_\textrm{indirect}}\right)^{-1}.
\end{equation}

On BETTII, the bulk of the phase noise comes from the uncertainties in co-adding consecutive scans (timescale~3), as the estimator uses an indirect method and never really measures the absolute phase for low-SNR targets. For high-SNR targets, the method described above can serve as a fringe tracker that not only is useful for calibration, but can also substantially improve the phase estimator's stability over long periods of time by preventing drifts.

%% Appendix Template

\chapter{A Fringe-Tracker for BETTII} % Main appendix title

\label{AppendixB} % Change X to a consecutive letter; for referencing this appendix elsewhere, use \ref{AppendixX}

\section*{Abstract}
We present the design of a fringe tracking system for the Balloon Experimental Twin Telescope for Infrared Interferometry (BETTII). BETTII is a balloon-borne, far-infrared, 8~m-baseline interferometer with two 50~cm siderostats. Beams from the two arms are combined in the pupil plane to enable double-Fourier, spatio-spectral interferometry. To maintain the phase stability of the system, we need to actively correct of the optical path difference (OPD) between the two arms. The fringe-tracking system will work in the near-infrared and will use a reference star within the field of view to achieve two goals: overlap the beams coming from the two siderostats, and track the location of the central fringe packet, which is a measure of the OPD. The fringe tracker will share most of the optical train with the science instrument. This system is part of the overall control architecture that feeds fast steering tip/tilt mirrors and a warm delay line to ensure proper beam combination and OPD control for the science instrument. This paper investigates the different sources of perturbations that are expected at float altitude, and derives the sensitivity of the fringe-tracking system. We show progress on validating our design using a visible light, broadband Mach-Zehnder interferometer that was developed at NASA/GSFC. This system demonstrates the viability of our OPD determination approach and provides a means of testing and characterizing several OPD determination and control algorithms.  


\section{Introduction} \label{sec:INTRO}
BETTII is a balloon-borne payload aimed at pioneering a new generation of free-flying interferometers. The science instrument is based on the concept of spatio-spectral interferometry \citep{Mariotti:1988vea}, which can be summarized in three steps: 1) interfere the beams from two light collectors separated by a distance $B$; 2) scan the delay between those two beams to create interferograms and obtain the spectrum of each source in the field through Fourier transform spectrometry; 3) repeat the process for several orientations and baselines to fill the UV-plane. The Wide-field Imaging Interferometer Testbed \citep{Rinehart:2010hq} (WIIT) is a laboratory testbed at NASA/GSFC which has demonstrated the viability of this method in reconstructing complex spatio-spectral scenes \citep{Lyon:2008cna}. This method is planned to be used in space on future missions like SPIRIT \citep{Leisawitz:2007if} and SPECS \citep{Harwit:2006hl}.


BETTII implements this method over a 2'$\times$2' field of view (FOV) in two simultaneous science bands: 30-50~$\um$ (the short band), and 60-90~$\um$ (the long band). These bands are inaccessible from the ground due to absorption and emission from the atmosphere. At 35~km altitude, above 99.6\% of the atmosphere, one can do science in these bands. With a fixed baseline size $B=8$~m, BETTII will provide an angular resolution $\lambda/B\sim 1$" and $\sim 2$" in the short and long band, respectively. The main science goal of BETTII's first flight is the study of star formation in the centers of nearby clusters, where the protostars are very bright at our wavelengths and have typical angular separations on the order of arcseconds. 

It is critical to observe these star-forming regions at very high angular resolution in order to understand the physical processes driving the star formation and to discriminate between the various theories currently explaining clustered star formation. BETTII will help answer these questions by simultaneously resolving individual emission regions separated only by $\sim 1$", and by providing low resolution ($R\sim 35$) spectra of these regions.

BETTII consists of an 8~m carbon fiber structure with two flat mirrors (the siderostats) with a 50~cm diameter collecting area on the sky. They redirect light towards the telescopes, composed of a primary and a secondary mirror, located on each side of a central dewar (see Fig. \ref{fig:Structure}). The afocal telescope achieves a 20:1 compression ratio. The dewar is composed of an external liquid nitrogen volume at 77~K, and an internal liquid helium volume at 4~K, in order to provide the appropriate environment for the far-infrared science detectors. %The mechanism that scans the delay between the two beams to produce science interferograms is located inside the 4~K volume, hence is referred to as the Cold Delay Line (CDL).

Spatio-spectral interferometry requires precise knowledge of the optical path difference (OPD) between the incoming beams for two reasons. First, as we scan the delay line and measure the intensity modulation (or fringes) on the science detectors, it is important to be able to match a given intensity measurement with a given OPD, in order to convert the interferogram to a spectrum, and to coherently add multiple interferograms together to build up the signal-to-noise ratio. Second, as we look at the same scene under different orientations, one needs to be able to reference all interferograms to the same phase reference in order to properly reconstruct the scene.

\begin{figure}[ht!]
\begin{center}
\includegraphics[width=1\textwidth]{Figures/BETTII_Top_Level1.png}
%{Two snapshots of BETTII's carbon fiber gondola and external optics. To the left, one early design is shown, without most of the holding structures for the optics. To the right, a more recent model is shown, with additional elements to connect the optics and to connect to the balloon train. This structure has exceptional thermal properties, and very high frequency vibration modes.}
\caption{BETTII's gondola.}%The 25 mm collimated beam then is sent to a delay line (in one arm) and to a K-mirror (on the right). The K-mirror rotates along with the siderostats to produce proper field rotation. The collimated beam is then sent into a 77 K cold volume of the cryostat, where it is split between the science channel and the fringe tracking unit. The science instrument is in a 4K volume. The science detectors operate at 450 mK. Within the dewar, a cold delay line (CDL) constantly scans the OPD, while the warm delay line (WDL) outside the dewar tries to keep ZPD within the appropriate range for the CDL to cover our total field of view.}
\label{fig:Structure}
\end{center}
\end{figure}

One solution for obtaining this path knowledge is to monitor the OPD using interferometric fringes on a reference star within the field of view. On BETTII, this will be achieved with a near-infrared (NIR) interferometer, part of the Fringe Tracking Unit (FTU). The FTU will combine the two beams of the interferometer to achieve NIR fringe tracking using real-time OPD measurement and compensation. The
BETTII implementation has a warm delay line (WDL) to compensate for path changes in the optical train outside the dewar, and a cold delay line (CDL) inside the dewar to create the path delay scanning required for the spatio-spectral imaging.

The science case is discussed elsewhere\cite{Rinehart:2010p2318}. The details of the payload design are discussed in Rinehart et al., 2012 (these proceedings). Benford et al., 2012 (these proceedings) discuss the environment at float in some detail. This paper will focus on the FTU, and is organized as follows. In section \ref{sec:FTU}, we describe the FTU's role, concept and design. In section \ref{sec:PARAMS}, we detail its parameters and derive the sensitivity of the tracking system. In section~\ref{sec:TESTBED}, we show results from a prototype testbed developed at NASA/GSFC to implement several fringe tracking algorithms for BETTII.

\section{THE FRINGE TRACKING UNIT}\label{sec:FTU}

\subsection{Enabling BETTII science}
\subsubsection{Requirements derived from the science}
\label{subsubsec:Reqs}

Two main requirements derive from the science. First, in order to minimize fringe visibility loss, the two incoming beams need to stay overlapped to better than $\sim$~1.5". Second, in order to interpret the science interferograms, the OPD needs to be known to within $\sim 2~\um$, 1/20 of the shortest science wavelength. It is also important to maintain the location of Zero Path Difference (ZPD) close to the center of the CDL, in order to always be scanning the whole field of view at the science wavelengths.

\subsubsection{Expected perturbations}
The relative optical path lengths along the two arms are sensitive to motions of the gondola, telescope pointing errors, and motions generated within BETTII itself. Several other balloon missions \citep{Fixsen:1996kha} have identified multiple pendulum modes of the payload, which are to be expected at float. They have characteristic periods from 2 to 30 seconds and amplitudes of up to $\sim 10$' relative to the gravity vector. Despite the large amplitudes of these modes, these same balloon missions have shown the ability to robustly stay pointed at the target to within an arcminute, because these motions are very regular and smooth.

Pointing errors will impact the OPD because the baseline vector (the vector connecting the centers of the two siderostats) is not always in the plane perpendicular to the line of sight to the science source. With a baseline $B=8$~m, if the baseline vector is tilted by 1' with respect to this plane, it creates a $2.4$~mm OPD perturbation, which is a very large value compared to our science wavelength. A tilt of 1" creates a 40~$\um$ OPD error.

However, these pointing errors can be measured in real time, and the OPD they generate can be compensated by the WDL in order to meet the science requirements. This advantage allows BETTII to provide arcsecond level resolution without requiring arcsecond level pointing stability of the gondola.

Other perturbations also come into play, such as the ones caused by thermal gradients, the momentum wheels, the actuators themselves. These perturbations are being minimized through careful engineering design and by maximizing the symmetry of the two optical arms of BETTII; this takes advantage of the fact that interferometry is only sensitive to the relative path difference, not the absolute path. To the best of our knowledge, \textit{all high frequency ($\gg  1$~Hz) perturbations will be excited by the payload itself}. In principle all these perturbations can be measured during testing on the ground.


\subsection{Description of the fringe tracking unit}\label{subsec:DESCRIPTION}

The FTU has two components: an Angle Tracker (AT), which overlaps the two beams with high accuracy; and a Fringe Tracker (FT), which measures and corrects for the net OPD up to a dichroic located inside the dewar. That dichroic splits the NIR light from the science bands. 
\begin{figure}[ht!]
\begin{center}
\includegraphics[width=1\textwidth]{Figures/Dewar_FT2.png}
%{Two snapshots of BETTII's carbon fiber gondola and external optics. To the left, one early design is shown, without most of the holding structures for the optics. To the right, a more recent model is shown, with additional elements to connect the optics and to connect to the balloon train. This structure has exceptional thermal properties, and very high frequency vibration modes.}
\caption{The Fringe Tracking Unit on BETTII. Left: Optical path from the primaries to the FTU detectors. In the dewar, represented by the gray rectangle, only the FTU optics are shown. The optics for the science instrument go above. The right picture is a zoom on the FTU optics inside the dewar. A: H1RG detector; B: 50/50 Beamsplitter; C: Dichroic splitting the AT from the FT; D: prism. The instrument is in three dimensions, with the 50/50 beamsplitter (B) located in the center of the dewar. Circular optics on this picture are flat mirrors, square optics are off-axis parabolas.}
\label{fig:FTU}
\end{center}
\end{figure}

Because we want the FTU to work in a wide FOV to optimize our chances to find an appropriate guide star, we use an HAWAII-1RG detector that will be located in the 77~K volume of the dewar. The detector has enough pixels to allow fields of view of several arcminutes, and is sensitive tat wavelengths up to $\sim$~2.5~$\um$. In order to optimize the use of NIR photons the AT will work in a band from 1~$\um$ to 1.5~$\um$, while the FT will work in a band from 1.6 $\um$ to 2.4 $\um$.




The AT will look at the images from individual arms and control fast-steering tip/tilt mirrors located just before the entrance into the dewar. The FT will interfere the beams from the arms and control the warm delay line to operate fringe tracking. The FTU has a total of four outputs, two for the AT, two for the FT.


Fig. \ref{fig:FTU} shows our optical design for the AT and FT package. The FT is on the bottom level, and the beam combination occurs at the center of the dewar. The AT is on the top level. Several fold mirrors redirect the four beams onto four optical quadrants on a single H1RG detector (see Fig. \ref{fig:quad}). Two quadrants correspond to the angle tracker and represent an image of the field of view as seen through each arm, in the wavelength range 1-1.5~$\um$. The two other quadrants each represent an image of the interfered field of view. One of them is dispersed by a prism.
%The AT needs at least to ensure beam combination better than 1.5" to achieve science fringes. However, in order for the FT to work, we also need to ensure proper beam combination for this $2\um$ interferometer to see fringes.

%We now present the design principle of the angle-and-fringe tracking system that we name the Fringe Tracking Unit (FTU).
%The FTU will be located inside the 77 K compartment of the dewar and will share all the external optical path with the science channel, up to a dichroic located inside the dewar. It will be sensitive to any relevant change in the external optics that would impact the interferometric signal in the science channels. It will not monitor any change of the science path inside the dewar, but these optics are not expected to experience any change over the duration of the flight since the temperature of the dewar remains constant.

%The FTU, shown in Fig \ref{fig:FTU}, has two tasks: 1) overlap the beams from both arms with the Angle Tracker (AT); 2) measure the OPD with the fringe tracker (FT) and keep it within the range of the cold delay line. The first of these is necessary to obtain science fringes; the second provides the critical knowledge of OPD that is needed for accurate interpretation of the science fringes. The FT part of BETTII is one of the very top challenges of this mission.

%The BETTII FTU design uses NIR light from 1-2.4 $\um$ with a single HAWAII-1RG detector. The AT works in a band from 1 $\um$ to 1.5 $\um$, while the FT works in a band from 1.6 $\um$ to 2.4 $\um$. The AT is required to provide 0.1" pointing knowledge in each arm at 100 Hz, while the FT is required to provide knowledge of the OPD to within 1$\um$ at 25 Hz. The relatively loose requirement on the FT comes from the fact that the OPD needs to be known to a small fraction of a science wavelength (40 $\um$). Even with this loose requirement, in order to enable fringe tracking at 2 $\um$, we must place more stringent requirements on the external optics and structure than would be required for the science channels alone.


%\subsection{FTU package}\label{subsec:FTUPACKAGE}




\begin{figure}[ht!]
\begin{center}

\includegraphics[width=0.2\textwidth]{Figures/Quad.png}
%{Two snapshots of BETTII's carbon fiber gondola and external optics. To the left, one early design is shown, without most of the holding structures for the optics. To the right, a more recent model is shown, with additional elements to connect the optics and to connect to the balloon train. This structure has exceptional thermal properties, and very high frequency vibration modes.}
\caption{Single H1RG with four quadrants.}
\label{fig:quad}
\end{center}
\end{figure}

%The FTU can operate over fields of view larger than the science channels, if the full quadrants of the H1RG are readout. However, as we look at the edges of the field of view, the aberrations will become a problem, especially in the FT. Our actual usable field of view will be determined after the optical design is completed and all issues are resolved.
%Working in the NIR gives us two advantages: first, we are not using the science photons and are not subject to the considerable background noise present at long wavelengths; second, we are naturally more sensitive to OPD: \textit{Achieving, even poorly, angle and fringe tracking in the NIR would guarantee a good science return}, since the long wavelengths of the science channels are naturally more forgiving than the NIR channels. However, for the fringe tracker to operate at this level, we must place more stringent requirements on the external optics (and structure) than would be required for the science channels alone.

\subsection{Angle tracker concept}\label{subsec:ANGLE}


The AT feeds an error signal to fast steering tip/tilt mirrors that compensate for any movement of the star with respect to a given location in each quadrant. In OBSERVE mode, we baseline a sensing frequency of 100 Hz, two orders of magnitudes faster than any expected external perturbation on our system.

We baseline simple centroiding algorithms and image motion sensing algorithms \citep{Hardy:1998ul}. As we continue work with the FTU testbed in the laboratory, we will investigate additional algorithm options to optimize the performance of the FTU.

%The algorithms used for identifying and tracking the guiding star are presently under investigation. To identify the guiding star, we might want to use a wide field of view and a star pattern-recognition software, similar to the one used for the star camera. For tracking, we will likely use simple centroiding algorithms along with image motion sensing algorithms.

The control architecture in OBSERVE mode is the following. The H1RG reads out only one small region of interest in each arm, located around the spot where we have driven the guide star, and generates an error signal. This signal is fed to the tip/tilts to correct the angle deviation through a simple control loop. The correction signal is also used by the siderostats and the azimuth control to correct the pointing of the overall gondola in azimuth and elevation, at a slower frequency, in order to maintain the tip/tilts near the center of their range. 

%Several options for the tip/tilt stages and actuators are still under investigation. Our baseline design uses Newport goniometer stages and piezo-actuators, although it is unclear if these will withstand the high frequency motion and the conditions at float. A vacuum chamber is presently under construction that will allow thorough testing of the candidates under adequate pressure and temperature conditions. 

\subsection{Fringe tracker concept}\label{subsec:FRINGETRACKER}
%\subsubsection{On the important role of a fringe tracker}
%BETTII's fringe tracker shares the optical train with the science interferometer, so it will monitor all the net OPD perturbations caused by the combination of pointing errors and any type of internal perturbations. It will also allow us to have an absolute phase reference. In addition, the FT ensures that ZPD always stays within the range of the CDL, so that we are properly scanning through ZPD for the whole field of view.

%The second aspect is that we see the 2$\um$ FT as a path forward to the future of free-flying interferometers. Enabling this type of technology would pave the way to a wide field of science experiments, that would benefit from a clear atmosphere (hence allowing long integration times) while still having relatively long baselines (good spatial resolution), a combination that is challenging on the ground because of atmospheric turbulence. Our FT has inherent interferometric resolution of $\lambda/B\sim$ 50mas, so one can imagine all sorts of science fields that would benefit from low-background, low phase-noise observations at these wavelengths. 

%However, making the FT work tightens all the requirements by an order of magnitude, since now we are constructing a free-flying, 2$\um$ interferometer with an 8m baseline, which needs to be robust to 100K of thermal shift between the environment in which it is aligned and the environment in which it will operate.

\subsubsection{Design details}

The bulk of the impact of the pointing errors on the OPD will be measured by the gyroscopes and the AT. This can be fed directly to the WDL in order to compensate for it, by converting the angular rate to a delay line speed, with appropriate coordinate system transformations. This approximately ``freezes" the fringe motion due to pointing errors.

However, this only gives us a blind, open-loop correction of the OPD since we have no feedback on what the actual OPD; we only know how much it varies due to pointing errors. The FT comes into play here by closing this loop and providing a direct measurement of the net OPD. We will use simple, robust schemes that operate fringe-scanning and fringe-tracking in order to find and keep the fringes in a known position. Note that for our science, \textit{it is only necessary to know where the 2 $\um$ fringes are, not necessarily to stay on the central fringe}.

\subsubsection{Fringe finding and tracking algorithms}

%Given the environmental conditions described in section \ref{sec:PERT}, the gondola will be subject to a sum of low-frequency sinewave motions that are generated by pendulation. Our control system (Benford et al. 2012, these proceedings) should be able to stabilize the gondola to better than 10 "/s. Taking this value as a maximum pointing drift in pure cross-elevation, it will make ZPD move at 400 $\um$/s in delay space, or 200 fringes/s. Although this is too fast for the H1RG to read and measure any fringe modulation, this rate can be estimated by the gyroscopes and the AT, and fed directly to the WDL through a PID loop.

%However, the above only gives us a blind, open-loop correction of the movement of ZPD. Although this ``freezes" the fringes, there could still be an offset between the actual position of our delay line and the real location of ZPD. There will also be an error due to non-ideal correction and estimation errors due to gyroscope drift, etc. The 2 $\um$ FT comes into play here, closing this loop. We will use simple, robust schemes that operate fringe-scanning and fringe-tracking in order to find and keep the fringes in a known position.  Hence, one can imagine a scheme where we scan back and forth through the fringe packet and interpolate the location of ZPD at each step.

%\begin{figure}[ht!]
%\begin{center}
%\includegraphics[width=0.5\textwidth]{Images/HT.pdf}
%\caption{Hilbert Transform of idealized interferogram.}
%{Two snapshots of BETTII's carbon fiber gondola and external optics. To the left, one early design is shown, without most of the holding structures for the optics. To the right, a more recent model is shown, with additional elements to connect the optics and to connect to the balloon train. This structure has exceptional thermal properties, and very high frequency vibration modes.}
%\label{fig:HT}
%\end{center}
%\end{figure}
We baseline a Hilbert transform algorithm to give us the fringe envelope, and a sliding-parabola fitting routine to give us the peak the envelope, which is a measure of the center of the fringe packet. The position of the WDL that corresponds to the center of the fringe packet is the point of ZPD. At this point we are sure that all net external OPD effects have been corrected, and now we can robustly interpret the movement of the CDL in the science instrument.

The Hilbert transform algorithm will be used for scans larger than the fringe packet, and in particular for initial determination of the fringe location. We baseline a version of the 4-bucket algorithm (also called ABCD \citep{Colavita:2010ce}) to lock on the fringes once we found them, with multiple backup options if that algorithm fails. This algorithm also uses the dispersed output to provide group delay information at a slower frequency \citep{Colavita:2010ce}.

Once we lock on the fringes, we make measurements of the OPD at 25~Hz (4 measurements at 100~Hz are required to make one estimation of the OPD). If the locking algorithm fails, we can use the Hilbert transform to scan back and forth through the fringe packet, determine the location of the envelope peak for each scan, and interpolate between scans. We can expect this algorithm to work at 1-3~Hz.

Hence in OBSERVE mode the WDL will implement two movements at once: the first has a slow frequency and large amplitude to follow gondola pendulation and mispointing, and occurs whether or not fringes are found with the 2~$\um$ FT; this uses the gyroscopes and AT signals to freeze the OPD. The second movement scans the OPD to look for fringes and lock on them once they are found. Residual errors from our gyroscopes and AT will likely prevent the WDL from being able to completely freeze ZPD for a whole scan duration: this will create dynamic variations of our 2~$\um$ fringe sampling when scanning for fringes.
%Note that during a given scan of the WDL (which always occurs on top of the broad motion that compensates for smooth pointing errors), there will be a residual error in our attitude rate estimation, as well as multiple other unpredicted perturbations. The error in the rate estimation will create a dynamic variation of the the effective sampling of our fringes, which can be seen, to first order, as a phase chirp. 
The Hilbert transform, although computationally heavier than other algorithms, is insensitive to sampling variations and inhomogeneities since it has an infinite bandwidth and affects all frequencies in the signal. The ABCD, on the other side, is a filter matched to one given sampling frequency, which is less appropriate to this problem.

%\subsubsection{Envelope determination by Hilbert transform}

%The Hilbert transform (HT) of a complex data sample consists in a multiplication by $-i\times\textrm{sign}(f)$ in the frequency domain. This has the effect of shifting all frequencies in the signal by $\pi/2$. In particular cosines are transformed to sines, and interferograms $I(\phi)=V(\phi)\cos(\phi)$ are transformed into $I_{\pi/2}(\phi)=V(\phi)\sin(\phi)$. The square envelope is estimated as $V^2(\phi)=I^2(\phi)+I_{\pi/2}^2(\phi)$. It is also interesting to note that the noise statistics are unchanged through the HT, so $I$ and $I_{\pi/2}$ have the same noise variance $\sigma^2_I$. The mean of the square envelope is connected to the interferogram's variance by: $\langle V^2\rangle = 2\sigma^2_I$. 




%\section{BASELINE INTERFEROMETRY FROM 35 KM ALTITUDE} \label{sec:PERT}
%\subsection{On the scientific potential of free-flying interferometers}\label{subsec:POT}
%Should I explain here all the ideas I have in case BETTII really works? (2$\um$ interferometry, astrometry, etc) Or should I keep it all a secret?

%BETTII's concept is inherited from proposed space missions SPECS (reference) and SPIRIT (reference), which featured movable baseline to provide extended UV coverage. BETTII's science goals are more modest, but the mission aims at paving the way for future balloon-borne and space-based interferometric missions. 

%If BETTII is a success, it opens up the possibilities of doing high resolution interferometry at long wavelengths that are inaccessible from the ground. In addition, if BETTII's fringe tracker works, it opens up many possibilities for near-IR interferometry. 

%We investigate some of the potential perturbations that are important for a NIR interferometer. Here, we focus on the effects of the atmosphere and the payload pendulum modes. Other perturbations, such as the ones caused by thermal gradients, are discussed in Benford et al., 2012 and Rinehart et al., 2012 (these proceedings). 

%\subsection{The environment at 35 km}\label{subsec:ENV}
%temperature, pressure; cite someone here
%At an altitude of 35 km the typical temperature is $\sim$220-230 K, the typical pressure is $\sim$0.01 bar, and the density of air is $\sim$0.02 kg.m$^{-3}$.
%The temperature is an important element because it will determine the dominant amount of far-infrared background radiation, by setting the temperature of the warm optics. It is expected that, over the course of one night ($\sim$8 hrs), the temperature of the optics will constantly be dropping, from $\sim$310 K at launch to $\sim$226 K at float, and that we will never reach thermal equilibrium.

%Using standard off-the-shelf computers, electronics, and actuators is challenging under these conditions, as they are typically neither designed for nor tested in this type of environment. Rinehart et al., 2012 (these proceedings) describe the test test chamber we are building to test the candidate hardware.

%The structure has exceptional properties, such as a lowest resonant mode above 20 Hz. To the best of our knowledge, there is no external perturbation that can excite structural modes at such a high natural frequency. There are several identified low-frequency pendulum modes that we describe briefly in section \ref{subsec:PENDULUM}, but these are all $<$0.5 Hz. Thus, in the design of BETTII and its fringe tracking unit, we work under the assumption that \textit{all high frequency vibrations will be excited by the payload itself}. Hence, a meticulous test plan will be used to robustly identify all sources of self-generated perturbations when BETTII is fully up and running in the lab. 

%For an interferometer, it is also important to consider other aspects of the environment, and in particular the stability of the overall system for the relevant time scales. Although measurements have been made at rather large scales (arcminute scale, see section \ref{subsec:PENDULUM}), there is very little literature about the stability of systems to the sub-arcsecond level. However, to the best of our knowledge, there is no physical mechanism to excite high-frequency ($>$10Hz), arcsecond-level gondola jitters. 

%\subsection{The atmosphere above the observatory}\label{subsec:ATMO}

%\subsubsection{Fried parameter and coherence time}
%There has been extensive studies on atmospheric turbulence and its impacts on the performance of ground-based telescopes\cite{Hardy:1998p2181}. The atmosphere distorts the wavefronts  due to inhomogeneities in temperature which in turn create inhomogeneities in the index of refraction of the air along the line of sight. As two telescopes separated by some distance look at the same target on the sky, the atmosphere above each these telescopes is slightly different. As a result the two wavefronts will have residual piston errors (in addition to other effects), that can be changing rapidly and considerably complicate the fringe tracking process.

%Usually, the level of atmospheric disturbances is represented by a characteristic turbulent scale size, $r_0$, also called the Fried parameter\cite{Fried:1966p2197}, which can be understood as the aperture size over which the mean square wavefront error is 1 square radian\cite{Hardy:1998p2181}. The Fried parameter synthesizes the effect of the integrated variations over the light path within the atmosphere. On the ground $r_0$ is usually less then a few tens of centimeters in the optical and NIR, even at the best facilities.

%However, above the tropopause, winds are much more stable and atmospheric turbulence has decreased by orders of magnitude. At 30 km and above, there is very little atmosphere left and its integrated contribution on the wavefront is now negligible. We can estimate what $r_0$ will be, as well as how fast our system needs to operate in order not to be affected by atmospheric phase noise between our two apertures separated by $B=8$ m. We use models\cite{Perlot:2009p1124} that are based on space-based measurements of stellar scintillation through the atmosphere\cite{Gurvich:2007p1117} to determine the temperature gradient through the upper atmosphere. The variation in index of refraction is directly related to the temperature structure. We use $C_n^2$ as the refractive index structure parameter\cite{Hardy:1998p2181}. We use the following model for the upper atmosphere:
%\begin{equation}
%C_n^2(h) = \left[2.7\times 10^{-4}e^{-h/H_0}\right]^2\times 10^{-10}e^{-h/H_1},
%\end{equation}
%where $H_0$ is the traditional atmospheric scale height, $H_0\sim 7$ km, and $H_1=10$ km\cite{Perlot:2009p1124}. We can now write\cite{Hardy:1998p2181}:
%\begin{equation}
%r_0(\theta) = \left[0.423\left(\frac{2\pi}{\lambda}\right)^2\frac{1}{\cos(\theta)} \int_{\textrm{alt}}^{\infty} C_n^2(h)dh\right]^{-3/5},
%\end{equation}
%where $\theta$ corresponds to the angle between the line of sight and the zenith. One obtain a typical cell size $r_0\sim 12$ km for 45 degrees elevation and an altitude alt=35 km. The characteristic frequency of the atmospheric turbulence is called the Greenwood frequency\cite{Hardy:1998p2181}, and an approximation of it is:
%\begin{equation}
%f_G=0.427\frac{v}{r_0},
%\end{equation}
%with $v$ the wind speed, that we can take as being $\sim$ 6 m.s$^{-1}$. We obtain a typical Greenwood frequency $\sim 0.2$ mHz with the same model used before\cite{Perlot:2009p1124}. 

%One other important consideration is the typical coherence time, that is, the integration time after which the rms wavefront error between both apertures has reached 1 rad. We use\cite{Colavita:1999p153} $T_c=0.815r_0/v$, and we find that $T_c\sim 1600$ s. 

%To test the robustness of these parameters, let's assume that the models underestimate the integrated amount of atmospheric turbulence by three orders of magnitude. In that case, $r_0 \sim 200$ m, $f_G=10$ mHz and $T_c=25$ s, which are still very comfortable values to design a proper instrument. We conclude that \textit{even if the float conditions are significantly worse than expected, the atmosphere does not provide any fundamental impediment for balloon-borne, NIR interferometry.}



%This demonstrates that balloon-borne interferometers are not subject to atmospheric turbulence perturbations over any reasonable timescale and thus do not require any adaptive optics system, even at short wavelengths ($r_0$ scales as $\lambda^{6/5}$). This is a great advantage for balloon-borne facilities trying to achieve baseline interferometry, but it is also excellent for regular telescopes trying to undertake observations that want to minimize seeing (which also depends on $r_0$).

%\subsubsection{Absorption bands in the NIR}
%{Two snapshots of BETTII's carbon fiber gondola and external optics. To the left, one early design is shown, without most of the holding structures for the optics. To the right, a more recent model is shown, with additional elements to connect the optics and to connect to the balloon train. This structure has exceptional thermal properties, and very high frequency vibration modes.}
%We ran models using MODTRAN of the atmospheric transmission both at a usual 4 km altitude, where one can distinguishes J, H, K bands, and at balloon altitude. At float, the transmission is excellent and the wavelength bands of our instruments are not constrained by the atmosphere.

%\subsection{Pendulum modes} \label{subsec:PENDULUM}
%The previous sections indicate that the atmosphere is not going to be a limitation in the design of a 2 $\um$, 8-m baseline interferometer. However, the more significant issue is the lack of complete control of the gondola.
%Indeed, the large-scale, low-frequency motions that are observed\cite{Fixsen:1996p737} are due to the geometric configuration of the balloon-payload assembly. As the payload hangs from a long ladder attached to a large and massive balloon, the geometry implies natural pendulum modes that have been measured by several balloon missions\cite{Fixsen:1996p737}. These were shown to be fairly predictable even with simple models. The periods of these modes range between $\sim$ 2 s and $\sim$ 30 s, and the amplitudes of up to $\sim$ 5-10 arcminutes. 

%The pendulum modes are excited right after ascent, as well as after each large slew. Since the coupling to the ambient, low-density air is very low, these modes are only slowly damped (they have typical $Q$ factors of about 30) and thus one needs to be actively controlling them to stay pointed.
%Despite the large amplitudes of these modes, previous balloon missions have shown the ability to stay pointed at a given target better than an arcminute, and we will use this figure for design purposes. A preliminary, rigid-body model of BETTII and its basic control system model already provides better results for the moment, but several parameters still remain to be tuned.

%While BETTII will achieve arcsecond-level angular resolution, this does not require gondola pointing at the arcsecond level.  By using an interferometer, pointing errors can be directly translated to OPD errors (see Fig. \ref{fig:pendulum}), and can be taken out in a delay line. In theory, therefore, BETTII partially decouples the resolution of the instrument from the gondola pointing, eliminating the requirement for sub-arcsecond pointing stability.


%\begin{figure}[ht!]
%\begin{center}
%\includegraphics[width=0.4\textwidth]
%{Images/BETTII_Pendulation_CrossElevation.png}
%{Two snapshots of BETTII's carbon fiber gondola and external optics. To the left, one early design is shown, without most of the holding structures for the optics. To the right, a more recent model is shown, with additional elements to connect the optics and to connect to the balloon train. This structure has exceptional thermal properties, and very high frequency vibration modes.}
%\caption{A cross-elevation pendulum mode changes the OPD by making one arm slightly longer than the other. With an 8 m baseline, 1" of pointing error in the cross-elevation direction corresponds to $\sim$40 $\um$ of OPD. To compensate for this effect, BETTII needs to control both azimuth and elevation. Note that pendulum modes in the elevation direction do not affect the OPD. These are corrected by the siderostat elevation only.} 
%\label{fig:pendulum}
%\end{center}
%\end{figure}

%\subsection{Thermal effects} \label{subsec:THERMAL}
%The thermal environment of the gondola needs to be considered carefully, and is one of the heaviest modelling efforts. Although the temperature of the air can be expected to be uniform across the gondola, multiple components will have a deviation from ideal and could create challenging thermal gradients. In particular, as all the electronics, batteries and computers are embedded in the central aluminum section of the gondola, there will be a natural outward thermal gradient that will need to be modelled and potentially controlled.
%In addition, there will be the natural gradient between the bottom of BETTII, which sees the warm Earth, and the top of BETTII, which sees the very cold space. Although this is a symmetrical gradient in that it will affect both arms the same way, we are still planning on having a large mylar shield under the payload. Thermal gradients can get more complicated if the Sun is up, but we design BETTII to have nominal operations during the night. The effect of the Sun in terms of thermal effects and in terms of the star camera sensitivity will be modelled later down the road.

%The structure will shrink of several hundreds of microns over the course of the flight before it reaches thermal equilibrium. Although this can be very problematic for OPD purposes, the truss has excellent symmetry and thus we can expect very little differential motion between the two sides. This is the result of an on-going effort to design the system as symmetrical as possible, so self-generated perturbations and thermal gradient impact both arms in the same way.



%\subsection{Cosmic rays}
%One other downside of being that high up in the atmosphere is that we are not protected from cosmic rays as we are down on the ground. This complicates the software design as it needs to be as robust as possible to glitches induced by cosmic rays. 
 
%\subsection{Other perturbations} \label{subsec:OTHERPERT}
%Other types of perturbations are discussed in Rinehart et al and Benford et al. (these proceedings). Among them, the thermal behavior of the gondola is critical for long-term optical stability. The momentum wheel assembly used to control the azimuth of the payload is expected to inject high-frequency vibrations in the gondola, and they will be thoroughly measured in the lab. However, BETTII is designed to be robust to perturbations that occur symmetrically in both arms, which 
%Among the other significant perturbations that we anticipate are the ones that could be created by the momentum wheels that we are using to control the azimuth of the payload. Two wheels are used along with stepper motors and can inject high-frequency perturbations into the whole gondola, if the wheel assembly is not properly isolated from the rest of the structure, or if the perturbations propagate differently in both arms. Although the details of these perturbations are still to be worked out, this is in principle testable on the ground.

%Other events, such as dropping a ballast and temperature-induced structural ``popping", will have less predictable effects. It will be natural for BETTII to have an on-board calibration procedure that resets the position of zero optical path difference within the gondola.

%\subsection{The importance of synchronization and the one-clock paradigm}
%Since the perturbations are self-generated, it is of paramount importance to synchronize every moving and sensing element on the gondola. In order to achieve this challenge, we are working with a one-clock paradigm, which has proven to work on previous balloon missions undertaken by members of the team\cite{Fixsen:1996p737}. This consists in having one single master clock for the whole payload, which is distributed to every subsystem after being divided by an integer number. Although this is possible in principle, it creates difficulties as we are considering buying off-the-shelf actuators and sensors, which sometimes come with their own electronics and sometimes their own clock.

%\subsection{Autonomy of the payload}
%Although this is not strictly speaking a feature of the environment, it is important to note that two-way communication with the payload will be limited. The up/downlink will actually change as the payload drifts away from the launch facility. It will not be possible, for example, to stream down large quantities of data, or completely upload a new version of the flight software. This is possibly where BETTII differs the most from ground-based interferometers, in the sense that it needs to be able to make a lot of decisions on its own, and should be baselined to operate completely autonomously by default. In practice, we will have ways to make executive decisions such as switching between operating modes, especially for our first flight (e.g., if we don't find NIR fringes, etc). This is important to note, as \textit{we do not design BETTII to be a remote-controlled instrument}, but instead a real autonomous payload that will implement a pre-determined flight plan and can decide whether or not it can move on to the next step.





\section{FTU PARAMETERS AND SENSITIVITY}\label{sec:PARAMS}
\subsection{Top-level Parameters}
\begin{table}[!ht]
\begin{center}
\caption{FTU parameters. }
\label{tb:FTUparams}
\begin{tabular}{|c||c|c|}
\hline 
Parameter & Angle Tracker & Fringe tracker \\ 
\hline 
\hline
Central wavelength $\lambda_0$ & 1.25 $\um$ & 2 $\um$ \\ 
\hline 
Bandpass $\Delta\lambda$ & 0.5 $\um$ & 0.8 $\um$ \\ 
\hline 
Plate scale & 0.6"/px & 1"/px \\ 
\hline 
Airy disk diameter & 2~px & 2~px \\ 
\hline 
Throughput efficiency & $\sim$35\% & $\sim$35\% \\ 
\hline 
Integration time per read & 10 ms & 10 ms \\ 
\hline 
Required performance & 0.1" & 1 $\um$ \\ 
\hline 
Range & 3' (on the sky) & 1 cm \\ 
\hline 
Minimum step size & 0.05" (on the sky) & 0.5 $\um$ \\ 
\hline 
Nominal sensing frequency & 100 Hz & 25 Hz \\ 
\hline 
\end{tabular} 
\end{center}
\end{table}

%The interferometric visibility specification is an on-going debate. We are presently designing the system to provide a  minimum visibility of $V=0.25$. We describe in section \ref{subsec:VIS} the details of the visibility budget.
The top-level design parameters are gathered in Table \ref{tb:FTUparams} for both the AT and the FT in a nominal OBSERVE mode.
The throughput efficiency assumes 12 mirror reflections with 99\% reflectivity, a detector quantum efficiency of 0.7, and about 55\% of additional loss due to dichroics and bandpass filters. The sensing frequency is the frequency at which we track the angular deviation (for the AT) or the OPD variations (for the FT). For the FT, if we can never lock on fringes, we can track them with the Hilbert transform at $\sim~1-3$~Hz.


\subsection{Background noise}\label{subsec:Noise}
At 2~$\um$, the dominant background noise is caused by the atmospheric airglow, generated by the OH lines at $\sim$~90~km altitude\cite{Hofmann:1977p2231}. Hence, our balloon experiment will be subject to this noise just like ground facilities. Although there has been inconsistency in the literature on the atmospheric radiance at 2.4~$\um$ \citep{Matsumoto:1994io,Mandolesi:1998wt} we use a conservative value for the radiance over both our bands, of $R_\textrm{atmo}=10$ nW.cm$^{-2}$.sr$^{-1}$. This corresponds to approximately 500 photons per second in the AT and 1600 photons per second in the FT. 

Since the AT and FT read out at 100 Hz in OBSERVE mode, these noise levels have standard deviations of 2 and 4~e$^{-}$~rms respectively, which will be negligible compared to the H1RG read noise of $\sigma_\textrm{CDS}$=18~e$^{-}$~rms. 

Other sources of noise of NIR background noise have been investigated, but none will have a significant contribution. Hence, the detector read noise will be our limiting factor for faint stars.

\subsection{Atmospheric effects}

\subsubsection{Transmission}
Balloon altitudes provide significantly better atmosphere transmission in the NIR wavelength region, compared to ground observatories. Fig. \ref{fig:trans} illustrates this difference using a modelling software called MODTRAN. The transmission from an altitude of 4~km shows transmission windows (J, H, K bands) that would limit the design of a ground-based interferometer. At float, the bands are not limited by the atmospheric transmission and thus we can use larger bands than the traditional J, H and K in order to optimize our photon signal.

\begin{figure}[ht!]
\begin{center}
\includegraphics[width=0.6\textwidth]{Figures/trans.pdf}
\caption{Model atmospheric transmission}
\label{fig:trans}
\end{center}
\end{figure}

\subsubsection{Phase noise}

There has been extensive studies on atmospheric turbulence and its impacts on the performance of ground-based telescopes \citep[e.g.,][]{Hardy:1998ul}. The atmosphere distorts the wavefronts  due to inhomogeneities in temperature which in turn create inhomogeneities in the index of refraction of the air along the line of sight. As two telescopes separated by some distance look at the same target on the sky, the atmosphere above each these telescopes is slightly different. As a result the two wavefronts will have arrival time differences, that can be changing rapidly and considerably complicate the fringe tracking process. For BETTII's FTU, it is important to check if the atmospheric noise has significant impact on our fringe tracking capability.

Usually, the atmospheric disturbances are represented by a characteristic turbulent scale size, $r_0$, called the Fried parameter \citep{Fried:1966tk}, which can be understood as the aperture size over which the mean square wavefront error is 1 square radian \citep{Hardy:1998ul}. The Fried parameter synthesizes the effect of the integrated variations over the light path within the atmosphere. On the ground, $r_0$ is usually less than a few tens of centimeters in the optical and NIR, even at the highest facilities.

However, above the tropopause, winds are stable and atmospheric turbulence has decreased by orders of magnitude. At 30 km and above \citep{Perlot:2009jqa}, there is very little atmosphere left and its integrated contribution on the wavefront is nearly negligible. We use models for the upper atmopshere \citep{Perlot:2009jqa} to give us $C_n^2$, the refractive index structure parameter \citep{Hardy:1998ul}:
\begin{equation}
C_n^2(h) = \left[2.7\times 10^{-4}e^{-h/H_0}\right]^2\times 10^{-10}e^{-h/H_1},
\end{equation}
where $H_0$ is the traditional atmospheric scale height, $H_0\sim 7$ km, and $H_1=10$ km is another scale height\cite{Perlot:2009p1124}. This is a critical parameter needed to determine the Fried parameter \citep{Hardy:1998ul}:
\begin{equation}
r_0(\theta) = \left[0.423\left(\frac{2\pi}{\lambda}\right)^2\frac{1}{\cos(\theta)} \int_{\textrm{h=35~km}}^{\infty} C_n^2(h)dh\right]^{-3/5},
\end{equation}
where $\theta$ corresponds to the angle between the line of sight and the zenith. At $\lambda=2~\um$, one obtains a typical cell size $r_0\sim 12$ km for 45~degrees elevation and an altitude of 35~km. The characteristic frequency of the atmospheric turbulence is called the Greenwood frequency\cite{Hardy:1998p2181}, $f_G\sim0.427 v /r_0\sim 0.2$ ~mHz, where $v\sim 6$~m.s$^{-1}$ is the wind speed.

One other important consideration is the typical coherence time for a two-telescope interferometer\cite{Colavita:1999p153}, $T_c\sim 0.815r_0/v\sim 1600$~s. This corresponds to the integration time during which the rms wavefront error about the interval mean has reached 1~radian~rms.

With a baseline $B\ll r_0$, an operating FT frequency of 1~Hz $\gg f_G$ in the worst case (25~Hz nominal), and integration times of 10~ms $\ll T_c$, our system is very robust to atmospheric phase noise. In order for the phase noise to degrade our performance, $\int_{\textrm{h}}^{\infty} C_n^2(h)dh$ would have to be larger than the model by five orders of magnitude, which would give $r_0\sim 12$~m, $f_G=0.2$~Hz, and $T_c=1.6$~s. \textit{We conclude that the atmosphere is not a limitation for balloon-borne, NIR interferometry.}


\subsection{Faintest detectable star and projected performance}\label{subsec:FAINTEST}
Using the background noise levels described in section \ref{subsec:Noise}, we can estimate the limiting magnitude of the AT and the FT. We pick J-magnitude reference points for the AT band, and K-magnitude reference point for the FT band, a reasonably conservative estimate since our guide stars will usually be brighter at shorter wavelengths, and our bands extend shortward of the regular J and K bands.

\begin{table}[!ht]
\begin{center}
\caption{Limiting magnitudes }
\label{tb:LimMag}
\begin{tabular}{|c||c|c|}
\hline 
Parameter & Angle Tracker & Fringe tracker \\ 
\hline 
\hline
Specified SNR per 10 ms read & 10 & 10 \\ 
\hline 
Minimum flux density (mJy) & 70 & 70 \\ 
\hline 
Limiting Magnitude & 10.9 (Jmag) & 10 (Kmag)\\ 
\hline 
\end{tabular} 
\end{center}
\end{table}

Table \ref{tb:LimMag} shows the limiting magnitudes for the desired SNR ratio. We baseline a SNR of 10 on the flux estimate for the AT. We are confident that it will allow the required sub-pixel centroid accuracy, although this will be robustly assessed after thorough testing in the next couple of months.

%Although the research on algorithms for centroiding and image motion is still ongoing, we baseline a SNR of 5 on the flux estimate in the AT, which likely will be able to provide a sub-pixel centroid accuracy.

The SNR of 10 on the flux estimate for the FT is derived from a specified interferometric SNR of 2.5 and a visibility $V=0.25$. We are confident that fringe tracking can be achieved with this SNR, based on our experience in the lab described in section \ref{sec:TESTBED}.

\subsection{Visibility budget}\label{subsec:VIS}

The visibility budget for the FT fringes is shown in table \ref{tb:Budget}. The static contributors to the budget need to be met at all times, and are composed of the differential wavefront errors (WFE) due to mirror surface quality and system misalignments, the amplitude mismatch between the two beams, the polarization effect and the pupil shear at beam combination \citep{Lawson:2000vf}. The differential WFE effect from misalignments is by far the strongest contributor to the budget. 

Dynamical effects will also degrade the visibility, as the FT tries to integrate the interferometric signal. Hence these need to be met every 10~ms, our typical integration time. The uncorrected OPD corresponds to the residual OPD after the WDL compensates for pointing errors. This is the combination of an estimation error (the gyroscopes/AT are not perfect) and a control error (the WDL will not correct perfectly the signal from the gyroscopes/AT). This will be met if the WDL can correct the real pointing-induced OPD error to better than $\sim 15~\um/\textrm{s}$, or 0.4"/s in equivalent pointing error. Estimation of the angular rate to this precision is achievable by both the gyroscopes and the AT. The differential tip/tilt corresponds to the angle between the two incoming beams, and determines how well the AT needs to achieve centroiding. This requirement feeds through the AT, and is rather loose as its contribution is small. Finally, the unallocated piston jitter acts as our margin. It is possible, for example, that some perturbations such as the one created by the momentum wheels can excite asymmetric vibrations that would hence create some piston jitter over 10~ms time scales.

%The pure piston jitter acts as our margin, since there is no predicted perturbation that should create piston jitter at $>$100~Hz. The error in azimuthal mispointing correction describes how well we need to estimate and compensate for the azimuth rate. The WDL needs to achieve an OPD compensation equivalent to this level. Assuming a perfect WDL with infinite bandwidth, this is equivalent to saying that our attitude rate estimate needs to be better than 0.4"/second, which will be possible with our gyroscopes and our AT. By far the largest contributor to our visibility loss is the differential wavefront error (WFE), due both to the mirror surface quality and to inaccuracies in alignment of the optical system. While the mirror WFE budget is met by having 12 mirrors with $\lambda/20$ surface quality at optical wavelengths, meeting the other differential WFE budget is very challenging and is discussed in section \ref{subsec:SENSITIVITY}. The differential tip/tilt allocation tells us how well we need to overlap the two Airy disks, and this requirement is soft since the impact in the visibility loss is less. The amplitude mismatch is expected to be achieved easily, as is the polarization effect. The pupil overlap allocation is challenging because it is strongly dependent of where we put the pupil in the FTU. Satisfying this budget and appropriately all effects is still an on-going effort and will likely require several more iterations.

\begin{table}[!ht]
\begin{center}
\caption{Visibility budget}
\label{tb:Budget}
\begin{tabular}{|c||c|c|c|c|c|}
\hline 
Term & Symbol & Alloc. & Units & Effect on visibility \citep{Lawson:2000vf} & $V_\textrm{loss}$  \\ 

\hline 
\hline 
\multicolumn{6}{|c|}{\textbf{Static contributors}} \\ 
\hline
WFE in mirror surface & $\sigma_\textrm{WFE,mir}$ & 0.1 & $\um$ rms & $\exp(-[2\pi\sigma_\textrm{WFE,mir}/\lambda]^2)$ & 0.9 \\ 
\hline 
Other differential WFE & $\sigma_\textrm{WFE}$ & 0.3 &$\um$ rms & $\exp(-[2\pi\sigma_\textrm{WFE}/\lambda]^2)$ & 0.4 \\ 
\hline 
Amplitude mismatch & $R$ & 90 & \% & $2/(R^{1/2}+R^{-1/2})$ & 0.99 \\ 
\hline 
Polarization effect & $\theta$ & 12 & degrees & $\cos(\pi\theta/180/2)$ & 0.99 \\ 
\hline 
Pupil area overlap & $f_\textrm{overlap}$ & 95 & \% & $f_\textrm{overlap}$ & 0.95 \\ 

\hline
\hline
\multicolumn{6}{|c|}{\textbf{Dynamical contributors (valid for each 10~ms integration)}} \\ 
\hline
Uncorrected OPD & $\sigma_\textrm{mp}$ & 0.15 & $\um$ rms & $\exp(-[2\pi\sigma_{\textrm{mp}}/\lambda]^2)$ & 0.8 \\ 
\hline 
Differential tip/tilt & $\sigma_\textrm{tt}$ & 0.1 & asec rms & $2J_1(\pi D\sigma_\textrm{tt,rad}/\lambda)/(\pi D\sigma_\textrm{tt,rad}/\lambda)$ & 0.98 \\ 
\hline 
Unallocated piston jitter & $\sigma_\textrm{piston}$ & 0.1 & $\um$ rms& $\exp(-[2\pi\sigma_\textrm{piston}/\lambda]^2)$ & 0.9  \\ 
\hline 
\hline
\multicolumn{4}{|r|}{\textbf{Total visibility over 10~ms}} & $\prod (V_\textrm{loss})$ & 0.25 \\ 
\hline

\end{tabular} 
\end{center}
\end{table}

\subsection{Alignment sensitivity study: the tough spots of BETTII}\label{subsec:SENSITIVITY}

The visibility budget for the FT drives the alignment sensitivity analysis, which tells us how well we need to align the optics, and how well we need to maintain their relative positions and angles. While the complete details of this analysis are beyond the scope of this paper, it has led to several important points worthy of mention. First, the strongest  requirements are all on the external optics, which are also more challenging to align and stabilize because of the flight environment to which they will be exposed.

The most challenging alignment is between the primary and the secondary mirrors. With a 20:1 beam compression, aberrations created by position shifts of the secondary are dramatic. Although the design is very symmetric and thus will generate mostly symmetric perturbations in both arms, the fields are flipped before entering the telescopes and thus symmetrical perturbations will not create symmetrical WFE. Symmetrical WFE would have no effect on the visibility as it would degrade the wavefront identically in both arms. 

We are minimizing these effects by implementing a homologous design, building the primary, secondary, and supports from a single material, and actively maintaining the structure at a uniform temperature to within 0.5~K.

Alignment effects on the visibility, both in pupil shear and differential WFE, are field dependent. In the middle of the field, the FT has a sweet spot where it is less sensitive to several misalignments. Hence, we choose to put our guide star nominally at the center of the FTU field of view. This is allowed since, even if the science target is pushed to the edges of the field of view, the effects of these misalignments are not as severe on the science visibility, due to the long wavelength of the science instrument.


%Our idea to mitigate this effect is to build the whole primary-secondary assembly out of aluminum, and run a liquid through the assembly in order to keep it isothermal. As the temperature changes while staying isothermal across the assembly, the latter will contract in a homologous fashion, thus not creating wavefront errors.



\section{A fringe-tracking testbed} \label{sec:TESTBED}
\subsection{The challenge of testing BETTII on the ground}
BETTII poses several challenges for system-level ground testing. In particular, the FT is very hard to test due to its sensitivity to atmospheric effects, both over 8~m in the lab, and even more if we look at an astronomical target through the turbulent atmosphere. Our hardware cannot accommodate the very short interferometric coherence time that we will have both in Greenbelt, Maryland and at the balloon launch facility in Texas or New Mexico. 


Full-system testing of the FTU is a problem, so a comprehensive test plan was developed for each subsystem. As a start, a first generation testbed was built out of hardware reused from another project. This effort was done in parallel to the FTU design phase and sensitivity analysis. The testbed was designed to gain experience with fringe tracking algorithms. We describe its concepts and results in the following sections.


%\subsection{A testbed to develop OPD determination algorithms}\label{subsec:TESTBED}
\subsection{Description}
A schematic and picture of the testbed is shown Fig. \ref{fig:layout}; it is essentially a traditional Mach-Zehnder interferometer at optical wavelengths. A halogen white light source is fed to a pinhole and collimated before it is injected in the interferometric core. It is split once, after which each arm goes through an independently controlled delay line. The two beams are then recombined, one output is directly imaged, and the second is dispersed then imaged. 

\begin{figure}[ht!]
\begin{center}
\includegraphics[width=0.8\textwidth]{Figures/layout.png}
%{Two snapshots of BETTII's carbon fiber gondola and external optics. To the left, one early design is shown, without most of the holding structures for the optics. To the right, a more recent model is shown, with additional elements to connect the optics and to connect to the balloon train. This structure has exceptional thermal properties, and very high frequency vibration modes.}
\caption{Optical layout, schematic and picture}
\label{fig:layout}
\end{center}
\end{figure}

Multiple filter setups are available for this testbed, but we mostly use the 20\% (618$\pm$60~nm) and 50\%  (800$\pm$200~nm) bandpass setups. The step size is about 60~nm. Note that we did not spend extensive efforts trying to align the interferometer to optimize the visibility. In particular, there are residual misalignments, differential WFE, and intensity mismatch that create a constant bias to the visibility. However, since we are only interested in the location of the peak center, optimizing the visibility is not be a priority for this first-generation testbed. In addition, we have not incorporated any absolute encoder yet.

\subsection{OPD measuring algorithms}
Three main algorithms have been explored, A, B, and C. A and B use interferogram scans several times the size of the fringe packet to find the peak location. Algorithm A computes the Hilbert transform of the interferogram, obtaining the fringe envelope function squared. The peak is identified after smoothing the envelope and using a sliding parabola fit. This method is insensitive to step size errors. The two regimes of operation are a high-SNR mode, where SNR~$\sim$~85, and a low-SNR mode with SNR~$\sim$~3 (see Fig \ref{fig:FirstScan}). 

\begin{figure}[ht!]
\begin{center}
\includegraphics[width=0.8\textwidth]{Figures/FirstScan2.png}
%{Two snapshots of BETTII's carbon fiber gondola and external optics. To the left, one early design is shown, without most of the holding structures for the optics. To the right, a more recent model is shown, with additional elements to connect the optics and to connect to the balloon train. This structure has exceptional thermal properties, and very high frequency vibration modes.}
\caption{512 steps interferograms obtained under two regimes: SNR $\sim 85$, left; SNR $\sim 3$, right. The line above each interferogram corresponds to the smoothed envelope obtained with the Hilbert transform.}
\label{fig:FirstScan}
\end{center}
\end{figure}

Algorithm B uses the phase term in the Fourier transform of the interferogram \citep{Pedretti:2004ti}. This algorithm is sensitive to step size errors, as they change the location of the signal peak in Fourier space. Hence, the interferogram peak location is extracted by averaging the phases in the FFT over a broad range of bins. The bins that do not contain any signal will see their phases average out.

Both algorithms scan through the fringe packet, compute an identified center, and feed it back to a control loop with unity gain. The location of the next scan window is set to put the identified peak at the center of the window. We use the estimated squared visibility as our metric for successful tracking. For example, when the visibility suddenly decreases significantly from one scan to the other, we consider that we have lost the fringes. Our ability to track the fringes depends on the size of the scan window, which in turn determines the frequency at which we can track, provided a set integration time per data point.

Algorithm C implements dispersed fringe tracking over short $\sim 2$-wave scans around the central fringe. It computes the phase using the Hilbert transform in each wavelength channel (each detector pixel along the dispersion direction), and computes the slope of the phases with respect to wavenumber. The center of the next scan window is adjusted by using the measured slope times some control gain.

\subsection{Results}

\subsubsection{Description of the fringe tracking loop}
Our nominal fringe packet is about 80 steps wide ($\sim 4.8~\um$), so we choose a nominal scan size of 256 steps. We read these interferograms at 200 steps per second, which gives us a tracking frequency of a little less than 1~Hz, similar to what BETTII could have if the ABCD algorithm does not work. The peak is identified robustly through algorithms A and B simultaneously, and throws a flag if the two results are very different. The error signal is the difference between the identified peak location and the center of the scan window. This signal is fed back with unity gain in order to adjust the next scan. The control loop contains a two-point moving average on the error signal for each actuator direction. This allows proper treatment of the hysteresis/backlash and of linear changes in OPD, which are both dependent on the direction the actuator is moving.

\subsubsection{Peak estimation results}
We use A and B to estimate the location of the peak of the envelope in all scans. Both algorithms estimate the same peak location to within 3~steps~rms over multiple runs of 50 scans in the high-SNR regime. In the low-SNR regime, the two algorithms deviate of 5.5~steps~rms. This effectively corresponds to about half a wavelength of estimation error between scans, and we take this as our uncertainty on the real position of ZPD. This will be more robustly determined once we obtain absolute encoders. 

\subsubsection{Effects of hysteresis and backlash}

The actuators in the present testbed have significant hysteresis and backlash. With no absolute encoder, we rely on counting actuator steps, which is not robust. As we scan through the fringe packet and determine the step number of the envelope peak, when the actuator turns around and gets back to the same step, the peak has shifted sometimes of several tens of steps. In addition this effect is different whether the actuator goes in one direction or the other. 

Measurements with our hardware find that this effect is repeatable, and the moving average can compensate for it better than $\sim 6$~steps~rms after about 7 scans in each direction, in the high-SNR regime. In the low-SNR regime, we can correct this effect to about $\sim 11$~steps~rms.

Before disturbing the system with one delay line, we always need a phase where the system learns how to compensate for hysteresis and backlash. This phase usually lasts 20 scans, 10 in each direction.

\subsubsection{Response to an OPD ramp}

After 20 scans, in the high-SNR regime, the peak now falls always at the center of the scan window to within $\pm$~6~steps. Now, we turn on the second delay line and move it in one direction at constant speed, creating an OPD ramp. In practice, we show robust fringe tracking for perturbations slower than 40~steps/s in both directions ($\sim$~4 wavelengths per second, which on BETTII would correspond to $\sim$~0.2"/s). This lasts for 20 scans, which is enough iterations for the algorithm to compensate for the ramp. Results are shown for two different speeds in low-SNR regime in the left panel of Fig. \ref{fig:ramps}, by tracking the distance of estimated peak from the center of the window, in number of actuator steps. Note that the starting position of the first scan with respect to the center of the fringe packet is not necessarily the same for each experiment. For perturbations of 60~steps/s in one given direction, fringe tracking consistently fails as the peak walks out of the scan window too fast for the algorithm to compensate.

In the right panel of Fig. \ref{fig:ramps}, after 20~scans of hysteresis/backlash compensation and 20~scans of OPD ramp, we change the sign of the ramp, and continue for 20 more scans. For this mode, which is our reference mode, we show results in both high-SNR and low-SNR.





%We achieve fringe tracking at low frequency, in both high-SNR regime and low-SNR regime. We predict and correct for most of the actuator backlash and hysteresis, although there is some residual, even at high SNR, see Fig. \ref{fig:windows}. We progressively decrease the width the scan window down to less than a fringe packet, and try to keep the peak at the center of the scan window.

\begin{figure}[ht!]
\begin{center}
\includegraphics[width=1\textwidth]{Figures/ramps.png}
%{Two snapshots of BETTII's carbon fiber gondola and external optics. To the left, one early design is shown, without most of the holding structures for the optics. To the right, a more recent model is shown, with additional elements to connect the optics and to connect to the balloon train. This structure has exceptional thermal properties, and very high frequency vibration modes.
\caption{Left: Fringe tracking results for one single OPD ramp, showing the distance from the estimated peak from the center of the window in number of steps. The algorithm tries to always keep the peak in the center of the window. The first 20 scans have no perturbation of OPD, and show the algorithm trying to compensate for hysteresis/backlash. Scans 20-40 show the algorithm compensating for an OPD ramp. Right: Fringe tracking results for the two consecutive 20~steps/s OPD ramps of opposite directions, for the two regimes. Note that in this particular run, the initial hysteresis/backlash is not optimally compensated during the first 20 scans in the high-SNR regime, due to initial conditions.}
\label{fig:ramps}
\end{center}
\end{figure}

Algorithm C succeeds at $\sim$~3~Hz tracking of the fringes, only in the high-SNR regime, and provided that we start close enough from the central fringe packet. In the low-SNR regime, dispersed fringes in individual channels have very small SNR and do not permit to identify the phase in each channel. We can only achieve $\sim$~3~Hz because if we operate faster the hysteresis/backlash becomes problematic.



\subsection{Limitations and discussion}
In the current testbed, the reused hardware is our main limitation. The actuators we are using have no absolute encoder, and are turn-screw motors with significant backlash and hysteresis. In addition, synchronization cannot be achieved to a good level, mainly because we are working on a basic Windows machine. The OPD is shown to be stable to the environment when no actuator is in movement (70~nm~rms over 20 seconds, or about 1~step~rms), however, residual noise is injected by vibrational modes caused by the delay lines themselves as they are moved and as they change direction.

Despite these hardware limitations, we successfully achieve robust fringe tracking at low frequency, using broad scans across the fringe packet as shown in the previous section. The actuator backlash and hysteresis, however, prevent us from implementing a fast ABCD algorithm over one wave, since there is too much uncertainty in the step size and shifts in OPD of more than one wave are common. 

Our results are strongly dependent on the size of the fringe packet and on the size of the scan window we choose. Even though our nominal set of parameters achieves sufficient fringe tracking, we do not claim they are optimal. In particular, there is a clear signature of the algorithm that can be seen in Fig. \ref{fig:ramps} Left and Right. Investigating the parameter space to search for optimal values will be done only for flight actuators and sensors in their respective control loops. However, all algorithms developed for the present testbed are generic functions that can easily be transposed to future generations of testbeds and facilitate the search for our optimal set of parameters.

Experimentally, we find that the low-SNR regime is very robust for fringe tracking. This makes us comfortable that we will be able to implement this slow-frequency fringe tracking on BETTII with SNR of about 2.5 on the fringes, as this is enough for the Hilbert algorithm to robustly identify the peak location.

%Hysteresis and backlash are significant and prevent us from implementing a simple ABCD algorithm over just one wave. Thus, our implementation of fringe tracking in preparation for BETTII is still incomplete.



\subsection{The new generation FTU testbed}\label{subsec:newFTU}

In order to enable the ABCD algorithm, and reproduce all operating frequencies used on flight, we are designing an upgraded testbed with candidate actuators and encoders selected for BETTII. This new generation testbed would mimic the AT and FT parts of the flight FTU, and will be controlled by one of BETTII's real-time computer to achieve good synchronization. Although it will have all the same parameters such as the plate scale, Airy disk size, etc., it will work at optical wavelengths for convenience and cost considerations. This new testbed will explore the ABCD algorithm in order to lock on the fringes, and will allow the team to tune almost all algorithms with gains and parameters that are relevant to the flight hardware, while we wait for the dewar to be commissioned. In particular, it will be used to test, align, and calibrate the WDL and the tip/tilt mirrors prior to their integration on the gondola.

\section{Conclusions}

In this paper we describe the design of a NIR fringe tracking unit that will implement angle tracking and fringe tracking for BETTII. The FTU will enable BETTII's science by ensuring optimal beam overlap and OPD knowledge prior the entrance to the dewar. 

The angle tracker will look at the fields from the two arms separately. It will find the guiding star and maintain it at a given position on the detector. The fringe tracker will control the OPD by implementing two motions: the first is an open-loop compensation, fed by the signal from the gyroscopes and the angle tracker. This approximately freezes the fringes. The second motion scans the OPD to find fringes and track any uncorrected OPD. We propose algorithms to achieve $\sim~1$~Hz envelope peak finding and tracking (Hilbert transform), and fast fringe tracking at 25~Hz. One of the two outputs of the fringe tracker is dispersed to provide group delay knowledge.

We built a simple lab setup where we tested different fringe finding and tracking algorithms. This features a traditional Mach-Zehnder interferometer that operates at optical wavelengths. We prove the viability of all algorithms except the ABCD, which requires improved actuators. This testbed is presently being utilized to characterize several candidates for our actuators and sensors at room temperature, before a proper environmental test chamber is built.

A new generation testbed is being designed, that will provide significantly enhanced capabilities, including angle tracking. It will feature improved actuators and absolute encoders, as well as a real-time operating system. This new testbed is designed to allow testing and characterization of BETTII flight tip/tilt mirrors and warm delay line. It will also allow subsystem level testing of the FTU computer and electronics, as well as proper tuning of all relevant algorithms for the different operating modes that BETTII will use in flight.
\chapter{Attitude representation in three dimensions}
\label{sec:attituderepresentation}
%\subsection{Definitions and conventions}

There are three common representations of the orientation, or \textit{attitude}, of an object in a 3-dimensional Euclidian reference frame: in the following we will discuss the Tait-Bryan angles (which are very similar to, and sometimes confused with proper Euler angles), rotation matrices, and quaternions. All of them can be understood as a rotation of the initial reference frame $I = \{\I,\J,\K\}$ into the object's local reference frame $L = \{\i,\j,\k\}$. The reference frame $I$ is assumed to be fixed while $L$ is allowed to move. We can write each unit vector as follows: $\I = \fromto{}{I}[1,0,0]^T$, $\J = \fromto{}{I}[0,1,0]^T$, $\K = \fromto{}{I}[0,0,1]^T$, and $\i = \fromto{}{L}[1,0,0]^T$, $\j = \fromto{}{L}[0,1,0]^T$, $\k = \fromto{}{L}[0,0,1]^T$. $\{\I,\J,\K\}$ and $\{\i,\j,\k\}$ are  orthonormal bases to $I$ and $L$, respectively. The subscript before the vector indicates in which reference frame the vector is expressed, and the $T$ after the vector indicates the transpose operation. We will keep this formalism for all vectors and matrices in this work.

\section{Tait-Bryan/Euler angles}
\label{sec:Tait-Bryan}
The Tait-Bryan formalism [CITE] corresponds to a sequence of three angles, each corresponding to a rotation about one of the object's main axes: these are also called "intrinsic" rotations. They differ from "extrinsic" rotation, sometimes called "Euler angles", which correspond to a rotation about one of the axes of the global (fixed) reference frame. In the following, we will focus on using exclusively intrinsic rotations, as they are more intuitive. Note that sometimes people call this formalism "Euler angles" as well, so it is important to understand how this works.
With this formalism, we start in the global reference frame and rotate the reference frame three times to end up in the 
\textit{body} reference frame, which describes the final orientation of an object. We will most often choose a well-known sequence of rotation such as the $z-y'-x''$ order, which corresponds to the angles used to describe the heading, elevation and bank of an aircraft with respect to a reference frame attached to the Earth, for example the North-East-Down reference frame. The first rotation about $\k$ will transform $I$ into $L'$. The second rotation, about the $\j$ axis of the rotated frame $L'$, transforms $L'$ into $L''$. The third and last rotation, about the $\i$ axis of $L''$, will transform $L''$ into the final orientation, $L$, of the object (see Fig.~\ref{fig:3simpleRotate}).

This sequence of rotation can be used to represent the rotation matrix that describes the attitude of an image of the sky. Celestial coordinates are usually given in terms of right ascension, declination. To fully describe the image of a patch of sky, we need another degree of freedom, which is the roll of the image about the boresight. When given these three angles: RA, DEC, and ROLL, one can reconstruct the attitude using the Tait-Bryan angles in the $z-y'-x''$ order, where the first, second and third elementary rotations correspond to the rotations in right ascension, declination and roll, respectively.



\section{Rotation matrices}
\label{sec:rotationMatrices}
Perhaps the most common way to express the orientation of an object within a given reference frame is to use the matrix that describes the rotation from one reference frame to the other. Since rotations are linear transformations of $\RealNumbers^3$, there always exists a matrix to represent it. If we choose an orthonormal basis to $\RealNumbers^3$, matrices representing rotations are $3\times 3$ orthogonal matrices. When given the traditional matrix multiplication operation, $3\times 3$ orthogonal matrices with determinant of $+1$ form a group which is an isomorphism of the group of all 3-D rotations of Euclidian space (subsequently called SO(3) for "special orthogonal group") [REFERENCE?]: it means that each rotation can always be represented by exactly one $3\times 3$ orthogonal matrix. This theorem is the mathematical translation of the sometimes obvious intuition that rotation matrices always exist, are unique for a given rotation, and that the composition of two rotations is still a rotation. It also expresses the requirement that the corresponding rotation matrices have a determinant of $+1$, which can be useful when we consider numerical implementations of these matrices, as rounding errors might require a periodic normalization of the matrices to ensure they stay in this group. Note that the group of rotation is a cyclic group, since a rotation of an angle $\theta$ is the same as a rotation of $\theta+2\pi$.

We are interested in matrices describing rotations of entire coordinate systems, which are also called \textit{passive} rotations. This is different from matrices describing rotations of vectors within a given coordinate system (called \textit{active} rotations), and an important distinction that can often lead to confusion. Let's suppose that we have an initial coordinate system $I$ of basis $\{\I,\J,\K\}$, and a second coordinate system $L$ of basis $\{\i,\j,\k\}$. For example, this applies when $L$ is the body reference frame, and we want to understand its orientation with respect to an initial reference frame, such as the inertial reference frame. The basis vectors of $L$ can all be expressed by a linear combination of the basis vector of $I$. This transformation can be described using the \textit{direction cosine matrix}, which has the following expression:

\begin{equations}
\fromto{I}{L}\R = 
\begin{bmatrix}  \I\cdot\i & \J\cdot\i  & \K\cdot\i\\
                \I\cdot\j &\J\cdot\j  & \K\cdot\j \\
				\I\cdot\k & \J\cdot\k  & \K\cdot\k
\end{bmatrix}.
\end{equations}

The columns of this matrix correspond to the expression of the basis vectors of $I$ expressed in the basis of $L$. This is what we call the \textit{rotation matrix} between $I$ and $L$, and transforms vectors expressed in $I$ into their representation in $L$. With this convention, the matrix pre-multiplies the vector. For example, if we have some vector $\fromto{}{I}\vectors{u}$ expressed in the initial reference frame $I$, its expression in the reference frame $L$ will be $\fromto{}{L}\vectors{u} = \fromto{I}{L}\R \fromto{}{I}\vectors{u}$.

%Most often, a rotation matrix is defined as the matrix $\R$ that \textit{rotates} a vector from $L$ to $I$, and can be constructed by expressing the unit vectors $\i$, $\j$ and $\k$ of $L$ in the reference frame of $I$. These three vectors form the columns of the rotation matrix $\R$. So, if we construct the matrix $\R = [\fromto{}{I}\i,\fromto{}{I}\j,\fromto{}{I}\k]$, we notice that this matrix would multiply vectors expressed in the local reference frame $L$ and express them in the reference frame $I$. In the rest of this work we will always refer to $\fromto{L}{I}\R$ as this rotation matrix, where the subscript designates the starting reference frame (the reference frame in which the vectors that post-multiply the matrix will be expressed), and the superscript designates the reference frame in which the resulting vector will be expressed. 

Note that the rotation matrix $\fromto{I}{L}\R$ is an orthogonal matrix of determinant $+1$: each columns are orthogonal with each other and of unit norm. Hence, the inverse of this matrix is its transpose, which also corresponds to the rotation of a vector from frame $I$ to frame $L$: $(\fromto{I}{L}\R)^{-1} = (\fromto{I}{L}\R)^T = \fromto{L}{I}\R$.

%This convention is widespread in the literature, but can be source of confusion. A given vector can be thought of as being \textit{rotated} from one reference frame to another, as described above, which is also called an \textit{active transformation}; alternatively, one can think of the vector staying still, and the coordinate systems rotating, which is also called a \textit{passive transformation}. This is the representation that is the most useful for our purpose, and will lead to simplifications down the road, especially when this formalism is related to quaternions. 

%What is the matrix that corresponds to the \textit{passive} transformation from $I$ to $L$? In other words, what is the matrix that will pre-multiply vectors originally expressed in $I$ and express them in $L$? It is precisely the inverse (or transpose) of the matrix $\R = [\fromto{}{I}\i,\fromto{}{I}\j,\fromto{}{I}\k]$ described earlier, so that $\fromto{}{L}\vectors{u} = \fromto{I}{L}\R \fromto{}{I}\vectors{u}$. 

Let's take an example and consider the unit vector $\fromto{}{I}\vectors{u} = \fromto{}{I}(1, 0, 0)$, expressed in $I$ originally. Now, let's rotate the coordinate frame $I$ by an angle $\theta$ with respect to the axis $\k$. The new reference frame is $L' = \{\i',\j',\k'\}$. For simplification, let's consider that $\theta = +90$~degrees. It is clear that the vector $\i$ is now equal to $-\j'$, and $\fromto{}{L'}\i = \fromto{}{L'}(0,-1,0)$. 

\begin{figure}
\begin{center}
\tdplotsetmaincoords{0}{0}

%start tikz picture, and use the tdplot_main_coords style to implement the display 
%coordinate transformation provided by 3dplot
\begin{tikzpicture}[scale=4.5,tdplot_main_coords,label/.style={%
   postaction={ decorate,%transform shape,
   decoration={ markings, mark=at position .8 with \node #1;}}}]

\def\rvec{0.8}
\coordinate (O) at (0,0,0);
%draw the main coordinate system axes
\draw[line width=0.3mm,->] (0,0,0) -- (\rvec,0,0) node[below,align=left]{$\X$};
\draw[line width=0.3mm,->] (0,0,0) -- (0,\rvec,0) node[above left]{$\Y$};
\draw[line width=0.3mm] (0,0,0) -- (0,0,\rvec) node[anchor=north]{$\Z=\z'$};

\def\rotangle{30}
\tdplotsetrotatedcoords{0}{0}{\rotangle}
\draw[line width=0.3mm,->,tdplot_rotated_coords,color=blue] (0,0,0) -- (\rvec,0,0) node[below,align=left]{$\x'$};
\draw[line width=0.3mm,->,tdplot_rotated_coords,color=blue] (0,0,0) -- (0,\rvec,0) node[above left]{$\y'$};

\tdplotdrawarc[tdplot_main_coords,->]{(0,0,0)}{\rvec/3}{0}{\rotangle}{align=center,anchor=west}{$\theta$}


\end{tikzpicture}
\caption[Rotation of 30~degrees about $\Z$]{The $\{\x',\y',\z'\}$ reference frame (in blue) is rotated with respect to $\{\X,\Y,\Z\}$ (in black). The rotation is about the axis $\Z$ by an angle $\theta=30$~degrees.}
\label{fig:TopViewSimpleRotate}
\end{center}
\end{figure}

In the more general case, let's suppose that the local reference frame $L'$ is rotated by an angle $\theta$ about the $\k$ axis (Fig.~ \ref{fig:simpleRotate}) with respect to the reference frame $I$. The convention we adopt sets the rotation matrix for this transformation as being:
\begin{equations}
\fromto{I}{L'}\R = \R_\k(\theta) = 
\begin{bmatrix} \cos\theta & \sin\theta & 0 \\
				-\sin\theta & \cos\theta & 0 \\
                0 & 0 & 1
\end{bmatrix},
\end{equations}
where $\k$ indicates the third axis of the current basis ($\i$ and $\j$ represent the first and second axes, respectively). This will transform vectors from $I$ to $L'$. Suppose now that we further rotate our reference frame by an angle $\phi$ about the newly-rotated $\j'$ axis. The rotation for this elementary transformation is:
\begin{equations}
\fromto{L'}{L''}\R = \R_\j(\phi) = 
\begin{bmatrix} \cos\phi  & 0 & -\sin\phi\\
                0  & 1 & 0 \\
				\sin\phi  & 0 & \cos\phi
\end{bmatrix}.
\end{equations}
And let's do one last rotation about $\i''$, of an angle $\psi$, for which the transformation matrix is:

\begin{equations}
\fromto{L''}{L}\R = \R_\i(\psi) = 
\begin{bmatrix}  1& 0  & 0\\
                0 &\cos\psi  &  \sin\psi \\
				0 & -\sin\psi  &  \cos\psi
\end{bmatrix}.
\end{equations}

The matrix that corresponds to the active transformation of $I$ to $L$ will multiply vectors expressed in $I$ and express them in $L$. Hence, this matrix can be written:
\begin{equations}
\fromto{I}{L}\R = 
\fromto{L''}{L}\R\fromto{L'}{L''}\R\fromto{I}{L'}\R = \R_\i(\psi)\R_\j(\phi)\R_\k(\theta),
\end{equations}
where we pre-multiply the matrix for each consecutive rotation of reference frames. This corresponds to the "natural order" of rotations [CITE], and is especially relevant when related to quaternions. While the first axis of rotation, $\k$, is defined in the initial reference frame, it is important to realize that the axes corresponding to the second and third rotations are defined in the intermediate frames $L'$ and $L''$, respectively. We can understand this by thinking that the transformations follow the \textit{body}, as each rotation is done in the body reference frame, and is a particularly useful approach to our problem.


\def\rvec{0.8}
\begin{figure}
\begin{center}
\tdplotsetmaincoords{70}{110}

%start tikz picture, and use the tdplot_main_coords style to implement the display 
%coordinate transformation provided by 3dplot
\begin{tikzpicture}[scale=4.5,tdplot_main_coords,label/.style={%
   postaction={ decorate,%transform shape,
   decoration={ markings, mark=at position .8 with \node #1;}}}]


\coordinate (O) at (0,0,0);
%draw the main coordinate system axes
\draw[line width=0.3mm,->] (0,0,0) -- (\rvec,0,0) node[below,align=left]{$\X$};
\draw[line width=0.3mm,->] (0,0,0) -- (0,\rvec,0) node[above left]{$\Y$};
\draw[line width=0.3mm,->] (0,0,0) -- (0,0,\rvec) node[above] {$\K = \k'$} node[midway]{\AxisRotator[rotate=-90,->]};


\def\rotangle{-30}
\tdplotsetrotatedcoords{\rotangle}{0}{0}
\draw[line width=0.3mm,->,tdplot_rotated_coords,color=blue] (0,0,0) -- (\rvec,0,0) node[below,align=left]{$\x'$};
\draw[line width=0.3mm,->,tdplot_rotated_coords,color=blue] (0,0,0) -- (0,\rvec,0) node[above left]{$\y'$};

\tdplotdrawarc[tdplot_main_coords,->]{(0,0,0)}{\rvec/3}{0}{\rotangle}{align=center,anchor=north east}{$\theta$}


\end{tikzpicture}
\caption[Rotation of -30~degrees about $\Z$]{The $\{\x',\y',\z'\}$ reference frame (in blue) is rotated with respect to $\{\X,\Y,\Z\}$ (in black). The rotation is about the axis $\Z$ by an angle $\theta=-30$~degrees.}
\label{fig:simpleRotate}
\end{center}
\end{figure}
\begin{figure}
\begin{center}
\tdplotsetmaincoords{70}{110}

%start tikz picture, and use the tdplot_main_coords style to implement the display 
%coordinate transformation provided by 3dplot
\begin{tikzpicture}[scale=4.5,tdplot_main_coords,label/.style={%
   postaction={ decorate,%transform shape,
   decoration={ markings, mark=at position .8 with \node #1;}}}]


\coordinate (O) at (0,0,0);
%draw the main coordinate system axes
\draw[line width=0.3mm,->] (0,0,0) -- (\rvec,0,0) node[below,align=left]{$\X$};
\draw[line width=0.3mm,->] (0,0,0) -- (0,\rvec,0) node[above left]{$\Y$};
\draw[line width=0.3mm,->] (0,0,0) -- (0,0,\rvec) node[anchor=east]{$\Z=\z'$};

% first rotation about Z
\def\rotangleZ{-30}
\def\rotangleY{45}
\def\rotangleX{15}
\tdplotsetrotatedcoords{\rotangleZ}{0}{0}
\draw[line width=0.3mm,->,tdplot_rotated_coords,color=blue] (0,0,0) -- (\rvec,0,0) node[below,align=left]{$\x'$};
\draw[line width=0.3mm,->,tdplot_rotated_coords,color=blue] (0,0,0) -- (0,\rvec,0) node[below left]{$\y'=\y''$};

\tdplotdrawarc[tdplot_main_coords,->]{(0,0,0)}{\rvec/3}{0}{\rotangleZ}{align=center,anchor=north east}{$\theta$}
\tdplotsetrotatedthetaplanecoords{0}
\tdplotdrawarc[tdplot_rotated_coords,->]{(0,0,0)}{\rvec/2}{90}{90-\rotangleY}{align=center,anchor=east}{$\phi$}
%\draw[tdplot_rotated_coords,->] (\rvec/3,0,0) arc (0:90:\rvec/3);


% second rotation about y'
\tdplotsetrotatedcoords{\rotangleZ}{-\rotangleY}{0}
\draw[line width=0.3mm,->,tdplot_rotated_coords,color=red] (0,0,0) -- (\rvec,0,0) node[above,align=left]{$\x'' = \x$};
%\draw[line width=0.3mm,->,tdplot_rotated_coords,color=red] (0,0,0) -- (0,\rvec,0);
\draw[line width=0.3mm,->,tdplot_rotated_coords,color=red] (0,0,0) -- (0,0,\rvec) node[above right]{$\z''$};

\end{tikzpicture}
\caption[Two consecutive elementary rotations]{The $\{\x'',\y'',\z''\}$ reference frame (in red) is rotated with respect to $\{\x',\y',\z'\}$ (in blue). The rotation is about the axis $\y'$ by an angle $\phi=-45$~degrees.}
\label{fig:3simpleRotate}
\end{center}
\end{figure}
\input{Figures/3ConsecutiveRotate}


\section{Quaternions}
\label{sec:Quaternions}
Quaternions are a more modern way to describe the orientation of a reference frame with respect to another, and are today widely used to describe spacecraft orientation [QUOTE]. From a strictly mathematical point of view, quaternions form a normed algebra over the real numbers that is an extension of traditional complex numbers. The quaternion normed algebra has four dimensions, instead of just two for the complex numbers. At its fundamental level, the basis for the quaternion algebra consists of one real axis and three imaginary axes \{1, $\i$, $\j$, $\k$\}. Like complex numbers (which have a basis \{1, $\i$\}), there are fundamental relations between the basis elements that govern the multiplication operation, such as the well known identity $\i^2=-1$, that we will discuss at length later in this section. In this document, we will write a quaternion using one of the following equivalent notations (Coutsias 1999) and (Schmidt 2001) :
\begin{equations}
\quat{q} = q_r\times 1 + q_i\i + q_j\j + q_r\k = q_r + \vectors{q} =
\begin{bmatrix}
q_i\\
q_j\\
q_k\\
q_r
\end{bmatrix} = \begin{bmatrix}
\vectors{q}\\
q_r
\end{bmatrix} = \begin{bmatrix}  \vectors{q}^T & q_r\end{bmatrix}^T,
\end{equations}
where we make a clear distinction between the quaternion's real part $q_r$, and its 3-dimensional imaginary part that we choose to represent as a vector $\vectors{q} =q_i\i + q_j\j + q_r\k$. Like complex numbers, quaternion have a conjugate operation, which negates the imaginary part: $\quat{q}^* = \begin{bmatrix}  -\vectors{q}^T & q_r\end{bmatrix}^T$.

Quaternions are interesting beyond their pure mathematical definition because the subset of quaternions of unit norm can be used to represent a coordinate frame rotation in three dimensions. The Euler rotation theorem states that any coordinate frame rotation can be described by a rotation of an angle $\theta$ about an appropriately-chosen unit vector $\vectors{u} = x\i + y\j + z\k$ (also called the "Euler axis" or "Euler vector"). This formalism has 3 degrees of freedom, the minimum needed to describe a rotation between two reference frames: two degrees of freedom in the vector (which is constrained to be of unit norm), and one in the rotation angle. If we encode this information in a quaternion using Euler's exponential notation for vectors [need references here], this precisely defines the quaternion:
\begin{equations}
\quat{q} = \exp\left[\frac{\theta}{2}(x\i + y\j + z\k)\right] = \cos\frac{\theta}{2} + (x\i + y\j + z\k)\sin\frac{\theta}{2}.
\end{equations}
This quaternion completely describes the rotation between the two reference frames and has unit norm. Conversely, every quaternion of unit norm can be decomposed like this and represent a rotation in three-dimensional Euclidian space. Like rotation matrices, the unit quaternions form a group under the quaternion multiplication operation, which is isomorphic to the special unitary group SU(2) [reference]. It is known that SU(2) is a surjective 2:1 homomorphism of SO(3). This means that each element in SO(3) can be described by exactly two elements in SU(2), or equivalently, two distinct unit quaternions: the quaternion $\quat{q}$, and its opposite $-\quat{q}$.

Quaternions use 4 numbers to describe 3 degrees of freedom: an advantage over matrices (9 elements), but an apparent disadvantage over Tait-Bryan angles, which consist of an optimal number of 3 elements. However, Tait-Bryan angles can be shown to exhibit a phenomenon called \textit{gimbal lock} [references], which leads to a degeneracy when describing the set of angles corresponding to rotations when the pitch angle (second rotation angle, about $\j$) is $\pm\pi/2$. This creates situations where some rotations and sequences of rotation would have to be avoided by fear of creating numerical issues caused by gimbal lock [footnote: for an example of gimbal lock, refer to the Apollo program!]. Quaternions, while needing an extra number to represent the rotation, are free of this concern. This is one of the main reasons that they were originally preferred to Tait-Bryan angles early in the spaceflight era [WERTZ 1985?]. [talk more about the advantages of quaternions: number of multiplications, etc - see Shuster et al 1993]

\subsection{Quaternion multiplication}

In order to form the unit quaternion group, one has to define an appropriate multiplication operation. We warn that the formulation that we use and present in the next few paragraphs does not correspond to the commonly accepted rules for quaternion operations (also called "Hamilton notation", from W. R. Hamilton who is attributed the discovery of quaternions). We use a formalism that was popularized by Caley [CITE] and adopted in most of the aerospace community [cite JPL quaternion standard], mostly to describe the orientation of satellites in inertial space. Its main advantage is that consecutive transformations using quaternions consist of multiplying elementary quaternions in a "natural order", exactly in the same order as the rotation matrices mentioned at the end of the previous section. 

[Add more philosophical dwelling on the hows and whys of this notation; including references still on page]

To avoid confusion, we will not mention the original Hamilton rules in this work. Instead, we define the quaternion elementary multiplication rules as follows[CITE]:
\begin{equations}
\label{eq:QuaternionRules}
\begin{split}
 \i^2 = \j^2 = \k^2 &= -1;\\
 \j\i = -\i\j &= \k;\\
 \k\j = -\j\k &= \i;\\
 \i\k = -\k\i &= \j.
\end{split}
\end{equations}

Using the relations in Eq.~\ref{eq:QuaternionRules}, we define the general quaternion multiplication operator $\otimes$:
\begin{equations}
\begin{split}
\quat{p}\otimes\quat{q} & =  (p_r + p_i\i + p_j\j + p_r\k)\times(q_r + q_i\i + q_j\j + q_r\k) \\
& =  (p_rq_r - p_iq_i - p_jq_j- p_kq_k)\\
& \;\;\;\;\;\; +  (p_rq_i + p_iq_r - p_jq_k + p_kq_j)\i\\
& \;\;\;\;\;\; +  (p_rq_j + p_jq_r - p_kq_i + p_iq_k)\j\\
& \;\;\;\;\;\; +  (p_rq_k + p_kq_r - p_iq_j + p_jq_i)\k\\
\end{split}.
\end{equations}

To express a vector $\fromto{}{I}\vectors{v} = \fromto{}{I}(x,y,z)$ in the new frame $L$, we construct a purely imaginary quaternion from this vector: $\quat{q}_\vectors{v} = 0 + x\i + y\j + z\k$, and we use the quaternion multiplication to obtain:
\begin{equations}
\begin{bmatrix} \fromto{}{L}\vectors{v}\\ 0 \end{bmatrix}= \fromto{I}{L}\quat{q}\otimes\quat{q}_\vectors{v}\otimes\fromto{I}{L}\quat{q}^{-1},
\end{equations} 
and extract the vector $\fromto{}{L}\vectors{v}$ from the resulting quaternion.

Note that the quaternion inverse operation for quaternions of unit norm is the same as the conjugate operation. 

\subsection{Relationship with matrices and elementary quaternions}
Using this formalism, a quaternion is behaving in the same way as the corresponding \textit{passive} transformation matrix to describe a reference frame rotation. This means that consecutive rotations are multiplying in the "natural order", which makes it more intuitive.

For example, let's consider the elementary rotation described in Fig~\ref{fig:simpleRotate} that represents a rotation of the initial reference frame $I$ into a reference frame $L$ about $\k$. Using the "left-hand" rule, the angle $\theta$ of rotation about $\k$ is now $\theta = +30$~degrees. This quaternion is $\fromto{I}{L}\quat{q} = \quat{q}_\k(\theta) = \cos\frac{\theta}{2} + \sin\frac{\theta}{2}\k$, and represents the same rotation as the passive rotation matrix $\R_\k(\theta)$ discussed in Section \ref{subsec:rotationMatrices}. [WORK OUT THE MATHS FOR THE VECTOR TRANSFORMATION] If the rotation of the reference frame is described by three consecutive rotations of angles $\theta$, $\phi$ and $\psi$ about $\k$, $\j'$, and $\i''$, respectively [see some figure], we can write:
\begin{equations}
\begin{split}
\fromto{I}{L}\quat{q} & =  \quat{q}_{\i}(\psi)\quat{q}_{\j}(\phi)\quat{q}_\k(\theta) \\
& = \begin{bmatrix}
0\\
\sin\frac{\psi}{2}\\
0\\
\cos\frac{\psi}{2}
\end{bmatrix}\begin{bmatrix}
0\\
0\\
\sin\frac{\phi}{2}\\
\cos\frac{\phi}{2}
\end{bmatrix}\begin{bmatrix}
0\\
0\\
\sin\frac{\theta}{2}\\
\cos\frac{\theta}{2}
\end{bmatrix}\\
\end{split},
\end{equations}

which forms a quaternion that is equivalent to the rotation matrix multiplication $ \R_\i(\psi)\R_\j(\phi)\R_\k(\theta).$
Note that the order of the quaternions is the same as the order of the matrices. This is one of the advantages of choosing this "natural order" convention [cite Shuster].


\section{Quaternion derivative and integration}

Properly defining the derivative and integral of quaternions is necessary for our purpose. We will need a derivative to describe our dynamic system as its orientation changes over time; and we will need to integrate (or \textit{propagate}) those equations to find a numerical solution to the attitude estimation problem.

In the following, we consider the body reference frame $L(t)$ which evolves as a function of time with respect to a fixed, inertial reference frame $I$. 

The mathematical derivations leading to those results can be found elsewhere [CITE]. Over an infinitesimal time step $dt$, the local frame is rotating by an angular vector $\delta\boldsymbol{\theta}$. The instantaneous angular velocity, expressed in the body reference frame $\L(t)$, is $\fromto{}{L(t)}\gyroVec(t) = \lim_{dt\to 0}\frac{\delta\boldsymbol{\theta}}{\delta t}$. It can be shown [CITE TRAWNY] that with this formalism, the quaternion derivative is defined using either a quaternion multiplication, or an equivalent matrix multiplication:
\begin{equations}
\fromto{I}{L(t)}\dotAttitude(t) = \frac{1}{2}\begin{bmatrix} \gyroVec \\ 0 \end{bmatrix}\otimes \fromto{I}{L(t)}\Attitude =  \frac{1}{2}\matOmega(\gyroVec)\fromto{I}{L(t)}\Attitude,
\label{eq:quaternionDerivation}
\end{equations}

where the matrix
\begin{equations}
\matOmega(\gyroVec) = 
\begin{bmatrix}
0 & \omega_z & -\omega_y & \omega_x \\
-\omega_z & 0 & \omega_x & \omega_y \\
\omega_y & -\omega_x & 0 & \omega_z \\
-\omega_x & -\omega_y & -\omega_z & 0
\end{bmatrix}
\label{eq:Omega}
\end{equations}

is going to play an important in the later sections. 

The integrator formulas are derived in []. The problem is to find a matrix $\thetaMat$ to integrate a quaternion $\fromto{I}{L(t)}\Attitude(t)$, and estimate attitude at time $t+\Deltat$, knowing the instantaneous angular velocity $\gyroVec(t)$: 
\begin{equations}
\Attitude(t+\Deltat) = \thetaMat(t,t+\Deltat)\Attitude(t)
\end{equations}
A zeroth-order solution assumes that the angular velocity $\gyroVec$ is a constant over the timestep $\Deltat$, an important special case since it describes the typical discrete representation that we will use in our software. The solution can be expressed as:

\begin{equations}
\thetaMat(t,t+\Deltat) \equiv \thetaMat(\Delta t) = \exp\left(\frac{1}{2}\matOmega(\gyroVec)\Deltat\right),
\label{eq:quaternionIntegration}
\end{equations}
where the matrix exponential is defined using a Taylor expansion [give reference for that]. 


A first-order solution is given in [TRAWNY] and uses knowledge of two previous $\gyroVec$ values to estimate the integral.

\section{Covariance matrices in different reference frames}
In the following, we will be describing our attitude using quaternions or rotation matrices in a Kalman filter with a state-space representation. This means that we will be make estimates of physical quantities, as well as estimates of our estimation error. These errors are represented using covariance matrices. 

Covariance matrices contain information about the cross-correlation of the variables in the state vector. The diagonal elements represent the auto-covariance of a given variable, while the terms off the diagonal indicate the degree of covariance (or correlation) between the different variables. For example, we will have three gyroscopes which will be mounted orthogonally from each other, each measuring the angular velocity about three different axes. In the ideal case, all gyroscopes are independent and the covariance matrix associated to the set of three angular velocities is diagonal, with the variances of each gyroscope on the diagonal.

If we rotate the gyroscopes' frame with a rotation matrix $\R$, the new covariance matrix $\noiseCovMat$ needs to be rotated as well: $\noiseCovMat' = \R\noiseCovMat\R^T$. 

\subsection{Small angle approximation}

Quaternions become more intuitive in the small angle approximation. Indeed, when all angles are small with respect to $\pi$, we can write:
\begin{equations}
\fromto{I}{L}\Attitude \approx \begin{bmatrix}  \frac{1}{2}\delta\boldsymbol{\theta} \\ 1 \end{bmatrix},
\end{equations}
where $\frac{1}{2}\delta\boldsymbol{\theta} = \frac{1}{2}[\delta\theta_i,  \delta\theta_j, \delta\theta_k]^T$ corresponds to three small rotations about all the three axes of the initial reference frame. Because we are in the small angle approximation, the order of the rotations does not matter. Hence, if the imaginary part of a quaternion has small values $q_i, q_j, q_k$, and if $q_r\approx 1$, this quaternion represents a rotation of the reference frame by an angle $\delta\theta_i = 2q_i$, then by an angle $\delta\theta_j = 2q_j$, and finally by an angle $\delta\theta_k = 2q_k$, where all the angles are expressed in radians.

More simplifications can also be found. For example, in the limit where $\gyroVec \to \boldsymbol{0}$, the matrix exponential defined in Eq.~\ref{eq:quaternionIntegration} simplifies into:
\begin{equations}
\thetaMat(\Deltat) \stackrel{\gyroVec \to \boldsymbol{0}}{=} \I_{4\times 4} + \frac{\Deltat}{2}\matOmega(\gyroVec).
\label{eq:quaternionIntegrationSimple}
\end{equations}



%\subsection{Examples} (believe me, examples are very very important when people read your stuff – if only other people had put examples in their write-ups!)
%\subsection{Implementation in software}
%As part of this effort, we wrote quaternion libraries in C++, Python, and Labview. In all cases, we use double floating point representation for the quaternion and its associated operations, to ensure that we do not lose information due to numerical round-off errors. Indeed

\documentclass{standalone}

\begin{document}

\chapter{The PID control loop}
\label{ap:PID}

Before we elaborate on the control architecture of the entire system, let's first discuss the elementary controls block: the PID.

A Proportional-Integral-Derivative (PID) control loop is one of the most basic, yet most used method to build systems with active control. The problem that these systems try to solve is simply to make an object reach a desired state: a sensor is used to measure the current state, and the difference between the desired state and the current state is fed to an apparatus capable of changing the state. Most commonly, this uses motors and either position or velocity sensors, but it can also be used for example for temperature control in a cryogenic environment, where heaters are used to change the temperature. For simplicity, in the rest of this work, we will always consider a loop with sensors and actuators. 

In its most simple expression, the PID can be reduced to a simple proportional loop. That is, the command is proportional to the error between the desired and measured state. The value of this proportional coefficient usually sets the dynamics of the response, as a large proportional gain $\Kp$ will mean that even a small deviation from our desired state will trigger a large response. Sometimes, a purely proportional system can lack stability.

A proportional-derivative loop adds the information of the speed at which the error varies. If the error is growing quickly, we can increase our command. If the error is being reduced quickly, it is time to slow down the command to avoid overshooting our target. This uses the time derivative of the error that multiplies a gain, $\Kd$, and has the effect to damp the motion. A PD loop usually will help with the system's stability.

But even then, a proportional-derivative does not guarantee that you will reach your desired state. We then complete the PID loop with an integral gain $\Ki$, which multiplies the integral of the error over some length of time. While the $\Kp$ and $\Kd$ gains mostly control the dynamics of the response, the integral term will control the steady-state error and ensure it converges to zero. While useful, this term needs to be considered with precaution, as some situations can lead to a diverging response.

\begin{figure}[!ht]
	\centering
	\includestandalone{Figures/SimplePID}
	\caption{}
	\label{fig:SimplePID}
    \end{figure}



A simple PID loop diagram is shown in Fig~\ref{fig:SimplePID}, with the desired input state at the entrance of the loop and the real state at the output of the loop. It is often the case that the state cannot be directly measured: this require the use of an \textit{estimator} or \textit{observer}, in which various indirect measurements will feed a mathematical model of the system to estimate its parameters. The relevant example for us is a scenario where we only measure a velocity measurement, while we want to close the loop on the position. Simply put, we know that the position has an integral relationship with the velocity, and the observer's role is to estimate the integration constants.

The estimator is also used to realize \textit{sensor fusion}. This consists of combining various types of measurements to provide the best estimate of the state to feed back to the control loop. The various measurements often happen at different discrete rates, with different lag times, which can lead to rather complex implementations. One of the most well-known estimation algorithms is the Kalman filter, which we will discuss at length in Section []. 

For BETTII, each subsystem has its own PID control loop. Each PID loop structure consists of 7 variables: the $\Kp$, $\Kd$, $\Ki$ gains, and overall scaling factor, an upper and a lower limit on the command, and a boolean value that is used to reset the content of the integral term used to multiply $\Ki$. 

\end{document}


\chapter{FORCAST DATA}

\label{ap:data}

\begin{landscape}
\renewcommand{\arraystretch}{0.5}
\tiny
%\begin{center}
\begin{longtable}{lccccccccccc}
\caption[Fitted parameters]{Fitted parameters for the 84 sources in our 42 SOFIA fields.\label{tab:AllFits}}\\


\toprule																											
SOFIA Name	&	Coordinates	&	Type	&	R37	&	$\alpha$	&	$R$	&	\Menv			&	Calc. \Menv	&	\Ltot			&	$\Lbol$	&	i	&	$\Av$	\\
	&	(J2000)	&		&		&		&		&	(\si{\Msun})			&	(\si{\Msun})	&	(\si{\Lsun})			&	(\si{\Lsun})	&	(\si{\degree})	&	(mag)	\\
\midrule																											
\endfirsthead				
\caption*{\tablename{} (continued)}\\																						\toprule																											
SOFIA Name	&	Coordinates	&	Type	&	R37	&	$\alpha$	&	$R$	&	\Menv			&	Calc. \Menv	&	\Ltot			&	$\Lbol$	&	i	&	$\Av$	\\
	&	(J2000)	&		&		&		&		&	(\si{\Msun})			&	(\si{\Msun})	&	(\si{\Lsun})			&	(\si{\Lsun})	&	(\si{\degree})	&	(mag)	\\	
\midrule
\endhead
\midrule																											
\multicolumn{3}{r}{{Continued on next page}} \\																											
\midrule																											
\endfoot																											
\bottomrule																											
\endlastfoot																											
CepA.1	&	22h56m06.9s +62d04m34.0s	&	Isolated	&	1.16	&	-0.75	&	0.66	&	0.015	$\pm$	0.006	&	--	&	39.9	$\pm$	6.0	&	6.1	&	19	&	7	\\
CepA.2	&	22h56m03.0s +62d02m58.0s	&	Extended	&	1.71	&	1.86	&	0.79	&	0.114	$\pm$	0.027	&	--	&	163.2	$\pm$	21.3	&	25.0	&	19	&	14	\\
CepA.3	&	22h56m14.0s +62d02m17.3s	&	Isolated	&	1.42	&	1.89	&	0.29	&	0.171	$\pm$	0.075	&	--	&	17.5	$\pm$	3.6	&	3.8	&	19	&	14	\\
CepA.4	&	22h56m19.0s +62d01m54.4s	&	Extended	&	2.61	&	2.40	&	9.92	&	22.17	$\pm$	1.776	&	--	&	374.4	$\pm$	0.0	&	2004.0	&	0	&	14	\\
CepC.1	&	23h05m45.8s +62d30m21.4s	&	Isolated	&	0.88	&	1.38	&	1.06	&	1.95	$\pm$	0.944	&	--	&	5.6	$\pm$	1.1	&	5.5	&	0	&	11	\\
CepC.2	&	23h05m47.9s +62d30m38.6s	&	Isolated	&	0.00	&	1.67	&	0.79	&	4.38	$\pm$	0.869	&	--	&	134.4	$\pm$	44.5	&	0.7	&	27	&	14	\\
CepC.3	&	23h05m40.0s +62d29m16.2s	&	Isolated	&	1.04	&	-0.30	&	1.89	&	0.001	$\pm$	0.000	&	--	&	111.1	$\pm$	21.3	&	7.3	&	51	&	12	\\
CepC.4	&	23h05m49.7s +62d30m00.8s	&	Extended	&	1.54	&	0.19	&	2.07	&	0.01	$\pm$	0.003	&	--	&	11.9	$\pm$	22.3	&	15.3	&	0	&	8	\\
IRAS20050.1	&	20h07m06.6s +27d28m48.0s	&	Clustered	&	--	&	0.07	&	0.74	&	0.004	$\pm$	0	&	--	&	128.0	$\pm$	15.3	&	14.9	&	65	&	9	\\
IRAS20050.2	&	20h07m06.2s +27d28m49.1s	&	Clustered	&	2.28	&	1.64	&	0.76	&	0.577	$\pm$	0.217	&	--	&	26.6	$\pm$	6.0	&	8.0	&	19	&	14	\\
IRAS20050.3	&	20h07m06.3s +27d28m56.6s	&	Clustered	&	2.00	&	1.13	&	0.73	&	0.256	$\pm$	0.114	&	--	&	48.5	$\pm$	6.3	&	12.8	&	27	&	5	\\
IRAS20050.4	&	20h07m05.9s +27d28m59.2s	&	Clustered	&	2.09	&	1.70	&	0.24	&	0.577	$\pm$	0.217	&	19.174	&	48.5	$\pm$	8.7	&	14.9	&	47	&	5	\\
IRAS20050.5	&	20h07m06.6s +27d28m53.1s	&	Clustered	&	2.07	&	0.54	&	0.78	&	0.01	$\pm$	0.003	&	--	&	49.4	$\pm$	6.2	&	5.8	&	43	&	14	\\
IRAS20050.6	&	20h07m02.2s +27d30m26.0s	&	Isolated	&	1.40	&	-0.34	&	2.22	&	0.004	$\pm$	0	&	--	&	201.6	$\pm$	32.1	&	19.3	&	81	&	14	\\
IRAS20050.7	&	20h07m07.9s +27d27m15.8s	&	Isolated	&	--	&	1.23	&	1.42	&	0.015	$\pm$	0.040	&	3.196	&	3.5	$\pm$	3.2	&	3.0	&	19	&	14	\\
NGC1333.1	&	03h29m07.7s +31d21m57.0s	&	Isolated	&	0.75	&	0.29	&	3.40	&	0.004	$\pm$	0.005	&	0.972	&	32.5	$\pm$	7.8	&	8.4	&	51	&	14	\\
NGC1333.2	&	03h29m10.3s +31d21m55.5s	&	Extended	&	2.23	&	1.24	&	1.77	&	22.168	$\pm$	9.901	&	--	&	7.7	$\pm$	1.1	&	27.8	&	0	&	2	\\
NGC1333.3	&	03h29m01.5s +31d20m20.5s	&	Isolated	&	0.90	&	0.71	&	3.39	&	0.004	$\pm$	0.027	&	1.122	&	3.5	$\pm$	2.1	&	8.1	&	0	&	14	\\
NGC1333.4	&	03h29m11.1s +31d18m30.8s	&	Isolated	&	1.10	&	2.00	&	0.83	&	2.919	$\pm$	0.447	&	1.496	&	2.3	$\pm$	0.4	&	3.1	&	19	&	11	\\
NGC1333.5	&	03h29m10.6s +31d18m19.6s	&	Isolated	&	1.62	&	1.78	&	0.77	&	1.297	$\pm$	0.327	&	1.496	&	1.3	$\pm$	0.3	&	2.8	&	19	&	14	\\
NGC1333.6	&	03h29m13.0s +31d18m13.8s	&	Isolated	&	0.95	&	0.87	&	1.21	&	0.001	$\pm$	0.001	&	0.471	&	7.5	$\pm$	1.2	&	1.5	&	27	&	14	\\
NGC1333.7	&	03h28m43.4s +31d17m34.8s	&	Isolated	&	1.19	&	0.93	&	1.83	&	0.001	$\pm$	0.001	&	--	&	9.6	$\pm$	1.8	&	1.4	&	58	&	0	\\
NGC1333.8	&	03h29m03.7s +31d16m03.9s	&	Isolated	&	0.77	&	1.15	&	1.06	&	1.946	$\pm$	0.746	&	2.020	&	17.0	$\pm$	2.4	&	35.1	&	0	&	13	\\
NGC1333.9	&	03h28m55.6s +31d14m36.6s	&	Isolated	&	0.80	&	2.88	&	2.62	&	2.919	$\pm$	0.354	&	1.721	&	17.0	$\pm$	2.4	&	24.3	&	19	&	14	\\
NGC1333.10	&	03h28m57.4s +31d14m15.0s	&	Isolated	&	0.80	&	1.79	&	1.16	&	0.256	$\pm$	0.178	&	0.449	&	5.6	$\pm$	0.9	&	4.8	&	19	&	14	\\
NGC1333.11	&	03h28m37.1s +31d13m30.0s	&	Isolated	&	1.02	&	1.67	&	0.99	&	0.577	$\pm$	0.133	&	0.269	&	7.7	$\pm$	0.7	&	7.5	&	19	&	14	\\
NGC2071.1	&	05h47m04.8s +00d21m43.1s	&	Isolated	&	1.17	&	2.31	&	2.83	&	22.168	$\pm$	2.597	&	13.521	&	43.7	$\pm$	3.3	&	297.2	&	0	&	14	\\
NGC2071.2	&	05h47m04.7s +00d21m48.2s	&	Isolated	&	2.13	&	2.21	&	1.28	&	22.168	$\pm$	14.238	&	20.464	&	74.1	$\pm$	20.7	&	199.9	&	19	&	14	\\
NGC2071.3	&	05h47m05.4s +00d21m50.3s	&	Isolated	&	--	&	1.01	&	1.51	&	0.171	$\pm$	0.317	&	6.212	&	28.8	$\pm$	7.9	&	113.7	&	19	&	14	\\
NGC2071.4	&	05h47m04.0s +00d22m10.5s	&	Isolated	&	0.96	&	1.01	&	1.31	&	0.001	$\pm$	0.001	&	--	&	39.9	$\pm$	6.1	&	21.4	&	38	&	14	\\
NGC2071.5	&	05h47m10.7s +00d21m14.0s	&	Isolated	&	1.11	&	0.32	&	0.96	&	0.002	$\pm$	0.001	&	--	&	39.9	$\pm$	5.1	&	14.9	&	27	&	14	\\
NGC2264.1	&	06h41m04.5s +09d36m20.5s	&	Isolated	&	0.88	&	1.86	&	0.45	&	1.297	$\pm$	0.286	&	--	&	43.7	$\pm$	3.9	&	8.9	&	19	&	11	\\
NGC2264.2	&	06h40m59.1s +09d35m50.5s	&	Isolated	&	1.13	&	1.72	&	0.93	&	0.01	$\pm$	0.004	&	--	&	49.4	$\pm$	13.6	&	6.7	&	0	&	5	\\
NGC2264.3	&	06h41m06.5s +09d35m54.2s	&	Isolated	&	1.61	&	1.76	&	1.01	&	1.95	$\pm$	0.331	&	--	&	14.5	$\pm$	1.7	&	1.9	&	19	&	3	\\
NGC2264.4	&	06h41m09.9s +09d35m40.5s	&	Isolated	&	1.07	&	2.15	&	0.98	&	2.93	$\pm$	0.515	&	--	&	32.3	$\pm$	3.8	&	2.9	&	19	&	12	\\
NGC2264.5	&	06h41m06.7s +09d34m45.9s	&	Isolated	&	1.23	&	0.10	&	0.79	&	0.007	$\pm$	0.002	&	--	&	59.5	$\pm$	7.3	&	16.7	&	0	&	7	\\
NGC2264.6	&	06h41m11.9s +09d35m33.8s	&	Isolated	&	1.40	&	1.37	&	0.42	&	0.022	$\pm$	0.010	&	--	&	11.0	$\pm$	4.3	&	2.4	&	0	&	14	\\
NGC2264.7	&	06h41m05.7s +09d34m06.9s	&	Isolated	&	1.70	&	1.32	&	0.47	&	0.114	$\pm$	0.268	&	--	&	11.9	$\pm$	5.3	&	1.9	&	27	&	13	\\
NGC2264.8	&	06h41m06.1s +09d34m08.5s	&	Isolated	&	1.87	&	1.37	&	0.58	&	6.568	$\pm$	2.243	&	--	&	163.2	$\pm$	52.9	&	1.4	&	27	&	2	\\
NGC2264.9	&	06h41m05.8s +09d35m29.8s	&	Isolated	&	1.80	&	0.29	&	0.47	&	0.004	$\pm$	0	&	--	&	65.5	$\pm$	6.8	&	7.2	&	27	&	13	\\
NGC2264.10	&	06h41m08.6s +09d35m42.1s	&	Isolated	&	1.30	&	2.23	&	1.87	&	2.92	$\pm$	0.475	&	--	&	9.2	$\pm$	1.9	&	1.4	&	19	&	14	\\
NGC2264.11	&	06h41m11.3s +09d29m05.6s	&	Isolated	&	1.07	&	1.72	&	0.04	&	0.171	$\pm$	0.098	&	--	&	26.6	$\pm$	2.0	&	3.3	&	19	&	11	\\
NGC2264.12	&	06h40m59.1s +09d33m23.9s	&	Isolated	&	--	&	1.37	&	0.21	&	0.022	$\pm$	0.010	&	--	&	3.3	$\pm$	7.4	&	2.3	&	0	&	14	\\
NGC2264.13	&	06h41m08.9s +09d29m44.9s	&	Isolated	&	1.26	&	1.35	&	0.59	&	0.114	$\pm$	0.021	&	--	&	147.2	$\pm$	13.9	&	19.9	&	72	&	13	\\
NGC2264.14	&	06h41m10.2s +09d29m33.3s	&	Isolated	&	1.05	&	0.41	&	1.57	&	14.78	$\pm$	1.908	&	--	&	374.4	$\pm$	0.0	&	1856.1	&	0	&	0	\\
NGC2264.15	&	06h41m12.7s +09d29m04.5s	&	Isolated	&	1.03	&	0.72	&	1.77	&	0.015	$\pm$	0.004	&	--	&	374.4	$\pm$	31.3	&	86.3	&	27	&	13	\\
NGC2264.16	&	06h41m02.8s +09d36m14.7s	&	Extended	&	2.12	&	0.41	&	1.03	&	0.022	$\pm$	0.007	&	--	&	166.4	$\pm$	27.8	&	37.8	&	51	&	9	\\
NGC2264.17	&	06h41m06.8s +09d33m31.5s	&	Isolated	&	0.98	&	1.39	&	1.25	&	2.919	$\pm$	0.880	&	--	&	10.4	$\pm$	2.0	&	1.2	&	19	&	14	\\
NGC2264.18	&	06h41m04.3s +09d34m59.3s	&	Isolated	&	1.38	&	0.80	&	0.51	&	0.004	$\pm$	0.001	&	--	&	38.0	$\pm$	7.1	&	2.9	&	43	&	14	\\
NGC2264.19	&	06h41m01.8s +09d34m33.7s	&	Isolated	&	1.55	&	0.44	&	0.39	&	0.007	$\pm$	0.002	&	--	&	26.6	$\pm$	3.6	&	4.0	&	55	&	14	\\
NGC2264.20	&	06h41m06.3s +09d33m50.4s	&	Isolated	&	1.49	&	0.99	&	0.67	&	22.17	$\pm$	16.130	&	--	&	374.4	$\pm$	267.4	&	1.4	&	19	&	0	\\
NGC2264.21	&	06h41m09.3s +09d30m25.8s	&	Isolated	&	1.10	&	0.89	&	0.93	&	0.004	$\pm$	0.002	&	--	&	26.6	$\pm$	4.1	&	3.9	&	0	&	14	\\
NGC7129.1	&	21h43m06.4s +66d06m55.4s	&	Extended	&	--	&	0.28	&	1.54	&	0.05	$\pm$	0.014	&	--	&	249.6	$\pm$	31.1	&	126.3	&	38	&	0	\\
NGC7129.2	&	21h43m01.8s +66d07m08.7s	&	Extended	&	1.60	&	1.19	&	0.64	&	6.568	$\pm$	3.028	&	--	&	72.3	$\pm$	9.0	&	62.7	&	0	&	14	\\
NGC7129.3	&	21h42m59.7s +66d06m11.3s	&	Extended	&	2.57	&	2.20	&	0.38	&	0.384	$\pm$	0.153	&	--	&	331.2	$\pm$	28.1	&	60.4	&	19	&	14	\\
NGC7129.4	&	21h42m50.2s +66d06m36.1s	&	Extended	&	1.63	&	0.50	&	1.20	&	0.022	$\pm$	0.014	&	--	&	128.0	$\pm$	29.0	&	27.6	&	90	&	0	\\
NGC7129.5	&	21h43m06.9s +66d06m42.1s	&	Isolated	&	--	&	-0.13	&	1.32	&	0.007	$\pm$	0.003	&	--	&	201.6	$\pm$	28.6	&	12.4	&	68	&	14	\\
Oph.1	&	16h27m10.3s -24d19m12.9s	&	Isolated	&	0.92	&	0.28	&	0.67	&	0.01	$\pm$	0.002	&	0.040	&	7.9	$\pm$	1.3	&	3.6	&	78	&	3	\\
Oph.2	&	16h26m44.2s -24d34m48.2s	&	Isolated	&	0.93	&	0.82	&	1.97	&	0.004	$\pm$	0	&	0.051	&	8.0	$\pm$	1.2	&	1.2	&	75	&	14	\\
Oph.3	&	16h27m09.4s -24d37m18.3s	&	Isolated	&	0.99	&	0.54	&	1.54	&	0.004	$\pm$	0.002	&	0.040	&	85.0	$\pm$	19.7	&	13.4	&	0	&	14	\\
Oph.4	&	16h27m02.5s -24d37m27.6s	&	Extended	&	1.80	&	0.19	&	2.22	&	0.004	$\pm$	4E-04	&	0.066	&	14.3	$\pm$	2.7	&	4.5	&	38	&	14	\\
Oph.5	&	16h27m06.8s -24d38m15.4s	&	Isolated	&	1.31	&	0.35	&	1.36	&	0.001	$\pm$	0	&	0.032	&	4.3	$\pm$	0.5	&	0.5	&	81	&	14	\\
Oph.6	&	16h27m15.7s -24d38m45.8s	&	Isolated	&	1.29	&	2.36	&	0.90	&	0.001	$\pm$	0.001	&	0.020	&	26.6	$\pm$	6.6	&	0.8	&	90	&	13	\\
Oph.7	&	16h27m28.0s -24d39m33.8s	&	Isolated	&	0.97	&	1.35	&	1.39	&	0.015	$\pm$	0.002	&	0.026	&	26.6	$\pm$	3.5	&	6.5	&	72	&	14	\\
Oph.8	&	16h27m37.2s -24d30m34.8s	&	Isolated	&	1.02	&	0.55	&	1.13	&	0.007	$\pm$	0.002	&	0.026	&	17.7	$\pm$	3.4	&	5.0	&	78	&	12	\\
Oph.9	&	16h27m21.8s -24d29m53.7s	&	Isolated	&	--	&	0.49	&	2.08	&	0.001	$\pm$	0	&	0.009	&	11.8	$\pm$	1.2	&	1.0	&	81	&	14	\\
Oph.10	&	16h27m17.5s -24d28m55.0s	&	Isolated	&	1.26	&	0.45	&	1.31	&	0.002	$\pm$	0.001	&	0.004	&	2.9	$\pm$	0.3	&	0.6	&	81	&	14	\\
Oph.11	&	16h26m59.2s -24d35m00.2s	&	Extended	&	2.60	&	2.04	&	0.74	&	0.034	$\pm$	0.010	&	--	&	11.9	$\pm$	1.9	&	4.0	&	78	&	14	\\
Oph.12	&	16h26m34.0s -24d23m40.7s	&	Extended	&	2.97	&	3.37	&	1.05	&	0.076	$\pm$	0.025	&	--	&	39.9	$\pm$	5.0	&	10.3	&	87	&	14	\\
Oph.13	&	16h27m30.1s -24d27m43.3s	&	Isolated	&	--	&	-0.39	&	2.23	&	0.001	$\pm$	0	&	0.009	&	17.7	$\pm$	5.5	&	1.5	&	81	&	14	\\
Oph.14	&	16h27m28.4s -24d27m21.1s	&	Isolated	&	1.89	&	-0.15	&	1.00	&	0.001	$\pm$	0.001	&	0.021	&	4.3	$\pm$	0.6	&	1.0	&	81	&	14	\\
Oph.15	&	16h27m29.4s -24d39m16.6s	&	Isolated	&	1.25	&	0.02	&	1.12	&	0.004	$\pm$	0.001	&	0.019	&	3.3	$\pm$	0.4	&	0.6	&	27	&	14	\\
Oph.16	&	16h26m24.1s -24d24m48.3s	&	Isolated	&	1.80	&	-0.74	&	1.87	&	0.001	$\pm$	0	&	--	&	17.7	$\pm$	2.9	&	2.2	&	78	&	10	\\
Oph.17	&	16h26m23.6s -24d24m39.4s	&	Isolated	&	0.96	&	-0.10	&	1.21	&	0.001	$\pm$	0	&	--	&	5.3	$\pm$	0.6	&	1.3	&	81	&	14	\\
Oph.18	&	16h26m17.2s -24d23m45.1s	&	Isolated	&	1.18	&	0.56	&	1.15	&	0.003	$\pm$	0.002	&	0.036	&	2.8	$\pm$	1.0	&	0.3	&	81	&	14	\\
Oph.19	&	16h26m30.5s -24d22m59.9s	&	Isolated	&	2.51	&	0.43	&	0.84	&	0.001	$\pm$	0.001	&	0.009	&	5.3	$\pm$	1.0	&	1.2	&	72	&	14	\\
S140.1	&	22h19m32.7s +63d19m24.4s	&	Isolated	&	0.91	&	1.82	&	1.25	&	4.38	$\pm$	2.347	&	--	&	25.0	$\pm$	3.6	&	22.9	&	0	&	14	\\
S140.2	&	22h19m20.9s +63d18m28.8s	&	Isolated	&	--	&	1.20	&	0.11	&	0.015	$\pm$	0.038	&	--	&	8.0	$\pm$	9.4	&	6.6	&	0	&	14	\\
S140.3	&	22h19m19.8s +63d18m49.6s	&	Clustered	&	3.02	&	1.68	&	1.97	&	14.78	$\pm$	2.345	&	--	&	331.2	$\pm$	27.3	&	631.6	&	0	&	13	\\
S140.4	&	22h19m18.2s +63d19m03.9s	&	Clustered	&	2.24	&	3.56	&	3.77	&	14.78	$\pm$	1.946	&	--	&	331.2	$\pm$	24.9	&	455.8	&	0	&	13	\\
S140.5	&	22h19m18.1s +63d18m47.0s	&	Extended	&	1.13	&	1.48	&	7.28	&	22.17	$\pm$	1.639	&	--	&	374.4	$\pm$	0.0	&	4129.7	&	0	&	0	\\
S140.6	&	22h19m22.4s +63d18m04.5s	&	Isolated	&	1.04	&	1.50	&	0.15	&	0.076	$\pm$	0.040	&	--	&	10.4	$\pm$	3.1	&	2.0	&	19	&	14	\\
S140.7	&	22h19m14.7s +63d19m00.0s	&	Isolated	&	--	&	1.49	&	0.07	&	0.384	$\pm$	0.321	&	--	&	26.6	$\pm$	10.5	&	2.4	&	19	&	10	\\
S171.1	&	00h03m59.8s +68d35m05.8s	&	Isolated	&	1.03	&	0.45	&	2.56	&	0.004	$\pm$	0	&	--	&	201.6	$\pm$	35.8	&	23.5	&	78	&	14	\\
S171.2	&	00h04m02.0s +68d34m33.3s	&	Isolated	&	--	&	-0.77	&	0.60	&	0.001	$\pm$	4E-04	&	--	&	48.4	$\pm$	5.6	&	7.8	&	43	&	6	\\
\end{longtable}																																	
%\caption*{\textbf{Note:} The complete version of this table is made available electronically}			
\end{landscape}			
																						

%----------------------------------------------------------------------------------------
%	BIBLIOGRAPHY
%----------------------------------------------------------------------------------------

\printbibliography

%----------------------------------------------------------------------------------------

%\includepdf[pages=-]{CV.pdf}
\end{document}  
