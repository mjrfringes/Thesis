\chapter{Star Formation in Clustered environments with SOFIA FORCAST}

\label{chap:SOFIA}

\section{Introduction}
Most stars in the Galaxy form in cluster environments of sizes 2-4 pc, often containing more than 100 young stellar objects (YSOs), with typical separations of $<$0.05~pc between stars near their centers \citep{Porras:2003kxa, Allen:2007wqa, Gutermuth:2009gca}.
Previous studies have been effective in elucidating the young stellar content and distribution in clouds on large scales (parsec down to 0.05~pc) \citep{Evans-ARAA2012}, but young cluster cores, born in dense portions of molecular clouds, are more difficult to observe. They are obscured at optical through near-IR wavelengths. At mid-IR through far-IR wavelengths, the material surrounding YSOs and involved in the stellar birth process emits due to heating by the young stars, but the resolution to date has not been sufficient to isolate individual stars in the cores of most nearby young clusters.


%\begin{enumerate}
%\item Description of the sample and the goals
%\item Description of the data we used: 2MASS, Spitzer, Herschel; SOFIA
%\item Instrument characterization
%\item Explain the data reduction process; include comparison with Megeath for Spitzer photometry; justify aperture correction with "total cluster" measurements
%\item Data products: maps, SEDs; spectral index distribution, Lbol, Tbol
%\item SED fitting method \& grid description (e.g. color-color diagram?)
%\item Focus on IRAS20050 and NGC2071 + Ophiuchus? SED fitting; our method, our results; analysis of the results similar to furlan?
%\item 
%\end{enumerate}
%SOFIA has a \SI{2.7}{\meter} primary mirror which is a significant size improvement over \Spitzer. The instrument we have used, FORCAST, provides unprecedented high angular resolution of \ang{;;2}-\ang{;;3.5} in multiple continuum bands from \SI{5.5}{\micro\meter} to \SI{37.1}{\micro\meter}, which allows us to probe a relatively unexplored region of phase space. We discuss our SOFIA project in Chapter~[].



\Spitzer has tremendously advanced our understanding of star formation, by providing sensitive observations in continuum bands from \SI{3.6}{\um} to \SI{160}{\micro\meter}. In particular, the IRAC instrument (with 4 bands from 3.6 to 8.4 $\mu$m) and MIPS \SI{24}{\micro\meter} channel provided a robust way to determine the spectral index of YSOs, hence leading to dramatic improvement of understanding of the YSO population in molecular clouds \citep[e.g.,][]{Gutermuth:2009gca,Gutermuth:2011he}.

However, the most dense regions of clusters presented a challenge for the MIPS instrument, as the YSOs are too bright and/or in too close proximity, which led to saturation and confusion, as exhibited in Fig.~\ref{fig:NGC2071saturated}. In this figure, we show the same region seen by the IRAC \SI{3.6}{\um} band, the MIPS \SI{24}{\um} band, and the \Herschel PACS \SI{70}{\um}, from left to right. While the IRAC instrument can clearly distinguish multiple objects within the region, the MIPS image is completely saturated, while the PACS image is confused and cannot properly resolve the individual objects due to the lower resolution of the \Herschel telescope at \SI{70}{\um}. Note that these YSOs are much closer to each other than it is typical in clusters (\SI{0.01}{\pc} instead of a typical value of \SI{0.04}{\pc}), however this scale of projected separation is not unusual at the centers of clusters.

\begin{figure}[!h]
\begin{center}
\includegraphics[width=\textwidth]{Figures/NGC2071_saturated_mosaic.png}
\vspace{-0.5cm}
\caption[Saturation and confusion]{Saturation and confusion in NGC2071.}
\label{fig:NGC2071saturated}
\end{center}
\end{figure}

Future instruments like BETTII will be able to tackle the confusion problem at wavelengths from 30 to \SI{100}{\micro\meter}, and be complementary to \Herschel observations of star-forming regions. In the meantime SOFIA, the Stratospheric Observatory For Infrared Astronomy, can already start studying these dense regions, providing 2-\ang{;;3.5} resolution between 10 and \SI{37}{\micro\meter}, without the saturation problems present in the \Spitzer data. This corresponds to a factor of 2-3 improvement in angular resolution over \Spitzer at \SI{24}{\um}. 

\section{Sample description and scientific goals}


This chapter reports the results of a survey of nearby star-forming cluster cores with the SOFIA FORCAST instrument \citep{Herter:2012hv}. The clusters were selected from a list of dense young clusters within \SI{1}{\kilo\pc} of the Sun derived from works by \citet{Porras:2003kxa} and \citet{Gutermuth:2009gca}. From their lists we selected clusters that were: (1) north of -\SI{25}{\degree} declination so that they could be observed from a northern hemisphere SOFIA flight; (2) included membership of $>$50 YSOs; and (3) included bright 8-\SI{24}{\micro\meter} sources within the dense cores based on \Spitzer and/or WISE data. 

We observed in four FORCAST science continuum bands: 11.1, 19.7, 31.5 and \SI{37.1}{\micron}, which covered the wavelength range available to the instrument at the time of proposal (2012). This wavelength coverage is complementary to archival data from \Spitzer and \Herschel. Our selection of bright regions spread all across the sky is convenient for SOFIA, as our project could be observed as a gap-filler between the primary science flight legs of other projects.

The main objectives of the survey are to gather statistics on the YSO content of the \Spitzer saturated regions, and fill the SED gap between {\Spitzer}'s bands and {\Herschel}'s bands, when the latter are available. While most of our targets have valid \Spitzer IRAC data, often the data from the MIPS instrument is unavailable due to saturation or confusion. \Herschel photometry usually is not published in the literature for our sources, but maps of our regions are sometimes available so we can retrieve the far-infrared fluxes for some sources. For the targets without MIPS or \Herschel data, these SOFIA observations are the best information available between the longest IRAC band at $\SI{8}{\um}$ and the shortest submillimeter bands from ground-based telescopes. Thus our data provide important constraints the SED of very clustered YSOs in these regions to infer their physical properties.


The data analysis and scientific interpretation are presented in the next few sections. First, we describe our observations, as well as the archival datasets that we use to complement them. Second, we characterize the systematics of the FORCAST instrument and their variations over multiple science flights spanning multiple years. Third, the data reduction process is explained, followed by a snapshot of the data products themselves. Fourth, we discuss our SED fitting strategy, and fit the SEDs of some of our clusters to derive the physical properties of their embedded YSOs. Finally, we focus on the case of the young stellar cluster IRAS~20050+2720, and discuss of our FORCAST data helps us understand the physics of such embedded regions.  


\section{Observations}
\label{subsec:SOFIAObservations}

The FORCAST camera \citep{Herter:2012hv} has two separate $256\times 256$ pixel infrared arrays that can image multiple bands in the wavelength range from 5.5-\SI{37}{\um} with $\ang{;;0.768}\times\ang{;;0.768}$ pixels. The two arrays can observe simultaneously through a dichroic beam splitter that divides the wavelength range shortward and longward of \SI{26}{\um}. Alternatively, the long wavelength array can be used by itself with the dichroic removed from the light path, gaining a sensitivity factor of $\sim 2.5$. We observe the 11.1 and \SI{37.1}{\um} together (hereafter "mode 1") and the 19.7 and \SI{31.5}{\um} together (hereafter "mode 2") . We set the 1$\sigma$ sensitivity threshold of the observations such that a a \SI{1.5}{\Lsun} source with a moderately rising SED would be detected at all wavelengths. The integration times were scaled appropriately for the distance to the cluster (see Table~\ref{tab:DesiredSensitivities}). This is allows is to probe the same luminosity limit at all distances and obtain a consistent sample of YSOs. 

\renewcommand{\arraystretch}{1.5}
\begin{table}[!h]
\scriptsize
\caption{List of desired sensitivities for different distances}
\vspace{-0.5cm}
\begin{longtable}{c|cccc|c}
\toprule
Distance & \multicolumn{4}{c|}{1$\sigma$ minimum detectable flux (Jy)} &  Corresponding minimum\\
(pc) & \SI{11}{\um}& \SI{19}{\um}& \SI{31}{\um}& \SI{37}{\um}& \si{\Lsun} \\
\hline
   200.0& 0.1& 0.1& 0.32& 0.7&$\sim$0.5\\
   400.0& 0.1& 0.1& 0.32& 0.6&$\sim$1.5\\
   600.0& 0.05& 0.04& 0.18& 0.25&$\sim$1.5\\
   800.0& 0.02& 0.02& 0.1& 0.12&$\sim$1.5\\
1,000.00& 0.01& 0.01& 0.06& 0.1&$\sim$1.5\\

\bottomrule																																		\end{longtable} 
%\caption*{List of desired sensitivities for different distances.}
\label{tab:DesiredSensitivities}
\end{table}


For the most nearby clusters ($<$ 300 pc), the required observing time was so short that the overhead from the observatory was very costly. Hence, we put a lower threshold to the integration time of \SI{30}{\second}. Similarly, the sensitivity of the \SI{37}{\um} band is such that in order to be consistent with our sensitivity target, this band was heavily driving the observing time in mode 1. Hence, we observe in this mode as long as is required to meet the sensitivity target for the \SI{11}{\um} band, and obtained additional observations in the \SI{37}{\um} band with the dichroic removed (hereafter "mode 3"). This allowed us to request less total observing time while achieving our sensitivity goals. A summary of our sensitivities for various distances is shown in Table~\ref{tab:DesiredSensitivities}.

Several observing strategies are available to the FORCAST user to deal with background subtraction. The most robust techniques are very costly in terms of time overhead for the observatory, so we decided to request the cheapest observing mode: the Chop-Nod mode (C2N), combined with 9 ditherings for each field, which dramatically helps when co-adding images together. Most of our data was processed by the SOFIA automated pipeline that provided calibrated Level 2 images, except for the data from the first few flights, for which we received the help of FORCAST's Principal Investigator, Dr.~Joe~Adams, who processed the raw data through his own instrument pipeline.

%\begin{figure}[!h]
%\begin{center}
%\includegraphics[width=\textwidth]{Figures/SOFIA_bands.pdf}
%\label{fig:SOFIAbands}
%\caption[SOFIA bands]{SOFIA FORCAST bands.}
%\end{center}
%\end{figure}

The data were acquired over 10 SOFIA flights spanning multiple years, with the last batch dating from February 2015. The actual observing times for each band and cluster is shown in Table~\ref{tab:times}. In that table, we have estimated the time for the \SI{37}{\um} band using a composite formula that levels the observing time from mode 3 to that of mode 1, considering their respective sensitivities. We obtained about \SI{10}{\hour} total of on-sky data, and 10 out of our 12 original target clusters were observed.

\renewcommand{\arraystretch}{1.5}
\begin{table}[!h]
\scriptsize
\caption{List of targets}
\vspace{-0.5cm}
\begin{longtable}{cP{4cm}P{2cm}P{1cm}P{0.5cm}P{0.5cm}P{0.5cm}P{0.5cm}P{0.5cm}}
\toprule																																			
Cluster 	&	 Coordinates 	&	 SOFIA 	&	 $N_\textrm{Fields}$	&	$d $	&	$T_{11} $  	&	$T_{19}  $&	$T_{31}  $&	$T_{37}  $\\
	&	(J2000)	&	Flight IDs	&		&	(pc)	&	(s)	&	(s)	&	(s)	&	(s)	\\
\midrule																	
Cepheus A	&	 22h56m10s +62d03m26s 	&	 F132 F109 	&	2	&	730	&	206	&	234	&	235	&	490	\\
Cepheus C	&	 23h05m45s +62d30m05s 	&	 F132 	&	1	&	730	&	150	&	121	&	121	&	286	\\
IRAS20050	&	 20h07m05s +27d28m51s 	&	 F166 F131 	&	2	&	700	&	321	&	224	&	256	&	266	\\
NGC1333 	&	 03h29m00s +31d17m20s 	&	 F129 F193 F190 	&	9	&	240	&	530	&	558	&	467	&	446	\\
NGC2071 	&	 05h47m06s +00d21m45s 	&	 F192 	&	2	&	420	&	36	&	25	&	33	&	42	\\
NGC2264 	&	 06h41m07s +09d33m35s 	&	 F156 	&	4	&	913	&	495	&	300	&	331	&	587	\\
NGC7129 	&	 21h43m07s +66d06m42s 	&	 F109 	&	1	&	1000	&	383	&	214	&	214	&	709	\\
Ophiuchus 	&	 16h27m05s -24d30m29s 	&	 F157 	&	11	&	150	&	396	&	468	&	501	&	365	\\
S140 	&	 22h19m23s +63d18m44s 	&	 F129 	&	1	&	900	&	322	&	393	&	393	&	568	\\
S171 	&	 00h04m01s +68d34m50s 	&	 F132 	&	1	&	850	&	253	&	219	&	219	&	476\\	\bottomrule																																		\end{longtable} 
\caption*{\textbf{Notes}: For each cluster, we list the SOFIA flights on which the data was taken, the number of individual fields within the cluster, the distance, and the total integration time for each of the 4 observation bands, including all fields. The \SI{37}{\micro\meter} time quoted is a composite time calculated by combining the exposure time of mode 1 with that of mode 3, as discussed in the text.}
%List of our 12 proposed targets, with approximate RA and Dec, distance $d$ in parsecs, peak number density in \# stars/pc$^{2}$ from \citep{Gutermuth:2009p1325}, whether the image saturates in Spitzer/in WISE, the number of different fields for each target, the number of YSOs above our threshold level derived from WISE photometry, and the requested time in minutes on source that we request. Note that the latter DOES NOT include overheads.}
\label{tab:times}
\end{table}

To complement our SOFIA observations, we obtained publicly available \Spitzer, and \Herschel images. Most of our targets have already published \Spitzer IRAC and/or MIPS photometry \citep[mostly from][]{Gutermuth:2009gca,Megeath:2012cn,Evans:2009bka}, which we use in the relevant cases. In the cases where no IRAC photometry was available, we applied our own photometry algorithms to publicly available archival images. We could not find published photometry for the targets with available \Herschel images, hence we also used our own photometry pipeline to derive fluxes from archival images. In some cases, we found published submillimeter continuum measurements to help constrain the long-wavelength behavior of the SEDs.


\section{SOFIA FORCAST characterization}

In addition to the science images, a number of calibrators were observed during each flight for different dichroic settings and wavelength bands. These calibrators are usually bright stars which are point sources for SOFIA's angular resolution, and have known mid-IR fluxes, so they can be used both for flux and PSF calibration. We use them for two purposes: the first is to obtain a robust metric to determine whether sources are extended or not; the second is to determine the aperture correction factor which will be used for aperture photometry of science sources. 

\subsection{PSF size}
The size of the PSF can be defined in multiple ways. We adopt the approach of characterizing the PSF using its encircled energy distribution. Fig~\ref{fig:averageEE} shows the average of the normalized encircled energy distribution of the PSF, measured on all the calibrators observed during our flights which use each filter settings. Each curve represents one of the five different combinations of bandpass filter and dichroic setting that we use for our observations. For each radius, the total energy is the sum of the pixels within the circular aperture of that radius, to which we subtract an estimate of the background in an annulus around the source (see Section~\ref{sec:sourceFluxExtraction} for details on the background subtraction methods). 

\begin{figure}[!h]
\begin{center}
\includegraphics[width=\textwidth]{Figures/average.png}
\vspace{-0.5cm}
\caption[PSF size]{Average PSF encircled energy distribution profile for all calibrator observations.}
\label{fig:averageEE}
\end{center}
\end{figure}


As expected, the PSF at \SI{37.1}{\micro\meter} is larger than the PSFs at shorter wavelengths, but by less than the traditional diffraction limit rule. This indicates that additional PSF smearing is occurring at short wavelengths, likely due to telescope jitter and pointing errors, which is consistent with what other authors have found \citep[e.g.][]{Herter:2013by}. Throughout all the flights, point source calibrators have the same encircled energy distribution shape within $\sim 4\%$ rms. 

To look at the behavior of the PSF in more detail, we can use the full width at half maximum of the encircled energy distribution, \Rfifty, as a proxy for PSF size. The variation of this quantity for the various flights, bandpass/dichroic setting, and calibrators used is showed in Fig.~\ref{fig:Rfiftydist}. This shows the flight-to-flight differences and, for some calibrators, the in-flight variability. We find that the latter is usually small, except for the SOFIA flight on 05-02-2014, for which the spread is quite considerable and could have been caused by instrumental malfunction or abnormal levels of water vapor in the atmosphere. The variation from flight to flight is larger than the variation within a given flight, which indicates variability in the observing conditions, systematics, or thermal radiation environment of the observatory between different flights. Even considering the flight-to-flight and calibrator-to-calibrator variations, the overall spread in \Rfifty for a given observation setting is almost always less then 10\%, making this metric a useful reference to compare with scientific data. In our analysis we will compute \Rfifty for our sources and compare it to the \Rfifty from the current flight for the same filter setting, if the calibration file exists. If no calibration observation exists for a given setting, we use the mean \Rfifty for that setting from calibrators observations in other flights. The ratio $\beta_{37}=\Rfifty_\textrm{source}/\Rfifty_\textrm{cal}$ helps quantify the extension of the source, to within $\sim 10\%$ confidence level. 

\begin{figure}[!h]
\begin{center}
\includegraphics[width=\textwidth]{Figures/R50.png}
\vspace{-0.5cm}
\caption[PSF size of calibrators]{Distribution of the \Rfifty for all calibrators observations within each bandpass. Lower wavelengths have lower \Rfifty. In red: \SI{11}{\um} band, with dichroic; in green: \SI{19}{\um} band, with dichroic; in blue: \SI{31}{\um} band, with dichroic; in yellow: \SI{37}{\um} band, with dichroic; in purple: \SI{37}{\um} band, no dichroic. Down triangles: $\alpha$ Boo; Pentagons: $\alpha$ Cet; Diamonds: $\alpha$ Tau;  Up triangles: $\beta$ And; Hexagons: $\beta$ Peg; Circles: $\beta$ UMi.}
\label{fig:Rfiftydist}
\end{center}
\end{figure}

\subsection{Aperture correction factor}
\label{subsec:apcorr}
In Fig.~\ref{fig:averageEE}, we observe that the encircled energy does not vary much by an aperture with radius of 12 pixels, so we consider this fiducial aperture as our "total flux" aperture. The goal of aperture photometry is to estimate the amount of flux in this large aperture, which we consider to be the total amount of flux from the source, by only measuring flux within a much smaller aperture. This has the advantage of reducing contamination from other sources, and increases the signal-to-noise ratio of the flux estimate since the pixels near the tail of the PSF usually contain more noise than signal. 
In Fig~\ref{fig:apercorr}, we plot the aperture correction factor that we compute from the ratio of the flux measured within an aperture of 3 pixels radius and this 12-pixel radius aperture.  Not surprisingly, this graph follows very closely the plot of $\Rfifty$ from Fig~\ref{fig:Rfiftydist}, showing the close link between the aperture correction factor and the shape of the calibrator's PSF. We match each observation in our data to the mean of the aperture correction factors for the same observation setting and flight.

\begin{figure}[!h]
\begin{center}
\includegraphics[width=\textwidth]{Figures/Aper_corr.png}
\vspace{-0.5cm}
\caption[aperture correction]{Instrumental response and aperture correction. The color code and marker shape is the same as in Fig.~\ref{fig:Rfiftydist}. Lower wavelengths usually have smaller aperture correction.}
\label{fig:apercorr}
\end{center}
\end{figure}

\subsection{Instrument response and overall uncertainty}
To validate our approach, we take a look at the calibrator fluxes after normalization by the calibration factor, which is provided directly by the FORCAST pipeline. This calibration factors converts the pixel digital value a physical flux density unit, and presumably is determined using the flux from calibrator stars as well. Here we re-measure the flux from each calibrator for each observation setting and each flight, using our standard aperture photometry method and background subtraction. Ideally, we would always obtain the same flux for each setting and calibrator, independently of the flight, an assertion we find true to within $\sim 5\%$ r.m.s (Fig~\ref{fig:response}). The in-flight errors are typically lower than this. This validates our aperture photometry method, and we can trust that the instrument's systematics are well-behaved to within these levels. 

This would suggest that we can adopt systematic $1\sigma$ uncertainties of $\sim 5\%$, a value which is consistent with the published uncertainties of $3\sigma \approx 20\%$ \citep{DeBuizer:2012ie}. In an effort to be conservative, we chose to follow those authors and adopt a $1\sigma\approx 7\%$ systematic measurement uncertainty.


\begin{figure}[!h]
\begin{center}
\includegraphics[width=\textwidth]{Figures/Phot_val.png}
\vspace{-0.5cm}
\caption[Instrumental response]{Instrumental response, showing decreased calibrator fluxes with longer wavelengths, which is expected since all calibration targets are evolved stars. The color code and marker shape is the same as in Fig.~\ref{fig:Rfiftydist}. In this plot, the variation across multiple flights for a given marker type of a given color is usually less than $\sim$5\%. Note that for the bottom green triangles ($\alpha$ Boo at \SI{19}{\um}), there seems to be a systematic change between flights occurring before and after 2013-09-19.}
\label{fig:response}
\end{center}
\end{figure}


\section{Data reduction and photometry}

The data were processed through various versions of the online pipeline to yield Level 2 data products available on the archive \citep{Herter:2013by}. We apply our own reduction procedure and photometry pipeline on those products to derive final images, source positions, fluxes and sensitivities. Our software makes extensive use of the Python \textit{astropy} package \citep{2013A&A...558A..33A} and its associated modules \textit{photutils} and \textit{APLpy}. 

\subsection{Pre-treatment}
Some manual treatment of each image was necessary before it could be analyzed by our software. We followed this procedure: a) visually align the WCS coordinate system, often 10-20" off, using point sources and archival data from other wavelengths and facilities such as IRAC \SI{8}{\micro\meter}; b) crop the images to clean off the nodded fields, and c) identify the coordinates of each source, both point-like and extended.

After these manual steps, the Level 2 images are multiplied by the calibration factor provided by the online pipeline, which converts them to Jy/pixel. We do not proceed to any systematic color correction, but the effects on the fluxes are very small \citep{Herter:2013by}.
%\begin{enumerate}
%\item Adjust WCS coordinates: use images at other wavelengths (2MASS, IRAC, MIPS, WISE) to re-align the (RA, DEC) position of the field. We estimate that this process is good to within one SOFIA pixel (\ang{;;0.768}) for the fields where one or more point sources can be identified. Extended fields are less trustworthy, since matching the extended emission to other wavelengths is harder. The rotation of the field produced by the SOFIA pipeline is correct for all of our data. 
%\item Crop each image, remove chopped fields, remove artifacts.
%\item Identify and categorize sources: isolated point sources, clustered point sources, and extended sources. For extended sources, a circular or elliptical aperture is used to try to encompass the entirety of the emission.
%\item Manually identify a location in the field that corresponds to a representative background.
%\end{enumerate}

\subsection{Source flux extraction}
\label{sec:sourceFluxExtraction}

We fed the adjusted files to our photometry pipeline. For each identified source, we determine its flux in all bands using aperture photometry with local background subtraction. The aperture correction factor we used is the one determined from the calibrators observed for the same observation setting during the same flight as the one when the data was taken. If a calibrator is not available during the flight, we use the average aperture correction factor taken over 9 of our 10 flights (we choose to exclude the flight on 05/02/2014 which seems to have abnormal behavior).

We distinguish between 3 types of sources after manual identification: \textit{isolated}, which are point sources with no nearby objects; \textit{clustered}, which are point sources with nearby objects; and \textit{extended}, which are not consistent with being point sources based on visual inspection. 

For point sources that are isolated, we use our standard aperture of 3 pixels at all wavelengths. We consider an annulus surrounding the source extending from 12 to 20 pixels radius (24 to 40 pixels for clustered sources): the local background is determined from the mode of the pixels in the annulus, while the sensitivity is calculated by measuring the standard deviation of the flux values within 3-pixel apertures spread over that annulus \citep{Shimizu:2016if}. We apply the aperture correction derived from the calibrator observations taken during that flight.

For extended sources, an elliptical aperture is determined manually from the \SI{37}{\micro\meter} images. The local background is determined from the mode of an elliptical annulus, with an inner boundary at the elliptical aperture and an outer boundary corresponding to an ellipse 20\% larger. The sensitivity quoted is the point source sensitivity, and is determined following the same method as for point sources, using the standard deviation of apertures spread across the elliptical annulus. 

The photometry from sources that were observed in different flights is then combined to increase the signal-to-noise ratio. This combination takes into account the sensitivity of each source by appropriately weighing each image.

The noise level calculated for the observation is added in quadrature to the systematic uncertainty of the instrument, for which we follow the recommendation from \citep{Herter:2012hv} and adopt a 7\%, $1\sigma$ uncertainty. 

\renewcommand{\arraystretch}{1.5}
\begin{table}[!h]
\scriptsize
\caption{SOFIA photometry comparison} \label{tab:SOFIAPhotometryHarvey}
\vspace{-0.5cm}
\begin{longtable}{l|P{1cm}P{1cm}|P{1cm}|P{1cm}P{1cm}|P{1cm}P{1cm}}
\toprule															
SOFIA name	&	F11	&	F11L	&	F19	&	F31	&	F31L	&	F37	&	F37L	\\
	&	Jy	&	Jy	&	Jy	&	Jy	&	Jy	&	Jy	&	Jy	\\
\midrule															
S140.3	&	10.28	&	9.70	&	101.49	&	419.41	&	401.00	&	525.90	&	669.00	\\
S140.4	&	3.80	&	4.00	&	88.95	&	337.22	&	368.00	&	352.07	&	485.00	\\
S140.5	&	110.57	&	110.00	&	830.97	&	2065.13	&	1585.00	&	2278.61	&	2176.00	\\
\midrule															
Sum of sources in cluster	&	124.65	&	123.70	&	1021.40	&	2821.76	&	2354.00	&	3156.58	&	3330.00	\\
Total cluster emission	&	135.20	&	145.00	&	1194.57	&	4449.46	&	3780.00	&	5840.64	&	6730.00	\\
Ratio	&	1.08	&	1.17	&	1.17	&	1.58	&	1.61	&	1.85	&	2.02	\\
\bottomrule					
	\end{longtable} 
\caption*{Comparison of SOFIA four-band photometry from \citet{Harvey:2012kw} on S140 (columns with 'L'). All fluxes are in Janskies. The authors' "total emission" actually represents the total emission in the entire field of view, whereas out measurement corresponds to a manually-selected source region encompassing only the dense core. The total emission in the entire field of view is less representative, as it could include contribution from other sources as well as areas of negative flux from the chopping and nodding steps. In this cluster, there is a large amount of emission which is not clearly associated to the three identified sources.}

\end{table}


To validate our flux extraction method, we compare our results with data from \citet{Harvey:2012kw} who observed one of the sources in our sample, S140. Their photometry (shown in their Table 1) of IRS 1, 2 and 3 (corresponding to our targets S140.5, S140.4, and S140.3, respectively) is compared to our photometry in Table~\ref{tab:SOFIAPhotometryHarvey}. We find reasonable agreement between our fluxes and theirs, although some differences are larger than expected in the longer wavelength bands. We attribute this to differences in the exact centroid location of the sources, which could be due to a different aperture size. Centroid errors have more impact at longer wavelengths, where the flux is larger and the PSF wings more extended.

\subsection{Image sensitivity}

\renewcommand{\arraystretch}{1.5}
\begin{table}[!h]
\scriptsize
\caption{FORCAST Sensitivities}
\vspace{-0.5cm}
\begin{longtable}{c|P{0.5cm}P{0.5cm}P{0.5cm}|P{0.5cm}P{0.5cm}P{0.5cm}|P{0.5cm}P{0.5cm}P{0.5cm}|P{0.5cm}P{0.5cm}P{0.5cm}|P{1cm}}
\toprule																			
Cluster 	&	\multicolumn{3}{c|}{F11}					&	\multicolumn{3}{c|}{F19}					&	\multicolumn{3}{c|}{F31}					&	\multicolumn{3}{c|}{F37}					&	 Sources 	\\
	&	$\sigman$ 	&	$\sigstd$ 	&	$\sigth$	&	$\sigman$ 	&	$\sigstd$ 	&	$\sigth$	&	$\sigman$ 	&	$\sigstd$ 	&	$\sigth$	&	$\sigman$ 	&	$\sigstd$ 	&	$\sigth$	&		\\
\midrule																											
CepA 	&	0.07	&	0.04	&	0.05	&	0.11	&	0.05	&	0.05	&	0.19	&	0.07	&	0.16	&	0.26	&	0.09	&	0.34	&	4	\\
CepC 	&	0.03	&	0.03	&	0.04	&	0.10	&	0.05	&	0.04	&	0.19	&	0.06	&	0.16	&	0.16	&	0.09	&	0.30	&	4	\\
IRAS20050 	&	0.04	&	0.03	&	0.04	&	0.08	&	0.04	&	0.05	&	0.13	&	0.05	&	0.16	&	0.30	&	0.11	&	0.32	&	7	\\
NGC1333 	&	0.12	&	0.04	&	0.07	&	0.07	&	0.07	&	0.07	&	0.22	&	0.08	&	0.25	&	0.48	&	0.13	&	0.52	&	11	\\
NGC2071 	&	0.19	&	0.10	&	0.12	&	0.32	&	0.15	&	0.15	&	0.21	&	0.22	&	0.49	&	0.45	&	0.28	&	0.81	&	6	\\
NGC2264 	&	0.07	&	0.03	&	0.05	&	0.19	&	0.05	&	0.06	&	0.28	&	0.07	&	0.20	&	0.21	&	0.09	&	0.43	&	21	\\
NGC7129 	&	0.07	&	0.03	&	0.03	&	0.10	&	0.04	&	0.03	&	0.26	&	0.09	&	0.12	&	0.17	&	0.08	&	0.19	&	5	\\
Ophiuchus 	&	0.11	&	0.05	&	0.08	&	0.16	&	0.07	&	0.08	&	0.31	&	0.09	&	0.27	&	0.41	&	0.18	&	0.65	&	19	\\
S140 	&	0.04	&	0.03	&	0.03	&	0.16	&	0.03	&	0.03	&	0.21	&	0.07	&	0.09	&	0.35	&	0.11	&	0.21	&	7	\\
S171 	&	0.04	&	0.03	&	0.03	&	0.07	&	0.04	&	0.03	&	0.07	&	0.05	&	0.12	&	0.16	&	0.06	&	0.23	&	2	\\
\bottomrule					
	\end{longtable} 
\caption*{\textbf{Notes}: For each band F11, F19, F31 and F37, we measure the 1$\sigma$ sensitivity \sigman and \sigstd in each field from the data using two different methods (see text), and present here the median of all fields. The theoretical sensitivity \sigth corresponds to the expected sensitivity for the actual integration time, using the SOFIA FORCAST observation planning tools and assuming moderate water vapor content. All sensitivity values are in Janskies.}
%List of our 12 proposed targets, with approximate RA and Dec, distance $d$ in parsecs, peak number density in \# stars/pc$^{2}$ from \citep{Gutermuth:2009p1325}, whether the image saturates in Spitzer/in WISE, the number of different fields for each target, the number of YSOs above our threshold level derived from WISE photometry, and the requested time in minutes on source that we request. Note that the latter DOES NOT include overheads.}
\label{tab:SOFIASensitivity}
\end{table}

In order to determine the absolute sensitivity in the image, we use two methods. First, we manually determine a region near each cluster that visually appears devoid of sources. We calculate the sensitivity as if this background region was a source, by patching apertures in an annulus around this background location and calculating the standard deviation of the obtained fluxes. We call this sensitivity measurement $\sigman$. The main downside of this method is that it requires a manual operation to select the appropriate background field, and hence could have more variation depending on which field we select. Second, we use a routine that iteratively isolates the pixel values above $2\sigma$ of the image, in order to remove the contamination from our actual sources. The standard deviation of the resulting image is then calculated, and is multiplied by the square root of the number of pixels in an aperture of 3 pixel radius. This corresponds to a floor sensitivity $\sigstd$. We present our results in Table~\ref{tab:SOFIASensitivity}, where we also compare this sensitivity with the expected sensitivity $\sigth$ obtained using the online calculator with the actual exposure time of our images. We note that usually, the theoretical values are more in agreement with our first method for F31 and F37, while more in agreement with our second method for F11 and F19. 



\subsection{Other photometry}

SOFIA provides mid-IR photometry. We looked in the literature for published fluxes on our targets in order to reconstruct more complete SEDs. In addition to our four SOFIA bands, We collected data from 2MASS, \Spitzer, and other instruments. Photometry from these sources is published in online catalogs, which we programmatically cross-reference with the positions of our targets. The closest target that corresponds to a Vizier location query is selected to be the correct catalog match. For the 2MASS data, the location of the target was required to be less than \ang{;;2} away from our coordinates for point sources, and \ang{;;5} for extended sources. For the \Spitzer data, the matching radius is \ang{;;3} for point sources and \ang{;;10} for extended sources. In addition to automated online catalog searches, we add values for sources in NGC2071 from \citet{vanKempen:2012fb}.

For our two most clustered regions, the cores of NGC~2071 and IRAS~20050+2720, the published catalogs do not have all available fluxes. We assume that the sources are so clustered that the source extraction software from the authors do not register them as point sources, due to confusion or saturation effects. Hence we adapt our own photometry routines for these clustered environments and obtain the fluxes directly from the calibrated Level 3 images, which are all available on the archive. In Table~\ref{tab:SpitzerPhotometry}, we compare our photometry results with published fluxes from \citet{Megeath:2012cn} and \citet{Gutermuth:2009gca} for isolated sources elsewhere in these same fields of view. We use the \Spitzer handbook recommendations for aperture photometry on \Spitzer archival images (\ang{;;2.4} aperture with and an annulus that extends from 12 to \ang{;;20}). We find that our results are within 10\% of the other authors' results for isolated sources, which can reflect a simple difference in exact aperture centroiding position. 

\renewcommand{\arraystretch}{1.5}
\begin{table}[!h]
\scriptsize
\caption{Spitzer photometry comparison}
\vspace{-0.5cm}
\begin{longtable}{l|P{1cm}P{1cm}P{1cm}P{1cm}}
\toprule																			
SOFIA name	&	i1	&	i2	&	i3	&	i4	\\
	&	Jy	&	Jy	&	Jy	&	Jy	\\
\midrule									
NGC2071.1	&	0.060	&	0.056	&	0.004	&	-0.021	\\
NGC2071.3	&	0.018	&	-0.010	&	-0.004	&	-0.047	\\
NGC2071.4	&	0.090	&	-0.054	&	0.036	&	-0.066	\\
NGC2071.5	&	-0.130	&	-0.109	&	-0.144	&	-0.139	\\
\midrule									
IRAS20050.1	&	0.020	&	0.039	&	0.017	&	0.131	\\
IRAS20050.3	&	0.181	&	0.122	&	0.082	&	0.121	\\
IRAS20050.6	&	-0.044	&	-0.046	&	-0.092	&	-0.056	\\
\bottomrule					
	\end{longtable} 
\caption*{\textbf{Note:} Fractional difference between our own aperture photometry on \Spitzer archival images and published \Spitzer photometry from \citet{Megeath:2012cn} for NGC2071, and \citet{Gutermuth:2009gca} for IRAS20050+2720. When values are negative, it means that their photometry is lower than ours.}
\label{tab:SpitzerPhotometry}
\end{table}

In some cases, we also found archival Herschel images, although no published photometry was available for most our sources. We then apply our same aperture photometry routines for those calibrated Herschel images, using aperture and background subtraction parameters from \citet{Shimizu:2016if} for the PACS and SPIRE. We find also very good agreement between our photometry results for the PACS~\SI{70}{\um} band and the published \Spitzer MIPS~\SI{70}{\um} for some of these sources. %Because of the very large beams of Herschel compared to FORCAST, the SPIRE bands are considered upper limit fluxes for sources that are further than \SI{300}{\pc}.

\section{Data products}

We identify 70 point sources and 14 extended sources in our sample. We produced three types of data products: the mosaic images in each band for all the clusters we observed; the photometry catalogs which can be used to make SEDs; and the fitted physical parameters for the point sources, which are determined from our radiative transfer model, explained in Section~\ref{sec:SEDFitting}.

\subsection{Mosaics}
The SOFIA FORCAST images consist of $\sim 200$ individual images, each representing a field at a given wavelength. Some fields are revisited multiple times when the entire observation could not be completed in a single flight leg. These individual fields are processed and mosaiced together to form one single map for each wavelength and each cluster. 

Before mosaicing the fields, we did a 2D background subtraction. This method divides the images into sections of $50\times 50$ pixels, estimates the median in each cell, and fits a 2D function to these median values. This function is used to construct a smooth background, which is then removed from the image. Each background-subtracted image is calibrated (using the calibration factor that is supplied by the FORCAST pipeline), and weighed by its exposure time before it is co-added into a mosaic in the WCS coordinate reference frame. Although these maps are useful for view the source distribution and spot artifacts, the actual photometry described in the previous sections uses each individual raw field, before the mosaicing and without our background subtraction (some level of background subtraction is already done by the SOFIA pipeline on the archive). If a source is present in multiple fields, the photometry from each of these fields is combined to provide a better flux estimate.

In Fig.~\ref{fig:varietySources} we present four maps from our cluster sample. Each map is a three-color image (red: \SI{37}{\micron}, green: \SI{31}{\micron} and blue: \SI{19}{\micron}), and the scale and stretch of each color is adjusted to balance each color. The three bands have resolutions of
6.4~pixels (\ang{;;4.9}), 6~pixels (\ang{;;4.6}) and 5~pixel (\ang{;;3.8}) FWHM for 37, 31 and \SI{19}{\um} respectively.

\begin{figure}[!h]
\begin{center}
\includegraphics[width=\textwidth]{Figures/RGBmosaic.png}
\vspace{-1cm}
\caption[RGB images of select sample of sources]{RGB images of selected sample of sources (red: \SI{37}{\micron}, green: \SI{31}{\micron} and blue: \SI{19}{\micron}). In these images, the three bands have 6.4, 6 and 5 pixels FWHM for 37, 31 and \SI{19}{\um} respectively.}
\label{fig:varietySources}
\end{center}
\end{figure}

\subsection{Photometry catalog}
%We also produce enhanced SEDs for all of our sources, where we combine archival data with our new SOFIA photometry. The SED plots are complemented with snapshots of the source seen with IRAC and SOFIA. In addition, a snapshot of the corresponding FORCAST \SI{37}{\um} calibrator is shown, which allows to quickly determine the degree of spatial extension of the sources.

\begin{landscape}
\begin{table}
\tiny
\caption[NGC1333 photometry]{Extract of NGC1333 photometry used for SED modeling.}
\label{tab:OphiuchusNGC1333}
\vspace{-0.5cm}
\begin{longtable}{llrrrrrrrrrrrrrrrrrrrrrrrrrrrrrrrrrrrrrrrrrrrrrrr}																\toprule																														
SOFIA name	&	Coordinates		&	R37	&	Lbol	&	j	&	e\_j	&	h	&	e\_h	&	ks	&	e\_ks	&	i1	&	e\_i1	&	i2	&	e\_i2			\\
\midrule																														
NGC1333.1	&	03h29m07.7s	+31d21m57.0s	&	0.746	&	8.385	&	0.0012	&	0.0001	&	0.0031	&	0.0003	&	0.0450	&	0.004	&	0.696	&	0.070	&	1.800	&	0.180			\\
NGC1333.2	&	03h29m10.3s	+31d21m55.5s	&	2.232	&	27.832	&	0.2853	&	0.0285	&	0.6539	&	0.0654	&	0.9010	&	0.090	&	0.637	&	0.064	&	0.446	&	0.045			\\
NGC1333.3	&	03h29m01.5s	+31d20m20.5s	&	0.904	&	8.104	&	0.0008	&	0.0001	&	0.0029	&	0.0003	&	0.0296	&	0.003	&	0.544	&	0.054	&	1.090	&	0.109			\\
NGC1333.4	&	03h29m11.1s	+31d18m30.8s	&	1.103	&	3.056	&	0.0007	&	0.0007	&	0.0009	&	0.0009	&	0.0015	&	0.002	&	0.001	&	0.0007	&	0.004	&	0.0004			\\
NGC1333.5	&	03h29m10.6s	+31d18m19.6s	&	1.623	&	2.786	&	0.0007	&	0.0007	&	0.0009	&	0.0009	&	0.0015	&	0.002	&	0.002	&	0.0002	&	0.007	&	0.001			\\
NGC1333.6	&	03h29m13.0s	+31d18m13.8s	&	0.951	&	1.155	&	0.0007	&	0.0007	&	0.0009	&	0.0009	&	0.0015	&	0.0004	&	0.046	&	0.005	&	0.180	&	0.018			\\
\midrule																														
	&			&	i3	&	e\_i3	&	i4	&	e\_i4	&	F11	&	e\_F11	&	F19	&	e\_F19	&	M24	&	e\_M24	&	F31	&	e\_F31	&	F37	\\
\midrule																														
NGC1333.1	&	03h29m07.7s	+31d21m57.0s	&	3.060	&	0.306	&	2.550	&	0.255	&	0.225	&	0.169	&	1.502	&	0.208	&	--	&	0.260	&	6.886	&	0.640	&	10.994	\\
NGC1333.2	&	03h29m10.3s	+31d21m55.5s	&	0.448	&	0.080	&	0.913	&	0.128	&	8.414	&	0.596	&	36.517	&	2.562	&	--	&	--	&	106.490	&	7.457	&	135.723	\\
NGC1333.3	&	03h29m01.5s	+31d20m20.5s	&	1.690	&	0.211	&	3.060	&	0.306	&	1.681	&	0.131	&	6.902	&	0.493	&	--	&	0.069	&	9.256	&	0.656	&	9.406	\\
NGC1333.4	&	03h29m11.1s	+31d18m30.8s	&	0.005	&	0.001	&	0.004	&	0.0004	&	0.097	&	0.060	&	0.076	&	0.115	&	0.607	&	0.061	&	1.785	&	0.209	&	3.040	\\
NGC1333.5	&	03h29m10.6s	+31d18m19.6s	&	0.010	&	0.001	&	0.011	&	0.001	&	0.114	&	0.093	&	0.150	&	0.119	&	0.771	&	0.077	&	1.946	&	0.234	&	2.166	\\
NGC1333.6	&	03h29m13.0s	+31d18m13.8s	&	0.274	&	0.027	&	0.320	&	0.032	&	0.160	&	0.035	&	0.570	&	0.093	&	0.735	&	0.074	&	1.446	&	0.180	&	1.806	\\
\midrule																														
	&			&	e\_F37	&	M70	&	e\_M70	&	H70	&	e\_H70	&	H160	&	e\_H160	&	H70	&	e\_H70	&	H160	&	e\_H160	&	H250	&	e\_H250	\\
\midrule																														
NGC1333.1	&	03h29m07.7s	+31d21m57.0s	&	0.948	&	49.300	&	4.930	&	52.724	&	5.272	&	66.529	&	35.197	&	52.724	&	5.272	&	66.529	&	35.197	&	71.541	&	14.258	\\
NGC1333.2	&	03h29m10.3s	+31d21m55.5s	&	9.507	&	--	&	--	&	70.039	&	7.004	&	77.574	&	20.036	&	70.039	&	7.004	&	77.574	&	20.036	&	87.661	&	15.014	\\
NGC1333.3	&	03h29m01.5s	+31d20m20.5s	&	0.695	&	23.400	&	2.340	&	20.218	&	2.022	&	78.316	&	7.832	&	20.218	&	2.022	&	78.316	&	7.832	&	101.472	&	18.943	\\
NGC1333.4	&	03h29m11.1s	+31d18m30.8s	&	0.341	&	--	&	--	&	16.609	&	1.661	&	53.689	&	5.369	&	16.609	&	1.661	&	53.689	&	5.369	&	57.215	&	6.293	\\
NGC1333.5	&	03h29m10.6s	+31d18m19.6s	&	0.377	&	20.600	&	2.060	&	14.627	&	1.463	&	49.868	&	4.987	&	14.627	&	1.463	&	49.868	&	4.987	&	52.536	&	6.166	\\
NGC1333.6	&	03h29m13.0s	+31d18m13.8s	&	0.345	&	4.290	&	0.429	&	1.527	&	3.883	&	4.702	&	13.332	&	1.527	&	3.883	&	4.702	&	13.332	&	29.105	&	6.272	\\
\midrule																														
	&			&	H350	&	e\_H350	&	H500	&	e\_H500	&	S850	&	e\_S850	&	F1100	&	e\_F1100	&	S1300	&	e\_S1300	&	$\alpha$	&	e\_$\alpha$	&		\\
\midrule																														
NGC1333.1	&	03h29m07.7s	+31d21m57.0s	&	45.559	&	17.857	&	24.264	&	16.301	&	--	&	--	&	1.300	&	0.130	&	--	&	--	&	0.280	&	0.564	&		\\
NGC1333.2	&	03h29m10.3s	+31d21m55.5s	&	51.506	&	16.114	&	24.742	&	13.062	&	--	&	--	&	--	&	--	&	--	&	--	&	1.243	&	--	&		\\
NGC1333.3	&	03h29m01.5s	+31d20m20.5s	&	70.907	&	17.371	&	40.867	&	11.474	&	--	&	--	&	1.500	&	0.150	&	--	&	--	&	0.714	&	0.385	&		\\
NGC1333.4	&	03h29m11.1s	+31d18m30.8s	&	38.449	&	6.033	&	18.594	&	4.666	&	--	&	--	&	2.000	&	0.200	&	--	&	--	&	1.864	&	0.458	&		\\
NGC1333.5	&	03h29m10.6s	+31d18m19.6s	&	36.232	&	6.189	&	18.007	&	4.878	&	--	&	--	&	2.000	&	0.200	&	--	&	--	&	1.705	&	0.273	&		\\
NGC1333.6	&	03h29m13.0s	+31d18m13.8s	&	34.781	&	8.007	&	21.255	&	6.628	&	--	&	--	&	0.630	&	0.063	&	--	&	--	&	1.001	&	0.501	&		\\
\bottomrule																														
\end{longtable}																																	\caption*{\textbf{Note:} The table contains the source name, coordinates in J2000, the ratio of R37$=\Rfifty_\textrm{source}/\Rfifty_\textrm{cal}$, the bolometric luminosity determined by integrating the data points in log-log space, followed by the photometry and its 1$\sigma$ error in the 2MASS bands (j, h, k$_s$ at 1.3, 1.6 and \SI{2.2}{\um} respectively), IRAC bands (i1, i2, i3, i4 at 3.6, 4.5, 5.8 and \SI{8}{\um} respectively), the FORCAST bands (F11, F19, F31, F37), the \Spitzer MIPS bands (M24 and M70), the Herschel PACS and SPIRE bands (H70, H160, H250, H350), the SCUBA band (S850), the BOLOCAM band (F1100) and the SMA continuum band (S1300). The number following capital letters in the band denomination indicates the band's central wavelength. Flags are used to designate whether or not a source is considered an upper limit, and are not shown in this table for clarity. The fluxes that are upper limit can be seen in the SED images, Fig.~\ref{fig:NGC1333_SEDs} and Fig~\ref{fig:Oph_SEDs}. The complete version of this table is made available electronically.}
\end{table}																														\end{landscape}

We produced a consolidated list of fluxes for our clusters, where we gather 2MASS, \Spitzer, FORCAST, \Herschel, SCUBA, and SMA data, when available, for $\sim 90$ sources. Most sources are point sources for the SOFIA FORCAST \SI{37}{\um} band, but some sources present a certain spatial extension which was not known before. 


A few other parameters are determined from the FORCAST data and shown in the catalog: the $R_{37}$ ratio, which is the ratio of \Rfifty for the source and \Rfifty for the last observed calibrator; the spectral index and its uncertainty, computed from the \SI{2.2}{\um} - \SI{37}{\um} fluxes; and the bolometric luminosity for each source. Note that the bolometric luminosity is the integration of the observed emission across the observed wavelength; as such, it is an observed quantity but it is generally not the true luminosity of the source due to extinction (which is not corrected) and viewing angle corrections. An excerpt of the final table containing just the results for a few NGC~1333 sources is shown in Table~\ref{tab:OphiuchusNGC1333}.

\begin{figure}[!h]
\begin{center}
\includegraphics[width=\textwidth]{Figures/NGC1333_SEDs.png}
\caption[NGC1333 SEDs]{SEDs of the point sources in NGC1333. The red curve represents the best fit. The grey curves represent all the fits with $R$ within 0.5 of the best fit (see Section~\ref{subsec:fittingmethod} for details about the fitting process). The white arrows show which data point is considered an upper limit for the fitting routine. Note that 2MASS J- and H-band measurements, as well as \Spitzer MIPS \SI{24}{\um} and \SI{70}{\um} are plotted, but never used for fitting. Red triangles: 2MASS. Green diamonds: \Spitzer (our photometry or data from existing catalogs). Dark blue triangles: FORCAST (our data). Purple stars: \Herschel (our photometry). Green triangles: SCUBA \SI{850}{\um} and SMA \SI{1.3}{\milli\meter} data from \citep{vanKempen:2009ku} and \citep{vanKempen:2012fb}. Left-pointing blue triangles: \SI{1.1}{\milli\meter} data from \citet{Enoch:2009ch}.}
\label{fig:NGC1333_SEDs}
\end{center}
\end{figure}

The SEDs for our most complete clusters, NGC1333 and Ophiuchus, are shown in Fig~\ref{fig:NGC1333_SEDs} and Fig~\ref{fig:Oph_SEDs}. These show the data points in various color codes and marker types, as well as the best fit models for each source, as determined using our custom fitting routine, described in Section~\ref{sec:SEDFitting}. The R-value, indicated for each fit, is a metric that indicates how well the software was able to find a match between the data points and a pre-computed grid of models. Lower values of R generally indicate better fits.

Many SEDs are well fitted.
Some of the sources show poor fits (R>3), where it seems difficult to find a model that both satisfies the long-wavelength measurements and the IRAC measurements. This indicated that none of the models in the grid fit well. We think this could be explained either by a mismatch of the spatial resolution for the different measurement bands, in which case the long-wavelength bands sample flux that is not necessarily associated with the protostar, but rather is associated with another, nearby source or extended dense cloud emission; or by an excess flux from the IRAC bands that could be explained by the proximity to an outflow \citep{NoriegaCrespo:2004fe,Hudgins:2004wa}. Fig.~\ref{fig:NGC1333_Confusion} shows an example of a poor fit which could be attributed to excess IRAC emission due to a nearby outflow. 

%\begin{figure}[!h]
%\begin{center}
%\includegraphics[width=\textwidth]{Figures/NGC1333_6_saturated_mosaic.png}
%\caption[Confusion in NGC1333]{Example of NGC1333.6, a poorly-fitting source that exhibits confusion. The Herschel is done at the same location as the sources identified using \Spitzer and FORCAST, hence could lead to situation where the long-wavelength comes from emission that is either extended, or too close from another source. [NOW THAT WE KNOW THAT, SHOULDN' I JUST REDO THE FIT WITH THE HERSCHEL AS AN UPPER LIMIT?]}
%\label{fig:NGC1333_Confusion}
%\end{center}
%\end{figure}

\begin{figure}[!h]
\begin{center}
\includegraphics[width=\textwidth]{Figures/NGC1333_1_saturated_mosaic.png}
\caption[Confusion in NGC1333]{Fields centered on NGC1333.1, a poorly-fitting source that shows a mismatch between the IRAC fluxes and the longer-wavelength fluxes.  The \Herschel SPIRE \SI{250}{\um} (right) is taken at the same location as the sources identified using \Spitzer IRAC \SI{3.6}{\um} (left) and FORCAST \SI{37}{\um} (middle). The resolution of the SPIRE beam is not shown on the figure, because it has a radius of \ang{;;22}. In this particular case, it appears that the long-wavelength emission is associated with the source. However, this source is in close proximity to NGC1333.2, an extended source of our sample, which shows a bow shock structure in the southwest of its center seen in IRAC \SI{3.6}{\um}, as well as diffuse emission that extends all the way to NGC1333.1.}
\label{fig:NGC1333_Confusion}
\end{center}
\end{figure}


\begin{figure}[!h]
\begin{center}
\includegraphics[width=\textwidth]{Figures/Oph_SEDs.png}
\caption[Ophiuchus SEDs]{SEDs of the point sources in the Ophiuchus cluster. Same legend as Fig.~\ref{fig:NGC1333_SEDs}}
\label{fig:Oph_SEDs}
\end{center}
\end{figure}


\subsection{Fitted physical parameters}

The spectral index ($\alpha \equiv (d\log(\lambda F_\lambda)/d\log\lambda$) distribution for the point sources in our sample is shown on the left of Fig.~\ref{fig:SpectralIndex}. Most sources have strongly positive spectral indices, indicative of a rise in the SED with increasing wavelength and a significant contribution to the total luminosity by long-wavelength emission. These objects are more embedded, and thought to be younger than objects with negative spectral index. Note however that the emission generally
peaks a little shortward of 100 $\mu$m. A closer inspection of the distribution of our sources reveals that the targets with negative index mostly lie in the Ophiuchus cluster, and are consistent with late Class I objects which have already cleared a significant fraction of their envelopes \citep{Jorgensen:2008gz}. 

\begin{figure}[!h]
\begin{center}
\includegraphics[width=\textwidth]{Figures/SpectralIndex.pdf}
\vspace{-1cm}
\caption[Spectral Index distribution of point sources]{Spectral index distribution of all point sources in our sample. \textit{Left}: standard determination of the spectral index, using 2MASS and \Spitzer from \SI{2}{\micron} to \SI{24}{\micron}, when data is available. \textit{Right}: Determination of the spectral index using data from 2MASS, \Spitzer and our FORCAST data up to \SI{37}{\micron}. The distribution changes significantly when you account for the longer fluxes in these clustered regions.}
\label{fig:SpectralIndex}
\end{center}
\end{figure}

The data tables also include all of the physical parameters derived using the technique from Section~\ref{sec:SEDFitting}, as well as their uncertainties.

\section{SED fitting}
\label{sec:SEDFitting}

This section looks more closely at the SED fitting process: examining its value and its common shortcomings. First, we need to recognize that SED fitting is prone to many degeneracies \citep[see e.g.][for an introduction on the degeneracies of SED fitting]{Robitaille:2007dl} unless one has a great deal of spatial and spectral information about the source, which is usually not the case. In order to make physically plausible models, there are
usually many geometrical and physical parameters in detailed radiative transfer models, but only a handful of measurement points are available to fit, leading to a dramatically under-constrained problem. The goals of our fitting procedures are then to reduce the number of parameters to those which have a significant quantitative impact on the SED, to identify the families of model parameters that fit the SED, and to define the "best fit" model and its "uncertainty" which represents the range in the model parameters with "reasonable" fits.

As our starting point of our investigation of fitting SEDs to our sources, we used the \textit{sedfitter} tool from \citep{Robitaille:2006cb}. These authors computed a large grid of tens of thousands of SED models using a radiative transfer code by \citep{Whitney:2003ke}, by varying 14 geometrical and physical parameters in the dust density grid such as the size of the disk, the accretion rates, the radius and mass of the envelope, etc. The models are then evaluated in the bands corresponding to our data, and a $\chi^2$ metric is evaluated for each model. By exploring the distribution of $\chi^2$, we noticed, as expected, the very large correlations between the parameters which is indicative of many local minimas in the 14-dimensional grid. Hence, inferring geometrical and physical parameters from such a grid can be misleading.

\subsection{A custom grid of models}

We use Hyperion \citep[][see also in Section~\ref{subsubsec:radiative}]{Robitaille:2011fc} to develop our own capability of calculating SEDs and understand the sensitivity of these parameters on the SED shape of our sources. Based on our investigation, the degeneracy between viewing angle and multiple geometrical parameters is considerable. The sensitivity of our SED to disk properties is small, as most of our YSOs are younger objects with significant envelopes. Since no central star is visible, parameters describing the central source such as the mass, radius and temperature are primarily important when they are combined into one single term, which is the central luminosity. Similarly, the luminosity created by accretion onto the central object can not be distinguished from a more luminous central object and a non-accreting disk. Finally, we find that there is very little difference between Ulrich envelope models \citep{Ulrich:1976ho} and standard power-law envelopes (see for example Fig.~14 from \citet{Whitney:2013cw}), except that the latter can more easily be related to physical parameters such as the envelope mass. 

From these findings, we created a simplified grid of models by significantly reducing the number of parameters in Hyperion. Table~\ref{tab:SEDModelGrid} describes most of the geometric and physical parameters that are available in Hyperion: divided into the central source, the disk, the envelope and the bipolar cavity. We set most parameters to constants which we determined as average values using literature examples as well as our own investigations for the objects we try to study, which are primarily class 0 and I YSOs. The parameters which we varied in the fits are shown at the bottom of the table: the inclination angle, the central luminosity (irrespective of whether it is caused by the central star or by accretion), the envelope mass, the external extinction and a scaling factor (explained below). The only two physical parameters that we vary are the luminosity and the envelope mass. While others \citep[e.g.][]{Furlan:2016df} have also attempted to reduce the number of parameters for their fitting, they still include more parameters such as the disk radius, but generally conclude that they are not able to properly constrain all of their parameters. As will be discussed later, there are a few YSOs which are
not well fitted with are heavily reduced set of fitting parameters.

It is an important point to emphasize that we are not know the values of these "fixed" model parameters but rather that fixing them to a typical literature value does not have major impact on the fitted parameters and hence the SED fit. Disk mass are radius are two examples; in the presence of
an envelope of comparable or greater mass, the disk emission is a weak function of mass in the
wavelengths (<20 $\mu$m) where it contributes significantly to the SED because it is optically thick at those wavelengths. Similarly, the outer radius of the disk controls its contributions are longer
wavelengths (>50 $\mu$m) where the envelope usually emits effectively; significantly reducing the
disk emission at longer wavelengths requires small disk outer radii (10-30 AU) but has little impact on its shorter wavelength emission.

%LGM Add reference for use of multiple dust types.
Unlike most authors, who use multiple kinds of dust models for different regions of the SED which add complexity and number of parameters \citep{Whitney:2003kc,Robitaille:2006cb,Whitney:2013cw}, we choose to use the same dust model (OH5) for both the envelope and the disk, and assume a 1:100 dust-to-gas ratio. By doing so, we tend to overestimate the short-wavelength emission from SEDs, because the OH5 model assumes isotropic scattering whereas most dust grains appear to be forward-scattering \citep{Draine:2011tr}.


\renewcommand{\arraystretch}{1.5}
\begin{table}[!h]
\scriptsize
\caption[SED model grid]{SED model grid.}
\label{tab:SEDModelGrid}
\vspace{-0.5cm}
\begin{longtable}{lP{5cm}P{3cm}P{2cm}}
\toprule																			
Parameter	&	Description	&	Values	&	Units	\\
\midrule							
\midrule							
\multicolumn{4}{c}{\textbf{Constant parameters}}							\\
\midrule							
\multicolumn{4}{c}{Central source}							\\
\Mstar	&	Stellar mass	&	1	&	\si{\Msun}	\\
\Tstar	&	Stellar temperature	&	4000	&	K	\\
\midrule							
\multicolumn{4}{c}{Disk}							\\
Type	&	Flared or alpha disk	&	Flared	&		\\
\Mdisk	&	Disk mass	&	0.001	&	\si{\Msun}	\\
\Rdiskmax	&	Disk outer radius	&	100	&	\si{\au}	\\
\Rdiskmin	&	Disk inner radius	&	 sublimation radius	&	\si{\au}	\\
$\beta$	&	Flaring parameter	&	1.25	&		\\
$p$	&	Disk surface density exponent	&	-1	&		\\
$r_0$	&	Reference distance for scale height	&	\Rdiskmin	&	\si{\au}	\\
$h_0$	&	Disk scale height at $r_0$	&	0.01$\times$\Rdiskmin	&	\si{\au}	\\
$d$	&	Dust	&	OH5	&		\\
\midrule							
\multicolumn{4}{c}{Envelope}							\\
Type	&	Power-law or Ulrich	&	Power-law	&		\\
\Renvmin	&	Envelope inner radius	&	\Rdiskmin	&	\si{\au}	\\
\Renvmax	&	Envelope outer radius	&	5000	&	\si{\au}	\\
$\alpha$	&	Power	&	-1.5	&		\\
$r^\textrm{env}_0$	&	Reference radius	&	\Renvmin	&	\si{\au}	\\
$d$	&	Dust	&	OH5	&		\\
\midrule							
\multicolumn{4}{c}{Cavity}							\\
$r^\textrm{cav}_0$	&	Cavity outer radius	& 	\Renvmax	&	\si{\au}	\\
$\theta_0$	&	Opening angle at $r^\textrm{cav}_0$	&	10	&	degrees	\\
	&	Flaring exponent	&	1.5	&		\\
$\rho_0$	&	Density at $r^\textrm{cav}_0$	&	0	&	\si{\gram\per\centi\meter}	\\
$\alpha_e$	&	Density profile exponent	&	0	&		\\
\midrule							
\midrule							
\multicolumn{4}{c}{\textbf{Fitted parameters}}							\\
\midrule							
$i$	&	Inclination angle	&	0 to 90 in 10 constant increments of $\cos i$	&	degrees	\\
\Lstar	&	Central luminosity	&	$5\times 1.5^p$ for $p=-4, -3, \dots, 10$ (from 0.99 to 288)	&	\si{\Lsun}	\\
\Menv	&	Envelope mass	&	$0.01\times 1.5^p$ for $p=-2, -1, \dots, 19$ (from 0.001 to 22.17)	&	\si{\Msun}	\\
\Av	&	External extinction	&	$0, 1, \dots, 14$	&	mag	\\
$s$	&	Scaling	&	0.7, 0.85, 1, 1.5, 1.3	&		\\
\bottomrule					
	\end{longtable} 
\end{table}

To facilitate the calculation of models, we constructed a wrapper program that can run the Hyperion software for the range of parameters given in Table~\ref{tab:SEDModelGrid} to create our model grid.

 Because of time and resource limitations, a moderate number of photons was chosen in the Monte Carlo calculation, which can increase the noise at short wavelengths. The details of our modeling parameters, which will be familiar to the Hyperion user, are described in Table~\ref{tab:HyperionParams}. Note that models of more than \SI{1}{\Msun} are actually run with more photons (\num{1e6} instead of \num{2e5}) for imaging, in order to obtain acceptable \SNR at short wavelengths.

\renewcommand{\arraystretch}{1.5}
\begin{table}[!h]
\scriptsize
\caption[Hyperion simulation parameters]{Hyperion simulation parameters.}
\label{tab:HyperionParams}
\vspace{-0.5cm}
\begin{longtable}{lP{4cm}}
\toprule																			
Number of photons (initial)	&	\num{2e5}	\\
Number of photons (imaging)	&	\num{2e5}	\\
Number of photons (raytracing sources)	&	\num{1e6}	\\
Number of photons (raytracing dust)	&	\num{1e6}	\\
Lucy max iterations	&	6	\\
Max photon interactions	&	\num{1e5}	\\
Geometrical grid parameters (radial, theta and azimuthal)	&	400, 199, 2	\\
MRW	&	True	\\
\bottomrule					
	\end{longtable} 
\end{table}

The grid is composed of $\sim 418$ models which are calcuated with Hyperion. For models with $\Menv>\SI{0.5}{\Msun}$, we interpolate the grid in mass by increments of 20\%, which allows for a finer sampling at higher masses, but increases the number of individual models to $958$. The interpolation is done at constant luminosity. Each model is sampled at 10 inclinations, 15 values for external extinction, and five different scaling factors, for a total of \num{718500} grid models. Each model is evaluated at all relevant observing bands, from the 2MASS bands all the way to the \SI{1.3}{\milli\meter} SMA bands. Given the sparsity of the grid, and the relatively simple model used, we do not apply color correction to the fluxes, nor do we convolve the model fluxes with the band transmission function: the resulting corrections fall within our approximations, and do not affect significantly the outcome of the fitting.

The scaling factor in our fits is used to represent the uncertainty in the distance determination \citep[e.g.][]{Robitaille:2006cb}, but it can also be considered as a scaling to represent modestly different luminosities from the model value \citep{Furlan:2016df}. Indeed, \citet{Furlan:2016df} show that, to first order, changing luminosity by a small amount is approximately equivalent to scaling the SED in flux. In their grid, they use a scaling factor that ranges from 0.5 to 2.0, which allows them to have factors of 2.0 in the luminosity of their calculated models. We choose a more conservative approach  by actually running the grid at closer luminosity steps (factor of 1.5) and hence have a smaller range of scaling factors. 

The extinction parameter is used to represent extinction by material along the line of sight that is \textit{outside} of the core: foreground material which may be extended material within the cluster core or may be additional cloud along the line of sight. This parameter is essential to providing good fits, as most other authors have also found \citep[e.g.][]{Robitaille:2006cb,Furlan:2016df}. A discussion of the meaning and importance of this parameter is given in the following sections.

\subsection{Fitting method}
\label{subsec:fittingmethod}

In order to determine which model fits the data best, we adopt a metric defined by \citet{Fischer:2012dj,Furlan:2016df}:

\begin{equation}
R = \frac{1}{N}\sum_i w_i|\log[\Fobs(\lambda_i)] - \log[\Fmod(\lambda_i)]|,
\end{equation}
where $i$ are the indices of the valid data points, the weights $w_i$ correspond to the inverse of the fractional uncertainty of each measurement, \Fobs and \Fmod are the observed and model fluxes respectively, and $N$ is the number of valid measurements. For our models, we set the fractional uncertainty to a minimum of 10\%, to avoid having a few points completely drive the fit. Early versions of the fitting routines, which used the published $1\sigma$ uncertainties would completely skew the results by putting all the weight into a few flux measurements. This was most notable for the \Spitzer IRAC points, for which published uncertainties are sometimes only have a few percent. We chose to override these uncertainties, in large part because the assumptions of geometry and dust properties that go into a model calculation do not justify that level of confidence in the model output.

\citet{Furlan:2016df} discuss in more detail the meaning of this $R$ metric, which differs from a standard $\chi^2$ metric such as the one used by \citet{Robitaille:2007dl}. $R$ represents a weighted average of the logarithmic deviations between the observations and the model. It is important to note that, although it is normalized, it does not have a statistical interpretation like the standard $\chi^2$ metric. In particular, models with fewer data points or large measurement uncertainties will tend to have smaller values of $R$, even if the fit is poor. $R$ is only useful as a relative measure of the goodness of fit to the specific observations.

For each source, we calculate $R$ for each model in our grid. The model with the smallest value for $R$ is the best-fitting model by this metric, but given our sparse sampling and the errors of our observations, this is not necessarily the most likely model to best fit the data. We can consider two extremes: in the first case, the best fit has a value of $R$ which is much lower than other models. Then, it is clearly the best fit. In the second case, let's suppose that the \num{1000} best-fitting models lie very close to the best $R$. In this case, concluding that the model that best fits our observations (and from which will interpret physical quantities) is the one with the minimum $R$ is too strict and does not account for the uncertainties that are present. 

In practice, most of our models fall in the second case. After visual inspection of the fits, we conclude there is very little significant difference between values of $R$ which are separated by $\sim 0.5$. They all can be considered equally good (or bad) fits. Hence, for a robust measure of the best-fitting model parameters, we choose the mode (the most likely value) of the parameters from models which are within $R_\textrm{min}$ and $R_\textrm{min}+0.2$, in order to really pick the best possible fits. The error on the parameter is then estimated using the models within $R_\textrm{min}$ and $R_\textrm{min}+0.5$, since these models all similarly fit, and is described in the next section.

Because we use exclusively the OH5 dust model, which we know overestimates the short-wavelength fluxes, we expect to overestimate the extinction required to match the observations. For this reason, we choose to ignore the 2MASS J and H band data points, which drive the extinction values up dramatically and sometimes leads the fit towards non realistic solutions. However, we choose to keep the \SI{2.2}{\um} Ks Band data point to give some weight to the shorter wavelength data.

\subsection{Overview of derived parameters}

The distribution of the best fit solutions of the envelope mass and central luminosity is shown in Fig.~\ref{fig:MassLumHist} for the clusters for which we have long-wavelength data (Ophiuchus and NGC1333). Our sample covers a broad range of masses, but is naturally biased towards high luminosities given SOFIA's instrumental sensitivity and our cluster selection.


\begin{figure}[!h]
\begin{center}
\includegraphics[width=\textwidth]{Figures/MassLumHist.pdf}
\vspace{-1cm}
\caption[Fitted envelope mass and luminosity distribution]{Fitted envelope mass and luminosity distribution for all observed point sources in Ophiuchus and NGC~1333, where long-wavelength data is available.}
\label{fig:MassLumHist}
\end{center}
\end{figure}

From visual inspection, data with $R$ less or close to 1 appear to fit the data well. Larger $R$ show less good fits. The distribution of $R$ for all the isolated point sources is shown in Fig.~\ref{fig:Rdistr}. Note that targets where less data points are available, or where data points are more noisy, usually have lower $R$ than targets with a lot of available data points, even if the fits are not necessarily as good. This has also been observed by \citep{Furlan:2016df} and is one of the drawbacks of using the $R$ metric.

\begin{figure}[!h]
\begin{center}
\includegraphics[width=\textwidth]{Figures/Rdistr.pdf}
\vspace{-1cm}
\caption[Distribution of $R$]{$R$ distribution across all observed point sources in Ophiuchus and NGC1333, where long-wavelength data is available.}
\label{fig:Rdistr}
\end{center}
\end{figure}

%For our sample, we can compare the fitted central luminosity, \Ltot, with the integrated luminosity from the datapoints, \Lbol for our entire sample of point sources (see Fig.~\ref{fig:LbolVsLest}). This shows relatively good agreement, although a systematic excess in fitted central luminosity can be observed, which we attribute to the widespread choice of using an external extinction coefficient. By using this external extinction as a model parameter, we artificially reduce the emission at short wavelengths, which would tend to decrease the bolometric luminosity.
%\begin{figure}[!h]
%\begin{center}
%\includegraphics[width=\textwidth]{Figures/LbolVsLest.pdf}
%\vspace{-1cm}
%\caption[Estimated luminosity vs bolometric luminosity]{Estimated luminosity vs bolometric luminosity. The best fit line is shown in red, along with 95\% confidence intervals. The grey dashed line represents $\Ltot = \Lbol$. The excess modeled luminosity for smaller luminosities is caused by the external extinction, which absorbs a large fraction of the luminosity emitted by the central object but does not re-radiate it at longer wavelengths - this is one of the limitations of this exercise.}
%\label{fig:LbolVsLest}
%\end{center}
%\end{figure}
%
%The luminosity excess is more pronounced for lower masses, as the short wavelength emission represents a larger portion of the total emission from the source (Fig.~\ref{fig:LbolMinusLestVSMass}). 
%
%\begin{figure}[!h]
%\begin{center}
%\includegraphics[width=\textwidth]{Figures/LbolMinusLestVSMass.pdf}
%\vspace{-1cm}
%\caption[Luminosity excess as a function of envelope mass]{Luminosity excess as a function of envelope mass.}
%\label{fig:LbolMinusLestVSMass}
%\end{center}
%\end{figure}

%LGM is the rewrite below what you meant?
One major limitation and inconsistency in all broad fitting works to date is the inclusion of a foreground material which provide extinction without emission. For example, \citet{Furlan:2016df} fit for external extinction up to $\Av = 40$ for some of their sources, and use all of the 2MASS bands in their fitting. It is not consistent to assume that so much material is present along the line of sight, while not also being observed at longer wavelengths. Since the dust is optically thin at longer wavelengths, the far-infrared and submillimeter observations should see emission from this material which is obscuring the shortest wavelengths.

Our exploration with the fitting routine shows that limiting the external extinction forces more inclined geometries, where the light from the central star passes through the disk before reaching us. However, we were not able to account for the entirety of the short wavelength extinction by doing this, as the mid-infrared wavelength (IRAC and FORCAST bands) are also affected by more inclined geometries, which can compromise the fits. This could indicate a fundamental limitation in our geometrical representation of YSOs or assumed dust properties, since there is no possible way to account for both the low amount of far-IR emission seen, e.g. by \Herschel and high amount of extinction seen from the 2MASS and IRAC bands. The highly cited publications, that we have referred to, adopt an external extinction factor, the rigor of which we now strongly question. However, we have not been able to find an appropriate solution to circumvent this issue.  Hence, we choose to follow the examples of previous authors and adopt an external extinction factor. Unlike \citet{Furlan:2016df}, which consider \Av as high as 40~mag, we choose to limit it to \Av = 14~mag, a moderate value of the diffuse extinction in our clusters of interest. 

An external extinction of \Av = 14~mag means that fluxes at \SI{2.2}{\um} are reduced by a factor of $\sim 17$, while \SI{8}{\um} fluxes are reduced only by a factor of $\sim 2$. For most Class 0 and I sources, most of the emission has already been reprocessed by dust out to longer wavelengths, and the 2MASS \SI{2.2}{\um} data points are usually extremely low, so a difference of a factor of 17 in this small region of the spectrum will not lead to significantly different luminosity estimates when compared to the contribution from other parts of the spectrum. In fact, a small exploration of our fits reveals that for these sources usually the extinction from within the envelope itself is already much larger than this factor. However, at $\Av = 40$, the flux reduction at \SI{2.2}{\um} is $>3000$, at which point we argue this could become a problem. Which such a large ratio, it is more difficult to claim that the fitted luminosity from the model is not overestimating the actual luminosity, since the external extinction reduces by many orders of magnitude the short-wavelength emission in order to fit the data. 

%LGM Why is this every true since you are using external extinction to mask some of the luminosity.
%   Think more about this paragraph.
For all sources we calculate the bolometric luminosity as the integral of all the available data points, even those which correspond to upper limits. This makes \Lbol an upper limit as well on the observed luminosity. Since most of the upper limits are from long-wavelength data points, this impacts sources with a larger envelope more. We note that the fitted luminosity \Lmod tends to be lower than the bolometric luminosity \Lbol for low inclination angles, but \Lmod tends to be higher than \Lbol for more inclined geometries. This is expected since for high inclinations a large fraction of the emission is not directed towards the observer \citep[see, e.g.][for a discussion]{Furlan:2016df}. On the contrary, when seen almost face-on, the observer sees both the emission from the source as well as the light scattered on the walls of the cavity and the disk.
%LGM see above comment. The last sentence is true if the external extinction is low. Is that systematically true?


%\begin{figure}[!h]
%\begin{center}
%\includegraphics[width=\textwidth]{Figures/massEstvsCalc.pdf}
%\vspace{-1cm}
%\caption[Estimated mass vs calculated mass]{Estimated mass vs calculated mass for the point sources which have sub-millimeter data.}
%\label{fig:massEstvsCalc}
%\end{center}
%\end{figure}

%For the clusters which do have submillimeter data points, we can use the traditional mass estimate described in Section~\ref{subsec:DustExtinction} with the \SI{1.1}{\milli\meter} or \SI{1.3}{\milli\meter} fluxes. For this calculation, we use an effective dust temperature of \SI{20}{\kelvin}, assuming an opacity of \SI{0.0114}{\raiseto{2}\centi\meter\per\gram} and \SI{0.009}{\raiseto{2}\centi\meter\per\gram} for \SI{1.1}{\milli\meter} and \SI{1.3}{\milli\meter}, respectively. Note that this measurement is very sensitive on these assumptions; for example, lowering the dust temperature estimate to \SI{10}{\kelvin} increases the mass estimate by a factor of 3.

%We find that our fitted mass estimates agree with this traditional method of deriving masses, as shown in Fig~\ref{fig:massEstvsCalc}. For lower masses, however, our fits tend to underestimate the mass compared to the derived quantities. This could be explained since most of the lower-mass envelopes belong to more evolved objects, which can thus have higher dust temperatures due to less opacity: the derived mass, assuming a temperature of \SI{20}{\kelvin}, would then overestimate the amount of material in the envelope.



%There are few caveats in determining the luminosity of these sources. First,  Second, the best-fitted extinction \Av is large for most of our fits. In fact, it is usually right at the maximum value we sample of 14. This means that the luminosity of the source, \Lmod, has to be dramatically extincted, which can lead to high model luminosities. Whether these model luminosities are correct or not depends on exactly where the shorter wavelengths are being absorbed and whether the heating from the absorption of this energy is being properly accounted for in the model.

Finally, for the clusters which have submillimeter data points, we calculate the traditional mass estimate described in Section~\ref{subsec:DustExtinction} using the \SI{1.1}{\milli\meter} or \SI{1.3}{\milli\meter} fluxes. For this calculation, we use an effective dust temperature of \SI{20}{\kelvin}, assuming an opacity of \SI{0.0114}{\raiseto{2}\centi\meter\per\gram} and \SI{0.009}{\raiseto{2}\centi\meter\per\gram} for \SI{1.1}{\milli\meter} and \SI{1.3}{\milli\meter}, respectively, based on the
expected emissivity for OH5 dust. These values for \Menv are shown in Table~\ref{tab:FittedParameters}.
Note that this measurement is very sensitive to these assumptions; for example, lowering the dust temperature estimate to \SI{10}{\kelvin} increases the mass estimate by a factor of 3. In addition, these measurements can be overestimates by a large amount if the 1.1 and \SI{1.3}{\milli\meter} fluxes are
measured with single-dishes, and hence upper limits on the flux from the YSO environment.

A summary of our fit results for Ophiuchus and NGC1333 is shown in Table~\ref{tab:FittedParameters}. All fits results for all our sources are shown in Appendix~\ref{ap:data}. Note that the luminosity that is used in this analysis is always the luminosity multiplied by the scaling factor $s$, under the assumption that the SED scales for small changes in luminosity. This scaling factor improved the fit but it must be remembered that the true luminosity is dependent on the distance. For example, a distance error of 10\% would cause a luminosity estimate that would differ by 20\%.


\begin{landscape}
\begin{table}[!h]
\scriptsize
\caption[Fitted parameters]{Fitted parameters for the point sources in Ophiuchus and NGC1333 where long-wavelength photometry is available.}
\label{tab:FittedParameters}
\vspace{-0.5cm}
\hspace*{-1.5cm}
\begin{center}
\begin{longtable}{lcccccccccc}

\toprule																									
SOFIA Name	&	Coordinates	&	R37	&	$\alpha$	&	$R$	&	\Menv			&	Calc. \Menv	&	\Ltot			&	$\Lbol$	&	i	&	$\Av$	\\
	&	(J2000)	&		&		&		&	(\si{\Msun})			&	(\si{\Msun})	&	(\si{\Lsun})			&	(\si{\Lsun})	&	(\si{\degree})	&	(mag)	\\
\midrule																									
NGC1333.1	&	03h29m08s +31d21m57s	&	0.75	&	0.28	&	3.40	&	0.004	$\pm$	0.005	&	0.97	&	32.5	$\pm$	7.8	&	8.4	&	51	&	14	\\
NGC1333.3	&	03h29m02s +31d20m21s	&	0.90	&	0.71	&	3.39	&	0.004	$\pm$	0.03	&	1.12	&	3.5	$\pm$	2.1	&	8.1	&	0	&	14	\\
NGC1333.4	&	03h29m11s +31d18m31s	&	1.10	&	1.86	&	0.83	&	2.919	$\pm$	0.45	&	1.50	&	2.3	$\pm$	0.4	&	3.1	&	19	&	11	\\
NGC1333.5	&	03h29m11s +31d18m20s	&	1.62	&	1.70	&	0.77	&	1.297	$\pm$	0.33	&	1.50	&	1.3	$\pm$	0.3	&	2.8	&	19	&	14	\\
NGC1333.6	&	03h29m13s +31d18m14s	&	0.95	&	1.00	&	1.21	&	0.001	$\pm$	0.0007	&	0.47	&	7.5	$\pm$	1.2	&	1.5	&	27	&	14	\\
NGC1333.7	&	03h28m43s +31d17m35s	&	1.19	&	1.08	&	1.83	&	0.001	$\pm$	0.001	&	--	&	9.6	$\pm$	1.8	&	1.4	&	58	&	0	\\
NGC1333.8	&	03h29m04s +31d16m04s	&	0.77	&	1.14	&	1.06	&	1.946	$\pm$	0.75	&	2.02	&	17.0	$\pm$	2.4	&	35.1	&	0	&	13	\\
NGC1333.9	&	03h28m56s +31d14m37s	&	0.80	&	2.79	&	2.62	&	2.919	$\pm$	0.35	&	1.72	&	17.0	$\pm$	2.4	&	24.3	&	19	&	14	\\
NGC1333.10	&	03h28m57s +31d14m15s	&	0.80	&	1.83	&	1.16	&	0.256	$\pm$	0.18	&	0.45	&	5.6	$\pm$	0.9	&	4.8	&	19	&	14	\\
NGC1333.11	&	03h28m37s +31d13m30s	&	1.02	&	1.69	&	0.99	&	0.38	$\pm$	0.18	&	0.27	&	7.7	$\pm$	0.8	&	7.5	&	19	&	14	\\
Oph.1	&	16h27m10s -24d19m13s	&	0.92	&	0.27	&	0.67	&	0.01	$\pm$	0.002	&	0.04	&	7.9	$\pm$	1.3	&	3.6	&	78	&	3	\\
Oph.2	&	16h26m44s -24d34m48s	&	0.93	&	0.83	&	2.08	&	0.001	$\pm$	0.002	&	0.05	&	32.3	$\pm$	21.1	&	1.2	&	84	&	14	\\
Oph.3	&	16h27m09s -24d37m18s	&	0.99	&	0.57	&	1.54	&	0.004	$\pm$	0.002	&	0.04	&	85.0	$\pm$	19.7	&	13.4	&	0	&	14	\\
Oph.5	&	16h27m07s -24d38m15s	&	1.31	&	0.31	&	1.36	&	0.001	$\pm$	0	&	0.03	&	4.3	$\pm$	0.5	&	0.5	&	81	&	14	\\
Oph.6	&	16h27m16s -24d38m46s	&	1.29	&	2.54	&	0.93	&	0.001	$\pm$	0.001	&	0.02	&	26.6	$\pm$	6.4	&	0.8	&	90	&	13	\\
Oph.7	&	16h27m28s -24d39m34s	&	0.97	&	1.35	&	1.39	&	0.015	$\pm$	0.002	&	0.03	&	26.6	$\pm$	3.5	&	6.5	&	72	&	14	\\
Oph.8	&	16h27m37s -24d30m35s	&	1.02	&	0.55	&	1.13	&	0.007	$\pm$	0.002	&	0.03	&	17.7	$\pm$	3.4	&	5.0	&	78	&	12	\\
Oph.9	&	16h27m22s -24d29m54s	&	--	&	0.51	&	2.08	&	0.001	$\pm$	0	&	0.01	&	11.8	$\pm$	1.2	&	1.0	&	81	&	14	\\
Oph.10	&	16h27m18s -24d28m55s	&	1.26	&	0.50	&	1.38	&	0.003	$\pm$	0.0006	&	0.004	&	5.0	$\pm$	1.6	&	0.6	&	81	&	14	\\
Oph.13	&	16h27m30s -24d27m43s	&	0.00	&	-0.37	&	2.23	&	0.001	$\pm$	0	&	0.01	&	17.7	$\pm$	5.5	&	1.5	&	81	&	14	\\
Oph.14	&	16h27m28s -24d27m21s	&	1.89	&	-0.13	&	1.00	&	0.001	$\pm$	0.0009	&	0.02	&	4.3	$\pm$	0.6	&	1.0	&	81	&	14	\\
Oph.15	&	16h27m29s -24d39m17s	&	1.25	&	0.10	&	1.12	&	0.004	$\pm$	0.0007	&	0.02	&	3.3	$\pm$	0.4	&	0.6	&	27	&	14	\\
Oph.16	&	16h26m24s -24d24m48s	&	1.80	&	-0.76	&	1.87	&	0.001	$\pm$	0	&	--	&	17.7	$\pm$	2.9	&	2.2	&	78	&	10	\\
Oph.17	&	16h26m24s -24d24m39s	&	0.96	&	-0.09	&	1.21	&	0.001	$\pm$	0	&	--	&	5.3	$\pm$	0.6	&	1.3	&	81	&	14	\\
Oph.18	&	16h26m17s -24d23m45s	&	1.18	&	0.63	&	1.34	&	0.003	$\pm$	0.004	&	0.04	&	2.8	$\pm$	0.9	&	0.3	&	81	&	14	\\
Oph.19	&	16h26m30s -24d23m00s	&	2.51	&	0.57	&	0.89	&	0.001	$\pm$	0.001	&	0.01	&	5.3	$\pm$	1.0	&	1.2	&	75	&	14	\\
\bottomrule																									
\end{longtable}																																	
%\caption*{\textbf{Note:} The complete version of this table is made available electronically}			
\end{center}																						
\end{table}	
\end{landscape}			


%\begin{figure}[!h]
%\begin{center}
%\includegraphics[width=\textwidth]{Figures/massVSalpha.pdf}
%\vspace{-1cm}
%\caption[Mass versus spectral index]{Mass versus spectral index for the point sources which have sub-millimeter data, and for which the fits have $R<2$.}
%\label{fig:massVSalpha}
%\end{center}
%\end{figure}
%
%By examining the envelope mass fitted values with respect to the measured spectral index $\alpha$, we note that there is a correlation (Fig.~\ref{fig:massVSalpha}). This correlation appears to be good for $\alpha <1$, but much more loose for $\alpha>1$.
%Of the notable relationship that can be seen among our parameters, we can see for example the envelope mass being correlated with the spectral index (see Fig.~\ref{fig:massVSalpha}) - this indicates a possibility that the SOFIA data points might become a tool to predict the envelope mass, once a sufficient amount of statistics have been gathered (to lower the scatter in that figure). 


\subsection{Estimating parameter uncertainty}

It is important to estimate the uncertainty in fitted parameters
 to quantify the confidence in a given fit. Without uncertainties, no meaningful conclusion can be drawn about the physical meaning of the fits. This estimation is also one of the most difficult aspect of the fitting process, since it really depends on the method used and the modeling strategy. It is also difficult to compare results with the findings of other authors who used a different approach to their fitting. 

In this work, we propose a novel methodology to derive the uncertainty on the best fit. First, we determine the best fit for a given parameter as the mode of the parameter values from the models that fit within $[\Rmin,\Rmin+0.2]$, where \Rmin is the minimum value of $R$ in the entire grid. This is statistically more robust than picking simply the model with the lower $R$, since, given our uncertainties and approximations, there is no statistically-significant difference between models that fit within that range.

Once this best fit value is determined for all parameters, the uncertainty is determined using all models that fit within $[\Rmin,\Rmin+0.5]$. We determine three quantities from these models: the standard deviation from the best fit; the median absolute deviation from the best fit; and the skewness of the distribution. %All of these values are given the data table.
%LGM Are they in the data table in the thesis. They should be. In a number of places, you have made reference to releasing infomation in some data or fit tables to be releases but the thesis should be self contained unless referring to already published work. I have deleted these references in places where I saw them.

The choice of the $R$ intervals are empirical based on our fitting experience. Since the metric $R$ is not model-dependent but instead is a distance of \textit{distance} between models and observations, we think that similar values will still lead to reasonable parameter and uncertainty estimates in other future works. One limitation could occur from the density of the grid: if the models are so sparse that there are only a handful of model within each interval used in the uncertainty estimation, this could lead to poor estimate of uncertainties. 


%Another important consideration is the cross-correlation between parameters. This is a phenomenon that we encountered heavily using the \textit{sedfitter} function from \citet{Robitaille:2006cb}, which attempts to fit 14-parameter models (plus extinction and scaling factor). This cross-correlation is apparent when the same observations can be fitted with models with wildly different sets of parameters, indicative of multiple local minimas in the grid.

%By limiting the number of parameters in the model, we dramatically reduce this effect. However, some cross-correlations...



\subsection{Discussion}

Several factors have been omitted for simplicity our model fitting. First, the models we use have an axisymmetric geometry which is unlikely to account for realistic mass distributions in the envelope and the disk. Second, we ignore the surrounding medium and consider it devoid of emission. In reality, the transition to the surrounding medium is likely smooth and its emission relevant at the longest wavelengths. Third, we assume that the only heating source is located at the center of the YSO. The heating source consists of both the light from the star, and from the accretion luminosity, which can not be distinguished from our point of view. It is important to realize that external heating can also play a role in raising the dust temperature and changing the SED signature in the cluster environment. The impact of the interstellar heating is explored in \citet{Furlan:2016df}, who show that it can have a substantial effect on the SED - but they nevertheless do not include this parameter in their grid, since it is too case-specific. The Hyperion radiative transfer code that is used to model our grid could accommodate external radiation fields as well, and this could become a future addition to our modeling. Finally, the observations that form our SEDs were not taken simultaneously, so it is possible for the YSO flux to change over the period of years. This YSO variability has been shown to be fairly common at the
10 to 20\% level at near infrared wavelengths \citep{Rebull:2014iw} and larger optical outbursts in luminosity are known to occur in a small sub-class of T Tauri stars called FU Ori stars \citep{Hartmann:1996gd}. However, we do not anticipate that YSO variability would drastically change the fit results, given typical variability amplitudes modest \citep[e.g.][]{Poppenhaeger:2015hm}.

Given the relative simplicity of our model grid, most of observations are fit reasonably well and the fitted parameters have acceptable uncertainties for a large fraction of sources. Our range of $R$ values is similar to that of \citet{Furlan:2016df} in their analysis, although they used more free parameters than just the luminosity and the envelope mass. 
%LGM The mention of 330 Oph source here will naturally lead to the question of whether he did your sources and how his answer compared to yours....
This further confirms the degeneracies that exist when trying to fit for too much physics into very elaborate models. The difference in the resolutions, sensitivity, and photometric techniques for each wavelength in the SED limits the value of a more thorough analysis, especially when located in very clustered environment when extended emission and nearby sources can contaminate the measurements. 

We argue that more complex models would not help in estimating the physical parameters of YSOs - but instead, this work highlights the need for higher angular resolution at wavelengths longward of \SI{37}{\um}. Such data can be obtained in the future at arcsecond and sub-arcsecond resolution at millimeter and submillimeter wavelengths with ALMA at $>\SI{350}{\um}$. The new continuum cameras for large radio telescopes like the Green  Bank 100-meter telescope and the 50-meter Large Millimeter Telescope can produce $\sim$5 arcsecond images of the extended envelope and surrounding cloud material to provide strong constraints on the external extinction issue. It is also essential to improve the resolution of observations from 30 to 200 $\mu$m to constrain the disk and envelope masses, and refine our knowledge of dust properties in the different regions of the circumstellar environment.
%Sources with only data up to \SI{37}{\um} are likely to have poorly constrained masses, suggesting that SEDs could have the same near- to mid-IR response while having substantially different long wavelength response (see Section~\ref{sec:IRAS}). 

In our models, we have used OH5 dust as it was recommended by various authors \citep[e.g.][]{Dunham:2010bx}. However, the models for OH5 do not include scattering properties, which might jeopardize the accuracy of the models at short wavelengths, which are dominated by scattered light. However, it seems to have a opacity that fits best the clustered environments (Huard et al., 2016, in prep.). By ignoring the fact that grains are preferentially forward-scattering,  we could cause the model to fit for larger extinction values or higher inclination angles than desired. We are considering using other types of dust, such as the one used by \citet{Furlan:2016df}, which has fully detailed scattering properties, but less long-wavelength opacity. Using this type of dust would require running a new model grid with considerably more photons for each model, which we estimate would take $\sim 4$ weeks on the UMD 8-core computer we have been using. 

Finally, the issue of external extinction needs much more investigation. To date, we have not found in the literature a proper treatment of this problem. In order to re-establish self-consistency, we suggest exploring ways to add constraints on the long-wavelength flux when adding more extinction. Since the material that causes truely external extinction is presumably far away from the source, we can assume that is it at the temperature of the surrounding molecular cloud. The extinction \Av can be converted to a column density of material by assuming a dust composition and knowing its opacity at short wavelengths. By knowing the amount of material along the line of sight and assuming a temperature, we could infer how much emission is expected at long wavelengths and test if it is consistent with \Herschel mesurements along the line of sight. A second approach is to add the flux from the extinction material to the long wavelength emission from the model as part of the fitting process. This suggestion has not yet been tested or implemented.


%\begin{figure}[!h]
%\begin{center}
%\includegraphics[width=\textwidth]{Figures/incVSmass.pdf}
%\vspace{-1cm}
%\caption[Inclination versus mass]{The correlation between inclination angle and envelope mass might indicate a degeneracy of the modeling software. [I AM GOING TO DELETE THIS PICTURE]}
%\label{fig:incVSmass}
%\end{center}
%\end{figure}
%
%
%
%[I AM GOING TO DELETE THESE TWO PARAGRAPHS, AS THEY DON'T TELL MUCH STORY] Similarly to \citet{Furlan:2016df}, we find that the distribution of inclination angles for the best fits is not uniform, which is not intuitive. There is no reason why protostars should have a selection effect in their inclination angle with respect to us. This is indicative of an artifact of the fitting process, and possibly a degeneracy between inclination angle and envelope mass (see Fig.~\ref{fig:incVSmass}), which is much more prominent when no long-wavelength data is available.
%
%Finally, we observe that there is no statistically significant relationship between the spectral index and the total luminosity of the object. This is perhaps not too surprising, as the total luminosity of the object is not expected to significantly change with its evolutionary stage. In addition, this also points out that there are no significant cross-correlation between  the luminosity and envelope mass in our model. 







%\begin{figure}[!h]
%\begin{center}
%\includegraphics[width=\textwidth]{Figures/slumVSalpha.pdf}
%\label{fig:slumVSalpha}
%\vspace{-1cm}
%\caption[Luminosity versus spectral index.]{There is no apparent correlation between fitted luminosity and spectral index.}
%\end{center}
%\end{figure}

\section{A close look at IRAS~20050+2720}
\label{sec:IRAS}

In this section, we focus our attention on IRAS~20050+2720 which shows very clustered sources that are resolved for the first time in the mid-IR with our SOFIA FORCAST observations. The fields that were observed are shown in Fig.~\ref{fig:NGC2071_IRAS20050_RGB}, superimposed with IRAC 3-color images to provide some context.

\begin{figure}
\begin{center}
\includegraphics[width=0.7\textwidth]{Figures/IRAS20050_RGB.png}

\caption[IRAS~20050+2720]{IRAC 3-color images of IRAS~20050+2720. The two SOFIA fields corresponds approximately to the two white squares in the image.}
\label{fig:NGC2071_IRAS20050_RGB}
\end{center}
\end{figure}




\subsection{Overview}
IRAS~20050+2720 is part of an active site of intermediate-mass star formation in the Cygnus Rift located at \SI{700}{\pc} \citep{Wilking:1989el}, with the particularity that it doesn't seem to contain any massive stars \citep{Gunther:2012dq}. The main cluster core is associated with water and methanol masers \citep{Palla:1991up,Fontani:2010cf} and multipolar molecular outflows observed at millimeter wavelengths \citep{Bachiller:1995cy,Anglada:1998uu,Beltran:2008gu}, suggesting that the region might have experienced a episode of star formation in the past 0.1 Myr which contrasts with the average age of the cluster of 1 Myr \citep{Chen:1997tb,Gutermuth:2005hx}. \cite{Gutermuth:2009gca} have identified $>170$ YSOs surrounding the core and measured their continuum fluxes up to \SI{8}{\micro\meter} with the \Spitzer IRAC instrument. While measurements at longer wavelengths are able to provide estimates of the total luminosity of the cluster \citep[e.g. using IRAS,][\SI{388}{\Lsun}]{Molinari:1996td}, the measurements are confused in the densest region and it has not been possible to properly associate the far-IR emission with its short wavelength counterpart because of the small separation between IRAC-detected protostars. 
%LGM has it been spelled out what IRAS was? Should be somewhere
The IRAS point source was classified as a luminous class 0 protostar \citep{Bachiller:1996ja}, and its emission associated with the bright millimeter source MMS1 to the northwest of the core \citep{Chini:2001fa}, also called OVRO1 in \citet{vanKempen:2012fb}. \citet{Beltran:2008gu} show strong evidence that this region has multiple generations of stars, and suggest that a group of low-mass stars first completed their main accretion phase, before the birth of new intermediate-mass stars at the core of this cluster. A recent study by \citet{Poppenhaeger:2015hm} investigated the YSO variability in the IRAC 3.6 and \SI{4.5}{\um} fluxes in the region, and found that a large fraction exhibit variability, some as large much as 0.55~mag over periods ranging from a few days to $\sim 30$~days. 


\subsection{Observations and discussion}

We observed two fields within the cluster (see Fig.~\ref{fig:NGC2071_IRAS20050_RGB}), including the brightest core at $20^h 07^m 06.70^s +\ang{27;28;54.5}$. Multiple sources in the core can be distinguished in the IRAC maps, but the core appears extended in \Spitzer MIPS at \SI{24}{\micro\meter}, and is identified as a single source with WISE. No high resolution far-infrared continuum data longward of \SI{24}{\micro\meter} was available for this source. To our knowledge, our observations are the only mid-IR observations available that can properly resolve the YSOs in the dense region. 

\subsubsection{A clustered region with an outflow}

\begin{figure}
\begin{center}
\includegraphics[width=\textwidth]{Figures/IRAS20050_core.png}
\caption[IRAS~20050+2720 core]{\SI{37}{\um} observations of the IRAS~20050+2720 core, with the 5 identified objects. The blue contours are from a \SI{2.7}{\milli\meter} continuum emission observed by the OVRO array \citep{Beltran:2008gu} at levels from 10 to \SI{46}{\milli\Jy\per\beam} by increments of \SI{4}{\milli\Jy\per\beam}. The resolution of the \SI{2.7}{\milli\meter} beam is $\sim\ang{;;4.8}$, while the r.m.s noise is \SI{1.5}{\milli\Jy\per\beam}. The dashed line is the axis of a bipolar outflow identified by \citet{Bachiller:1995cy}. The beam shown at the bottom left represents the resolution of the FORCAST instrument.}
\label{fig:IRAS20050_core}
\end{center}
\end{figure}
We find 5 separate sources in the core which appear to share an envelope of extended emission at \SI{37}{\um}. These sources are labeled in Fig.~\ref{fig:IRAS20050_core}, and their IRAC and FORCAST photometry is summarized in Table~\ref{tab:IRAS20050fluxes}. IRAS20050.4 is coincident with the source at the northwestern end of the cluster, which is named OVRO1 in \citet{Beltran:2008gu}. Two more sources are identified with the blue millimeter wavelength continuum contours from \citet{Beltran:2008gu}, to the south and east of OVRO1, but they do not appear to correlate with our SOFIA sources. The CO outflow axis \citep[Outflow "A",][]{Bachiller:1995cy}, which is associated with IRAS20050.4, appears to be aligned with extended emission that is visible to the east of the 5 sources. This extended emission is visible in both IRAC and FORCAST, and coincides with blue-shifted CO emission in the velocity maps from \citet{Beltran:2008gu}. The emission, totalling $\sim\SI{6}{\Jy}$ at \SI{37}{\um}, appears diffuse and not connected to any particular YSO. A multi-wavelength view of the region is shown in Fig.\ref{fig:IRAS20050_mosaic}.

\renewcommand{\arraystretch}{1.5}
\begin{table}
\scriptsize
\caption[Source fluxes in IRAS~20050+2720's dense core]{Sources fluxes in IRAS~20050+2720.}
\label{tab:IRAS20050fluxes}
\vspace{-0.5cm}
\begin{longtable}{lP{2cm}P{0.7cm}P{0.7cm}P{0.7cm}P{0.7cm}P{0.7cm}P{0.7cm}P{0.7cm}P{0.7cm}P{0.7cm}}
\toprule																			
SOFIA name	&	Coordinates	&	ks			&	i1			&	i2			&	i3			&	i4			&	F11			&	F19			&	F31			&	F37			\\
	&	J2000	&	Jy			&	Jy			&	Jy			&	Jy			&	Jy			&	Jy			&	Jy			&	Jy			&	Jy			\\
\midrule																																		
IRAS20050.1	&	20h07m06.6s +27d28m48.0s	&	0.214	$\pm$	0.021	&	0.489	$\pm$	0.049	&	0.57	$\pm$	0.057	&	0.731	$\pm$	0.073	&	0.858	$\pm$	0.086	&	0.64	$\pm$	0.07	&	1.93	$\pm$	0.20	&	4.50	$\pm$	0.35	&	6.32	$\pm$	0.59	\\
IRAS20050.2	&	20h07m06.2s +27d28m49.1s	&	0.002	$\pm$	0.002	&	0.041	$\pm$	0.004	&	0.142	$\pm$	0.014	&	0.264	$\pm$	0.026	&	0.308	$\pm$	0.031	&	0.06	$\pm$	0.06	&	1.45	$\pm$	0.19	&	9.31	$\pm$	0.72	&	11.96	$\pm$	1.19	\\
IRAS20050.3	&	20h07m06.3s +27d28m56.6s	&	0.028	$\pm$	0.003	&	0.09	$\pm$	0.009	&	0.218	$\pm$	0.022	&	0.339	$\pm$	0.034	&	0.429	$\pm$	0.043	&	0.18	$\pm$	0.06	&	2.58	$\pm$	0.27	&	12.53	$\pm$	0.94	&	19.34	$\pm$	1.41	\\
IRAS20050.4	&	20h07m05.9s +27d28m59.2s	&	0.002	$\pm$	0.002	&	0.023	$\pm$	0.003	&	0.039	$\pm$	0.004	&	0.053	$\pm$	0.008	&	0.055	$\pm$	0.008	&	0.06	$\pm$	0.05	&	0.25	$\pm$	0.20	&	8.54	$\pm$	0.80	&	12.85	$\pm$	1.25	\\
IRAS20050.5	&	20h07m06.6s +27d28m53.1s	&	0.042	$\pm$	0.004	&	0.118	$\pm$	0.012	&	0.176	$\pm$	0.018	&	0.235	$\pm$	0.024	&	0.32	$\pm$	0.032	&	0.19	$\pm$	0.05	&	1.03	$\pm$	0.21	&	2.97	$\pm$	0.33	&	5.65	$\pm$	0.65	\\
IRAS20050.6	&	20h07m02.2s +27d30m26.0s	&	0.155	$\pm$	0.016	&	0.537	$\pm$	0.054	&	0.771	$\pm$	0.077	&	1.113	$\pm$	0.111	&	1.805	$\pm$	0.181	&	1.81	$\pm$	0.13	&	2.29	$\pm$	0.17	&	1.64	$\pm$	0.14	&	1.22	$\pm$	0.38	\\
IRAS20050.7	&	20h07m07.9s +27d27m15.8s	&	0.002	$\pm$	0.002	&	0.004	$\pm$	0.004	&	0.024	$\pm$	0.002	&	0.06	$\pm$	0.006	&	0.072	$\pm$	0.007	&	0.06	$\pm$	0.05	&	0.11	$\pm$	0.06	&	1.15	$\pm$	0.14	&	2.09	$\pm$	0.31	\\			
\bottomrule	
	\end{longtable} 
\end{table}

% : this requires a mechanism to keep the dust emitting at these wavelengths, since no viable heating source is available to heat this material at these distances (many thousands of au from the nearest YSO).

%Since the emission appears associated with the outflow, one possible scenario is that  the material was recently ejected from the central clump of YSOs by this powerful outflow. This could be material from the diffuse envelope which seem to surround the 5 sources, or material from one given YSO's gravitationally bound envelope. The gas and dust being ejected at high velocities \citep{Bachiller:1995cy}, it might not yet have time to completely thermalize with the surrounding medium (at which point it would not emit at these wavelengths). This scenario could be confirmed with high sensitivity sub-millimeter maps of the region, with a focus on dense gas tracers that would follow the mass in these regions. The existing maps from \citet{Beltran:2008gu} do not have sufficient sensitivity or resolution to properly identify the velocity field from the gas associated with this continuum emission.
\begin{landscape}
\begin{figure}
\begin{center}
%\hspace*{-1.2in}
\includegraphics[width=1.3\textwidth]{Figures/IRAS20050.png}
\caption[Multi-wavelength view of the IRAS~20050+2720 core]{The core of IRAS20050+2720 is seen in the four bands of the \textit{Spitzer} IRAS instrument, as well as with the four FORCAST bands. The increased resolution of FORCAST compared to previous instruments allows to match the long-wavelength emission with its short wavelength counterpart. The stretch in each image is adjusted for optimal readability. The red contours correspond to the FORCAST \SI{37}{\micro\meter} emission at 0.03, 0.07, 0.13, 0.2, 0.3 and 0.4 Jy.}
\label{fig:IRAS20050_mosaic}
\end{center}
\end{figure}
\end{landscape}



A likely explanation for this emission is that the outflow from IRAS20050.4 (MMS1/OVRO1) is colliding with cold material in the surrounding cloud, creating a shock layer and heating the dust a few hundreds of K. This could explain the arc shape of the emission. The hypothesis is supported by the SED for the extended emission (Fig.~\ref{fig:IRAS20050_ext_SED}) which exhibits strong features at 5.8 and \SI{8}{\um} (IRAC bands 3 and 4 respectively). These bands are known to have multiple PAH broad emission lines, and significant PAH emission has been found to be associated with other outflows \citep{NoriegaCrespo:2004fe,Hudgins:2004wa} as the shock energy and the direct starlight hitting the dust through the outflow cavity excite these emission features. PAH emission is significantly weaker in IRAC band 1, while it is expected to be non-existent in band 2 \citep{NoriegaCrespo:2004fe}. The presence of excess emission in bands 3 and 4 indicates that the emission is not purely thermal.


\begin{figure}[!h]
\begin{center}
\includegraphics[width=0.6\textwidth]{Figures/IRAS20050_extended.jpg}
\caption[IRAS20050+2720 extended emission SED]{SED from extended emission to the east of the cluster, using photometry from IRAC and FORCAST in a \ang{;;2.4} radius aperture. Excess emission at 5.8 and \SI{8}{\um} (IRAC bands 3 and 4 respectively) can be attributed to PAHs excited by the shock and/or by the radiation from the outflow.}
\label{fig:IRAS20050_ext_SED}
\end{center}
\end{figure}




\subsubsection{SEDs and fitted parameters}
The 5 sources in the densest part of the cluster shown in Fig.~\ref{fig:IRAS20050_mosaic} are all highly extincted based on the slopes of the emission in the 2MASS bands and the depth of the \SI{10}{\um} silicate absorption feature (see Fig.~\ref{fig:IRAS20050_SEDs}). IRAS20050.1 has a flat spectrum out to \SI{37}{\um}, unlike the four other sources which are rising. IRAS20050.4 is the most steeply rising source, and is weak in the IRAC bands, suggesting that it is the most embedded source, which is corroborated by the fact that it is coincident with the strongest millimeter continuum source in the region. 

The fitted parameters for the 7 identified sources in our two fieldsare shown in Table~\ref{tab:IRAS20050params}. When the mass error is 0, it means that there is only one mass parameter for all the fits which are within 0.5 of \Rmin. Since no long-wavelength data is available, the envelope masses are not very well constrained. Sources 6 and 7 are far away from the main core which was discussed previously, and do not appear to be associated with the first 5 sources. We interpret these results in several ways:
\renewcommand{\arraystretch}{1.5}
\begin{table}[!h]
\scriptsize
\caption[Fitted parameters in IRAS~20050+2720]{Fitted parameters of sources in IRAS~20050+2720.}
\label{tab:IRAS20050params}
\vspace{-0.5cm}
\begin{longtable}{lccccccccc}
\toprule																			
SOFIA Name	&	Coordinates	&	$\alpha$	&	R	&	\Menv			&	\Ltot			&	$i$	&	$\Av$	\\
	&	(J2000)	&		&		&	(\si{\Msun})			&	(\si{\Lsun})			&	(\si{\degree})	&	(mag	)\\
\midrule																			
IRAS20050.1	&	20h07m06.6s +27d28m48.0s	&	0.07	&	0.74	&	0.004	$\pm$	0	&	128	$\pm$	15.3	&	65	&	9	\\
IRAS20050.2	&	20h07m06.2s +27d28m49.1s	&	1.65	&	0.77	&	0.58	$\pm$	0.22	&	26.6	$\pm$	6.0	&	19	&	14	\\
IRAS20050.3	&	20h07m06.3s +27d28m56.6s	&	1.13	&	0.73	&	0.26	$\pm$	0.11	&	48.5	$\pm$	6.3	&	27	&	5	\\
IRAS20050.4	&	20h07m05.9s +27d28m59.2s	&	1.71	&	0.27	&	0.38	$\pm$	0.32	&	48.5	$\pm$	8.8	&	43	&	5	\\
IRAS20050.5	&	20h07m06.6s +27d28m53.1s	&	0.54	&	0.78	&	0.01	$\pm$	0	&	49.4	$\pm$	6.2	&	43	&	14	\\
IRAS20050.6	&	20h07m02.2s +27d30m26.0s	&	-0.34	&	2.22	&	0.004	$\pm$	0	&	201.6	$\pm$	32.1	&	81	&	14	\\
IRAS20050.7	&	20h07m07.9s +27d27m15.8s	&	1.29	&	1.50	&	0.015	$\pm$	0.09	&	3.5	$\pm$	3.6	&	0	&	14	\\
\bottomrule	
\end{longtable} 
\caption*{\textbf{Notes}: The envelope masses are not well constrained due to the lack of long-wavelength emission. When the mass error is 0, it means that there is only one mass parameter for all the fits which are within 0.5 of \Rmin.}
\end{table}

\begin{itemize}
\item Generally, the best-fitting luminosity is better constrained than the masses ($10-25\%$ uncertainties in luminosity for $R<1$);
\item The sum of the protostellar luminosities is $\sim \SI{300}{\Lsun}$, which is consistent with IRAS luminosity measurements of the entire region of \SI{388}{\Lsun}.
\item Sources 1 and 5 appear to be at a later stage of their evolution, with a lower spectral index and much lower envelope mass. However, less models in our grid fit these models well, as all well-fitting models have the same mass at the very edge of our range of grid parameters;
\item Sources 2, 3 and 4 are more embedded, with steeply rising SOFIA fluxes. They are consistent with having sub-solar mass envelopes, but the uncertainties on the envelope mass are large due to the lack of long-wavelength measurements;
\item Source 6 fits less well and appears to have a very low envelope mass, as the SOFIA fluxes are decreasing with increasing wavelength.
\item Source 7 has the most fractional scatter in terms of envelope mass, as well as in luminosity. The fits from this source do not allow to draw meaning conclusions. It would greatly benefit from having a high-resolution data point at long wavelength.
\end{itemize}

IRAS20050.6 is the poorest fit in this cluster, and it shows the limit of our grid of models. Judging by the shape of the SED and its negative spectral index, we can conclude that this object is not a Class 0 or Class I protostar with a large envelope. In fact, it is classified as a Class II YSO in \citet{Gutermuth:2009gca}, and our grid is not particularly well-suited to fit sources of this type. By doing some exploration of the parameter space around the best fit, it appears that the emission could potentially be fitted by a much smaller disk outer radius with no envelope. The smaller disk size is required to reduce long-wavelength emission, as all the dust stays warmer and emits only at shorter wavelengths. In order to by thorough, we might decide to run a larger grid to fit this type of objects as well. 

\begin{figure}
\begin{center}
\includegraphics[width=\textwidth]{Figures/IRAS20050_SEDs.png}
\caption[IRAS20050+2720 SEDs]{SEDs of the 7 sources in the two fields. }
\label{fig:IRAS20050_SEDs}
\end{center}
\end{figure}

\subsubsection{Diffuse emission}
In testing the various scenarios of star formation, it is useful to obtain a measure of how much mass is available for the YSOs to grow after their original collapse. For this, clustered regions such as this one are an ideal laboratory since the YSOs appear to share an envelope. In this cluster, the typical separations between the sources are \ang{;;6}-\ang{;;8}, which correspond to projected distances of \num{4200}-\SI{5600}{\au}. This strongly indicates that the envelopes of individual YSOs are interacting with each other.

\renewcommand{\arraystretch}{1.5}
\begin{table}[!h]
\scriptsize
\caption[Clustered sources in IRAS~20050+2720's dense core]{Clustered sources in the densest region of IRAS~20050+2720.}
\label{tab:IRAS20050sum}
\vspace{-0.5cm}
\begin{longtable}{lP{2cm}P{2cm}P{2cm}P{2cm}}
\toprule																			
SOFIA name	&	F11	&	F19	&	F31	&	F37	\\
	&	Jy	&	Jy	&	Jy	&	Jy\\
\midrule									
IRAS20050.1	&	0.64	&	1.93	&	4.50	&	6.32	\\
IRAS20050.2	&	0.06	&	1.45	&	9.31	&	11.96	\\
IRAS20050.3	&	0.18	&	2.58	&	12.53	&	19.34	\\
IRAS20050.4	&	0.06	&	0.25	&	8.54	&	12.85	\\
IRAS20050.5	&	0.19	&	1.03	&	2.97	&	5.65	\\
\midrule									
Sum of point sources in cluster	&	1.13	&	7.24	&	37.84	&	56.11	\\
Total cluster emission	&	1.79	&	7.07	&	37.36	&	49.33	\\
Ratio	&	1.58	&	0.98	&	0.99	&	0.88	\\
\bottomrule					
	\end{longtable} 
	\caption*{\textbf{Notes}: The "total cluster" emission corresponds to the entirety of the region shown in Fig.~\ref{fig:IRAS20050_mosaic}, which is then background-subtracted.}
\end{table}

However, measuring the flux from each individual source in these clustered regions is challenging, since the sources are so close together. With an aperture of \ang{;;2.4} (3-pixel radius on FORCAST), we managed to put non-overlapping apertures for all the 5 sources in IRAS~20050+2720. However, since the aperture correction was derived considering a "total flux" aperture to be $\sim$12 pixel radius, we are accounting for the same flux multiple times, even if the apertures are not overlapping. We estimate the \SI{37}{\um} flux from the eastern extended emission to total $\sim\SI{6}{\Jy}$, we obtain about 22\% of excess \SI{37}{\um} flux when comparing the sum of the point sources and the total emission from the cluster (see Table~\ref{tab:IRAS20050sum}). At \SI{31}{\um}, the flux excess is only about 10\%. At \SI{19}{\um} and below, the extended emission is within the noise uncertainty of the map. 

This excess flux can only partially be explained by the tails of the PSF extending well below the aperture size (see Fig.~\ref{fig:averageEE}), with 10-15\% of the total energy still existing in the annulus outward of 8~pixels (\ang{;;6}) from the aperture center. However, the contribution of a source to any given other source is only a fraction of this since it would only correspond to the amount of flux within a 3-pixel aperture. We conclude that the PSF shape is not responsible for the bulk of the observed excess flux at both wavelengths.

One possible explanation would be that diffuse thermal emission occurs across the entire region. This could be caused by heating internal to the cluster (powered by the outflow, for example, like the eastern extended emission) or by a population of stochastically heated very small grains, which are not in LTE. The high outflow activity in this region could carve out multiple cavities which facilitate short-wavelength photons from the individual stars to reach out to larger distances within the envelopes and the shared mass reservoir. At \SI{37}{\um}, the level of diffuse emission required to account for the excess flux is about \SI{0.05}{\Jy\per\pixel}, which is the same as the average diffuse emission in the eastern region. Such an explanation would also help account for the high amount of external extinction that is needed to fit most of the SEDs in this region.

This tends to favor a scenario where protostars are fragmenting from a cloud and continue accreting material within that original envelope. The envelopes of neighboring YSOs interact, and possibly can exchange material as some YSOs become more massive (competitive accretion). 


\subsubsection{Conclusions on IRAS~20050+2720}

We have determined the photometry for 7 objects in IRAS~20050+2720. 5 of these objects are highly clustered and our FORCAST data is the first mid- to far-IR photometry for these sources. Our findings can be summarized as follows:
\begin{itemize}
\item Fitted luminosities for the 5 clustered sources are between 26 and \SI{128}{\Lsun}, with estimated scatter ranging from 10 to 25\%. 
\item IRAS20050.1 and IRAS20050.5 have smaller envelope mass estimates compared to the other 3 sources, which is consistent with the different in their spectral index. Fitted masses are less robust and show more scatter for the most embedded sources 2, 3 and 4, which are limited by the lack of far-IR, high-resolution data points.
\item We detect extended emission to the east of the main core, which is strongest in the 31 and \SI{37}{\um} images. It appears to be associated with the blue lobe of the outflow coming from IRAS20050.1 (MMS1/OVRO1). We argue that the emission is arising from shock-heated material where the outflow is impacting the cloud, and we suggest that the enhanced IRAC band 3 and 4 fluxes are a signature of PAHs emission, which can be characteristic of such outflow regions.
\item The grid might need to be expanded to lower masses and/or to smaller disk and envelope sizes to manage to fit Class II sources such as IRAS20050.6.
\item Finally, the inconsistency between the sum of point source fluxes and the total cluster emission at 31 and \SI{37}{\um} could be explained by the presence of an extended, diffuse component the 5 clustered sources appear to share. This is consistent with competitive accretion theory of clustered star formation in which multiple cores will attempt to accrete mass from a same, shared envelope.
%\item Our findings support the work by \citet{Beltran:2008gu} who suggest that there are multiple generations of star formation coexisting in the same cluster, judging by the spectral indices and the fitted masses.
\end{itemize}


%This highlights the complexity of the physical processes in these regions, which 
%LGM Your luminosities fits do not support that 1 and 5 are lower luminosity that 2,3,4 so why are you repeating Beltran story?

%LGM STOPPED READING HERE
%
%\subsection{NGC 2071}
%
%\subsubsection{Context}
%The NGC~2071 star-forming region is one of several active areas of star formation in the northern part of L1630 giant molecular cloud which is located at a distance of \SI{422}{\pc} \citep{vanDishoeck:2011em}. NGC~2071 itself is a reflection nebula.
%The NGC~2071 infrared cluster, located about 4' north of the reflection nebula, is a region of intermediate mass star formation \citep{Strom1976, Persson1981, Butner1990}. Maps of the cloud in CO and its isotopomers \citep{Buckle2010} show a large scale clump with $\sim\SI{1000}{\Msun}$ associated with the cluster. Dust continuum emission at $\lambda$=0.85 and \SI{1.3}{\milli\meter} peaks on center of the cluster extending ~1' in diameter containing ~\SI{30}{\Msun} of gas and dust \citep{Johnstone2001,Mitchell2001,Launhardt1996}. Emission from CS in the J=2-1 through J=7-6 indicate that the gas in this region is centrally condensed with a density of $sim\SI{1e6}{\raiseto{-3}\centi\meter}$ \citep{Zhou1990}. 
%
%There are a number of near infrared surveys of the young cluster \citep[e.g.,][]{Strom1976,Lada1991,Megeath2012,Spezzi2015}. \citet{Spezzi2015} identify 52 YSOs associated with the NGC~2071 cluster, with the majority Class II sources. \citet{Flaherty2008} estimate an age of $\sim$\SI{2}{\mega\year} for the cluster, consistent with the large fraction of Class II sources \citep{Evans2009}. The brightest far infrared emission from the cluster is associated with the IRS1 region \citep{Harvey1979,Butner1990}, which has an estimated total luminosity of \SI{520}{\Lsun}. The immediate region of IRS 1 is, in fact, home to a number of YSOs that are infrared, X-ray, and radio sources \citep{Skinner2009,C-G2012,Kempen2012}. The radio \citep{C-G2012} and H$_2$ emission line imaging indicate that IRS~1, IRS~2, IRS~3, and, perhaps, VLA~1 are YSOs with outflows. The larger scale molecular outflow associated with this region is well studied in a number of molecules \citep{Bally1982,Chernin1993,Stoji2008}.
%
%Figure \ref{fig:NGC2071_Lee} shows the Spitzer 3.1~$\mu$m image of the IRS~1 region on the left \citep[image from Spitzer Archive:][]{Megeath2012} and the Herschel 70~$\mu$m image on the right (image from Herschel Archive: Gould Belt Project, P.I. Andr\'e). The plus marks in both panels indicate the position of the brighter YSOs: IRS~1, IRS~2, IRS~3, IRS~4, and VLA~1. The inner red circle with a diameter of \ang{;;26} indicates the extend of the saturated region in the Spitzer MIPS \SI{24}{\um} image; the outer red circle, diameter \ang{;;60}, encompasses the region with strong imaging artifacts in the MIPS \SI{24}{\um} image.
%The right panel shows Herschel 70~$\mu$m image which does not resolve the emission from IRS~1, IRS~2, IRS~3, and VLA~1. The centroid of the \SI{24}{\um} and \SI{70}{\um} emission is between IRS~1 and VLA~1 indicating that several of the sources are contributing to the total observed emission. Interferometric observations show that the millimeter wavelength dust emission is dominated by envelopes associated with IRS~1 and IRS~3, with estimated masses of 8.2 and \SI{12.3}{\Msun} material, respectively \citep{Kempen2012}. The millimeter emission also reveals the presence of disks with radii $\le$\SI{100}{\au} associated with IRS~1 and IRS~3 \citep{Kempen2012}.
%
%The luminosities and masses of the individual source, IRS~1, IRS~2, IRS~3, and VLA~1, are not known. The Spectral Energy Distributions (SEDs) shortward of 10~$\mu$m support their identification as embedded YSOS \citep{Skinner2009}. \citet{Skinner2009} gives a clear discussion of the possibilities for IRS~1 and concludes that it is likely a mid-to late B~star. \cite{Kempen2012} find luminosities of 10, 3.4, and $\le$\SI{27}{\Lsun} for IRS~1, 2, and 3, respectively, and stellar masses of $\le$\SI{1}{\Msun} for each, based on SED fitting. These masses and luminosities are not consistent with estimate of the total luminosity of the region of \SI{520}{\Lsun} \citep{Butner1990}. The far infrared images from Herschel reveal that IRS~1 alone does not totally dominate, as seen in Fig.~\ref{fig:NGC2071_Lee}; IRS~1, VLA~1, and IRS~3 likely make substantial contributions to the emission with lesser emission from IRS~2 and IRS~4.
%
%\subsubsection{Observations and discussion}
%
%\begin{figure}
%\begin{center}
%\includegraphics[width=\textwidth]{Figures/Lee_NGC2071.png}
%
%
%\caption[Multi-wavelength view of the NGC~2071 core]{The core of NGC2071 is seen in two bands of the \textit{Spitzer} IRAC instrument ("I1" and "I4"), as well as with the four FORCAST bands. The increased resolution of FORCAST compared to previous instruments allows to match the long-wavelength emission with its short wavelength counterpart. The stretch in each image is adjusted for optimal readability. The red contours correspond to the FORCAST \SI{37}{\micro\meter} emission, between 0.1 and 2.4~Jy by increment of 0.25~Jy. }
%\label{fig:NGC2071_Lee}
%\end{center}
%\end{figure}
%
%
%\begin{landscape}
%\begin{figure}
%\begin{center}
%\includegraphics[width=1.4\textwidth]{Figures/NGC2071_mosaic.png}
%
%
%\caption[Multi-wavelength view of the NGC~2071 core]{The core of NGC2071 is seen in two bands of the \textit{Spitzer} IRAC instrument ("I1" and "I4"), as well as with the four FORCAST bands. The increased resolution of FORCAST compared to previous instruments allows to match the long-wavelength emission with its short wavelength counterpart. The stretch in each image is adjusted for optimal readability. The red contours correspond to the FORCAST \SI{37}{\micro\meter} emission, between 0.1 and 2.4~Jy by increment of 0.25~Jy. }
%\label{fig:NGC2071_mosaic}
%\end{center}
%\end{figure}
%\end{landscape}
%
%
%\renewcommand{\arraystretch}{1.5}
%\begin{table}[!h]
%\scriptsize
%\caption[Sources in NGC2071's dense core]{Sources in the densest region of NGC2071.}
%\label{tab:NGC2071sum}
%\vspace{-0.5cm}
%\begin{longtable}{lP{2cm}P{2cm}P{2cm}P{2cm}}
%\toprule																			
%SOFIA name	&	F11	&	F19	&	F31	&	F37	\\
%	&	Jy	&	Jy	&	Jy	&	Jy	\\
%\midrule									
%NGC2071.1	&	10.07	&	72.041	&	167.93	&	234.93	\\
%NGC2071.2	&	0.38	&	11.207	&	56.70	&	89.55	\\
%NGC2071.3	&	0.19	&	3.027	&	19.97	&	37.56	\\
%\midrule									
%Sum of point sources in cluster	&	10.65	&	86.28	&	244.61	&	362.03	\\
%Total cluster emission	&	13.523	&	94.16	&	280.14	&	362.99	\\
%Ratio	&	1.27	&	1.09	&	1.15	&	1.00	\\
%\bottomrule					
%	\end{longtable} 
%\end{table}
%
%
%\renewcommand{\arraystretch}{1.5}
%\begin{table}[!h]
%\scriptsize
%\caption[Fitted parameters in NGC2071]{Fitted parameters of sources in NGC2071.}
%\label{tab:NGC2071params}
%\vspace{-0.5cm}
%\begin{longtable}{lcccccccccc}
%\toprule																					
%SOFIA Name	&	Coordinates	&	$\alpha$	&	R	&	\Menv			&	\Ltot			&	\Lbol	&	$i$	&	$\Av$	\\
%	&	J2000	&		&		&	\si{\Msun}			&	\si{\Lsun}			&	\si{\Lsun}	&	\si{\degree}	&	mag	\\
%\midrule																					
%NGC2071.1	&	05h47m04.8s +00d21m43.1s	&	2.31	&	2.83	&	22.17	$\pm$	2.597	&	43.7	$\pm$	3.3	&	297.2	&	0	&	14	\\
%NGC2071.2	&	05h47m04.7s +00d21m48.2s	&	2.22	&	1.31	&	1.30	$\pm$	0.294	&	74.1	$\pm$	12.7	&	199.8	&	19	&	13	\\
%NGC2071.3	&	05h47m05.4s +00d21m50.3s	&	1.01	&	1.51	&	0.17	$\pm$	0.315	&	28.8	$\pm$	7.8	&	113.7	&	19	&	14	\\
%NGC2071.4	&	05h47m04.0s +00d22m10.5s	&	1.08	&	2.57	&	0.001	$\pm$	0.002	&	111.1	$\pm$	62.2	&	21.4	&	84	&	13	\\
%NGC2071.5	&	05h47m10.7s +00d21m14.0s	&	0.32	&	0.96	&	0.002	$\pm$	0.001	&	39.9	$\pm$	5.1	&	14.9	&	27	&	14	\\
%\bottomrule																																										
%	\end{longtable} 
%\end{table}
%
%Show sum of sources compared to cluster total
%
%
%\begin{figure}
%\begin{center}
%\includegraphics[width=\textwidth]{Figures/NGC2071_SEDs.png}
%\caption[NGC2071 SEDs]{SEDs of the 5 sources in the two fields. }
%\label{fig:NGC2071_SEDs}
%\end{center}
%\end{figure}



\section{Conclusion and future work}

We have used SOFIA FORCAST to image 42 fields in bright, nearby stellar clusters. We derive aperture photometry in 4 bands: \SI{11.1}{\um}, \SI{19.7}{\um}, \SI{31.5}{\um}, \SI{37.1}{\um}, for a total of 70 point sources and 14 extended sources. In many cases, our photometry is the only mid- to far-IR photometry available for these sources, since archival \Spitzer observations were either saturated or confused.

In most cases, we complete our SOFIA photometry using \Spitzer IRAC and 2MASS data to produce SEDs from \SI{1.2}{\um} to \SI{37}{\um}. In a limited number of cases, we also obtained \Herschel data. When the photometry catalogs cannot be found, we use the same photometry pipeline that we developed for SOFIA on the \Spitzer and \Herschel calibrated images.

%In our sample there were 15 cases of extended emission at \SI{37}{\um}. This spatial extension is not simple to model: with a FORCAST FWHM of $\sim\ang{;;3.5}$, an object with spatial extension has a size on the order of a few thousands of \si{\au} at \SI{500}{\pc}: we haven't been able to show that the central object can heat dust grains at this distance sufficiently for them to emit thermally at \SI{37}{\um}. Hence, another heating mechanism is responsible for this emission: we suggest that the emission could be due to a population of non-LTE, very small dust grains, or a proximity to outflows. Answering this question will require further analysis and more study.

We proceed to SED fitting of a subset of our sources, based on a radiative transfer code called Hyperion. Starting from a standard model, we argue that there are 4 primary parameters that are needed to model the SEDs for Class 0 and I YSOs: the central luminosity, the envelope mass, the inclination angle, and an external extinction component. A scaling factor can be used as a proxy for finer luminosity sampling in the grid. We argue that as a system approaches the boundary between Class I and Class II, a fifth parameter might need to be varied in the model: the disk/envelope outer radius. Reducing this radius is necessary to appropriately reducing the emission around \SI{100}{\um} while maintaining the fluxes at shorter wavelength, as exhibited for example in our source IRAS20050.6. 

Fits for most our sources are reasonable. The luminosity of the family of best fits is usually constrained well, with scatter usually less than 25\%. We attribute this good accuracy to the FORCAST 31 and \SI{37}{\um} bands which really sample the envelope. This typical scatter does not change when long-wavelength data points are not available, for example in IRAS~20050+2720. Envelope masses, however, are constrained much better when long-wavelength data (such as \Herschel) is available. Unfortunately, for most of our sources the long-wavelength data come from single-dish telescopes, which do not have sufficient angular resolution to guarantee that the measured flux is associated with the source; the measured emission could belong to an extended component, or to another nearby source. 

We find that the fitted luminosity is substantially different from the observationally-defined bolometric luminosity of our sources (which corresponds to the integral of the observed data points), with a high dependence on the inclination angle of the fit. We argue that the fitted luminosity is a more accurate measurement of the central luminosity (which is composed of the source's luminosity and the accretion luminosity). Indeed, the bolometric luminosity is highly geometry-dependent, as a source seen edge-on will exhibit a dramatically lower bolometric luminosity as opposed to sources seen through the throat of the cavity, because of the line of sight passes through the disk which has a lot of opacity. SED fitting allows to lift this degeneracy and provide a more robust estimate of the actual luminosity.

Finally, we discuss several caveats with the fitting methods that are traditionally employed, such as the use of an external extinction factor which is modeled purely as extinction with emission counterpart. While this could matter substantially for estimating masses, we argue that its impact on the luminosity determination is small, provided that the allowed range of extinction remains reasonable. 


%\subsection{NGC 2071}
%
%\subsubsection{Context}
%The NGC~2071 star-forming region is one of several active areas of star formation in the northern part of L1630 giant molecular cloud which is located at a distance of \SI{422}{\pc} \citep{vanDishoeck:2011em}. NGC~2071 itself is a reflection nebula.
%The NGC~2071 infrared cluster, located about 4' north of the reflection nebula, is a region of intermediate mass star formation \citep{Strom1976, Persson1981, Butner1990}. Maps of the cloud in CO and its isotopomers \citep{Buckle2010} show a large scale clump with $\sim\SI{1000}{\Msun}$ associated with the cluster. Dust continuum emission at $\lambda$=0.85 and \SI{1.3}{\milli\meter} peaks on center of the cluster extending ~1' in diameter containing ~\SI{30}{\Msun} of gas and dust \citep{Johnstone2001,Mitchell2001,Launhardt1996}. Emission from CS in the J=2-1 through J=7-6 indicate that the gas in this region is centrally condensed with a density of $sim\SI{1e6}{\raiseto{-3}\centi\meter}$ \citep{Zhou1990}. 
%
%There are a number of near infrared surveys of the young cluster \citep[e.g.,][]{Strom1976,Lada1991,Megeath2012,Spezzi2015}. \citet{Spezzi2015} identify 52 YSOs associated with the NGC~2071 cluster, with the majority Class II sources. \citet{Flaherty2008} estimate an age of $\sim$\SI{2}{\mega\year} for the cluster, consistent with the large fraction of Class II sources \citep{Evans2009}. The brightest far infrared emission from the cluster is associated with the IRS1 region \citep{Harvey1979,Butner1990}, which has an estimated total luminosity of \SI{520}{\Lsun}. The immediate region of IRS 1 is, in fact, home to a number of YSOs that are infrared, X-ray, and radio sources \citep{Skinner2009,C-G2012,Kempen2012}. The radio \citep{C-G2012} and H$_2$ emission line imaging indicate that IRS~1, IRS~2, IRS~3, and, perhaps, VLA~1 are YSOs with outflows. The larger scale molecular outflow associated with this region is well studied in a number of molecules \citep{Bally1982,Chernin1993,Stoji2008}.
%
%Figure \ref{fig:NGC2071_Lee} shows the Spitzer 3.1~$\mu$m image of the IRS~1 region on the left \citep[image from Spitzer Archive:][]{Megeath2012} and the Herschel 70~$\mu$m image on the right (image from Herschel Archive: Gould Belt Project, P.I. Andr\'e). The plus marks in both panels indicate the position of the brighter YSOs: IRS~1, IRS~2, IRS~3, IRS~4, and VLA~1. The inner red circle with a diameter of \ang{;;26} indicates the extend of the saturated region in the Spitzer MIPS \SI{24}{\um} image; the outer red circle, diameter \ang{;;60}, encompasses the region with strong imaging artifacts in the MIPS \SI{24}{\um} image.
%The right panel shows Herschel 70~$\mu$m image which does not resolve the emission from IRS~1, IRS~2, IRS~3, and VLA~1. The centroid of the \SI{24}{\um} and \SI{70}{\um} emission is between IRS~1 and VLA~1 indicating that several of the sources are contributing to the total observed emission. Interferometric observations show that the millimeter wavelength dust emission is dominated by envelopes associated with IRS~1 and IRS~3, with estimated masses of 8.2 and \SI{12.3}{\Msun} material, respectively \citep{Kempen2012}. The millimeter emission also reveals the presence of disks with radii $\le$\SI{100}{\au} associated with IRS~1 and IRS~3 \citep{Kempen2012}.
%
%The luminosities and masses of the individual source, IRS~1, IRS~2, IRS~3, and VLA~1, are not known. The Spectral Energy Distributions (SEDs) shortward of 10~$\mu$m support their identification as embedded YSOS \citep{Skinner2009}. \citet{Skinner2009} gives a clear discussion of the possibilities for IRS~1 and concludes that it is likely a mid-to late B~star. \cite{Kempen2012} find luminosities of 10, 3.4, and $\le$\SI{27}{\Lsun} for IRS~1, 2, and 3, respectively, and stellar masses of $\le$\SI{1}{\Msun} for each, based on SED fitting. These masses and luminosities are not consistent with estimate of the total luminosity of the region of \SI{520}{\Lsun} \citep{Butner1990}. The far infrared images from Herschel reveal that IRS~1 alone does not totally dominate, as seen in Fig.~\ref{fig:NGC2071_Lee}; IRS~1, VLA~1, and IRS~3 likely make substantial contributions to the emission with lesser emission from IRS~2 and IRS~4.
%
%\subsubsection{Observations and discussion}
%
%\begin{figure}
%\begin{center}
%\includegraphics[width=\textwidth]{Figures/Lee_NGC2071.png}
%
%
%\caption[Multi-wavelength view of the NGC~2071 core]{The core of NGC2071 is seen in two bands of the \textit{Spitzer} IRAC instrument ("I1" and "I4"), as well as with the four FORCAST bands. The increased resolution of FORCAST compared to previous instruments allows to match the long-wavelength emission with its short wavelength counterpart. The stretch in each image is adjusted for optimal readability. The red contours correspond to the FORCAST \SI{37}{\micro\meter} emission, between 0.1 and 2.4~Jy by increment of 0.25~Jy. }
%\label{fig:NGC2071_Lee}
%\end{center}
%\end{figure}
%
%
%\begin{landscape}
%\begin{figure}
%\begin{center}
%\includegraphics[width=1.4\textwidth]{Figures/NGC2071_mosaic.png}
%
%
%\caption[Multi-wavelength view of the NGC~2071 core]{The core of NGC2071 is seen in two bands of the \textit{Spitzer} IRAC instrument ("I1" and "I4"), as well as with the four FORCAST bands. The increased resolution of FORCAST compared to previous instruments allows to match the long-wavelength emission with its short wavelength counterpart. The stretch in each image is adjusted for optimal readability. The red contours correspond to the FORCAST \SI{37}{\micro\meter} emission, between 0.1 and 2.4~Jy by increment of 0.25~Jy. }
%\label{fig:NGC2071_mosaic}
%\end{center}
%\end{figure}
%\end{landscape}
%
%
%\renewcommand{\arraystretch}{1.5}
%\begin{table}[!h]
%\scriptsize
%\caption[Sources in NGC2071's dense core]{Sources in the densest region of NGC2071.}
%\label{tab:NGC2071sum}
%\vspace{-0.5cm}
%\begin{longtable}{lP{2cm}P{2cm}P{2cm}P{2cm}}
%\toprule																			
%SOFIA name	&	F11	&	F19	&	F31	&	F37	\\
%	&	Jy	&	Jy	&	Jy	&	Jy	\\
%\midrule									
%NGC2071.1	&	10.07	&	72.041	&	167.93	&	234.93	\\
%NGC2071.2	&	0.38	&	11.207	&	56.70	&	89.55	\\
%NGC2071.3	&	0.19	&	3.027	&	19.97	&	37.56	\\
%\midrule									
%Sum of point sources in cluster	&	10.65	&	86.28	&	244.61	&	362.03	\\
%Total cluster emission	&	13.523	&	94.16	&	280.14	&	362.99	\\
%Ratio	&	1.27	&	1.09	&	1.15	&	1.00	\\
%\bottomrule					
%	\end{longtable} 
%\end{table}
%
%
%\renewcommand{\arraystretch}{1.5}
%\begin{table}[!h]
%\scriptsize
%\caption[Fitted parameters in NGC2071]{Fitted parameters of sources in NGC2071.}
%\label{tab:NGC2071params}
%\vspace{-0.5cm}
%\begin{longtable}{lcccccccccc}
%\toprule																					
%SOFIA Name	&	Coordinates	&	$\alpha$	&	R	&	\Menv			&	\Ltot			&	\Lbol	&	$i$	&	$\Av$	\\
%	&	J2000	&		&		&	\si{\Msun}			&	\si{\Lsun}			&	\si{\Lsun}	&	\si{\degree}	&	mag	\\
%\midrule																					
%NGC2071.1	&	05h47m04.8s +00d21m43.1s	&	2.31	&	2.83	&	22.17	$\pm$	2.597	&	43.7	$\pm$	3.3	&	297.2	&	0	&	14	\\
%NGC2071.2	&	05h47m04.7s +00d21m48.2s	&	2.22	&	1.31	&	1.30	$\pm$	0.294	&	74.1	$\pm$	12.7	&	199.8	&	19	&	13	\\
%NGC2071.3	&	05h47m05.4s +00d21m50.3s	&	1.01	&	1.51	&	0.17	$\pm$	0.315	&	28.8	$\pm$	7.8	&	113.7	&	19	&	14	\\
%NGC2071.4	&	05h47m04.0s +00d22m10.5s	&	1.08	&	2.57	&	0.001	$\pm$	0.002	&	111.1	$\pm$	62.2	&	21.4	&	84	&	13	\\
%NGC2071.5	&	05h47m10.7s +00d21m14.0s	&	0.32	&	0.96	&	0.002	$\pm$	0.001	&	39.9	$\pm$	5.1	&	14.9	&	27	&	14	\\
%\bottomrule																																										
%	\end{longtable} 
%\end{table}
%
%Show sum of sources compared to cluster total
%
%
%\begin{figure}
%\begin{center}
%\includegraphics[width=\textwidth]{Figures/NGC2071_SEDs.png}
%\caption[NGC2071 SEDs]{SEDs of the 5 sources in the two fields. }
%\label{fig:NGC2071_SEDs}
%\end{center}
%\end{figure}



%\section{Conclusion and future work}
%
%We have used SOFIA FORCAST to image 42 fields in bright, nearby stellar clusters. We derive aperture photometry in 4 bands: \SI{11.1}{\um}, \SI{19.7}{\um}, \SI{31.5}{\um}, \SI{37.1}{\um}, for a total of 90 sources. In many cases, our photometry is the only mid- to far-IR photometry available for these sources, since archival \Spitzer observations were either saturated or confused.
%
%In multiple cases, we complete our SOFIA photometry using \Spitzer IRAC as well as \Herschel data. When the catalogs cannot be found, we use the same photometry pipeline that we developed for SOFIA on the \Spitzer and \Herschel calibrated images.
%
%We also proceed to SED fitting of our sources, based on a radiative transfer code called Hyperion. Using a simple grid, we produce estimates for physical parameters of these YSOs, and carefully approximate the error in the parameter estimate.
%
%We take a closer look at two special clusters: IRAS~20050+2720, which contains five close-by YSOs sharing what appears to be an extended envelope, favoring a competitive accretion scenario; and NGC2071 [insert conclusions here]
%
%In our sample there were 15 cases of extended emission at \SI{37}{\um}. This spatial extension is not simple to model: with a FORCAST FWHM of $\sim\ang{;;3.5}$, an object with spatial extension has a size on the order of a few thousands of \si{\au} at \SI{500}{\pc}: we haven't been able to show that the central object can heat dust grains at this distance sufficiently for them to emit thermally at \SI{37}{\um}. Hence, another heating mechanism is responsible for this emission: we suggest that the emission could be due to a population of non-LTE, very small dust grains. Answering this question will require further analysis and more study.
%\section{Introduction}
%Most stars in the Galaxy form in cluster environments of sizes 2-4 pc, often containing more than 100 young stellar objects (YSOs), with typical separations of $<$0.05~pc between stars near their centers \citep{Porras:2003kxa, Allen:2007wqa, Gutermuth:2009gca}.
%Previous studies have been effective in elucidating the young stellar content and distribution in clouds on large scales (parsec down to 0.05~pc) \citep{Evans-ARAA2012}, but young cluster cores, born in dense portions of molecular clouds, are more difficult to observe. They are obscured at optical through near-IR wavelengths. At mid-IR through far-IR wavelengths, the material surrounding YSOs and involved in the stellar birth process emits due to heating by the young stars, but the resolution to date has not been sufficient to isolate individual stars in the cores of most nearby young clusters.
%
%Space telescopes such as \textit{Spitzer} and WISE have tremendous sensitivity, which made them so scientifically productive, but it limits their utility in the densest regions of star-forming clusters because of detector saturated and imaging artifacts (See \citep{2008ApJ...672.1013P} for examples). This is particularly problematic at wavelengths of \SI{24}{\micro\meter} and beyond, where a bright cluster star can dominate a region 3-5 nominal resolution elements out from the star. In fact, it is often difficult for \textit{Spitzer} and WISE to provide good flux estimates for even the brightest YSOs in the cores of clusters.
%
%The FORCAST instrument on SOFIA provides the opportunity to study cluster cores at 10 to \SI{37}{\micro\meter} with better angular resolution than \textit{Spitzer} and WISE, without saturation even on the brightest sources. Although it is less sensitive than space-based instrument at comparable wavelengths due to the large thermal background noise at 13~km altitude, FORCAST images can lift degeneracies in assigning flux by separating sources that were previously unresolved or hidden by saturation artifacts. The mid-infrared fluxes of very clustered objects are essential contraints on their YSO's spectral energy distribution (SED) which is used to determine luminosity and evolutionary state.
%
%This paper presents the results for two clusters, IRAS~200050
%and NGC~2071, which were observed as part of a FORCAST survey program to observe bright, nearby star-forming cluster cores for which the \textit{Spitzer} and WISE archival data show extensive amounts of saturation and source confusion based on near infrared images. 
%
%Section 2 provides an overview of what is know about the two clusters. In section 3, we describe our data and reduction methods in detail, and discuss systematic of the FORCAST instrument. Section 4 presents our SOFIA images and discusses them in the context of other observations. Section 5 discusses flux measurements for cluster sources at other wavelengths and outlines our procedures for deriving improved fluxes where applicable. Section 6 presents Spectral Energy Distributions and fits for the SOIFA sources, and section 7 discusses our findings.
%
%\section{Target Clusters}
%IRAS~20050+2720 and NGC~2071 are embedded young stellar clusters with total luminosities in the cores that are characteristic of intermediate mass YSOs. The following two subsections provide overviews of each cluster and its environment.
%
%\subsection{IRAS20050+2720}
%
%IRAS~20050+2720 is part of an active site of intermediate-mass star formation in the Cygnus Rift located at 700~pc \citep{Wilking:1989el}, with the particularity that it doesn't seem to contain any massive stars \citep{Gunther:2012dq}. The main cluster core is associated with water and methanol masers \citep{Palla:1991up,Fontani:2010cf} and multipolar molecular outflows observed at millimeter wavelengths \citep{Bachiller:1995cy,Anglada:1998uu,Beltran:2008gu}, suggesting that the region might have experienced a recent episode of star formation in the past 0.1 Myr which contrasts with the average age of the cluster of 1 Myr \citep{Chen:1997tb,Gutermuth:2005hx}. \cite{Gutermuth:2009gca} have identified $>170$ YSOs surrounding the core and measured their continuum fluxes up to \SI{8}{\micro\meter} with IRAC. While measurements at longer wavelengths were able to provide estimates of the total mass of the cluster \citep[e.g. using IRAS,][388~$L_\odot$]{Molinari:1996td}, the measurements are confused in the densest region and it has not been possible to properly associate the far-IR emission with its short wavelength counterpart because of the small separation between IRAC-detected protostars. The IRAS point source was classified as a luminous class 0 protostar \citep{Bachiller:1996ja}, and its emission associated with the bright millimeter source MMS1 to the northwest of the core \citep{Chini:2001fa}. \cite{Beltran:2008gu} show strong evidence that this region has multiple generations of stars, and suggest that a group of low-mass stars first completed its main accretion phase, before setting the stage for the birth of new intermediate-mass stars at the core of this cluster.
%
%We have observed two fields within the cluster (see Fig.~\ref{fig:IRAS20050_RGB}), including the brightest core at $20^h 07^m 06.70^s +\ang{27;28;54.5}$. Multiple sources in the core can be distinguished in the IRAC maps, but the core appears extended in Spitzer MIPS at \SI{24}{\micro\meter}, and is identified as a single source with WISE. No good high resolution far-infrared continuum data longward of \SI{24}{\micro\meter} was available for this source.
%
%%Cite also: \citep{Kumar:2006jo} if we want to talk about multiple generations of star formation.
%
%\begin{figure*}
%\begin{center}
%\includegraphics[width=\textwidth]{Figures/RGB.png}
%\label{fig:RGB}
%\caption{
%%\textit{Left:} The two white squares correspond to the FORCAST fields that we observed in around the core of IRAS~20050+2720. The background RGB image is a composition of \textit{Spitzer} IRAC 8~$\um$ (red), \textit{Spitzer} IRAC 5.7 $\um$ (green) , and IRAC 3.6~$\um$ (blue). The dashed red square at the center of the image correspond to the core of the cluster, displayed in greater detail in  the picture to the right. \textit{Right:} This RGB picture shows the IRAC 1, 3, and 4 bands of the core at the center of Fig.~\ref{fig:IRAS20050_RGB}. The white contours represent the contours of the FORCAST 37~$\um$ maps, and the red circles show the FORCAST-identified point sources. An infrared nebulosity can be seen to the East of the core with physical projected size of about 0.015~pc, surrounding a cavity with slightly smaller size. The nebulosity and its cavity can be seen all the way up to 37~$\um$. The far-IR emission is mapped well onto the IRAC sources, except for SOF4, for which almost no emission can be seen at shorter wavelengths. SOF4 matches the location of the bright millimeter source MMS1 \citep{Chini:2001fa}.}
%}
%\end{center}
%\end{figure*}
%
%% \begin{figure*}
%% \begin{center}
%% \includegraphics[width=5in]{}
%% \label{fig:IRAS20050_RGB_core}
%% \caption{}
%% \end{center}
%% \end{figure*}
%
%
%[Include a discussion about de-reddening towards that region?]
%
%\subsection{NGC 2071}
%The NGC~2071 star-forming region is one of several active areas of star formation in the northern part of L1630 giant molecular cloud which is located at a distance of 390 pc \citep{A-T1982}. 
%NGC~2071 itself is a reflection nebula.
%The NGC~2071 infrared cluster, located about 4' north of the reflection nebula, is a region of intermediate mass star formation \citep{Strom1976, Persson1981, Butner1990}. Maps of the cloud in CO and its isotopomers \citep{Buckle2010} show a large scale clump with $\sim$1,000 M$_\odot$ associated with the cluster. Dust continuum emission at $\lambda$=0.85 and 1.3 mm peaks on center of the cluster extending ~1' in diameter containing ~30 M$_\odot$ of gas and dust \citep{Johnstone2001,Mitchell2001,Launhardt1996}. Emission from CS in the J=2-1 through J=7-6 indicate that the gas in this region is centrally condensed with a density of ~10$^6$ cm$^{-3}$ \citep{Zhou1990}. 
%
%There are a number of near infrared surveys of the young cluster \citep[e.g.,][]{Strom1976,Lada1991,Megeath2012,Spezzi2015}. \cite{Spezzi2015} identify 52 YSOs associated with the NGC~2071 cluster, with the majority Class II sources. \cite{Flaherty2008} estimate an age of $\sim$2 Myr for the cluster, consistent with the large fraction of Class II sources (\cite{Evans2009}. The brightest far infrared emission from the cluster is associated with the IRS1 region \citep{Harvey1979,Butner1990}, which has an estimated total luminosity of 520~L$_\odot$. The immediate region of IRS 1 is, in fact, home to a number of YSOs that are infrared, X-ray, and radio sources \citep{Skinner2009,C-G2012,Kempen2012}. The radio \citep{C-G2012} and H$_2$ emission line imaging indicate that IRS~1, IRS~2, IRS~3, and, perhaps, VLA~1 are YSOs with outflows. The larger scale molecular outflow associated with this region is well studied in a number of molecules \citep{Bally1982,Chernin1993,Stoji2008}.
%\begin{figure*}
%\begin{center}
%\includegraphics[width=\textwidth]{Figures/NGC2071_saturated_mosaic.png}
%\label{fig:n2071saturated}
%\caption{%NGC 2071 IRS~1 Region: The left panel show the Spitzer IRAC 3.1~$\um$ image. The right panel is the Herschel 70~$\um$ image. The green plus marks indicate the positions of IRS~1, IRS~2, IRS~3, and VLA~1. The inner red circle show the extend of the saturated region in the Spitzer MIPS 24~$\um$ image and the outer red circle encompasses the region strong affected by imaging artifacts. The gray circle on the lower right of the right panel is the resolution the 70~$\um$ image.
%NGC2071 seen with IRAC, MIPS and Herschel.}
%\end{center}
%\end{figure*}
%
%Figure \ref{fig:n2071overview} shows the Spitzer 3.1~$\mu$m image of the IRS~1 region on the left \citep[image from Spitzer Archive:][]{Megeath2012} and the Herschel 70~$\mu$m image on the right (image from Herschel Archive: Gould Belt Project, P.I. Andr\'e). The plus marks in both panels indicate the position of the brighter YSOs: IRS~1, IRS~2, IRS~3, IRS~4, and VLA~1. The inner red circle with a diameter of 26" indicates the extend of the saturated region in the Spitzer MIPS 24~$\mu$m image; the outer red circle, diameter 60", encompasses the region with strong imaging artifacts in the MIPS 24~$\mu$m image.
%The right panel shows Herschel 70~$\mu$m image which does not resolve the emission from IRS~1, IRS~2, IRS~3, and VLA~1. The centroid of the 24~$\mu$m and 70~$\mu$m emission is between IRS~1 and VLA~1 indicating that several of the sources are contributing to the total observed emission. Interferometric observations show that the millimeter wavelength dust emission is dominated by envelopes associated with IRS~1 and IRS~3, with estimated masses of 8.2 and 12.3~M$_\odot$ material, respectively \citep{Kempen2012}. The millimeter emission also reveals the presence of disks with radii $\le$100~AU associated with IRS~1 and IRS~3 \citep{Kempen2012}.
%
%The luminosities and masses of the individual source, IRS~1, IRS~2, IRS~3, and VLA~1, are not known. The Spectral Energy Distributions (SEDs) shortward of 10~$\mu$m support their identification as embedded YSOS \citep{Skinner2009}. \cite{Skinner2009} gives a clear discussion of the possibilities for IRS~1 and concludes that it is likely a mid-to late B~star. \cite{Kempen2012} find luminosities of 10, 3.4, and $\le$27~L$_\odot$ for IRS~1, 2, and 3, respectively, and stellar masses of $\le$1~M$_\odot$ for each, based on SED fitting. These masses and luminosities are not consistent with estimate of the total luminosity of the region of 520~L$_\odot$ \citep{Butner1990}. The far infrared images from Herschel reveal that IRS~1 alone does not totally dominate, as seen in Figure N; IRS~1, VLA~1, and IRS~3 likely make substantial contributions to the emission with lesser emission from IRS~2 and IRS~4.
%
%\section{SOFIA Observations}
%NGC 2071 and IRAS 200050+2720 were observed with the FORCAST instrument on SOFIA as part of a survey of selected nearby ($\le$1~kpc) bright star-forming clusters which show high protostellar density \citep{Gutermuth:2009gca} and are located in the northern hemisphere. The survey focussed on the clusters that contain one or more saturated or confused region in the archival \textit{Spitzer} and WISE data.
%
%\subsection{Data Acquisition}
%The IRAS 200050 and NGC~2071 observations were collected over three flights out of the 10-flight survey campaign. A summary is shown in Table~\ref{tab:obssummary}. Because the clusters are dominated by bright sources, the observations fit into small flight segments which filled gaps in the flight schedule. The source coordinate in Table~\ref{tab:obssummary} correspond to the centers of the green areas in Fig.~\ref{fig:IRAS20050_RGB} and Fig.~\ref{fig:n2071overview}. In each cluster, two fields separated by $\sim$ 3 arcminutes were observed to focus on two saturated regions. All fields were observed using the chop-and-node C2N mode from FORCAST, with 9-point dithering to reduce the flat field issues.
%
%%\capstartfalse
%%\begin{deluxetable*}{ccccccccccc}
%%\tablecaption{target list}
%%\tablenum{1}
%%\tablehead{\colhead{Cluster} & \colhead{RA} & \colhead{DEC} &  \colhead{Flight} & \colhead{Fields} & \colhead{Dist.} & \colhead{Time/Field} & \colhead{Sen{\_}11} & \colhead{Sen${\_}$19} & \colhead{Sen${\_}$31} & \colhead{Sen${\_}$37} \\
%%\colhead{} & \colhead{(deg)} & \colhead{(deg)} & \colhead{} & \colhead{}  & \colhead{(pc)} & \colhead{(s)} & \colhead{(Jy)} & \colhead{(Jy)} & \colhead{(Jy)} & \colhead{(Jy)}} 
%%\startdata
%%IRAS20050+2720 & 301.771 & 27.481 & F166, F131 & 2 & 700$^{1}$ & 253 & 0.026 & 0.039 & 0.068 & 0.127 \\
%%NGC2071 & 86.775 & 0.363 & F192 & 2 & 420$^{2}$ & 35 & 0.118 & 0.119 & 0.196 & 0.474
%%\enddata
%%\label{tab:obssummary}
%%\end{deluxetable*}
%%\capstarttrue 
%
%
%We observed the clusters in 4 bands: 11.1, 19.7, 31.5 and \SI{37.1}{\micro\meter}. The requested observation time in each band was calculated to detect YSOs with rising spectral energy distribution (SED) of same luminosity at the two distances of the clusters. We estimated integration time based on a rising-SED protostar model for $\sim 1.5\Lsun$. [HOW SHOULD I JUSTIFY THE FLUXES THAT WE SET AS OUR LIMITS?]
%
%Average observing time per field and average measured 1-sigma sensitivity levels are shown in Table~\ref{tab:obssummary} for the 4 bands. The measured background levels indicate the smallest detectable point source flux density at each wavelength, based on the noise measurements in the image itself. Bands 11.1 and \SI{37.1}{\micro\meter} were observed simultaneously using a dichroic filter, as were 19.7 and 31.5. However, in order to reach the required flux sensitivity for the \SI{37.1}{\micro\meter} band, we completed some of our observations with single-band observations at \SI{37.1}{\micro\meter}. 
%
%% \begin{longtable*}{ccccccc}
%% \hline
%% \hline
%% Cluster Name & Field & Field Coordinates & Band & Time (s) & Flight & Date \\
%% \hline
%
%% IRAS20050+2720 & 2 & 20h07m03s +27d30m38s & 11.1d & 135 & F131 & 2013-09-17 \\
%% IRAS20050+2720 & 2 & 20h07m03s +27d30m30s & 19.7d & 140 & F166 & 2014-05-02 \\
%% IRAS20050+2720 & 2 & 20h07m03s +27d30m47s & 31.5d & 160 & F166 & 2014-05-02 \\
%% IRAS20050+2720 & 2 & 20h07m04s +27d31m02s & 37.1d & 135 & F131 & 2013-09-17 \\
%% IRAS20050+2720 & 3 & 20h07m06s +27d28m12s & 11.1d & 135 & F166 & 2014-05-02 \\
%% IRAS20050+2720 & 3 & 20h07m06s +27d28m05s & 19.7d & 84 & F166 & 2014-05-02 \\
%% IRAS20050+2720 & 3 & 20h07m06s +27d28m13s & 31.5d & 96 & F166 & 2014-05-02 \\
%% IRAS20050+2720 & 3 & 20h07m06s +27d28m06s & 37.1d & 126 & F166 & 2014-05-02 \\
%% \hline
%
%% NGC2071 & 1 & 05h47m07s +00d17m49s & 11.1d & 18 & F192 & 2015-02-05 \\
%% NGC2071 & 1 & 05h47m07s +00d18m02s & 19.7d & 11 & F192 & 2015-02-05 \\
%% NGC2071 & 1 & 05h47m06s +00d17m55s & 31.5d & 11 & F192 & 2015-02-05 \\
%% NGC2071 & 1 & 05h47m07s +00d18m03s & 37.1d & 21 & F192 & 2015-02-05 \\
%% NGC2071 & 2 & 05h47m07s +00d21m30s & 11.1d & 18 & F192 & 2015-02-05 \\
%% NGC2071 & 2 & 05h47m07s +00d21m31s & 19.7d & 18 & F192 & 2015-02-05 \\
%% NGC2071 & 2 & 05h47m07s +00d21m34s & 31.5d & 24 & F192 & 2015-02-05 \\
%% NGC2071 & 2 & 05h47m07s +00d21m34s & 37.1d & 21 & F192 & 2015-02-05 \\
%
%% \caption{List of observations}
%% \end{longtable*}
%
%%Make sure to mention the distances picked and the references for it, as well as the cluster's age estimate
%%Need to find references to cite for details about each region
%\subsection{Data reduction}
%The data are processed through various versions of the online pipeline to yield Level 2 data products available on the archive \citep{Herter:2013by}. We apply our own reduction procedure and photometry pipeline on those products to derive final images, source positions, fluxes and sensitivities. The software utilizes the Python \textit{astropy} package \citep{2013A&A...558A..33A} and its associated modules \textit{photutils} and \textit{APLpy}. 
%
%\subsubsection{Pre-treatment}
%Some manual treatment of each image is necessary before it can be analyzed by our software, which follows this procedure: a) visually aligning the WCS coordinate system, often 10-20" off, using point sources and archival data from other wavelengths and facilities such as IRAC \SI{8}{\micro\meter}; b) cropping the images to clean off the nodded fields, and c) identify the coordinates of each source, both point-like and extended.
%
%After these manual steps, the Level 2 images are multiplied by the calibration factor provided by the online pipeline, which converts them to Jy/pixel. We do not proceed to any systematic color correction, but the effects on the fluxes are very small \citep{Herter:2013by}.
%\begin{comment}
%\begin{enumerate}
%\item Adjust WCS coordinates: use images at other wavelengths (2MASS, IRAC, MIPS, WISE) to re-align the (RA, DEC) position of the field. We estimate that this process is good to within one SOFIA pixel (0.768") for the fields where one or more point sources can be identified. Extended fields are less trustworthy, since matching the extended emission to other wavelengths is harder. The rotation of the field produced by the SOFIA pipeline is correct for all of our data. 
%\item Crop each image, remove chopped fields, remove artifacts.
%\item Identify and categorize sources: isolated point sources, clustered point sources, and extended sources. For extended sources, a circular or elliptical aperture is used to try to encompass the entirety of the emission.
%\item Manually identify a location in the field that corresponds to a representative background.
%\end{enumerate}
%\end{comment}
%\subsubsection{Source flux extraction and calibration}
%\begin{comment}
%We feed the adjusted FITS and associated metadata files to a photometry pipeline. The pipeline first processes all the calibrator stars that are observed during each flight, with each filter setting, and derives a new aperture correction factor for each image, based on an aperture of 4 pixels radius (3.072") and an aperture of 12 pixels radius. We consider that the latter aperture contains all the flux from a given point source.
%
%We distinguish between 3 types of sources after manual identification: \textit{isolated}, which are point sources with no nearby objects; \textit{clustered}, which are point sources with nearby objects; and \textit{extended}, which are not consistent with being point sources. [THIS PARAGRAPH MAY GO AWAY IF WE DON'T WANT TO TALK ABOUT GENERALITIES ABOUT THE PIPELINE]
%
%
%For point sources, we use our standard aperture of 4 pixels at all wavelengths. We consider an annulus surrounding the source extending from 12 to 20 pixels radius (24 to 40 for clustered sources): the local background is determined from the mode of the pixels in the annulus, while the sensitivity is calculated by measuring the standard deviation of 4-pixel apertures within that annulus [Cite Taro's paper and the Herschel photometry paper that Tracy gave us]. We apply the aperture correction derived from the calibrator observations taking during that flight.
%
%For extended sources, an elliptical aperture is determined manually from the \SI{37}{\micro\meter} images. The local background is determined from the mode of an elliptical annulus, with an inner boundary at the elliptical aperture and an outer boundary corresponding to an ellipse 20\% larger. The sensitivity quoted is the point source sensitivity, and is determined following the same method as for point sources, using the standard deviation of apertures spread across the elliptical annulus. [NO NEED TO MENTION THIS SINCE WE MIGHT NOT TREAT EXTENDED SOURCES]
%
%The photometry from sources that were observed in different flights is then combined to increase the signal-to-noise ratio. This combination takes into account the sensitivity of each source by appropriately weighing each image.
%
%Although we can compute source sensitivities based on the local noise, and we use the calibrators each flight to determine the aperture correction, observations with SOFIA have additional noise from the water vapor overburden and air mass, as well as from the flat field variations. These noise contributions usually amount to much higher than the sensitivities estimated from the local background, and we follow the recommendation from \citep{Herter:2012hv} to adopt a 20\% uncertainty for our flux estimates. 
%
%\end{comment}
%
%\subsection{Photometry at other wavelengths}
%
%\subsection{Instrumental Characterization}
%[THIS WHOLE DISCUSSION MIGHT FIT BETTER IN THE PAPER ABOUT THE WHOLE SURVEY, RATHER THAN THIS PAPER WHICH IS JUST ABOUT TWO CLUSTERS...]
%The three flights discussed here were part of the total of 10 data flights for the entire survey. The larger context of the entire survey allowed us to follow the evolution of the instrument properties throughout the two years of science observations. We discuss three metrics: the evolution of the aperture correction factor, the evolution of the instrument's residuals, and the evolution of the PSF size, through half width at half max of the encircled energy distribution, that we call $\Rfifty$. This is done in an attempt to assess the uncertainties in our flux determination, and our confidence in determining that sources are point-like or extended. %This study is based  on the large number of calibrator observations during our 10 science flights.
%
%\subsubsection{PSF size}
%
%Fig~\ref{fig:average_EE} shows the average of the normalized encircled energy distribution of the PSF, measured on all the calibrators of our sample. Each curve represents one of the five different combinations of bandpass filter and dichroic setting that we use for our observations: \SI{11.1}{\micro\meter}, \SI{19.7}{\micro\meter}, \SI{31.5}{\micro\meter}, \SI{37.1}{\micro\meter} with dichroic and \SI{37.1}{\micro\meter} without a dichroic (open). 
%
%As expected, the PSF at \SI{37.1}{\micro\meter} is larger than the PSFs at shorter wavelengths, but less that the traditional diffraction limit rule. This indicates that additional PSF smearing is occurring at short wavelengths, likely due to plane jitter and pointing errors. This was predicted and mentioned in the SOFIA Observer's Handbook. Throughout all the flights, the largest 1-sigma error occurs for the \SI{37}{\micro\meter} observations at about XX\%. CONCLUDE ON OUR ABILITY TO DETERMINE WHETHER A SOURCE IS EXTENDED OR NOT.
%
%To look at the behavior of the PSF in more detail, we can use the half width at half maximum of the encircled energy distribution, $\Rfifty$ as a proxy for PSF size. The variation of this quantity for the various flights, bandpass/dichroic setting, and calibrators used is showed in Fig~\ref{fig:Rfifty_dist}. This shows the flight-to-flight differences and, for some calibrators, the in-flight variability. We find that the latter is usually NN\% or less, except for the SOFIA flight on 05-02-2014, for which the spread is quite considerable. The variation from flight to flight is larger than the variation within a given flight, which indicates variability in the observing conditions, systematics, or thermal radiation environment of the observatory between different flights. Hence, we conclude that the extension of a source can be best determined by comparing $\Rfifty$ for that source with $\Rfifty$ for the corresponding calibrator measurement for that flight and dichroic setting, to within NN\%, 1-sigma confidence.
%\begin{figure}
%\begin{center}
%
%\includegraphics[width=0.45\textwidth]{Figures/average.png}
%\label{fig:average_EE}
%\caption{PSF size distribution}
%\end{center}
%\end{figure}
%
%\begin{figure}
%\begin{center}
%
%\includegraphics[width=0.45\textwidth]{Figures/fwhs.png}
%\label{fig:Rfifty_dist}
%
%\caption{PSF size distribution}
%\end{center}
%\end{figure}
%
%\subsubsection{Aperture correction}
%In Fig~\ref{fig:response}, we plot the aperture correction factor that we compute from the ratio of the flux measured within an aperture of 4 pixels, divided by the flux measured in an aperture of 12 pixel radius, which we consider to be encompassing the total flux in the source. Not surprisingly, this graph follows very closely the plot of $\Rfifty$ shown in Fig~\ref{fig:Rfifty_dist}, showing the close link between the aperture correction factor and the shape of the calibrator's PSF. For the aperture correction variability, we adopt a XX, 1-sigma uncertainty value on the flux estimate.
%
%
%
%\subsubsection{Instrument response}
%To further study the variability of the observing, we can look at the detector response and the aperture correction evolution after applying the calibration factor. In the ideal conditions, the calibration factor always leads to the same flux measurement of the calibrator source, within the pipeline's systematic errors and residual noise. The detector response is measured here by simply applying aperture photometry on the calibrators to measure their fluxes, using the same local background subtraction as the one described in the previous sections. Calibrator stars are good ways to correct for telescope and atmospheric variability, as their far-infrared fluxes are not expected to vary significantly over any relevant timescale. %We adopt a value of XX, 1-sigma value for the response variability, effectively representing the uncertainties in the observatory and the atmosphere.
%In Fig~\ref{fig:response}, we plot all the measured fluxes of the calibrators. The flux variability from flight to flight for a given calibrator is small (typically [CALCULATE THIS]), and the variability within the same flight is even smaller [QUANTIFY]. We adopt a XX, 1-sigma uncertainty value for the absolute flux measurement. This is the residual uncertainty after applying the systematic correction produced by the SOFIA FORCAST online pipeline.
%
%
%
%\begin{figure}
%\begin{center}
%
%\includegraphics[width=0.45\textwidth]{Figures/Phot_val.png}
%\label{fig:response}
%\caption{Instrumental response and aperture correction}
%
%\end{center}
%\end{figure}
%
%\begin{figure}
%\begin{center}
%\includegraphics[width=0.45\textwidth]{Figures/Aper_corr.png}
%\label{fig:aper_corr}
%
%\caption{Instrumental response and aperture correction}
%\end{center}
%\end{figure}
%
%%WHAT IS THE BOTTOM LINE FROM THIS DISCUSSION? IS A FLUX MEASUUREMENT LIMITED BY VARIATIONS IN THE PSF IF IT IS BRIGHT? SOMETHING ELSE? IT SEEMS LIKE THE CONCLUSION FROM THIS SECTION SHOULD BE A STATEMENT ABOUT THE SYSTEMATIC ERRORS ON ANY QUOTED FLUX MEASUREMENT AND A STATEMENT ABOUT LIMITATIONS ON KNOWING WHETHER A SOURCE IS EXTENDED RELATIVE TO A POINT SOURCE.
%\section{Observational results}
%\subsection{IRAS 200050}
%
%SOFIA photometry results, IRAC photometry issues and results; looking at the sources that are fit by guthermuth, we find a 10\% agreement between our photometry results and theirs.
%
%\subsubsection{Photometry results and maps}
%
%\subsubsection{SEDs}
%
%
%\subsubsection{Upper limits on other sources in the field}
%
%
%\begin{figure}
%\begin{center}
%\includegraphics[width=1\textwidth]{Figures/IRAS20050_SEDs.png}
%\label{fig:IRAS20050_SEDs}
%\caption{}
%\end{center}
%\end{figure}
%
%
%
%
%\subsection{NGC2071}
%
%
%\subsubsection{Photometry results and maps}
%
%\begin{figure}
%\begin{center}
%\includegraphics[width=1\textwidth]{Figures/NGC2071_mosaic.png}
%\label{fig:NGC2071_mosaic}
%\caption{The core of IRAS20050+2720 is seen in the four bands of the \textit{Spitzer} IRAS instrument, as well as with the four FORCAST bands. The increased resolution of FORCAST compared to previous instruments allows to match the long-wavelength emission with its short wavelength counterpart. The stretch in each image is adjusted for optimal readability. The white contours correspond to the FORCAST \SI{37}{\micro\meter} emission [mention the contour levels]. The red circles indicate the location of the five FORCAST point sources in the core.}
%\end{center}
%\end{figure}
%
%\begin{figure}
%\begin{center}
%\includegraphics[width=1\textwidth]{Figures/NGC2071_SEDs.png}
%\label{fig:NGC2071_SEDs}
%\caption{}
%\end{center}
%\end{figure}
%
%\subsubsection{SEDs}
%
%\subsubsection{Upper limits on other sources in the field}
%
%
%\subsection{Summary}
%
%Put here the table of photometry + flags
%
%
%
%\section{SED and dust modeling}
%
%YE05 assumed the dust opacities of Ossenkopf \& Henning
%(1994) appropriate for thin ice mantles after 105 year of coagulation at a gas density of 106 cm-3 (OH5 dust), which several recent studies have shown to be appropriate for cold, dense cores (e.g., Evans et al. 2001; Shirley et al. 2002; Young et al. 2003; Shirley et al. 2005) [LOOK AT REST OF DISCUSSION IN DUNHAM et al. 2010]
%
%\subsection{Radiative transfer models}
%
%We use radiative transfer models to fit the SEDs we observe and extract physical parameters. We explored the tool by \cite{Robitaille:2006cb} as a starting point for this process. While the tool provides results that fit the data well, the large number of parameters makes it difficult to draw meaningful conclusions on the real physics behind the observations. We have observed a large amount of cross-correlations between the model parameters, as well as many local minimas in the $\chi^2$ minimization scheme that is used.
%
%In an effort to understand the dependence of the observations with the physical parameters used in the models, we use the HYPERION radiative transfer code \citep{Robitaille:2011fc} and create our own grid of models by varying a small amount of meaningful parameters. HYPERION is a very versatile code with lots of options for various geometries, but we simplify the problem to its most essential components: a circularly symmetric disk with a power-law envelope.
%
%We explore the various geometries and parameters that are available in the code, and come to following conclusions:
%\begin{enumerate}
%\item Modeling accretion through an $\alpha$-disk instead of a flared disk is equivalent to increasing the central luminosity by an appropriate amount. Hence, we use only standard flared disks and quote a total central luminosity
%\item It is good enough to only vary the total luminosity of the central object, instead of varying its mass, radius and temperature
%\item Models are very insensitive to disk mass, when the envelope's mass is larger
%\item The model is sensitive to the envelope's mass distribution within a given radius, but not sensitive to the envelope's inner or outer radius
%\item The outer radius of the disk has no effect on the models
%\item The inner radius of the envelope 
%\end{enumerate}
%
%Based on our exploration of HYPERION, we proceed with the following simulation set up. We use a density structure composed of a standard flared disk of \SI{e-3}{\Msun} that extends from the dust sublimation radius out to 50~AU. The scale height at the dust sublimation radius is set to be 10\% of that radius. The disk's flaring power is a constant set to $\beta=1.1$. 
%
%We add an envelope with a central cavity. The envelope extends from 30~AU out to a variable radius, and has a variable mass and power law exponent. The inner cavity has a constant 25 degree opening angle. Setting this latter parameter has some effect that is correlated with the viewing angle.
%
%All elements in our model are using the same dust model by \cite{Draine:2003di}. We have experimented with various other types of dust models, notably with the OH5 dust [REFERENCE], as it was suggested in [CITE TRACY]. We found that the fits with the OH5 dust were much worse. In most cases, it was particularly difficult to fit the \SI{10}{\micro\meter} silicate absorption feature. 
%
%We use a variable foreground extinction also with the same dust model, as we anticipate that most of the extinction along the line of sight will occur in the cluster itself. We chose to not use any ambient medium, as they complicated the interpretation of the results.
%
%We run our simulations using $10^5$ photons, and spherical grids with 199 radial cells, 49 meridional cells, and 1 single azimuthal cell. We have tested these various parameters and sought an optimum of fast computing times and low statistical noise. With this, the calculated uncertainties are a few percent at long wavelengths and can be on the order of 10 to 15 percent at short wavelengths. This is acceptable since the short wavelength range is largely used in the fit to determine the overall external extinction magnitude. At the wavelenghts relevant for the IRAC fluxes, the uncertainties due to the simulation are on the order of 5\% - less than our estimated measurement error. With these parameters, a typical model takes about 5-10 minutes to run on a standard desktop computer. Our wrapper software allows us to run multiple different grids in a row and merge them into one single entity that we can feed to a minimization routine.
%
%In order to fit the data, we use the $\chi^2$ method described in \cite{Robitaille:2007dl}, with the exception of the overall optimal scaling step. We calculate the $\chi^2$ for our targets with all calculated models, inclinations, and a range of values of foreground extinction magnitudes.
%
%
%%Table of set parameters]
%
%%variable parameters: envelope mass, envelope density power law, source luminosity, inclination, extinction
%
%
%
%
%%This section discusses how we set up a grid of models to fit, and why we moved away from Robitaille's sedfitter. Maybe show some results from the Robitaille's fits?
%
%\subsection{Extracting physical parameters from the fits}
%This section discusses the results from the fits: best fits, parameter estimation and variation about the best fits, color-color diagrams, etc.
%\begin{itemize}
%\item \citep{Shirley:2000gh}: typical masses of protostars are a few tenths to \SI{2}{\Msun}. 
%
%\end{itemize}
%
%
%\section{Discussion}
%
%\section{Conclusion}
%blabla
%
%\begin{landscape}
%\renewcommand{\arraystretch}{1}
%
%\tiny
%
%\begin{longtable}{lllrrrrrrrrrrrrrrrrrrrrrrrrrrrrrrrrrrrrrrrrrrrrrrrrrrrrrrr}
%\toprule
%{} &               Coordinates &   Property &         j &       e\_j &         h &       e\_h &        ks &      e\_ks &        i1 &      e\_i1 &        i2 &      e\_i2 &         i3 &      e\_i3 &         i4 &      e\_i4 &        F11 &     e\_F11 &        F19 &     e\_F19 &        m1 &      e\_m1 &         F31 &      e\_F31 &         F37 &      e\_F37 &        m2 &      e\_m2 &         H70 &      e\_H70 &        H160 &     e\_H160 &        H250 &      e\_H250 &        H350 &      e\_H350 &       H500 &     e\_H500 &   S850 &  e\_S850 &  F1100 &  e\_F1100 &  S1300 &  e\_S1300 &       R37 &       Lbol &     alpha &         R &   env\_mass &  env\_mass\_std &  calc\_mass &       sLsun &  sLsun\_std &       Lbol &        inc &  ext &     s \\
%SOFIA\_name  &                           &            &           &           &           &           &           &           &           &           &           &           &            &           &            &           &            &           &            &           &           &           &             &            &             &            &           &           &             &            &             &            &             &             &             &             &            &            &        &         &        &          &        &          &           &            &           &           &            &               &            &             &            &            &            &      &       \\
%\midrule
%\endhead
%\midrule
%\multicolumn{3}{r}{{Continued on next page}} \\
%\midrule
%\endfoot
%
%\bottomrule
%\endlastfoot
%Oph.12      &  16h26m34.0s -24d23m40.7s &   Extended &  0.000700 &  0.000700 &  0.000900 &  0.000900 &  0.001500 &  0.002000 &        -- &        -- &        -- &        -- &         -- &        -- &         -- &        -- &   1.622251 &  0.234589 &  49.015377 &  3.439944 &        -- &        -- &  185.704775 &  13.021490 &  244.997024 &  17.197742 &        -- &        -- &   38.022248 &  16.367280 &   31.874880 &  31.874880 &  160.940056 &  160.940056 &  119.104440 &  119.104440 &  63.397558 &  63.397558 &     -- &      -- &     -- &       -- &     -- &       -- &  2.969460 &  10.280410 &  3.369796 &  1.086848 &   0.076000 &      0.025482 &         -- &   39.899998 &   5.133392 &  10.280410 &  18.671719 &   14 &  0.70 \\
%NGC1333.2   &  03h29m10.3s +31d21m55.5s &   Extended &  0.285291 &  0.028529 &  0.653883 &  0.065388 &  0.901012 &  0.090101 &  0.637000 &  0.063700 &  0.446000 &  0.044600 &   0.448000 &  0.079800 &   0.912525 &  0.127753 &   8.414433 &  0.595773 &  36.516910 &  2.562018 &        -- &        -- &  106.489879 &   7.456661 &  135.723202 &   9.507282 &        -- &        -- &   70.038748 &   7.003875 &   77.573890 &  20.036312 &   87.660813 &   15.014373 &   51.505937 &   16.114088 &  24.741970 &  13.061772 &     -- &      -- &     -- &       -- &     -- &       -- &  2.232105 &  27.831752 &  1.242967 &  1.773727 &  22.167999 &      9.901331 &         -- &    7.700000 &   1.112940 &  27.831752 &   0.000000 &    2 &  0.70 \\
%NGC1333.8   &  03h29m03.7s +31d16m03.9s &   Isolated &  0.034236 &  0.003424 &  0.140832 &  0.014083 &  0.360023 &  0.036002 &  0.904000 &  0.090400 &  0.359000 &  0.068700 &   2.750000 &  0.331000 &   5.690000 &  0.569000 &   4.125824 &  0.292108 &  25.042361 &  1.757174 &        -- &  0.434000 &   99.119629 &   6.940325 &  106.511012 &   7.463216 &    125.00 &    12.500 &  187.822184 &  18.782218 &  228.025009 &  22.802501 &  156.558418 &   15.655842 &   90.190510 &   14.084944 &  42.435647 &  13.650262 &     -- &      -- &  2.700 &   0.2700 &     -- &       -- &  0.770085 &  35.106406 &  1.144899 &  1.061540 &   1.946000 &      0.745726 &   2.019725 &   17.000000 &   2.389399 &  35.106406 &   0.000000 &   13 &  1.30 \\
%Oph.3       &  16h27m09.4s -24d37m18.3s &   Isolated &  0.000700 &  0.000700 &  0.038967 &  0.003897 &  0.928818 &  0.092882 &        -- &  0.021700 &        -- &        -- &  12.800000 &  1.350000 &         -- &  0.031200 &  20.696882 &  1.450637 &  57.194587 &  4.005008 &        -- &  0.480000 &   83.549816 &   5.856017 &   87.378670 &   6.163625 &     49.10 &     4.910 &   84.839941 &   8.483994 &   65.551102 &   6.804876 &   36.490880 &    9.759635 &   16.302506 &    9.272575 &   7.565188 &   7.565188 &  0.410 &  0.0410 &  0.057 &   0.0057 &  0.095 &   0.0095 &  0.992850 &  13.394719 &  0.574003 &  1.543482 &   0.004000 &      0.001816 &   0.040478 &   85.000000 &  19.741953 &  13.394719 &   0.000000 &   14 &  1.00 \\
%Oph.7       &  16h27m28.0s -24d39m33.8s &   Isolated &  0.000700 &  0.000700 &  0.003460 &  0.000346 &  0.047025 &  0.004703 &  0.731000 &  0.097100 &  1.830000 &  0.183000 &   2.940000 &  0.294000 &   2.320000 &  0.252000 &   2.927179 &  0.224177 &  30.721915 &  2.162581 &        -- &        -- &   70.940218 &   4.972795 &   70.480142 &   4.956635 &     34.70 &     3.470 &   65.516629 &   6.551663 &   44.630179 &   4.463018 &   37.363205 &   24.113419 &   23.657531 &   22.485618 &  15.313136 &  15.313136 &  0.360 &  0.0360 &  0.360 &   0.0360 &  0.060 &   0.0060 &  0.970612 &   6.474091 &  1.353049 &  1.392024 &   0.015000 &      0.002340 &   0.025565 &   26.579000 &   3.493491 &   6.474091 &  71.591522 &   14 &  0.70 \\
%Oph.8       &  16h27m37.2s -24d30m34.8s &   Isolated &  0.094643 &  0.009464 &  0.305000 &  0.030500 &  0.618202 &  0.061820 &  1.410000 &  0.141000 &  1.600000 &  0.160000 &   4.060000 &  0.406000 &   6.000000 &  0.600000 &   3.535605 &  0.258554 &  23.107697 &  1.620863 &        -- &        -- &   39.950535 &   2.798894 &   51.640811 &   3.625942 &     20.80 &     2.080 &   27.932258 &   2.793226 &   13.925723 &   2.408777 &    5.603603 &    5.603603 &    5.408533 &    5.408533 &         -- &         -- &  0.180 &  0.0180 &     -- &       -- &  0.060 &   0.0060 &  1.015303 &   5.017779 &  0.545431 &  1.133202 &   0.007000 &      0.001843 &   0.025565 &   17.716999 &   3.387279 &   5.017779 &  77.846802 &   12 &  0.70 \\
%Oph.11      &  16h26m59.2s -24d35m00.2s &   Extended &  0.000700 &  0.000700 &  0.000900 &  0.000900 &  0.001500 &  0.002000 &  0.143000 &  0.017000 &  0.487000 &  0.060400 &   0.564000 &  0.433000 &   0.553000 &  0.134000 &   0.839171 &  0.088616 &  12.550480 &  0.891950 &  5.640000 &  0.790000 &   40.712799 &   2.852379 &   47.672216 &   3.345748 &     47.60 &     4.760 &  101.124889 &  10.112489 &   34.731077 &   7.937487 &   19.808539 &    4.897385 &   19.421774 &    7.187933 &  17.293798 &   6.872309 &     -- &      -- &     -- &       -- &     -- &       -- &  2.600471 &   4.034299 &  2.039836 &  0.776784 &   0.034000 &      0.010207 &         -- &   11.900000 &   1.886536 &   4.034299 &  77.846802 &   14 &  0.70 \\
%NGC1333.9   &  03h28m55.6s +31d14m36.6s &   Isolated &  0.000700 &  0.000700 &  0.000900 &  0.000900 &  0.001500 &  0.002000 &  0.001050 &  0.000105 &  0.012600 &  0.001260 &   0.022700 &  0.002270 &   0.030217 &  0.003022 &   0.122895 &  0.070532 &   1.213009 &  0.134313 &  5.194206 &  0.519421 &   32.151835 &   2.262214 &   45.738104 &   3.238440 &        -- &        -- &  210.068777 &  21.006878 &  336.489802 &  33.648980 &  180.769192 &   18.076919 &   91.496328 &   18.612656 &  38.205550 &  17.115160 &     -- &      -- &  2.300 &   0.2300 &     -- &       -- &  0.799509 &  24.282985 &  2.793514 &  2.624300 &   2.919000 &      0.353591 &   1.720506 &   17.000000 &   2.412459 &  24.282985 &  18.671719 &   14 &  0.85 \\
%Oph.1       &  16h27m10.3s -24d19m12.9s &   Isolated &  0.506394 &  0.050639 &  1.017419 &  0.101742 &  1.368770 &  0.136877 &  1.300000 &  0.130000 &  1.170000 &  0.117000 &   1.280000 &  0.128000 &   1.660000 &  0.166000 &   1.780982 &  0.156661 &  12.230152 &  0.869434 &        -- &  0.113000 &   25.509892 &   1.788463 &   29.200074 &   2.093071 &     16.60 &     1.660 &   23.562913 &   2.356291 &   15.325085 &   1.532508 &    6.082145 &    4.223516 &    4.014879 &    4.014879 &   3.230003 &   3.230003 &  0.400 &  0.0400 &     -- &       -- &  0.095 &   0.0095 &  0.916298 &   3.633096 &  0.268865 &  0.666547 &   0.010000 &      0.001820 &   0.040478 &    7.875000 &   1.273325 &   3.633096 &  77.846802 &    3 &  0.70 \\
%NGC1333.11  &  03h28m37.1s +31d13m30.0s &   Isolated &  0.000700 &  0.000700 &  0.001183 &  0.000153 &  0.010128 &  0.002000 &  0.030100 &  0.003010 &  0.089200 &  0.008920 &   0.267000 &  0.026700 &   0.006630 &  0.028300 &   0.143330 &  0.156646 &   4.398746 &  0.323481 &  2.180000 &  0.739000 &   20.890854 &   1.483130 &   23.313519 &   1.649508 &        -- &        -- &   58.802758 &   5.880276 &   63.081860 &   6.308186 &   38.422454 &    3.842245 &   22.023228 &    2.202323 &   9.683953 &   1.694426 &     -- &      -- &  0.360 &   0.0360 &     -- &       -- &  1.016810 &   7.468708 &  1.693932 &  0.987321 &   0.384000 &      0.181414 &   0.269297 &    7.700000 &   0.802750 &   7.468708 &  18.671719 &   14 &  0.70 \\
%IRAS20050.3 &  20h07m06.3s +27d28m56.6s &  Clustered &  0.002014 &  0.000201 &  0.005833 &  0.000583 &  0.027947 &  0.002795 &  0.090188 &  0.009019 &  0.218415 &  0.021842 &   0.338759 &  0.033876 &   0.428667 &  0.042867 &   0.180523 &  0.059050 &   2.575909 &  0.265602 &        -- &        -- &   12.532365 &   0.940287 &   19.334657 &   1.412497 &        -- &        -- &          -- &         -- &          -- &         -- &          -- &          -- &          -- &          -- &         -- &         -- &     -- &      -- &     -- &       -- &     -- &       -- &  1.998293 &  12.813751 &  1.134668 &  0.732251 &   0.256000 &      0.113978 &         -- &   48.450001 &   6.340800 &  12.813751 &  26.525352 &    5 &  1.00 \\
%Oph.4       &  16h27m02.5s -24d37m27.6s &   Extended &  0.003443 &  0.000344 &  0.065933 &  0.006593 &  0.396580 &  0.039658 &  1.400000 &  0.140000 &  1.970000 &  0.197000 &   5.030000 &  0.503000 &  10.700000 &  1.460000 &   5.894930 &  0.417953 &   8.756463 &  0.618932 &        -- &        -- &   16.278797 &   1.143412 &   17.928454 &   1.308147 &     29.10 &     2.910 &   60.526791 &   7.333157 &   28.022688 &   2.802269 &          -- &          -- &          -- &          -- &         -- &         -- &     -- &      -- &  0.260 &   0.0260 &     -- &       -- &  1.801189 &   4.503086 &  0.185629 &  2.224143 &   0.004000 &      0.000439 &   0.066181 &   14.299999 &   2.702848 &   4.503086 &  37.863647 &   14 &  1.30 \\
%NGC1333.10  &  03h28m57.4s +31d14m15.0s &   Isolated &  0.000700 &  0.000700 &  0.000900 &  0.000900 &  0.001500 &  0.002000 &  0.031600 &  0.003160 &  0.104000 &  0.010400 &   0.262000 &  0.026200 &   0.370036 &  0.037004 &   0.116106 &  0.100473 &   1.594437 &  0.170151 &  4.650709 &  0.465071 &   10.144191 &   0.751700 &   15.630996 &   1.139252 &        -- &        -- &   27.238466 &   2.723847 &   30.473024 &   6.811199 &   53.872778 &   16.332824 &   54.379025 &   20.295716 &  32.772184 &  20.160709 &     -- &      -- &  0.600 &   0.0600 &     -- &       -- &  0.799389 &   4.822243 &  1.831622 &  1.155856 &   0.256000 &      0.178401 &   0.448828 &    5.600000 &   0.929834 &   4.822243 &  18.671719 &   14 &  0.70 \\
%IRAS20050.4 &  20h07m05.9s +27d28m59.2s &  Clustered &  0.000700 &  0.000700 &  0.000900 &  0.000900 &  0.001500 &  0.002000 &  0.023476 &  0.002743 &  0.039114 &  0.003911 &   0.052865 &  0.007526 &   0.054601 &  0.007638 &   0.059474 &  0.053307 &   0.251180 &  0.196063 &        -- &        -- &    8.537096 &   0.795371 &   12.845401 &   1.251667 &        -- &        -- &          -- &         -- &          -- &         -- &          -- &          -- &          -- &          -- &         -- &         -- &     -- &      -- &     -- &       -- &  1.800 &   0.2000 &  2.087205 &  14.943041 &  1.709257 &  0.266312 &   0.384000 &      0.324725 &  19.173673 &   48.450001 &   8.820868 &  14.943041 &  42.536900 &    5 &  1.30 \\
%IRAS20050.2 &  20h07m06.2s +27d28m49.1s &  Clustered &  0.000700 &  0.000700 &  0.000900 &  0.000900 &  0.001500 &  0.002000 &  0.040636 &  0.004064 &  0.142323 &  0.014232 &   0.263840 &  0.026384 &   0.308258 &  0.030826 &   0.059474 &  0.055310 &   1.449680 &  0.191045 &        -- &        -- &    9.309298 &   0.715988 &   11.959660 &   1.193237 &        -- &        -- &          -- &         -- &          -- &         -- &          -- &          -- &          -- &          -- &         -- &         -- &     -- &      -- &     -- &       -- &     -- &       -- &  2.280466 &   8.026514 &  1.643030 &  0.771839 &   0.577000 &      0.216318 &         -- &   26.600000 &   6.046117 &   8.026514 &  18.671719 &   14 &  0.70 \\
%NGC1333.1   &  03h29m07.7s +31d21m57.0s &   Isolated &  0.001240 &  0.000124 &  0.003087 &  0.000309 &  0.044950 &  0.004495 &  0.696000 &  0.069600 &  1.800000 &  0.180000 &   3.060000 &  0.306000 &   2.550000 &  0.255000 &   0.225415 &  0.169458 &   1.501840 &  0.207759 &        -- &  0.260000 &    6.886420 &   0.639683 &   10.993506 &   0.947935 &     49.30 &     4.930 &   52.723662 &   5.272366 &   66.528595 &  35.196775 &   71.541448 &   14.257565 &   45.559105 &   17.857228 &  24.263509 &  16.300584 &     -- &      -- &  1.300 &   0.1300 &     -- &       -- &  0.746145 &   8.384582 &  0.279856 &  3.398432 &   0.004000 &      0.004675 &   0.972460 &   32.500000 &   7.795440 &   8.384582 &  50.833290 &   14 &  1.30 \\
%NGC1333.3   &  03h29m01.5s +31d20m20.5s &   Isolated &  0.000801 &  0.000080 &  0.002865 &  0.000286 &  0.029562 &  0.002956 &  0.544000 &  0.054400 &  1.090000 &  0.109000 &   1.690000 &  0.211000 &   3.060000 &  0.306000 &   1.680629 &  0.130992 &   6.901880 &  0.493400 &        -- &  0.069000 &    9.255659 &   0.655862 &    9.406444 &   0.694632 &     23.40 &     2.340 &   20.217963 &   2.021796 &   78.315869 &   7.831587 &  101.472321 &   18.943177 &   70.906602 &   17.370831 &  40.866774 &  11.474091 &     -- &      -- &  1.500 &   0.1500 &     -- &       -- &  0.904212 &   8.104207 &  0.713901 &  3.393613 &   0.004000 &      0.027190 &   1.122069 &    3.500000 &   2.055202 &   8.104207 &   0.000000 &   14 &  1.15 \\
%Oph.6       &  16h27m15.7s -24d38m45.8s &   Isolated &  0.000700 &  0.000700 &  0.000900 &  0.000900 &  0.001500 &  0.002000 &  0.000653 &  0.001670 &  0.029700 &  0.002970 &   0.037400 &  0.011300 &   0.059200 &  0.022800 &   0.430812 &  0.140062 &   4.305059 &  0.331922 &  0.993000 &  0.355000 &    8.538799 &   0.616541 &    8.179484 &   0.625100 &  14000.00 &  1400.000 &   11.849532 &   1.184953 &    9.622496 &   4.264276 &   10.224295 &   10.224295 &   10.464819 &   10.464819 &   8.968175 &   8.968175 &  0.180 &  0.0180 &     -- &       -- &  0.047 &   0.0047 &  1.294131 &   0.767300 &  2.540989 &  0.934818 &   0.001000 &      0.001320 &   0.020026 &   26.579000 &   6.414977 &   0.767300 &  90.000000 &   13 &  0.70 \\
%Oph.2       &  16h26m44.2s -24d34m48.2s &   Isolated &  0.000700 &  0.000700 &  0.003208 &  0.000321 &  0.015231 &  0.002000 &  0.239000 &  0.023900 &  0.744000 &  0.074400 &   1.610000 &  0.161000 &   2.240000 &  0.224000 &   0.785942 &  0.102364 &   3.190894 &  0.294129 &        -- &  0.065300 &    7.548146 &   0.601857 &    7.941138 &   0.766174 &      8.12 &     0.812 &    8.830506 &   0.883051 &   10.640690 &   1.882673 &    8.527568 &    5.283352 &   10.097091 &    7.482095 &   8.455704 &   8.455704 &  0.220 &  0.0220 &     -- &       -- &  0.120 &   0.0120 &  0.925523 &   1.190757 &  0.828853 &  2.083560 &   0.001000 &      0.002054 &   0.051130 &   32.274502 &  21.100348 &   1.190757 &  83.957672 &   14 &  1.00 \\
%IRAS20050.1 &  20h07m06.6s +27d28m48.0s &  Clustered &  0.002113 &  0.000211 &  0.042102 &  0.004210 &  0.213959 &  0.021396 &  0.489159 &  0.048916 &  0.570047 &  0.057005 &   0.731244 &  0.073124 &   0.857892 &  0.085789 &   0.643347 &  0.067657 &   1.932588 &  0.202346 &        -- &        -- &    4.497465 &   0.352530 &    6.323872 &   0.593341 &        -- &        -- &          -- &         -- &          -- &         -- &          -- &          -- &          -- &          -- &         -- &         -- &     -- &      -- &     -- &       -- &     -- &       -- &  0.000000 &  14.907880 &  0.071313 &  0.741234 &   0.004000 &      0.000000 &         -- &  128.000000 &  15.305788 &  14.907880 &  65.098938 &    9 &  0.85 \\
%Oph.9       &  16h27m21.8s -24d29m53.7s &   Isolated &  0.000700 &  0.000700 &  0.000900 &  0.000900 &  0.031127 &  0.003113 &  0.467000 &  0.046700 &  0.925000 &  0.092500 &   1.440000 &  0.144000 &   1.730000 &  0.173000 &   1.167176 &  0.118828 &   3.424540 &  0.304219 &  4.360000 &  0.436000 &    6.027531 &   0.494175 &    5.806116 &   0.729619 &      5.11 &     0.521 &    4.564878 &   0.456488 &    3.191735 &   1.066859 &          -- &          -- &          -- &          -- &         -- &         -- &  0.030 &  0.0030 &     -- &       -- &  0.020 &   0.0020 &  0.000000 &   0.990687 &  0.505343 &  2.076775 &   0.001000 &      0.000000 &   0.008522 &   11.815999 &   1.159392 &   0.990687 &  80.915283 &   14 &  0.70 \\
%IRAS20050.5 &  20h07m06.6s +27d28m53.1s &  Clustered &  0.001892 &  0.000189 &  0.010053 &  0.001005 &  0.042455 &  0.004246 &  0.118129 &  0.011813 &  0.176229 &  0.017623 &   0.235331 &  0.023533 &   0.319871 &  0.031987 &   0.186287 &  0.051583 &   1.030396 &  0.212319 &        -- &        -- &    2.969365 &   0.325861 &    5.644845 &   0.651528 &        -- &        -- &          -- &         -- &          -- &         -- &          -- &          -- &          -- &          -- &         -- &         -- &     -- &      -- &     -- &       -- &     -- &       -- &  2.066712 &   5.844922 &  0.536668 &  0.781840 &   0.010000 &      0.002597 &         -- &   49.361000 &   6.241276 &   5.844922 &  42.536900 &   14 &  1.00 \\
%Oph.10      &  16h27m17.5s -24d28m55.0s &   Isolated &  0.000700 &  0.000700 &  0.001752 &  0.000175 &  0.015500 &  0.002000 &  0.127000 &  0.012700 &  0.206000 &  0.020600 &   0.286000 &  0.028600 &   0.268000 &  0.026800 &   0.164212 &  0.069534 &   0.668142 &  0.203877 &  0.780000 &  0.078000 &    1.718547 &   0.203886 &    3.740913 &   0.488908 &     12.20 &     1.220 &    6.927097 &   0.692710 &   18.971498 &   2.276556 &   15.879686 &    8.555977 &   11.114462 &   11.114462 &         -- &         -- &  0.093 &  0.0093 &  0.230 &   0.0230 &  0.010 &   0.0010 &  1.260146 &   0.563149 &  0.503234 &  1.378907 &   0.003000 &      0.000601 &   0.004261 &    5.000000 &   1.609805 &   0.563149 &  80.915283 &   14 &  1.00 \\
%Oph.5       &  16h27m06.8s -24d38m15.4s &   Isolated &  0.000700 &  0.000700 &  0.001958 &  0.000196 &  0.027260 &  0.002726 &  0.240000 &  0.024000 &  0.416000 &  0.041600 &   0.553000 &  0.055300 &   0.695000 &  0.069500 &   0.335569 &  0.065278 &   1.179738 &  0.156206 &  2.790000 &  0.279000 &    2.357097 &   0.231135 &    3.054466 &   0.429695 &      6.07 &     0.607 &    4.098763 &   1.070658 &    7.596331 &   3.955141 &    5.783070 &    5.783070 &    3.851165 &    3.851165 &   2.548819 &   2.548819 &  0.100 &  0.0100 &  0.160 &   0.0160 &  0.075 &   0.0075 &  1.307895 &   0.544752 &  0.311926 &  1.355876 &   0.001000 &      0.000000 &   0.031956 &    4.250000 &   0.463816 &   0.544752 &  80.915283 &   14 &  0.85 \\
%NGC1333.4   &  03h29m11.1s +31d18m30.8s &   Isolated &  0.000700 &  0.000700 &  0.000900 &  0.000900 &  0.001500 &  0.002000 &  0.000718 &  0.000072 &  0.003940 &  0.000394 &   0.005000 &  0.000500 &   0.004328 &  0.000433 &   0.097076 &  0.060302 &   0.076183 &  0.115364 &  0.607462 &  0.060746 &    1.784878 &   0.208835 &    3.039983 &   0.341298 &        -- &        -- &   16.609251 &   1.660925 &   53.689046 &   5.368905 &   57.215194 &    6.292760 &   38.448845 &    6.032870 &  18.593682 &   4.666282 &     -- &      -- &  2.000 &   0.2000 &     -- &       -- &  1.102551 &   3.055830 &  1.864111 &  0.830834 &   2.919000 &      0.446509 &   1.496092 &    2.331000 &   0.350655 &   3.055830 &  18.671719 &   11 &  0.70 \\
%NGC1333.7   &  03h28m43.4s +31d17m34.8s &   Isolated &  0.000700 &  0.000700 &  0.000900 &  0.000900 &  0.001500 &  0.002000 &  0.178000 &  0.017800 &  0.351000 &  0.035100 &   0.488000 &  0.048800 &   0.771000 &  0.077100 &   0.702686 &  0.072989 &   1.608865 &  0.177482 &  1.880000 &  0.188000 &    2.083729 &   0.234452 &    2.613522 &   0.346819 &      2.08 &     0.208 &    1.674428 &   0.167443 &    2.587022 &   0.590856 &    7.939325 &    2.613854 &    7.999988 &    3.217143 &   6.634023 &   2.677956 &     -- &      -- &     -- &       -- &     -- &       -- &  1.189289 &   1.356686 &  1.076204 &  1.832949 &   0.001000 &      0.000995 &         -- &    9.562500 &   1.758680 &   1.356686 &  58.243137 &    0 &  0.70 \\
%Oph.16      &  16h26m24.1s -24d24m48.3s &   Isolated &  0.057081 &  0.005708 &  0.333503 &  0.033350 &  0.784026 &  0.078403 &  0.755000 &  0.202000 &  2.100000 &  0.210000 &   2.440000 &  0.244000 &   2.690000 &  0.269000 &   1.780024 &  0.196095 &   1.821645 &  0.268900 &  2.390000 &  0.239000 &    1.602580 &   0.298194 &    2.370660 &   0.566745 &        -- &        -- &    3.481016 &   0.711042 &   18.760602 &  18.760602 &  150.022181 &  150.022181 &  103.094646 &  103.094646 &  51.331864 &  51.331864 &     -- &      -- &     -- &       -- &     -- &       -- &  1.797880 &   2.175843 & -0.755982 &  1.869389 &   0.001000 &      0.000000 &         -- &   17.716999 &   2.923584 &   2.175843 &  77.846802 &   10 &  0.70 \\
%NGC1333.5   &  03h29m10.6s +31d18m19.6s &   Isolated &  0.000700 &  0.000700 &  0.000900 &  0.000900 &  0.001500 &  0.002000 &  0.001840 &  0.000184 &  0.006690 &  0.000669 &   0.010300 &  0.001030 &   0.010900 &  0.001090 &   0.114497 &  0.092701 &   0.150284 &  0.118763 &  0.771000 &  0.077100 &    1.946230 &   0.233640 &    2.166457 &   0.377374 &     20.60 &     2.060 &   14.626733 &   1.462673 &   49.867882 &   4.986788 &   52.535950 &    6.166298 &   36.231726 &    6.188723 &  18.007443 &   4.878490 &     -- &      -- &  2.000 &   0.2000 &     -- &       -- &  1.623470 &   2.786059 &  1.704848 &  0.765793 &   1.297000 &      0.326535 &   1.496092 &    1.258000 &   0.261528 &   2.786059 &  18.671719 &   14 &  0.85 \\
%IRAS20050.7 &  20h07m07.9s +27d27m15.8s &   Isolated &  0.000700 &  0.000700 &  0.000900 &  0.000900 &  0.001500 &  0.002000 &  0.003867 &  0.000387 &  0.024170 &  0.002417 &   0.059890 &  0.005989 &   0.072165 &  0.007216 &   0.059474 &  0.052650 &   0.111262 &  0.057550 &  0.659962 &  0.065996 &    1.146484 &   0.143348 &    2.087927 &   0.311746 &        -- &        -- &          -- &         -- &          -- &         -- &          -- &          -- &          -- &          -- &         -- &         -- &     -- &      -- &     -- &       -- &  0.300 &   0.0600 &  0.000000 &   3.057660 &  1.273086 &  1.499519 &   0.015000 &      0.085220 &   3.195612 &    3.500000 &   3.609952 &   3.057660 &   0.000000 &   14 &  0.70 \\
%Oph.13      &  16h27m30.1s -24d27m43.3s &   Isolated &  0.001186 &  0.000119 &  0.025369 &  0.002537 &  0.163806 &  0.016381 &  0.740000 &  0.074000 &  1.190000 &  0.119000 &   1.580000 &  0.158000 &   2.040000 &  0.204000 &   1.382168 &  0.139436 &   1.806742 &  0.186310 &  1.720000 &  0.172000 &    2.014207 &   0.226905 &    1.984322 &   0.550785 &      7.04 &     0.704 &    4.928438 &   0.492844 &   37.384324 &  13.912175 &   62.016241 &   11.593029 &   48.591685 &   16.514025 &  36.513717 &  13.273109 &  1.200 &  0.1200 &  0.620 &   0.0620 &  0.020 &   0.0020 &  0.000000 &   1.491269 & -0.373531 &  2.225736 &   0.001000 &      0.000000 &   0.008522 &   17.716999 &   5.501830 &   1.491269 &  80.915283 &   14 &  0.70 \\
%NGC1333.6   &  03h29m13.0s +31d18m13.8s &   Isolated &  0.000700 &  0.000700 &  0.000900 &  0.000900 &  0.001506 &  0.002000 &  0.045600 &  0.004560 &  0.180000 &  0.018000 &   0.274000 &  0.027400 &   0.320000 &  0.032000 &   0.160283 &  0.034904 &   0.570316 &  0.092861 &  0.735000 &  0.073500 &    1.445675 &   0.180093 &    1.805938 &   0.344682 &      4.29 &     0.429 &    3.882886 &   3.882886 &   13.332256 &  13.332256 &   29.105069 &    6.271950 &   34.780780 &    8.007092 &  21.255056 &   6.628205 &     -- &      -- &  0.630 &   0.0630 &     -- &       -- &  0.950532 &   1.495050 &  1.001035 &  1.205061 &   0.001000 &      0.000698 &   0.471269 &    7.500000 &   1.188587 &   1.495050 &  26.525352 &   14 &  1.30 \\
%Oph.15      &  16h27m29.4s -24d39m16.6s &   Isolated &  0.000700 &  0.000700 &  0.009026 &  0.000903 &  0.077541 &  0.007754 &  0.172000 &  0.017200 &  0.271000 &  0.027100 &   0.402000 &  0.040200 &   0.411000 &  0.041100 &   0.681783 &  0.112742 &   1.527776 &  0.367109 &  0.639000 &  0.264000 &    2.012071 &   0.344345 &    1.631132 &   0.599339 &        -- &        -- &    3.615509 &   0.361551 &    9.898856 &   4.256740 &   20.072302 &   15.115748 &   18.185662 &   18.185662 &  15.017925 &  15.017925 &  0.180 &  0.0180 &     -- &       -- &  0.045 &   0.0045 &  1.250547 &   0.565569 &  0.098897 &  1.122236 &   0.004000 &      0.000699 &   0.019174 &    3.330000 &   0.410340 &   0.565569 &  26.525352 &   14 &  1.00 \\
%Oph.17      &  16h26m23.6s -24d24m39.4s &   Isolated &  0.001519 &  0.000152 &  0.012856 &  0.001286 &  0.054241 &  0.005424 &  0.297000 &  0.029700 &  0.431000 &  0.043100 &   0.561000 &  0.056100 &   0.619000 &  0.061900 &   0.452223 &  0.231609 &   0.968562 &  0.471099 &  1.010000 &  0.101000 &    1.410939 &   0.394304 &    1.457149 &   0.449171 &        -- &        -- &    3.772271 &   0.478031 &   46.248139 &  46.248139 &  144.052854 &  144.052854 &   95.975629 &   95.975629 &  52.648627 &  52.648627 &     -- &      -- &     -- &       -- &     -- &       -- &  0.955404 &   1.286765 & -0.087406 &  1.208607 &   0.001000 &      0.000000 &         -- &    5.250000 &   0.646469 &   1.286765 &  80.915283 &   14 &  0.70 \\
%Oph.19      &  16h26m30.5s -24d22m59.9s &   Isolated &  0.000700 &  0.000700 &  0.000900 &  0.000900 &  0.001500 &  0.002000 &  0.286000 &  0.028600 &  0.371000 &  0.037100 &   0.438000 &  0.043800 &   0.466000 &  0.046600 &   0.168926 &  0.157077 &   0.637756 &  0.124966 &  0.639000 &  0.161000 &    0.597403 &   0.399042 &    1.297142 &   0.884549 &        -- &        -- &          -- &         -- &   59.165540 &  59.165540 &  101.675272 &   48.875106 &   91.340358 &   91.340358 &  67.219849 &  50.880915 &  0.330 &  0.0330 &     -- &       -- &  0.020 &   0.0020 &  2.510181 &   1.241395 &  0.571762 &  0.893497 &   0.001000 &      0.000633 &   0.008522 &    5.250000 &   0.997850 &   1.241395 &  74.742477 &   14 &  0.70 \\
%Oph.18      &  16h26m17.2s -24d23m45.1s &   Isolated &  0.000700 &  0.000700 &  0.000900 &  0.000900 &  0.008386 &  0.002000 &  0.045300 &  0.004530 &  0.087700 &  0.008770 &   0.161000 &  0.016100 &   0.237000 &  0.023700 &   0.300678 &  0.148358 &   0.672159 &  0.191628 &  0.516000 &  0.051600 &    1.201269 &   0.274837 &    1.286545 &   0.539335 &        -- &        -- &    1.559830 &   1.559830 &          -- &         -- &   12.703012 &   12.703012 &   15.149902 &   15.149902 &         -- &         -- &  0.210 &  0.0210 &     -- &       -- &  0.085 &   0.0085 &  1.180654 &   0.281605 &  0.628201 &  1.341163 &   0.003000 &      0.003573 &   0.036217 &    2.830500 &   0.919428 &   0.281605 &  80.915283 &   14 &  0.85 \\
%IRAS20050.6 &  20h07m02.2s +27d30m26.0s &   Isolated &  0.077284 &  0.007728 &  0.094689 &  0.009469 &  0.155428 &  0.015543 &  0.537046 &  0.053705 &  0.770690 &  0.077069 &   1.112551 &  0.111255 &   1.805481 &  0.180548 &   1.809239 &  0.132991 &   2.292125 &  0.169155 &  2.288333 &  0.228833 &    1.635956 &   0.136628 &    1.215817 &   0.380688 &        -- &        -- &          -- &         -- &          -- &         -- &          -- &          -- &          -- &          -- &         -- &         -- &     -- &      -- &     -- &       -- &     -- &       -- &  1.404580 &  19.327547 & -0.372075 &  2.221856 &   0.004000 &      0.000000 &         -- &  201.599991 &  32.128288 &  19.327547 &  80.915283 &   14 &  0.70 \\
%Oph.14      &  16h27m28.4s -24d27m21.1s &   Isolated &  0.000807 &  0.000096 &  0.012198 &  0.001220 &  0.060748 &  0.006075 &  0.187000 &  0.018700 &  0.272000 &  0.027200 &   0.382000 &  0.038200 &   0.444882 &  0.044488 &   0.520242 &  0.115900 &   0.872317 &  0.234499 &  0.717000 &  0.071700 &    0.966333 &   0.298191 &    1.003625 &   0.487493 &  11000.00 &  1100.000 &    5.991281 &   0.757259 &   35.124328 &   8.697466 &   69.349126 &    8.553092 &   59.827719 &   10.566376 &  34.294671 &   9.572711 &  1.700 &  0.1700 &  0.950 &   0.0950 &  0.050 &   0.0050 &  1.894093 &   0.950774 & -0.131346 &  1.002063 &   0.001000 &      0.000943 &   0.021304 &    4.250000 &   0.559817 &   0.950774 &  80.915283 &   14 &  0.70 \\
%\end{longtable}
%\end{landscape}