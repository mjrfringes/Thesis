% Chapter 5

\chapter[Concluding remarks]{\setstretch{1}Concluding remarks} % Main chapter title
\label{chap:conclusion}


\epigraph{\setstretch{1}\small\itshape There is nothing like a dream to create the future.}{V. Hugo}

High angular resolution observations in the far-infrared are essential to determine the physical properties of young stars, which are preferentially forming in embedded stellar clusters. Using SOFIA FORCAST, we showed results of a survey of 10 nearby clusters, which observed 70 YSOs and 14 extended sources, between 11 and \SI{37}{\um}. The higher angular resolution of FORCAST allowed us to map these regions at an effective resolution which compares with \Spitzer IRAC at \SI{8}{\um}. We use a radiative transfer modelling tool to fit physical parameters of the point sources in our clusters with sufficient amount of data, and propose a detailed study of IRAS~20050+2720. The poor angular resolution at long wavelengths cause uncertainties in the SEDs, since it is not clear that the measured fluxes are associated with their short-wavelength counterparts. This is an inherent limitation which can cause over-interpretation of SED fitting results, and can be lifted through obtaining higher angular resolution between $\sim$\SI{30}{\um} and $\sim$\SI{300}{\um}, which no existing instrument can provide.

%In order to make progress in our understanding of clustered star-forming regions, increased angular resolution required at all wavelengths. With present instrumentation, it is not possible match observations between $\sim$\SI{25}{\um} (JWST) and $\sim$\SI{400}{\um} (ALMA) are low-resolution

BETTII is a far-infrared balloon-borne interferometer, and a pathfinder towards this type of instruments. It is schedule for its commissioning flight in the Fall of 2016. We discuss the design of the balloon payload and the instrument, and derive its expected sensitivity. As part of this sensitivity analysis effort, we developed a new technique to more accurately determine the spectral sensitivity of spatio-spectral interferometers such as BETTII. These sensitivity predictions are used as requirements to design every aspect of the mission.

In this work we detail our contribution to one aspect of the mission, which is the control system. As a remote-controlled, flying interferometer, achieving phase stability is challenging and requires detailed attention. We propose a control strategy that ensures appropriate phase control and allows us to gain appropriate phase knowledge to reconstruct the interferograms, which contain the scientific data. Finally, we discuss the practical implementation of this strategy, and some preliminary test results which occurred in the Spring of 2016.

\vspace{2cm}

Over the course of 5 years, the BETTII project went from paper drawings to its first flight campaign. I have had the opportunity to be involved in all aspects of the project, which provided me with a unique view of how to build instruments to address a specific scientific question. In addition to the day-to-day engineering challenges, a global vision of the process was acquired, which made this experience irreplaceable.

%I have used scientific data obtained from other far-IR observatories to understand and study star formation in nearby clusters. By trying to interpret the data at hands and relate them to the physics of star formation in clusters, I realized their limitations, and gained a much clearer understanding of the capabilities that are needed for future astronomical telescopes in the far-IR. 

The mechanical, cryogenic, optical, electrical, and software infrastructure developed from scratch for BETTII form a powerful pointed observatory platform that can host various instruments in the future. If BETTII succeeds and is able to obtain more funding over the years, the versatility of its subsystems make them relatively straightforward to repair, enhance, or adapt to future goals. 

The work and thoughts spent on BETTII during the past years, combined with the approaching Decadal Survey discussions, have converged towards a new concept for a potential Probe-class space telescope: the Space High Angular Resolution Probe for the InfraRed (SHARP-IR, pronounced "sharper"). This new concept, which is currently going through the Architecture Design Lab and soon through the Instrument Design Lab at NASA GSFC, could see the full potential of double-Fourier interferometry come to fruition, and provide transformational science in the far-infrared. The concept was unveiled for the first time at the SPIE conference in Edinburgh (Rinehart \textit{et al.} 2016), and synthesizes all of our lessons learned from designing and building BETTII.
%LGM I would not push SHARP-IR as much as you do here.. but that is up to you.




