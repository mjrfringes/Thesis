% Chapter 5

\chapter[Conclusion]{\setstretch{1}Conclusion} % Main chapter title
\label{chap:conclusion}


\epigraph{\setstretch{1}\small\itshape There is nothing like a dream to create the future.}{V. Hugo}


Over the course of 5 years, the BETTII project went from paper drawings to its first flight campaign. I have had the opportunity to be involved in all aspects of the project, which provided me with a unique view of how to build instruments to address a specific scientific question. In addition to the day-to-day engineering challenges, a global vision of the process was acquired, which made this experience irreplaceable.

I have used scientific data obtained from other far-IR observatories to understand and study star formation in nearby clusters. By trying to interpret the data at hands and relate them to the physics of star formation in clusters, I realized their limitations, and gained a much clearer understanding of the capabilities that are needed for future astronomical telescopes in the far-IR. 

Regardless of the success of BETTII's first flight campaign, the mechanical, electrical, and software infrastructure developed from scratch for BETTII form a powerful pointed observatory platform that can host various instruments in the future. If BETTII succeeds and is able to obtain more funding over the years, the versatility of its subsystems make them relatively straightforward to repair, enhance, or adapt to future goals.

The work and thoughts spent on BETTII during the past years, combined with the approaching Decadal Survey discussions, have converged towards a new concept for a potential Probe-class space telescope: the Space High Angular Resolution Probe for the InfraRed (SHARP-IR, pronounced "sharper"). This new concept, which is currently going through the Architecture Design Lab and soon through the Instrument Design Lab at NASA GSFC, could see the full potential of double-Fourier interferometry come to fruition, and provide transformational science in the far-infrared. The concept was unveiled for the first time at the SPIE conference in Edinburgh (Rinehart \textit{et al.} 2016), and synthesizes all of our lessons learned from designing and building BETTII.




