\section{Derivation of requirements from the science}

Chapter~\ref{chap:phasenoisepaper} sets the general background to double-Fourier interferometry when used mostly in spectroscopy mode. It sets the mathematical formalism to estimate the spectral sensitivity, given various sources of gaussian noises. 

In this section, we see more directly how this applies to BETTII, and how the system is designed to satisfy these requirements in order to guarantee good observations.

\subsection{	Relevant timescales (flows well because it is relevant to chapter 2)}
\subsection{	Expected perturbations}
\subsubsection{	High-altitude winds and pendulum motions}
Insert here the picture of the pendulum mode showing the error introduced due to mispointing
\subsubsection{	High-frequency embedded perturbations}
Show here a PSD of the noise spectrum recorded by the gyroscopes, when it is lifted and the wheels are on/off, and when it is on the ground with the wheels on
Need the wheels to be synchronized...
\subsection{	Perturbation rejection requirements }
\subsection{	Required control performance}
\subsection{	Required knowledge performance}
