% Chapter 1

\chapter{Introduction} % Main chapter title

\label{chap:introduction} % For referencing the chapter elsewhere, use \ref{Chapter1} 

%----------------------------------------------------------------------------------------


\section{Star formation in clustered environments}

\subsection{The physics of star formation}

In here, describe 

\subsection{Clustered environements}

\section{The Ballon Experimental Twin Telescope for Infrared Interferometry}

Start by describing the context for BETTII; interferometry as a means to getting high angular resolution
\subsection{A brief history of interferometry}
I have always wanted to talk about this!

\subsection{Basics of interferometry}

\subsubsection{Fourier transform spectroscopy}
Using Michelson interferometry to determine the spectrum of a source
\subsubsection{Aperture synthesis}
Mention here details of interferometry, start by using the appendix 3 of the paper (chap2). 

Pictures:
\begin{itemize}
\item Picture of single wavelength interferogram (use Python to generate values, and tikz to display?)
\item Picture of multi-wavelength interferogram, with envelope, etc.
- Optics diagram of two-aperture Michelson interferometer FTS vs single-aperture FTS
\end{itemize}

Watch out, we are being redundant with chapter 2. what do we do?

\subsubsection{Double-Fourier interferometry}

Explain how this comes together.

Pictures:
\begin{itemize}
\item Put here the picture of the detector plane (again, taken from the paper!)
\end{itemize}


\subsection{BETTII Instrument design}

Mention that we cover the control system in detail in chapter 3.

\subsubsection{Overview}
talk about how it works in general. Watch out not to be too redundant with controls chapter. Explain the controls-optics-dewar paradigm. Also, what are the scientific products we expect? SHow all the instrument's parameters (wavelength coverage, angular resolution, etc).

\subsubsection{Mechanical}
defer description of mechanisms to a later section
\subsubsection{Optics}

\subsubsection{Cryostat \& detectors}

Explain briefly here how TES work.

\subsubsection{Data products \& analysis}
Just explain how we get from our detector images to the science-ready products

\subsection{Sensitivity analysis}

\subsubsection{Far-IR background noise estimation}
Stack up all of the noise sources
\subsubsection{Interferometric visibility budget}
both in the science and the tracking channel
\subsubsection{Science channel estimated sensitivity}
make sure to add the formulas that we use, so that this is useful to others

\subsubsection{Tracking channel estimated sensitivity}
make sure to add the formulas that we use

\subsection{Scientific targets}

introduce the SOFIA work here