% Chapter 4

\chapter[Implementation and on-sky testing]{\setstretch{1}Implementation and on-sky testing} % Main chapter title


%\epigraph{\setstretch{1}\small\itshape Ever tried. Ever failed. No matter. \\ Try again. Fail again. Fail better.}{S. Beckett, \textit{Worstward Ho!}}

\section{Key pre-flight procedures}
\subsection{Inertia measurement}
While CAD models allowed to us to estimate the moment of inertia of the payload, this is only an approximation. For testing and for launch, the payload will be different than the model we have: we will either miss some components because they are not yet installed, or have additional components such as the ballasts, the crush pads, or the weights that are used to balance the payload.
\subsection{Sensor alignment and calibration}
\subsubsection{Gyroscope spectral analysis in flight configuration}
\subsubsection{Orthogonalization of gyroscope mount}

\subsubsection{Alignment of gyroscope mount to star camera}

\subsection{Star camera software}
\subsubsection{Tuning}
Discuss about tuning, catalog, filters, etc
\subsubsection{Calibration}

\section{Test setups and limitations}
\subsection{Talk about the way we test in the high bay, etc}
\subsection{List of test setups: gyro only, gyro+star camera, gyro+star camera+tip/tilts with CCD cameras, gyro+star camera with H1RG;}
\subsection{Explain the communication/data recording approach}


\documentclass{standalone}
\usepackage{tikz}
\usetikzlibrary{shapes,arrows}

\begin{document}
\tikzstyle{block} = [draw, fill=black!20, rectangle, 
    minimum height=3em, minimum width=6em,align=center]
\tikzstyle{input} = [node distance=1cm]
\tikzstyle{output} = [node distance=1cm]
\newpage
\begin{tikzpicture}[auto, >=latex',scale=0.8, every node/.style={transform shape}]
\linespread{1}


% Inputs
\node [input,name=measuredVelocity,shift={(2cm,0cm)}] {$\gyroVec^{\textrm{meas}}_k$};
\node [input,above of=measuredVelocity,name=bias] {$\EstBias_{k|k-N}$};
\node [input,above of=bias,name=attitude,node distance=1cm] {$\EstAttitude_{k-1|k-N}$};
\node [input,above of=measuredVelocity,name=covariance,node distance=3cm] {$\noiseCovMat_k$};
\node [input,below of=measuredVelocity,name=propagation matrices,node distance=4cm] {$\A_{k-1},\B_{k-1},\C_{k-1}$};

% line break
\node [input,name=line break left,above of=covariance,node distance=2cm] {};
\node [input,name=line break left2,below of=line break left,node distance=0.5cm,right] {\large \textbf{Kalman Filter: Prediction}};
\node [input,name=line break right,right of=line break left,node distance=15cm] {\large { }};
\draw[dashed] ([xshift=-1cm]line break left.north west) -- (line break right.north east);


% blocks
\node [block, right of=measuredVelocity,node distance=3cm] (estimate velocity) {Estimate \\ Velocity};
\node [block, right of=estimate velocity,node distance=4cm] (predict attitude) {Predict \\ Attitude};
\node [block, below of=predict attitude,node distance=2cm] (state transition) {State \\ transition};
\node [block, right of=state transition,node distance=4cm] (estimate covariance) {Estimate \\ covariance};
\node [block, below of=estimate covariance,node distance=2cm] (update propagation matrices) {Propagate \\ matrices};

% outputs
\node [output,right of=predict attitude,node distance=3cm,name=estattitude]{};
\node [output,right of=estimate covariance,node distance=3cm,name=estcovariance]{};
\node [output,right of=update propagation matrices,node distance=3cm,name=propmat]{};

% arrows
\draw[->] (bias.east) -| (estimate velocity.north);
\draw[->] (measuredVelocity.east) -- (estimate velocity.west);

\node[name=mid,right of=estimate velocity,node distance=2cm,draw,fill,minimum size=3pt,circle]{};
\node[above] at (mid){$\EstGyroVec_{k|k-N}$};
\draw[->] (estimate velocity.east) -- (mid) -- (predict attitude.west);
\draw[->] (attitude.east) -| (predict attitude.north);
\draw[->] (mid) |- (state transition.west) ;
\node[name=mid,right of=state transition,node distance=2cm,draw,fill,minimum size=3pt,circle]{};
\node[above] at (mid){$\StateTransitionMat_k$};
\draw[->] (state transition.east) -- (mid) -- (estimate covariance.west) ;
\node[name=mid2,below of=mid,node distance=1cm,outer sep=0cm,inner sep=0cm]{};
\draw[-] (mid) --  (mid2.north) ;
\draw[->] (mid2.north) -|  (update propagation matrices.north) ;

\node[name=mid,right of=estimate covariance,node distance=3cm]{};
\node[above] at (mid){$\stateCovMat_{k|k-N}$};
\draw[->] (estimate covariance.east) -- (estcovariance) ;
\draw[->] (covariance.east) -| (estimate covariance.north) ;

\node[name=mid,right of=predict attitude,node distance=3cm]{};
\node[above] at (mid){$\EstAttitude_{k|k-N}$};
\draw[->] (predict attitude.east) -- (mid);

\draw[->] (propagation matrices.east) -- (update propagation matrices.west);
\draw[->] (update propagation matrices.east) -- (propmat);
\node[name=mid,right of=update propagation matrices,node distance=3cm]{};
\node[above] at (mid){$\A_{k},\B_{k},\C_{k}$};


% Kalman Update
% line break
\node [input,name=line break left,below of=propagation matrices,node distance=1.5cm] {};
\node [input,name=line break left2,below of=line break left,node distance=0.5cm,right] {\large \textbf{Kalman Filter: Update}};
\node [input,name=line break right,right of=line break left,node distance=15cm] {\large { }};
\draw[dashed] ([xshift=-1cm]line break left.north west) -- (line break right.north east);

% Inputs
\node [input,name=measured attitude,below of=line break left,node distance=3cm] {$\Attitude^\textrm{meas}_{\starcam}$};
\node [input,name=final matrices,above of=measured attitude,node distance=1cm] {$\A_{k},\B_{k},\C_{k}$};
\node [input,name=covmat,below of=measured attitude,node distance=6cm] {$\stateCovMat_{k|k-N}$};
\node [input,name=meascovmat,below of=measured attitude,node distance=2cm] {$\measCovMat_{k}$};

% blocks
\node [block, right of=measured attitude,node distance=3cm] (propagate attitude) {Rotate \& \\ Propagate};
\node [block, below of=propagate attitude,node distance=2cm] (innovation) {Innovation};
\node [block, below of=innovation,node distance=2cm] (innovation covariance) {Innovation \\ covariance};
\node [block, below of=innovation covariance,node distance=2cm] (kalman gain) {Kalman \\ gain};
\node [block, right of=kalman gain,node distance=4cm] (calculate error) {Calculate \\ error};
\node [block, right of=calculate error,node distance=4cm] (update bias) {Update \\ bias};
\node [block, below of=update bias,node distance=2cm] (update attitude) {Update \\ attitude};
\node [block, above of=update bias,shift={(3cm,1cm)}] (update velocity) {Update \\ velocity};
\node [block, below of=update attitude,node distance=2cm] (update covariance) {Update \\ covariance};

% arrows
\draw[->] (measured attitude) -- (propagate attitude);
\draw[->] (final matrices) -| (propagate attitude);
\draw[->] (propagate attitude.south) -- (innovation.north);
\draw[->] (innovation covariance.south) -| (kalman gain.north);
\draw[->] (covmat.east) -- (kalman gain.west);
\draw[->] (covmat.north) |- (innovation covariance.west);
\draw[->] (meascovmat.east) -- (innovation.west);
\draw[->] (covmat.south) |- (update covariance.190);

% intermediary nodes
\node[name=mid,right of=kalman gain,node distance=2cm,draw,fill,minimum size=3pt,circle]{};
\node[above] at (mid){$\KalmanGain_{k}$};
\draw[->] (kalman gain.east) -- (mid) -- (calculate error.west);
\draw[->] (mid) |- (update covariance.165);

\node[name=mid,below of=propagate attitude,node distance=1cm]{};
\node[left] at (mid){$\Attitude^{\textrm{meas}}_{k}$};


\node[name=mid,below of=innovation,node distance=1cm,draw,fill,minimum size=3pt,circle]{};
\node[left] at (mid){$\zMeasurement_{k}$};
\draw[->] (mid) -| (calculate error.north);
\draw[->] (innovation.south) -- (mid) -- (innovation covariance.north);

\node[name=mid,below of=innovation covariance,node distance=1cm]{};
\node[left] at (mid){$\measErrCovMat_{k}$};


\node[name=mid,right of=calculate error,node distance=2cm,draw,fill,minimum size=3pt,circle]{};
\node[above] at (mid){$\ErrorState_{k|k}$};
\draw[->] (calculate error.east) -- (mid) -- (update bias.west);

\node [input,name=bias estimate,above of=mid,shift={(1cm,0cm)}] {$\EstBias_{k|k-N}$};
\draw[->] (bias estimate) -| (update bias.north);

\node [input,name=attitude estimate,below of=bias estimate,node distance=2cm] {$\EstAttitude_{k|k-N}$};
\draw[->] (attitude estimate) -| (update attitude.north);


\node [input,name=velocity estimate,above of=bias estimate,node distance=2cm] {$\EstGyroVec_{k|k-N}$};
\draw[->] (velocity estimate) -| (update velocity.north);
\draw[->] (mid) |- (update attitude.west);

\node [output,name=velocity,right of=update velocity,node distance=2cm]{}; \node [output,right of=update velocity,node distance=2cm,above] {$\EstGyroVec_{k|k}$};
\draw[->] (update velocity.east) -- (velocity);

% outputs
\node [output,name=mid,right of=update bias,node distance=3cm,draw,fill,minimum size=3pt,circle] {};
\draw[->] (update bias.east) -- (mid);
\draw[->] (mid) -- (update velocity.south);
\node [output,name=bias,right of=mid,node distance=2cm] {};
\node [output,right of=mid,node distance=2cm,above] {$\EstBias_{k|k}$};
\draw[->] (mid) -- (bias.west);
\node [output,name=attitude,right of=update attitude,node distance=2cm] {};
\node [output,right of=update attitude,node distance=2cm,above] {$\EstAttitude_{k|k}$};
\draw[->] (update attitude) -- (attitude);

\node [output,name=covariances,right of=update covariance,node distance=2cm] {};
\node [output,right of=update covariance,node distance=2cm,above] {$\stateCovMat_{k|k}$};
\draw[->] (update covariance) -- (covariances);

\end{tikzpicture}
\end{document}

\subsection{Autofocus algorithm and performance}

\section{Estimator implementation}
\subsection{	Gondola attitude estimator}
\subsubsection{Testing the Kalman filter software with simulated data}
\subsubsection{Test results when sitting on the ground}
\subsection{Telescope attitude estimator}
Link to cross-elevation
\subsection{Phase estimator [is that Arnab’s realm?]}

\section{Operating modes}
\subsection{Track mode}
\subsection{	Slew mode}
\subsection{	Acquire mode [this one is contingent on the telescopes working, and is not perfectly representative when only using one single side of the payload]}

\section{Pointing tests and performance results}
\subsection{	Gyros+star camera}
\subsection{	gyro+star camera+tip/tilts with CCD cameras}
\subsection{	gyro+star camera with H1RG;}

\section{	Using the test results to estimate the flight performance (have to think more about that section)}
\subsection{	Perturbation rejection estimates}
\subsection{	Pointing knowledge predictions}
\subsection{	Pointing control predictions}
